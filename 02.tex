\section{22/02/2024}

\subsection{Wielandt's theorem}

\begin{lemma}
	\label{lemma:4Wielandt}
	Let $G$ be a finite group and $H$ and $K$ be subgroups of $G$ 
        of coprime indices. Then $G=HK$ and $(H:H\cap K)=(G:K)$.
\end{lemma}

\begin{proof}
	Let $D=H\cap K$. Since 
	\[
	(G:D)=\frac{|G|}{|H\cap K|}=(G:H)(H:H\cap K),
	\]
	$(G:H)$ divides $(G:D)$. Similarly, $(G:K)$ divides 
	$(G:D)$. Since $(G:H)$ and $(G:K)$ are coprime, $(G:H)(G:K)$
	divides $(G:D)$. In particular, 
	\[
	\frac{|G|}{|H|}\frac{|G|}{|K|}=(G:H)(G:K)\leq (G:D)=\frac{|G|}{|H\cap K|} 
	\]
	and hence $|G|=|HK|$. Since 
 \[
 |G|=|HK|=|H||K|/|H\cap K|,
 \]
 we conclude that 
	$(G:K)=(H:H\cap K)$.
\end{proof}

\begin{definition}
\index{Normal closure}
Let $G$ be a group and $H$ be a subgroup of $G$. The 
\textbf{normal closure} $H^G$ of $H$ in $G$ is the subgroup 
$H^G=\langle xHx^{-1}:x\in G\rangle$.
\end{definition}

\begin{exercise}
	Let $G$ be a group and $H$ a subgroup of $G$. Prove that $H^G$ is normal in $G$ and that 
	$H^G$ is the smallest normal subgroup of $G$ containing $H$. 
\end{exercise}

% \begin{svgraybox}
% 	Es trivial demostrar que $H^G$ is normal in $G$.  Sea $N$ un subgrupo
% 	normal de $G$ tal que $H\subseteq N$. Since $xHx^{-1}\subseteq xNx^{-1}=N$
% 	para todo $x\in G$, $H\subseteq H^G\subseteq N$. 
% \end{svgraybox}

\begin{example}
Let $G=\Alt_4$ and $H=\{\id,(12)(34)\}$. The normal closure of $H$ in $G$
is 
 \[
 H^G=\{\id,(12)(34),(13)(24),(14)(23)\}\simeq C_2\times C_2.
 \]
\end{example}

\begin{theorem}[Wielandt]
	\label{theorem:Wielandt:solvable}
	Let $G$ be a finite group and $H$, $K$ and $L$ be 
        subgroups of $G$ with pairwise coprime indices. 
        If $H$, $K$ and $L$ are solvable, then $G$ is 
	solvable.
\end{theorem}

%%% TODO: revisar la demostración para escribirla mejor!

\begin{proof}
    Let $G$ be a minimal counterexample. Then $G\ne\{1\}$. Let $N$
	be a minimal normal subgroup of $G$ and $\pi\colon G\to G/N$ be the canonical map. Since $H$, $K$ and $L$ are solvable, 
	the subgroups $\pi(H)=\pi(HN)$, $\pi(K)=\pi(KN)$ and $\pi(L)=\pi(LN)$ of $\pi(G)=G/N$ are solvable. 
	By the correspondence theorem, $\pi(H)$, $\pi(K)$ and $\pi(L)$ 
        have pairwise coprime indices. 
% 	pues por ejemplo\footnote{El núcleo de la restricción
% 	$\ker(\pi|_H)=\ker \pi\cap N$ and entonces $\pi(H)\simeq H/N\cap H$.}
% 	\[
% 	(\pi(G):\pi(H))=(G/N:H/N\cap H)=(G:NH)
% 	\]
% 	divide a $(G:N)$. 
    By the minimality of $G$, $\pi(G)$ is solvable. If $H=\{1\}$, then 
    $|G|=(G:H)$ is coprime with $(G:K)$ and thus $G=K$ is solvable. If 
    $H\ne \{1\}$, let $M$ be a minimal normal subgroup of $H$. By Lemma~\ref{lemma:minimal_normal}, 
    $M$ is a $p$-group for some prime number $p$. 
    Without loss of generality, we may assume that $p$ does not divide $(G:K)$ (otherwise, if $p$ divides $(G:K)$, then 
    $p$ does not divide $(G:L)$ and we just need to change $K$ by $L$). 
    There exists $P\in\Syl_p(G)$ such that $P\subseteq K$. Sylow subgroups are conjugate, so there exists 
    $g\in G$ such that $M\subseteq
	gKg^{-1}$. Since $(G:gKg^{-1})=(G:K)$ is coprime with $(G:H)$, Lemma~\ref{lemma:4Wielandt} implies that
    $G=(gKg^{-1})H$. 
	
    We claim that all conjugates of $M$ are in $gKg^{-1}$. 
    If $x\in G$, write $x=uv$ some some $u\in 
	gKg^{-1}$ and $v\in H$. Since $M$ is normal in $H$, 
	\[
	xMx^{-1}=(uv)M(uv)^{-1}=uMu^{-1}\subseteq gKg^{-1}.
	\]
	In particular, $\{1\}\ne M^G\subseteq gKg^{-1}$ is solvable, as $gKg^{-1}$ is 
	solvable. By the minimality of $M$, the group $G/M^G$ is solvable. Hence 
        $G$ is solvable. 
\end{proof}

\subsection{Hall's theorem}

\begin{definition}
\index{$p$-complement}
Let $G$ be a finite group of order $p^{\alpha}m$, where $p$ is a prime number such that 
$\gcd(p,m)=1$. A subgroup 
$H$ of $G$ is said to be a \textbf{$p$-complement} if $|H|=m$. 
\end{definition}

\begin{example}
Let $G=\Sym_3$. Then $H=\langle (123)\rangle$ is a $2$-complement and 
$K=\langle (12)\rangle$ is a $3$-complement.
\end{example}

\begin{theorem}[Hall]
\label{theorem:Hall:solvable}
 \index{Hall's theorem}
Let $G$ be a finite group that admits a $p$-complement for every prime divisor $p$ of $|G|$. 
Then $G$ is solvable. 
\end{theorem}

\begin{proof}
	Let $|G|=p_1^{\alpha_1}\cdots
	p_k^{\alpha_k}$ with $p_1<\cdots<p_k$ prime numbers. We proceed by induction on $k$. 
	If $k=1$, then the claim holds, as $G$ is a $p$-group. If $k=2$, the result holds by
        Burnside's theorem. Assume now that 
	$k\geq3$. For $j\in\{1,2,3\}$, let $H_j$ be a $p_j$-complement in 
	$G$. Since $|H_j|=|G|/p_j^{\alpha_j}$, the subgroups $H_j$ have pairwise coprime indices.

	We claim that $H_1$ is solvable. Note that $|H_1|=p_2^{\alpha_2}\cdots
	p_k^{\alpha_k}$. Let $p$ be a prime number dividing $|H_1|$ and $Q$ be a 
	$p$-complement in $G$. 
	Since $(G:H_1)$ and $(G:Q)$ are
	coprime, Lemma~\ref{lemma:4Wielandt} implies that  
	\[
	(H_1:H_1\cap Q)=(G:Q). 
	\]
	Then $H_1\cap Q$ is a $p$-complement in $H_1$.  Therefore $H_1$ is
	solvable by the inductive hypothesis. Similarly, both $H_2$ and 
	$H_3$ are solvable.

	Since $H_1$, $H_2$ and $H_3$ are solvable of pairwise coprime indices, 
        the theorem follows from Wieland's theorem. 
\end{proof}

\subsection{Nilpotent groups}

For a group $G$ and $x,y,z\in G$, conjugation will be considered as a left action of $G$ on $G$ 
and we will use the following notation: $\prescript{x}{}y=xyx^{-1}$. The commutator between $x$ and $y$ 
will be written as 
\[
[x,y]=xyx^{-1}y^{-1}=(\prescript{x}{}y)y^{-1}.
\]

We will also use the following notation:  
$[x,y,z]=[x,[y,z]]$. For subgroups $X$, $Y$ and $Z$ of $G$, we write 
$[X,Y,Z]=\left[ X,[Y,Z] \right]$. Note that $[X,Y]=[Y,X]$. 

\begin{exercise}[The Hall--Witt identity]
	\label{exercise:HallWitt}
	\index{Hall--Witt identity}
	\index{Hall, P.}
	\index{Witt, E.}
	Let $G$ be a group and $x,y,z\in G$. Prove that 
	\begin{equation}
		\label{eq:HallWitt}
	\left(\prescript{y}{}[x,y^{-1},z]\right)\left(\prescript{z}{}[y,z^{-1},x]\right)\left(\prescript{x}{}[z,x^{-1},y]\right)=1.
	\end{equation}
\end{exercise}

\index{Jacobi, G.}
\index{Jacobi identity}
If $G$ is a group and $[G,G]$ is central in $G$,
then the Hall-Witt becomes Jacobi's identity.

%\begin{proof}
%	Como la demostración es simplemente un cálculo de rutina, la dejamos como ejercicio.
%	Calculamos 
%	\begin{align*}
%	&\prescript{y}{}[x,y^{-1},z]=yxy^{-1}zyz^{-1}x^{-1}zy^{-1}z^{-1},\\
%	&\prescript{z}{}[y,z^{-1},x]=zyz^{-1}xzx^{-1}y^{-1}xz^{-1}x^{-1},\\
%	&\prescript{x}{}[z,x^{-1},y]=xzx^{-1}yxy^{-1}z^{-1}yx^{-1}y^{-1}.
%	\end{align*}
%\end{proof}
\begin{lemma}[Hall's three subgroups lemma]
	\label{lemma:3subgrupos}
	\index{Lema!de los tres subgrupos}
	Let $X$, $Y$ and $Z$ be subgroups of $G$ 
 such that $[X,Y,Z]=[Y,Z,X]=\{1\}$.
	Then $[Z,X,Y]=\{1\}$.
\end{lemma}

\begin{proof}
Since $[x,y]\in C_G(z)$ implies $[X,Y]\subseteq C_G(Z)$, 
it is enough to prove that $[z,x^{-1},y]=1$ for all $x\in X$, $y\in Y$ and $z\in Z$. Since $[y^{-1},z]\in [Y,Z]$, $[x,y^{-1},z]\in [X,Y,Z]=\{1\}$. Thus $\prescript{y}{}[x,y^{-1},z]=1$. Similarly, $\prescript{z}{}[y,z^{-1},x]=1$. Using the Hall--Witt identity, we conclude that $[z,x^{-1},y]=1$.
\end{proof}

\begin{exercise}
\label{xca:3subgroups}
Let $N$ be a normal subgroup of $G$ and 
$X$, $Y$ and $Z$ be subgroups of $G$. If $[X,Y,Z]\subseteq N$ and $[Y,Z,X]\subseteq N$, then 
$[Z,X,Y]\subseteq N$.
\end{exercise}

%\begin{sol}{xca:3subgroups}
%	Sea $\pi\colon G\to G/N$ el morfismo canónico. Como $[X,Y,Z]\subseteq N$, 
%	\begin{align*}
%		\{1\}&=\pi([X,Y,Z])=\pi([ X,[Y,Z]])\\
%		&=[\pi(X),\pi([Y,Z])]=[ \pi(X),[\pi(Y),\pi(Z)]]=[\pi(X),\pi(Y),\pi(Z)]. 
%	\end{align*}
%	Similarmente $[\pi(Y),\pi(Z),\pi(X)]=\{1\}$. Entonces, gracias al lema de los
%	tres subgrupos, $[\pi(Z),\pi(X),\pi(Y)]=\{1\}$, es decir
%	$[Z,X,Y]\subseteq N$.
%\end{sol}

% \begin{svgraybox}
% 	Sean $h\in H$ y $g\in G$. $hKgK=gKhK$ si y sólo si $[h,g]\in K$.
% \end{svgraybox}

\begin{definition}
\index{Lower central series}
Let $G$ be a group. The \textbf{lower central series} 
is the sequence $\gamma_k(G)$ of subgroups defined inductively 
as 
\[
\gamma_1(G)=G,\quad
\gamma_{i+1}(G)=[G,\gamma_i(G)]\quad i\geq 1.
\]
\end{definition}

\begin{definition}
\index{Group!nilpotent}
\index{Nilpotency index}
A group $G$ is said to be \textbf{nilpotent} if there exists a positive integer $c$ such that 
$\gamma_{c+1}(G)=\{1\}$. The smallest $c$ with $\gamma_{c+1}(G)=\{1\}$ is 
the \textbf{nilpotency class} of $G$.
\end{definition}

\begin{exercise}
\label{xca:nilpotent=>solvable}
Prove that every nilpotent group is solvable. 
\end{exercise}

A group is nilpotent of nilpotency class one if and only if it is abelian. 

\begin{example}
The group $\Sym_3$ is solvable, as 
$\Sym_3\supseteq \Alt_3\supseteq\{1\}$ is a composition series 
with abelian factors. However, $\Sym_3$ is not nilpotent, as 
\[
\gamma_1(\Sym_3)=\Alt_3,\quad
\gamma_2(\Sym_3)=[\Alt_3,\Sym_3]=\Alt_3, 
\]
and therefore $\gamma_i(\Sym_3)\ne\{1\}$ for all $i\geq1$. 
\end{example}

\begin{example}
The group $G=\Alt_4$ is not nilpotent, as 
\[
\gamma_1(G)=G,\quad
\gamma_j(G)=\{\id,(12)(34),(13)(24),(14)(23)\}\simeq C_2\times C_2
\]
for all $j\geq2$. We can do this with the computer:
\begin{lstlisting}
gap> IsNilpotent(AlternatingGroup(4));
false
\end{lstlisting}
Let us do the calculation of the lower central series with the computer: 
\begin{lstlisting}
gap> List(LowerCentralSeries(AlternatingGroup(4)),\
StructureDescription);
[ "A4", "C2 x C2" ]
\end{lstlisting}
Here is an alternative:
\begin{lstlisting}
gap> G := AlternatingGroup(4);;
gap> gamma_1 := G;;
gap> gamma_2 := DerivedSubgroup(G);;
gap> gamma_3 := CommutatorSubgroup(gamma_2,G);;
gap> StructureDescription(gamma_1);
"A4"
gap> StructureDescription(gamma_2);
"C2 x C2"
gap> StructureDescription(gamma_3);
"C2 x C2"
\end{lstlisting}
\end{example}

\begin{exercise}
\label{xca:gamma}
Let $G$ be a group. Prove the following statements: 
\begin{enumerate}
\item Each $\gamma_i(G)$ is a characteristic subgroup of $G$. 
\item $\gamma_i(G)\supseteq\gamma_{i+1}(G)$ for all $i\geq1$.
\item If $f\colon G\to H$ is a surjective group homomorphism, then  
$f(\gamma_i(G))=\gamma_i(H)$ for all $i\geq1$.
\end{enumerate}
\end{exercise}

\begin{exercise}
\label{xca:HxK_nilpotente}
Prove that if $H$ and $K$ are nilpotent groups, then 
$H\times K$ is nilpotent. 
\end{exercise}

\begin{exercise}
\label{xca:nilpotente}
Let $G$ be a nilpotent group. Prove the following statements: 
\begin{enumerate}
\item Subgroups of $G$ are nilpotent. 
\item If $f\colon G\to H$ is a surjective homomorphism, then $H$ is nilpotent.  
\end{enumerate}
\end{exercise}

%\begin{sol}{xca:nilpotent}
%	La primera afirmación es cierta pues $\gamma_i(H)\subseteq\gamma_i(G)$ para
%	todo $i\geq1$. La segunda afirmación: si existe $c$ tal que $\gamma_{c+1}(G)=\{1\}$
%	entonces \[
%	\gamma_{c+1}(H)=f(\gamma_{c+1}(G))=f(\{1\})=\{1\}.\qedhere
%	\]
%\end{sol}

\begin{exercise}
   True or false? If $G$ is a nilpotent group and $N$ is normal 
   subgroup of $G$ such that $N$ and $G/N$ are nilpotent, then 
   $G$ is nilpotent. 
\end{exercise}

%\begin{sol}{xca:nilpotent_notexact}
%    The claim is false. Take $G=\Sym_3$
%    and $N=\langle(123)\rangle$. Then both $N\simeq C_3$ and $G/N\simeq C_2$ are nilpotent, but $G$ is not nilpotent. 
%\end{sol}

\begin{proposition}
\label{pro:nilpotent_pgroups}
Finite $p$-groups are nilpotent.
\end{proposition}

\begin{proof}
We proceed by induction on $|G|$. The case $G=\{1\}$ is trivial. 
Assume the result holds for $p$-groups of order $<|G|$. Since 
$G$ is a $p$-group, $Z(G)\ne\{1\}$. By the inductive hypothesis, 
$G/Z(G)$ is nilpotent. There exists $c$ such that 
$\gamma_{c+1}(G/Z(G))=\{1\}$. 
	
Let $\pi\colon G\to G/Z(G)$ be the canonical map.  
By Exercise~\ref{xca:gamma},
\[
\pi(\gamma_{c+1}(G))=\gamma_{c+1}(G/Z(G))=\{1\}. 
\]
Then $\gamma_{c+1}(G)\subseteq \ker\pi=Z(G)$. Hence
$G$ is nilpotent, as 
\[	
\gamma_{c+2}(G)=[G,\gamma_{c+1}(G)]=[G,Z(G)]=\{1\}.\qedhere
\]
\end{proof}

\begin{theorem}
\label{thm:gamma}
If $G$ is a group, then $[\gamma_i(G),\gamma_j(G)]\subseteq
\gamma_{i+j}(G)$ for all $i,j\geq1$.	
\end{theorem}

\begin{proof}
We proceed by induction on $i$. The case $i=1$ is trivial, as
$[G,\gamma_j(G)]=\gamma_{j+1}(G)$ by definition. Assume that
the result holds for some $i\geq1$ and all $j\geq 1$. 
	
First note that 
\begin{equation*}
[G,\gamma_i(G),\gamma_j(G)]\subseteq [G,\gamma_{i+j}(G)]=\gamma_{i+j+1}(G)
\end{equation*}
by the inductive hypothesis. Moreover, by the inductive hypothesis, 
\begin{equation*}
[\gamma_i(G),\gamma_j(G),G]=[\gamma_i(G),G,\gamma_j(G)]=[\gamma_{i}(G),\gamma_{j+1}(G)]\subseteq \gamma_{i+j+1}(G)
\end{equation*}
By using Exercise~\ref{xca:3subgroups} with $X=G$, $Y=\gamma_i(G)$ and $Z=\gamma_j(G)$,
we get that 
\[
[\gamma_j(G),G,\gamma_i(G)]\subseteq \gamma_{i+j+1}(G).
\]
Hence  
\[
[\gamma_{i+1}(G),\gamma_{j}(G)]=[\gamma_{j}(G),\gamma_{i+1}(G)]=[\gamma_j(G),G,\gamma_i(G)]\subseteq \gamma_{i+j+1}(G).\qedhere
\]
\end{proof}

\index{Weigth of a commutator}
We consider arbitrary commutators, not necessarily associated on the right.   
For example, both $[G,G,G]=[G,[G,G]]$ and $[[G,G],G]$ are commutators of \textbf{weight} three. 

\begin{corollary}
In a group $G$, every weight $n$ commutator is contained in 
$\gamma_n(G)$.
\end{corollary}

\begin{proof}
We proceed by induction on $n$. The case $n=1$ is trivial. Assume that $n\geq1$ and 
the result holds for all $k\leq n$. An arbitrary commutator of weight $n+1$ 
is of the form $[A,B]$, where $A$ is a commutator of weight $k$,
$B$ is a commutator of weight $l$ and $n+1=k+l$. Since $k<n$ and $l<n$, 
the inductive hypothesis implies that  $A\subseteq \gamma_k(G)$ and $B\subseteq
\gamma_l(G)$. Hence $[A,B]\subseteq [\gamma_k(G),\gamma_l(G)]\subseteq
\gamma_{k+l}(G)$ by the previous theorem.
\end{proof}

\begin{exercise}
\label{xca:inclusion}
    Let $G$ be a group. Prove that $G^{(k)}\subseteq G^{2^k}$ for all $k\geq1$. 
\end{exercise}

\begin{exercise}
\label{xca:derived_length}
    Let $G$ be a nilpotent group of class $m$. Prove that
    the derived length of $G$ is $\leq 1+\log_2m$.
\end{exercise}

The following lemma is important. It states that nilpotent groups satisfy 
the \textbf{normalizer condition}. 

\begin{lemma}[normalizer condition]
\label{lem:normalizadora}
\index{Normalizer condition}
Let $G$ be a nilpotent group. If $H$ is a proper subgroup of $G$, then 
$H\subsetneq N_G(H)$.
\end{lemma}

\begin{proof}
There exists $c$ such that $G=\gamma_1(G)\supseteq\cdots\supseteq\gamma_{c+1}(G)=\{1\}$. Since 
$\{1\}=\gamma_{c+1}(G)\subseteq H$ and $\gamma_1(G)\not\subseteq H$, 
let $k$ be the smallest positive integer such that  $\gamma_k(G)\subseteq H$. 
Since
\[
[H,\gamma_{k-1}(G)]\subseteq [G,\gamma_{k-1}(G)]=\gamma_k(G)\subseteq H,
\]
we obtain that  
	%$xhx^{-1}h^{-1}\in H$ para todo $x\in\gamma_{k-1}(G)$ y $h\in
	%H$. Esto implica que 
$xHx^{-1}\subseteq H$ for all $x\in\gamma_{k-1}(G)$,
that is $\gamma_{k-1}(G)\subseteq N_G(H)$. If $N_G(H)=H$, then
$\gamma_{k-1}(G)\subseteq H$, a contradiction to the minimality of $k$. 
\end{proof}

%\begin{example}
%	Un grupo $G$ es nilpotente de clase dos si y sólo $\gamma_2(G)

%\end{example}

For a group $G$, we define the sequence  $\zeta_0(G),\zeta_1(G),\dots$
recursively as follows: 
\[
	\zeta_0(G)=\{1\},\quad
	\zeta_{i+1}(G)=\{g\in G:[x,g]\in\zeta_{i}(G)\text{ for all $x\in G$}\},\quad i\geq 0.
\]
For example, $\zeta_1(G)=Z(G)$.

\begin{lemma}
\label{lem:central_ascendente}
Let $G$ be a group. For every $i\geq0$, the set $\zeta_i(G)$ 
is a normal subgroup of $G$. 
\end{lemma}

\begin{proof}
We proceed by induction on $i$. The case $i=0$ is trivial, as 
$\zeta_0(G)=\{1\}$. Assume the result holds for some $i$.
We claim that $\zeta_{i+1}(G)$ is normal subgroup of $G$. 
Let $g,h\in \zeta_{i+1}(G)$ and $x\in G$. By the inductive hypothesis, 
\begin{align*}
	&[x,g^{-1}]=(xg^{-1})[x^{-1},g](xg^{-1})^{-1}\zeta_i(G)(xg^{-1})^{-1}=\zeta_i(G),\\
	&[x,gh]=[x,h][hxh^{-1},g]\in \zeta_{i}(G).
\end{align*}
Since $1\in\zeta_{i+1}(G)$, we conclude that each $\zeta_i(G)$ is a subgroup of $G$. 
Moreover, $xgx^{-1}\in\zeta_{i+1}(G)$, as  
	\[
	[xgx^{-1},y]=x[g,x^{-1}yx]x^{-1}\in\zeta_{i}(G)
	\]
para todo $y\in G$.
\end{proof}


\begin{definition}
\index{Ascending central series}
Let $G$ be a group. The \textbf{ascending central series} of $G$ 
is the sequence 
\[
\{1\}=\zeta_0(G)\subseteq\zeta_1(G)\subseteq\zeta_2(G)\subseteq\cdots
\]
\end{definition}

\begin{definition}
\index{Group!perfect}
A group $G$ is said to be \textbf{perfect} if $[G,G]=G$.
\end{definition}

\begin{theorem}[Gr\"un]
\label{thm:Grun}
\index{Gr\"un's theorem}
If $G$ is a perfect group, then $Z(G/Z(G))=\{1\}$. 
\end{theorem}

\begin{proof}
    By definition, $[G,\zeta_2(G)]\subseteq Z(G)$ and 
    $[\zeta_2(G),G]\subseteq Z(G)$. Then 
    \[
    [G,G,\zeta_2(G)]=[G,\zeta_2(G),G]=\{1\}.
    \]
    By using the three subgroups lemma with $X=Y=G$ and $Z=\zeta_2(G)$, 
    \[
    [\zeta_2(G),G]=[\zeta_2(G),[G,G]]=[\zeta_2(G),G,G]=\{1\}.
    \]
    Thus $\zeta_2(G)\subseteq Z(G)$. 
    
    We aim to prove that $Z(G/Z(G))$ is trivial. Let $\pi\colon G\to G/Z(G)$ be the canonical map and 
    $g\in G$ be such that $\pi(g)$ is central. Since 
    \[
    1=[\pi(x),\pi(g)]=\pi([x,g])
    \]
    for all $x\in G$, $[x,g]\in Z(G)=\zeta_1(G)$ for all $x\in G$. Hence 
    $g\in\zeta_2(G)\subseteq Z(G)$. 
\end{proof}

For subgroups $H$ and $K$ of $G$, let 
\[
[H,K]=\langle [h,k]:h\in H,\,k\in K\rangle.
\]

\index{Normalizer}
Let $G$ be a group and $K$ be a subgroup of $G$. We say that $K$ \textbf{normalizes} 
$H$ if $K\subseteq N_G(H)$.
\index{Centralizer}
We say that $K$ \textbf{centralizes} 
$H$ if $K\subseteq C_G(H)$, that is if and only if $[H,K]=\{1\}$.

\begin{exercise}
Let $K$ and $H$ be subgroups of $G$ such that $K\subseteq H$ and $K$ is normal in $G$.
Prove that $[H,G]\subseteq K$ if and only if $H/K\subseteq Z(G/K)$. 
\end{exercise}

\begin{lemma}
\label{lem:gamma_zeta}
Let $G$ be a group. There exists an integer $c$ such that 
$\zeta_c(G)=G$ if and only if 
$\gamma_{c+1}(G)=\{1\}$. In this case, 
\[
\gamma_{i+1}(G)\subseteq\zeta_{c-i}(G)
\]
for all $i\in\{0,1,\dots,c\}$. 
\end{lemma}

\begin{proof}
Assume first that $\zeta_c(G)=G$. To prove that 
$\gamma_{i+1}(G)\subseteq\zeta_{c-i}(G)$ holds for all $i$, we proceed by induction. 
The case $i=0$ is trivial. So assume that the result holds for some $i\geq0$. If
$g\in\gamma_{i+2}(G)=[G,\gamma_{i+1}(G)]$, then     
\[
g=\prod_{k=1}^N [x_k,g_k],
\]
for some $g_1,\dots,g_N\in\gamma_{i+1}(G)$ and $x_1,\dots,x_N\in G$. By the inductive 
hypothesis, 
	\[
	g_k\in\gamma_i(G)\subseteq\zeta_{c-i}(G)
	\]
for all $k$. Hence $[x_k,g_k]\in\zeta_{c-i-1}(G)$ for all $k$. Therefore   
$g\in\zeta_{c-(i+1)}(G)$. 
	
We now assume that $\gamma_{c+1}(G)=\{1\}$. We aim to prove that 
$\gamma_{i+1}(G)\subseteq\zeta_{c-i}(G)$ holds for all $i$. We proceed by backwards induction on $i$. 
The case $i=c$ is trivial. So assume the result holds for some $i+1\leq c$. 
Let $g\in\gamma_{i}(G)$. By the inductive hypothesis, 
	\[
	[x,g]\in [G,\gamma_i(G)]=\gamma_{i+1}(G)\subseteq\zeta_{c-i}(G).
	\]
Thus $g\in\zeta_{c-i+1}(G)$ by definition. 
\end{proof}

\begin{example}
Let $G=\Sym_3$. Then $\zeta_j(G)=\{1\}$ for all $j\geq 0$: 
\begin{lstlisting}
gap> UpperCentralSeries(SymmetricGroup(3));
[ Group(()) ]
\end{lstlisting}
\end{example}

\begin{definition}
\index{Central series}
Let $G$ be a group. A \textbf{central series} for $G$ 
is a sequence 
	\[
		G=G_0\supseteq G_1\supseteq\cdots\supseteq G_n=\{1\}
	\]
of normal subgroups of $G$ such that 
for each $i\in\{1,\dots,n\}$, 
$\pi_i(G_{i-1})$ is a subgroup of $Z(G/G_i)$, where  $\pi_i\colon G\to
G/G_i$ is the canonical map.
\end{definition}

\begin{lemma}
\label{lem:central_series}
Let $G=G_0\supseteq G_1\supseteq\cdots\supseteq G_n=\{1\}$ be 
a central series of a group $G$. Then $\gamma_{i+1}(G)\subseteq G_i$ for all $i$.
\end{lemma}

\begin{proof}
We proceed by induction on $i$. The case $i=0$ is trivial. So assume the result holds for some
$i\geq0$. Let  
	$\pi_i\colon G\to
	G/G_i$ be the canonical map. 
	Then  
	\[
	\gamma_{i+1}(G)=[G,\gamma_i(G)]\subseteq [G,G_{i-1}].
	\]
	Since $\pi_i(G_{i-1})\subseteq Z(G/G_{i})$, 
	\[
        \pi_i([G,G_{i-1}])=[\pi_i(G),\pi_i(G_{i-1})]=\{1\}.
        \]
        Hence $\gamma_{i+1}(G)=[G,G_{i-1}]\subseteq G_i$. 
\end{proof}

% extender el lema para ver qué pasa con zeta_i

\begin{theorem}
A group is nilpotent if and only if it admits a central series. 
\end{theorem}

\begin{proof}
Let $G$ be a group. If $G$ is nilpotent, then the $\gamma_j(G)$ form a central series of 
$G$. Conversely, if $G=G_0\supseteq
G_1\supseteq\cdots\supseteq G_n=\{1\}$ is a central series of $G$, 
then, by the previous lemma,  
	\[
	\gamma_{n+1}(G)\subseteq G_n=\{1\}.
	\]
Hence $G$ is nilpotent. 
\end{proof}

\begin{exercise}
\label{xca:nilpotente_central}
Let $G$ be a group. Prove that if $K$ is a subgroup of $Z(G)$ such that 
$G/K$ is nilpotent, then $G$ is nilpotent. 
\end{exercise}

\subsection{Hirsch's theorem}

\begin{theorem}[Hirsch]
\label{thm:Hirsch}
\index{Hirsch's theorem}
Let $G$ be a nilpotent group. If $H$ is a non-trivial normal subgroup of $G$, 
then $H\cap Z(G)\ne\{1\}$. In particular, $Z(G)\ne\{1\}$. 
\end{theorem}

\begin{proof}
Since $\zeta_0(G)=\{1\}$ and there exists an integer $c$ such that $\zeta_c(G)=G$, 
there exists 
\[
m=\min\{k:H\cap\zeta_k(G)\ne\{1\}\}.
\]
Since $H$ is normal in $G$, 
\[
[G,H\cap\zeta_m(G)]\subseteq H\cap[G,\zeta_m(G)]\subseteq H\cap\zeta_{m-1}(G)=\{1\}.
\]
Therefore $\{1\}\ne H\cap\zeta_m(G)\subseteq H\cap Z(G)$. If $H=G$, then $Z(G)\ne\{1\}$. 
\end{proof}

\begin{exercise}
\label{xca:nilpotente_minimalnormal}
Let $G$ be a nilpotent group and $M$ be a minimal normal subgroup of $G$. Prove 
that $M\subseteq Z(G)$.
\end{exercise}

% \begin{svgraybox}
% 	Como $M\cap Z(G)$ es normal en $G$, la minimalidad de $M$ implica que hay
% 	dos posibilidades: $M\cap Z(G)$ es trivial o bien $M=M\cap Z(G)\subseteq Z(G)$.
% 	Por el teorema~\ref{theorem:Z(nilpotent)}, $M\cap Z(G)\ne 1$.
% \end{svgraybox}

\begin{definition}
\index{Subgroup|Maximal normal}
    Let $G$ be a group. A subgroup $M$ is said to be \textbf{maximal normal} in $G$
    if $M\ne G$ and $M$ is the only proper normal subgroup of $G$ containing $M$. 
\end{definition}

\begin{corollary}
Let $G$ be a non-abelian nilpotent group and $A$ be a maximal normal and abelian 
subgroup un subgrup of $G$. Then $A=C_G(A)$.
\end{corollary}

\begin{proof}
Since $A$ is abelian, $A\subseteq C_G(A)$. Assume that $A\ne C_G(A)$.
The centralizer $C_G(A)$ is normal in $G$, as, since $A$ is normal in $G$, 
\[
gC_G(A)g^{-1}=C_G(gAg^{-1})=C_G(A).
\]
for all $g\in G$. Let $\pi\colon G\to G/A$ be the canonical map. 
Then $\pi(C_G(A))$ is a non-trivial normal subgroup of $\pi(G)$. Since 
$G$ is nilpotent, $\pi(G)$ is nilpotent. By Hirsch's theorem, 
\[
\pi(C_G(A))\cap Z(\pi(G))\ne\{1\}.
\]
Let $x\in C_G(A)\setminus A$ be such that $\pi(x)$ is central in $\pi(G)$.  Then 
$\langle A,x\rangle$ is abelian, as $x\in C_G(A)$. Moreover,  $\langle
A,x\rangle$ is normal in $G$, as $A$ is normal in $G$ and 
$gxg^{-1}x^{-1}\in A$ for all  $g\in G$ (because $\pi(x)$ is central). Hence
	$gxg^{-1}\in \langle A,x\rangle$ and therefore $A\subsetneq \langle
	A,x\rangle\subsetneq G$, a contradiction.
\end{proof}

\begin{theorem}
Let $G$ be a nilpotent group. The following statements hold: 
\begin{enumerate}
\item Every minimal normal subgroup of $G$ has prime order and is central. 
\item Every maximal subgroup of $G$ is normal of prime index and contains $[G,G]$. 
\end{enumerate}
\end{theorem}

\begin{proof}\
\begin{enumerate}
    \item Let $N$ be a minimal normal subgroup of $G$. Since 
        $N\cap Z(G)\ne\{1\}$ by Hirsch's theorem, $N\cap Z(G)$ is a normal subgroup of 
        $G$ contained in $N$. Then $N=N\cap Z(G)\subseteq
	Z(G)$ by the minimality of $N$. In particular, $N$ is abelian. Since every subgroup of 
         $N$ is normal in $G$, $N$ is simple. Hence $N\simeq
	C_p$ for some prime number $p$.
 \item  If $M$ is a maximal subgroup, then $M$
is normal in $G$ by the normalizer condition (Lemma \ref{lem:normalizadora}). By the maximality
of $M$, the quotient $G/M$ contains no proper non-trivial subgroups. Thus 
$G/M\simeq C_p$ for some prime $p$. Since 
	$G/M$ is abeliano, $[G,G]\subseteq M$. \qedhere 
\end{enumerate}
\end{proof}

The previous theorem does not prove the existence of maximal subgroups. For example, 
$\Q$ is a nilpotent group (as it is abelian) 
that contains no maximal subgroups. 

\begin{proposition}
\label{pro:g^n}
Sea $G$ un grupo nilpotente y sea $H$ un subgrupo de $G$ de índice $n$. Si
$g\in G$ entonces $g^n\in H$.
\end{proposition}

%%% TODO: hacer inducción en el índice para escribirlo mejor

\begin{proof}
	Procederemos por inducción en $n$. El caso $n=1$ es trivial y el caso 
	$n=2$ se obtiene de la normalidad de $H$. Supongamos entonces que el resultado es válido para
	todo subgrupo de índice $<n$. Si $H$ es tal que $(G:H)=n$, entonces
	sean 
	$H_0=H$ y $H_{i+1}=N_G(H_i)$ para $i\geq0$. Por definición, $H_{i}$ es
	normal en $H_{i+1}$ y además, como $G$ es nilpotente, si $H_i\ne G$
	entonces $H_i\subsetneq H_{i+1}$ por la condición normalizadora. 
	Como $(G:H)$ es finito, existe $k$ tal que $H_k=G$. Por hipótesis 
	inductiva, como $(H_j:H_{j-1})<n$ para todo $j$, se tiene que
	$x^{(H_j:H_{j-1})}\in H_{j-1}$ para todo $x\in H_j$ y todo $j$. Luego  
	\[
		g^{(G:H)}=g^{(H_k:H_{k-1})(H_{k-1}:H_{k-2})\cdots (H_1:H_0)}\in H.\qedhere 
	\]
% 	El resultado es obvio en el caso en que $H$ sea un subgrupo normal.  Sea
% 	$H_0=H$ y $H_{i+1}=N_G(H_i)$ para $i\geq0$. Por definición, $H_{i}$ es
% 	normal en $H_{i+1}$ y además, como $G$ es nilpotente, si $H_i\ne G$
% 	entonces $H_i\subsetneq H_{i+1}$ por la condición normalizadora. 
% 	Como $(G:H)$ es finito, existe $k$ tal que $H_k=G$. Veamos que 
% 	\[
% 		g^{(G:H)}=g^{(H_k:H_{k-1})(H_{k-1}:H_{k-2})\cdots (H_1:H_0)}\in H.
% 	\]
% 	Observemos que $g^{(H_k:H_{k-1})}\in H_{k-1}$ pues $H_{k-1}$ es normal en $H_k=G$, y que, como 
% 	$g^{(H_k:H_{k-1})}\in H_k$, entonces 
% 	\[
% 	g^{(H_k:H_{k-2})}=g^{(H_k:H_{k-1})(H_{k-1}:H_{k-2})}=\left(g^{(H_k:H_{k-1})}\right)^{(H_{k-1}:H_{k-2})}\in H_{k-2}
% 	\]
% 	pues $H_{k-2}$ es normal en $H_{k-1}$. Al repetir este argumento, $g^{(G:H)}\in H$. 
\end{proof}

\begin{example}
	La proposición anterior no vale si el grupo $G$ no es
	nilpotente. Sea $G=\Sym_3$ y sea $H=\{\id,(12)\}$ de índice tres.  Si
	$g=(13)$ entonces $g^{3}=(13)\not\in H$.
\end{example}

El lema que daremos a continuación es una herramienta útil para hacer demostraciones por inducción
en grupos nilpotentes. 

\begin{lemma}
	\label{lemma:a[GG]}
	Sea $G$ un grupo nilpotente de clase $c\geq2$. Si $x\in G$ entonces el subgrupo 
	$\langle x,[G,G]\rangle$ es nilpotente de clase $<c$.
\end{lemma}

\begin{proof}
	Sea $H=\langle x,[G,G]\rangle$.  Si $x\in [G,G]$, el resultado es cierto.
	Supongamos entonces que $x\not\in [G,G]$. Observemos que 
	\[
		H=\{x^nc:n\in\Z,c\in [G,G]\},
	\]
	pues $[G,G]$ es normal en $G$. Para demostrar el lema  
	alcanza entonces con probar que $[H,H]\subseteq\gamma_3(G)$. Sean $h=x^nc,k=x^md\in H$
	con $c,d\in [G,G]$. 
	Como 
	\[
	[h,x^m]=[x^n,[c,x^m]][c,x^m]\in\gamma_4(G)\gamma_3(G)\subseteq\gamma_3(G),
	\]
	entonces 
	\begin{align*}
		[h,k]&=[h,x^m][x^m,[h,d]][h,d]\\
			&=[x^n,[c,x^m]][c,x^m][x^m,[h,d]][h,d]\in\gamma_3(G).\qedhere
	\end{align*}
%	Como
%	$[G,G]=\gamma_2(G)\subseteq\zeta_{c-1}(G)$, al usar la definición de $H$ se
%	obtiene que $[G,G]\subseteq \zeta_{c-1}(G)\cap H\subseteq\zeta_{c-1}(H)$.
%	Luego $H/\zeta_{c-1}(H)$ es cíclico generado por $a\zeta_{c-1}(H)$. Si
%	$\zeta_{c-1}(H)=H$, no hay nada para demostrar. En caso contrario,
%	$Z(H)\subseteq\zeta_{c-1}(H)$ implicaría que $H$ es abeliano y luego $H$ es
%	nilpotente de clase uno.
\end{proof}

Veamos un ejemplo, está hecho con la computadora. 

\begin{example}
	Sea $G=\D_{8}=\langle r,s:r^{8}=s^2=1,srs=r^{-1}\rangle$ el grupo diedral
	de orden 16. El grupo $G$ es nilpotente de clase tres
	y $[G,G]=\{1,r^2,r^4,r^6\}\simeq C_4$. El subgrupo $\langle
	s,[G,G]\rangle\simeq\D_4$ es nilpotente de clase dos.

	\begin{lstlisting}
gap> G := DihedralGroup(IsPermGroup,16);;
gap> gens := GeneratorsOfGroup(G);;
gap> r := gens[1];;
gap> s := gens[2];;
gap> D := DerivedSubgroup(G);;
gap> S := Subgroup(G, Concatenation(Elements(D), [s]));;
gap> StructureDescription(S);
"D8"
gap> NilpotencyClassOfGroup(G);
3
gap> NilpotencyClassOfGroup(S);
2
	\end{lstlisting}
\end{example}

Ahora una aplicación del lema. 

\begin{theorem}
	\label{theorem:T(nilpotent)}
	Si $G$ es un grupo nilpotente, entonces
	\[
	T(G)=\{g\in G:g^n=1\text{ para algún $n\gew1$}\}
	\]
	es un subgrupo de $G$. 
\end{theorem}

\begin{proof}
	Sean $a,b\in T(G)$ y sean
	\[
		A=\langle a,[G,G]\rangle,\quad
		B=\langle b,[G,G]\rangle.
	\]
	Como $A$ y $B$ son nilpotentes por el lema anterior, por hipótesis
	inductiva, $T(A)$ es un subgrupo de $A$ y $T(B)$ es un subgrupo de $B$.
	Como $T(A)$ es característico en $A$ y $A$ es normal en $G$, $T(A)$ es
	normal en $G$. Similarmente se demuestra que $T(B)$ es normal en $B$.  
	Veamos ahora que todo elemento de $T(A)T(B)$ tiene orden finito: si
	$x\in T(A)T(B)$, digamos $x=a_1b_1$ con
	$a_1$ de orden $m$, entonces $x$ tiene orden finito pues 
	\begin{align*}
	x^m=(a_1b_1)^m&=%(a_1b_1a_1^{-1})(a_1^2b_1a_1^{-2})\cdots (a_1^{m-1} b_1 a_1^{-m+1}b_1)\\
	(a_1b_1a_1^{-1})(a_1^2b_1a_1^{-2})\cdots (a^{m-1} b_1 a^{-m+1})b_1\in T(B).
	\end{align*}
	
	Para entender mejor el truco hagamos un caso concreto, digamos $m=3$. La fórmula en este caso queda así:
	\begin{align*}
	(a_1b_1)^3&=(a_1b_1)(a_1b_1)(a_1b_1)\\
	&=(a_1b_1a_1^{-1})(a_1^2b_1a_1^{-2})a_1^3b_1
	=(a_1b_1a_1^{-1})(a_1^2b_1a_1^{-2})b_1,
	\end{align*}
	pues $a_1^3=1$.
	
	El truco nos permite entonces demostrar que $ab$ y $a^{-1}$ tienen ambos 
	orden finito. Luego $T(G)$ es un
	subgrupo de $G$.
\end{proof}

Otra aplicación.

\begin{theorem}
	\label{theorem:a=b}
	Sea $G$ un grupo nilpotente y sin torsión y sean $a,b\in G$. Si existe
	$n\ne 0$ tal que $a^n=b^n$ entonces $a=b$.
\end{theorem}

\begin{proof}
	Procederemos por inducción en el orden de nilpotencia $c$ de $G$. El
	resultado es trivialmente cierto si $G$ es abeliano. Supongamos entonces
	que $G$ es nilpotente de índice $c\geq1$. Como $\langle a,[G,G]\rangle$ es un
	subgrupo de $G$ nilpotente de índice $<c$, y $bab^{-1}=[b,a]a\in \langle
	a,[G,G]\rangle$, por hipótesis inductiva, $ba=ab$ pues 
	\[
		a^n=(bab^{-1})^n=b^n.
	\]
	Luego $(ab^{-1})^n=a^nb^{-n}=1$ y por lo tanto, como $G$ no tiene torsión,
	se concluye que $a=b$.
\end{proof}

\begin{corollary}
	Sea $G$ un grupo nilpotente sin torsión. Si $x,y\in G$ son tales que
	$x^ny^m=y^mx^n$ para algún $n,m\ne 0$, entonces $xy=yx$.
\end{corollary}

\begin{proof}
	Sean $a=x$ y $b=y^nxy^{-n}$. Como $a^m=b^m$, el teorema anterior
	implica que $a=b$ y luego $xy^n=y^nx$. Al usar nuevamente ese 
	teorema, esta vez con con $a=y$ y $b=xyx^{-1}$, se
	concluye que $xy=yx$. 
\end{proof}

\begin{lemma}
	\label{lemma:fg}
	Sea $G$ un grupo finitamente generado y sea $H$ un subgrupo de índice
	finito. Entonces $H$ es finitamente generado.
\end{lemma}

\begin{proof}
	Supongamos que $G$ está generado por $\{g_1,\dots,g_m\}$. Podemos suponer, sin pérdida de generalidad, que
	para cada $i$ existe $k$ tal que $g_i^{-1}=g_k$. 
	
	Sea $\{1=t_1,\dots,t_n\}$ un transversal de $H$ en $G$, es decir un 
	conjunto de representantes de $G/H$. Para $i\in\{1,\dots,n\}$,
	$j\in\{1,\dots,m\}$, escribimos
	\[
		t_ig_j=h(i,j)t_{k(i,j)}.
	\]
	Vamos a demostrar que $H$ está generado por los $h(i,j)$. Sea $x\in H$.
	Escribamos 
	\begin{align*}
	x &=g_{i_1}\cdots g_{i_s}\\
	&= (t_1g_{i_1})g_{i_2}\cdots g_{i_s}\\
	&= h(1,i_1)t_{k_1}g_{i_2}\cdots g_{i_s}\\
	&= h(1,i_1)h(k_1,i_2)t_{k_2}g_{i_3}\cdots g_{i_s}\\
	&= h(1,i_1)h(k_1,i_2)\cdots h(k_{s-1},i_s)t_{k_s},
	\end{align*}
	donde $k_1,\dots,k_{s-1}\in\{1,\dots,n\}$. Como $t_{k_s}\in H$ pues $x\in H$, entonces
	$t_{k_s}=1\in H$ y luego $x$ está generado por los $h(i,j)$.
\end{proof}

\begin{theorem}
	\label{theorem:T(G)finito}
	Sea $G$ un grupo finitamente generado, de torsión y nilpotente. Entonces
	$G$ es finito. 	
\end{theorem}

\begin{proof}
	Procederemos por inducción en la clase de nilpotencia $c$. El caso $c=1$ es
	verdadero pues $G$ es abeliano. Supongamos entonces que el resultado es
	válido para $c\geq1$.  Como $[G,G]$ y $G/[G,G]$ son nilpotentes de clase
	$<c$, finitamente generados por el lema anterior y de torsión, por
	hipótesis inductiva se tiene que $[G,G]$ y $G/[G,G]$ son finitos. Luego $G$
	es también finito, de hecho puede probarse que $G$ es de orden $|[G,G]|(G:[G,G])$.
\end{proof}

%\begin{lemma}
%	\label{lemma:kgenerators}
%	Sea $G$ un grupo y sea $G=G_0\supseteq G_1\supseteq\cdots\supseteq G_k=1$
%	una sucesión de subgrupos de $G$ tal que cada $G_{i+1}$ es normal en $G_i$
%	y cada $G_{i}/G_{i+1}$ es cíclico. Todo subgrupo de $G$ es finitamente
%	generado por $k$ elementos.
%\end{lemma}
%
%\begin{proof}
%	Procedemos por inducción en $k$. Supongamos primero que $k=1$. Entonces
%	$G\simeq G_0/G_1$ es cíclico y luego todo subgrupo de $G$ está generado por
%	un elemento. Supongamos ahora que el resultado es válido para $k\geq1$. Sea
%	$H$ un subgrupo de $G$, sea $N=G_{1}$ y sea $\pi\colon G\to G/N$ el
%	morfismo canónico. El grupo 
%	\[
%		\pi(H)\simeq H/H\cap N
%	\]
%	es cíclico pues un un subgrupo del grupo cíclico $G_k/G_{k-1}=G/N$. Como
%	existe $h\in H$ tal que $\pi(H)$ está generado por $\pi(h)$, se concluye que 
%	$H=\langle \pi(h),H\cap N\rangle$. Por hipótesis
%	inductiva, $H\cap N$ está generado por $k-1$ elementos y luego $H$ está
%	generado por $k$ elementos.
%\end{proof}
%
%\begin{theorem}
%	Sea $G$ un grupo nilpotente y finitamente generado. Entonces $T(G)$ es
%	finito.
%\end{theorem}
%
%%%% aca hay que hacer producto tensorial para construir una serie con factores cíclicos
%%%% ver libro de Khukhro
%%%% Nilpotent Groups and Their Automorphisms
%\begin{proof}
%	Sabemos por el teorema~\ref{theorem:} que existe una sucesión
%	$G=G_0\supseteq G_1\supseteq\cdots\supseteq G_k=G$ de subgrupos normales de
%	$G$ con factores cíclicos. 
%\end{proof}




%\subsection{Grupos finitos nilpotentes}

El siguiente lema también resultará muy útil, especialmente en caso de trabajar con 
grupos finitos nilpotentes. 

\begin{lemma}
	\label{lemma:normalizador}
	Sean $G$ un grupo finito, $p$ un primo que divide a $|G|$ y
	$P\in\Syl_p(G)$. Entonces
	\[
	N_G(N_G(P))=N_G(P). 
	\]
\end{lemma}

\begin{proof}
	Sea $H=N_G(P)$. Como $P$ es normal en $H$, $P$ es el único $p$-subgrupo de
	Sylow de $H$. Para ver que $N_G(H)=H$ basta demostrar que $N_G(H)\subseteq
	H$. Sea $g\in N_G(H)$. Como 
	\[
	gPg^{-1}\subseteq gHg^{-1}=H,
	\]
	$gPg^{-1}\in\Syl_p(H)$ y $H$ tiene un único $p$-subgrupo de Sylow, 
	$P=gPg^{-1}$.  Luego $g\in N_G(P)=H$. 
\end{proof}

\begin{theorem}
	\label{thm:nilpotente:eq}
	Sea $G$ un grupo finito. Son equivalentes:
	\begin{enumerate}
		\item $G$ es nilpotente.
		\item Todo subgrupo de Sylow de $G$ es normal.
		\item $G$ es producto directo de sus subgrupos de Sylow.
	\end{enumerate}
\end{theorem}

\begin{proof}
	Veamos que $(1)\implies(2)$. Sea $P\in\Syl_p(G)$. Queremos ver que $P$ es
	normal en $G$, es decir $N_G(P)=G$.  Por el lema anterior, 
	$N_G(N_G(P))=N_G(P)$. La condición normalizadora implica entonces que $N_G(P)=G$.

	Veamos ahora que $(2)\implies(3)$. Sean $p_1,\cdots,p_k$ los factores
	primos de $|G|$ y para cada $i\in\{1,\dots,k\}$ sea $P_i\in\Syl_{p_i}(G)$.
	Por hipótesis, cada $P_j$ es normal en $G$.

	Vamos a demostrar que $P_1\cdots P_j\simeq P_1\times\cdots\times P_j$ para todo $j$.
	El caso $j=1$ es trivial. Supongamos entonces que el resultado vale para
	algún $j\geq 1$. Como 
	\[
	N=P_1\cdots P_j\simeq P_1\times\cdots\times P_j
	\]
	es normal en $G$ y tiene orden coprimo con $|P_{j+1}|$, $N\cap
	P_{j+1}=\{1\}$. Luego
	\[
		NP_{j+1}\simeq N\times P_{j+1}\simeq P_1\times\cdots\times P_j\times P_{j+1}
	\]
	pues $P_{j+1}$ es también normal en $G$. 

	Ahora que sabemos que $P_1\cdots P_k\simeq P_1\times\cdots\times P_k$ es un
	subgrupo de orden $|G|$, se concluye que $G=P_1\times\cdots\times P_k$.

	Para ver que $(3)\implies(1)$ simplemente hace falta observar que todo
	$p$-grupo es nilpotente (proposición~\ref{pro:nilpotent_pgroups}) y
	que el producto directo de finitos nilpotentes es nilpotente.
	%(ejercicio~\ref{exercise:HxK_nilpotente}).
\end{proof}

\begin{exercise}
	\label{xca:truco}
	Sea $G$ un grupo finito. Demuestre que si $P\in\Syl_p(G)$ y $M$ es un subgrupo de $G$ tal que
	$N_G(P)\subseteq M$ entonces $M=N_G(M)$. 
\end{exercise}

% \begin{svgraybox}
% 	Sea $x\in N_G(M)$. Como $P\subseteq M$ y $M$ es normal en $N_G(M)$,
% 	$xPx^{-1}\subseteq M$.  Como $P$ y $xPx^{-1}$ son $p$-subgrupos de Sylow de
% 	$M$, existe $m\in M$ tal que 
% 	\[
% 	mPm^{-1}=xPx^{-1}.
% 	\]
% 	Luego $x\in M$ pues
% 	$m^{-1}x\in N_G(P)\subseteq M$. 
% \end{svgraybox}

\begin{exercise}
	\label{xca:normalizadora}
	Sea $G$ un grupo finito. Son equivalentes:
	\begin{enumerate}
		\item $G$ es nilpotente.
		\item Si $H\subsetneq G$ es un subgrupo entonces $H\subsetneq N_G(H)$.
		\item Todo subgrupo maximal de $G$ es normal en $G$.
	\end{enumerate}
\end{exercise}

% \begin{svgraybox}
% 	Para demostrar que $(1)\implies(2)$ simplemente usamos el
% 	lema~\ref{lemma:normalizadora}. Para demostrar que $(2)\implies(3)$ hacemos
% 	lo siguiente: si $M$ es un subgrupo maximal, como $M\subsetneq N_G(M)$ por
% 	hipótesis, $N_G(M)=G$ por maximalidad. Finalmente demostremos que
% 	$(3)\implies(1)$.  Sea $P\in\Syl_p(G)$. Si $P$ no es normal en $G$,
% 	$N_G(P)\ne G$ y entonces existe un subgrupo maximal $M$ tal que
% 	$N_G(P)\subseteq M$. Como $M$ es normal en $G$, el
% 	ejercicio~\ref{exercise:truco} implica que $M=N_G(M)=G$, una contradicción.
% 	Luego $P$ es normal en $G$ y entonces $G$ es nilpotente por el
% 	teorema~\ref{theorem:nilpotente:eq}.
% \end{svgraybox}

% ejercicio: G finito. Es nilpotente si y solo si dos elementos de ordenes coprimos conmnutan
% 5.41 rotman

\begin{theorem}
	Sea $G$ un grupo finito nilpotente. Si $p$ es un primo que divide al orden
	de $G$, existe un subgrupo minimal-normal de orden $p$ y existe un subgrupo
	maximal de índice $p$.
\end{theorem}

\begin{proof}
	Supongamos que $|G|=p^{\alpha}m$, donde $p$ es un primo coprimo con $m$.
	Escribamos $G=P\times H$, donde $P\in\Syl_p(G)$.  Como $Z(P)$ es un
	subgrupo normal no trivial de $P$, todo subgrupo 
	de $Z(P)$ minimal-normal en $G$ tiene orden $p$ (y esos subgrupos 
	existen porque $G$ es un grupo finito). Por otro lado, como $P$ contiene un subgrupo
	de índice $p$, que resulta maximal. Luego $P\times H$ también contiene un
	subgrupo maximal de índice $p$.
\end{proof}

\begin{exercise}
	\label{xca:pgrupos}
	Sea $p$ un primo y sea $G$ un grupo no trivial de orden $p^n$.
	Demuestre las siguientes afirmaciones:
	\begin{enumerate}
		\item $G$ tiene un subgrupo normal de orden $p$.
		\item Para todo $j\in\{0,\dots,n\}$ existe un subgrupo normal de
			$G$ de orden $p^j$. 
	\end{enumerate}
\end{exercise}

% \begin{svgraybox}
% 	\begin{enumerate}
% 		\item Sabemos que $Z(G)\ne1$. Sea $g\in Z(G)$ tal que $g\ne 1$.
% 			Supongamos que el orden de $g$ es $p^k$ para algún $k\geq1$.
% 			Entonces $g^{p^{k-1}}$ tiene orden $p$ y luego genera un subgrupo
% 			central de orden $p$. 
% 		\item Procederemos por inducción en $n$. Si $n=1$ el resultado es
% 			trivial.  Supongamos entonces que el resultado vale para un cierto
% 			$n\geq2$. Por el punto anterior, $G$ posee un subgrupo normal $N$
% 			de orden $p$. Luego $G/N$ tiene orden $p^{n-1}$. Sea $\pi\colon G\to G/N$ el morfismo canónico. 
% 			Por hipótesis
% 			inductiva, para cada $j\in\{0,\dots,n-1\}$. Por el teorema de la
% 			correspondecia, cada subgrupo normal $S_j$ de $G/N$ de orden $p^j$ se
% 			corresponde con un subgrupo $\pi^{-1}(S_j)$ de $G$ de orden $p^{j+1}$ pues, como
% 			$\pi$ es sobreyectiva, se tiene $\pi(\pi^{-1}(S_j))=S_j$, y luego
% 			\[
% 			p^j=|S_j|=|\pi(\pi^{-1}(S_j))|=\frac{|\pi^{-1}(S_j)|}{|\pi^{-1}(S_j)\cap N|}=\frac{|\pi^{-1}(S_j)|}{|N|}=\frac{|\pi^{-1}(S_j)|}{p}.
% 			\]
% 	\end{enumerate}
% \end{svgraybox}

\begin{exercise}
\label{xca:nilpotente_equivalencia}
	Sea $G$ un grupo finito. Demuestre que las siguientes afirmaciones son
	equivalentes: 
	\begin{enumerate}
		\item $G$ es nilpotente.
		\item Cualesquiera dos elementos de órdenes coprimos conmutan. 
		\item Todo cociente no trivial de $G$ tiene centro no trivial.
		\item Si $d$ divide al orden de $G$, existe un subgrupo normal de $G$
			de orden $d$.
	\end{enumerate}
\end{exercise}

% \begin{svgraybox}
% 	Veamos que $(1)\implies(2)$. Sabemos que $G$ es producto directo de sus
% 	subgrupos de Sylow, digamos $G=\prod_{i=1}^k S_i$, donde los $S_i$ son los
% 	distintos subgrupos de Sylow de $G$.  Sean
% 	$x=(x_1,\dots,x_k),y=(y_1,\dots,y_k)\in G$. Como $|x|$ y $|y|$ son
% 	coprimos, para cada $i\in\{1,\dots,k\}$ se tiene $x_i=1$ o $y_i=1$. Luego
% 	\[
% 		[x,y]=([x_1,y_1],[x_2,y_2],\dots,[x_k,y_k])=1. 
% 	\]
% 	Demostremos ahora que $(2)\implies(1)$. Supongamos que
% 	$|G|=p_1^{\alpha_1}\cdots p_k^{\alpha_k}$, donde los $p_j$ son primos
% 	distintos y para cada $j$ sea $P_j\in\Syl_{p_j}(G)$. Como elementos de
% 	órdenes coprimos conmutan, la función $P_1\times\cdots\times P_k\to G$,
% 	$(x_1,\dots,x_k)\mapsto x_1\cdots x_k$, es un morfismo inyectivo de grupos.
% 	Como entonces $G\simeq P_1\times\cdots P_k$, y cada $P_j$ es nilpotente,
% 	$G$ es nilpotente. 

% 	Para demostrar que $(1)\implies(3)$ simplemente hay que observar que todo
% 	cociente de $G$ es nilpotente y luego utilizar el
% 	teorema~\ref{theorem:Z(nilpotent)}. Demostremos que $(3)\implies(1)$. Como
% 	todo cociente no trivial de $G$ tiene centro no trivial, en particular
% 	$Z_1=Z(G)$ es no trivial. Si $Z_1=G$ entonces $G$ es abeliano y no hay nada
% 	para demostrar. Si $Z_1\ne G$ entonces $G/Z_1\ne 1$; luego $Z(G/Z_1)\ne 1$.
% 	Si $\pi_1\colon G\to G/Z_1$ es el morfismo canónico,
% 	$Z_2=\pi_1^{-1}(Z(G/Z_1))$. Inductivamente, si tenemos construido el
% 	subgrupo $Z_i$, $Z_i\ne G$ y  $\pi_i\colon G\to G/Z_{i}$ es el morfismo
% 	canónico, se define el subgrupo $Z_{i+1}=\pi_i^{-1}(Z(G/Z_i))$. Por
% 	construcción, $Z_i\subseteq Z_{i+1}$ para todo $i$. Como $G$ es finito,
% 	existe $k$ tal que $Z_k=G$ y luego $G$ es nilpotente.

% 	Demostremos que $(1)\implies(4)$. Esta implicación es consecuencia
% 	inmediata del ejercicio~\ref{exercise:pgrupos}. 
% 	Como $G$ es nilpotente, $G$ producto
% 	directo de sus $p$-grupos de Sylow. Si $d=p_1^{\alpha_1}\cdots
% 	p_k^{\alpha_k}$ es un divisor del orden de $G$, basta tomar
% 	$H=H_1\times\cdots\times H_k$, 
% 	donde cada $H_j$ es un subgrupo normal del $p_j$-subgrupo de Sylow de $G$
% 	de orden $p_j^{\alpha_j}$. Para demostrar que $(4)\implies(1)$ simplemente
% 	se aplica la hipótesis a cada $p$-subgrupo de $G$ de orden maximal.
% \end{svgraybox}

El siguiente resultado, que puede demostrarse en forma
completamente elemental, fue descubierto en 2014. 

\begin{theorem}[Baumslag--Wiegold]
	\index{Teorema de!Baumslag--Wiegold}
	Sea $G$ un grupo finito tal que $|xy|=|x||y|$ si $x$ e $y$ son elementos de
	órdenes coprimos. Entonces $G$ es nilpotente.
\end{theorem}

\begin{proof}
	Sean $p_1,\dots,p_n$ los distintos primos que dividen al orden de $G$. Para cada
	$i\in\{1,\dots,n\}$ sea $P_i\in\Syl_{p_i}(G)$.  Primero vamos a demostrar
	que $G=P_1\cdots P_n$. La inclusión no trivial equivale a 
	demostrar que la función 
	\[
		\psi\colon P_1\times\cdots\times P_n\to G,\quad
		(x_1,\dots,x_n)\mapsto x_1\cdots x_n
	\]
	es sobreyectiva. Procederemos de la siguiente forma. Primero vemos que  
	la función $\psi$ es inyectiva. En efecto, si
	$\psi(x_1,\dots,x_n)=\psi(y_1,\dots,y_n)$, entonces 
	\[
		x_1\cdots
	x_n=y_1\cdots y_n. 
	\]
	Si $y_n\ne x_n$, entonces $x_1\cdots x_{n-1}=(y_1\cdots
	y_{n-1})y_nx_n^{-1}$. Pero $x_1\cdots x_{n-1}$ es un elemento de orden
	coprimo con $p_n$ y $y_1\cdots y_{n-1}y_nx_n^{-1}$ es un elemento de orden
	múltiplo de $p_n$, una contradicción. Entonces $x_n=y_n$ y luego, el mismo
	argumento, prueba que $\psi$ es inyectiva. Como $|P_1\times\cdots\times
	P_n|=|G|$, se concluye que $\psi$ es biyectiva. En particular, $\psi$ es sobreyectiva
	y luego $G=P_1\cdots P_n$.

	Veamos ahora que cada $P_j$ es normal en $G$.  Sea $j\in\{1,\dots,n\}$ y
	sea $x_j\in P_j$. Sea $g\in G$ y sea $y_j=gx_jg^{-1}$.  Como $y_j\in G$,
	podemos escribir $y_j=z_1\cdots z_n$ con $z_k\in P_k$ para todo $k$.  Como
	el orden de $y_j$ es una potencia del primo $p_j$, el elemento $z_1\cdots
	z_n$ tiene orden una potencia de $p_j$ y luego $z_k=1$ para todo $k\ne j$ y
	además $y_j=z_j\in P_j$. Como cada subgrupo de Sylow es normal en $G$, se
	concluye que $G$ es nilpotente.
\end{proof}


%\subsection{Grupos nilpotentes de clase dos}

\begin{lemma}
	\label{lemma:commutador}
	Si $x,y\in G$ son tales que $[x,y]\in C_G(x)\cap C_G(y)$, entonces
	\[
	[x,y]^n=[x^n,y]=[x,y^n]
	\]
	para todo $n\in\Z$.
\end{lemma}

\begin{proof}
	Procederemos por inducción en $n\geq0$. El caso $n=0$ es trivial. Supongamos entonces
	que el resultado vale para algún $n\geq0$. Entonces, como $[x,y]\in C_G(x)$, 
	\begin{align*}
		[x,y]^{n+1}&=[x,y]^n[x,y]
		=[x^n,y][x,y]=[x^n,y]xyx^{-1}y^{-1}=x[x^n,y]yx^{-1}y^{-1}=[x^{n+1},y].
	\end{align*}
	Para demostrar el lema en el caso $n<0$ basta observar que, como $[x,y]^{-1}=[x^{-1},y]$, 
	$[x,y]^{-n}=[x^{-1},y]^n=[x^{-n},y]$.
\end{proof}

\begin{lemma}[Hall]
    \index{Lema!de Hall}
	\label{lemma:Hall}
	Sea $G$ un grupo nilpotente de clase dos y $x,y\in G$. Entonces
	\[
		(xy)^n=[y,x]^{n(n-1)/2}x^ny^n
	\]
	para todo $n\in\N$.
\end{lemma}

\begin{proof}
	Procederemos por inducción en $n$. Como el caso $n=1$ es trivial,
	supongamos que el resultado es válido para algún $n\geq1$. Entonces,
	gracias al lema anterior, 
	\begin{align*}
		(xy)^{n+1} &= (xy)^n(xy)=[y,x]^{n(n-1)/2}x^ny^{n-1}(yx)y\\
		&=[y,x]^{n(n-1)/2}x^{n}[y^n,x]xy^{n+1}=[y,x]^{n(n-1)/2}[y,x]^nx^{n+1}y^{n+1}.\qedhere 
	\end{align*}
\end{proof}

\begin{lemma}
	\label{lemma:class2}
	Sea $p>2$ un número primo y sea 
	$P$ un $p$-grupo de clase de nilpotencia $\leq2$. 
	Si $[y,x]^p=1$ para todo $x,y\in P$ entonces $P\to [P,P]$,
	$x\mapsto x^p$, es un morfismo de grupos.
\end{lemma}

\begin{proof}
	Por lema de Hall,
	$(xy)^p=[y,x]^{p(p-1)/2}x^py^p=x^py^p$. 	
\end{proof}

\begin{theorem}
	\label{thm:class2}
	Sea $p>2$ un número primo y sea 
	$P$ un $p$-grupo de clase de nilpotencia $\leq2$. 
	Entonces $\{x\in P:x^p=1\}$ es un subgrupo de $P$.
\end{theorem}

\begin{proof}
	Como $P$ tiene clase de nilpotencia dos, los conmutadores son centrales.
	Para cada $x\in G$, la función $g\mapsto [g,x]$ es un morfismo de grupos
	pues
	\[
		[gh,x]=ghxh^{-1}g^{-1}x^{-1}=g[h,x]xg^{-1}x^{-1}=[g,x][h,x].
	\]
	En particular, si $x,y\in P$ con $x^p=y^p=1$, entonces
	\[
		[x,y]^p=[x^p,y]=[1,y]=1.
	\]
	Luego, al usar el lema de Hall, se concluye que
	$(xy)^p=[y,x]^{p(p-1)/2}x^py^p=1$.
\end{proof}
