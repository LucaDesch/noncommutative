\chapter{}

We now describe some very-well known open problems
in the theory of group rings and the connection between 
them. 

\begin{definition}
\index{Ring!reduced}
A ring $R$ is \textbf{reduced} if for all $r\in R$ such that 
$r^2=0$ one has $r=0$.
\end{definition}

% \topic{Andrunakevic--Rjabuhin's theorem}
%https://ysharifi.wordpress.com/2010/06/04/about-reduced-rings-1/
% \begin{exercise}
% \label{xca:reduced}
%     Let $R$ be a ring and $I$ be an ideal of $R$.
%     Prove that $I$ is prime if and only if $xRy\subseteq I$ implies
%     either $x\in I$ or $y\in I$. 
% \end{exercise}

% \begin{sol}{xca:reduced}
%     Let $A$ and $B$ be ideals such that $AB\subseteq I$. If 
%     $A\not\subseteq I$ and $B\not\subseteq J$, let 
%     $x\in A\setminus P$ and $y\in B\setminus P$. Then
%     $xRy\subseteq AB\subseteq I$, a contradiction. 
%     Conversely, if $xRy\subseteq I$ and $x\not I$ and $y\not\in I$, 
%     then $A=(x)\not\subseteq I$ and $B=(y)\not\subseteq P$.
% \end{sol}

Integral domains and boolean rings are reduced. $\Z/8$ 
and $M_2(\R)$ are not reduced. 

\begin{example}
    $\Z^n$ with $(a_1,\dots,a_n)(b_1,\dots,b_n)=(a_1b_1,\dots,a_nb_n)$
    is reduced. 
\end{example}

The structure of 
reduced rings is described by
Andrunakevic--Rjabuhin's theorem. It states
that a ring is reduced if and only if
it is a subdirect products of domains. See
\cite[3.20.5]{MR2015465} for a proof. 

% \begin{theorem}[Andrunakevic--Rjabuhin]
% \index{Andrunakevic--Rjabuhin's theorem}
% 	Let $R$ be a non-zero ring. Then $R$ is reduced if and only 
% 	if $R$ is a subdirect product of domains.
% \end{theorem}

% We shall need some lemmas. 

% \begin{lemma}
%     Let $P$ be a minimal prime ideal of $R$. Then 
%     $S=R\setminus P$ is multiplicatively closed if and only if
%     $s_1\cdots s_k\ne 0$ for all $k\geq 1$ and $s_1,\dots,s_k\in S$. 
% \end{lemma}

% \begin{proof}
%     Let $T=\{s_1\cdots s_k:k\geq 1,s_1,\dots,s_k\in S\}$. Clearly $T$ 
%     is multiplicatively closed and $S\subseteq T$. We claim that 
%     $T\subseteq S$. Let $X=\{I:\text{$I$ is an ideal of $R$ such that $I\cap T=\emptyset$}\}$. 
%     Then $X\ne\emptyset$, as $\{0\}\in X$. If $C$ is a chain in $X$, then
%     $\cup_{I\in C}I$ is an upper bound, so there exists a maximal element $Q\in X$ 
%     by Zorn's lemma. It is an exercise to show that $Q$ is prime. Since 
%     $Q\cap T=\emptyset$, it follows that $Q\cap S=\emptyset$, that is $Q\subseteq P$. 
%     Since $P$ is minimal, $P=Q\in X$. In particular, $T\subseteq S$. 
% \end{proof}

% \begin{lemma}
%     Let $P$ be a minimal prime ideal of $R$. 
%     Then $R/P$ is a domain.
% \end{lemma}

% \begin{proof}
    
% \end{proof}


% \begin{proof}
% 	If $R$ is reduced, then every prime ideal contains a minimal prime ideal. 
% 	Thus $\cap_{i\in I}P'_i=\{0\}$, where $\{P'_i:\in I\}$ is the collection
% 	of minimal prime ideals. Then each $R/P'_i$ is a domain and 
% 	there is an injective map $R\to\prod_{i\in I}R/P'_i$. 
	
% 	Supongamos ahora que $R$ es producto subdirecto de la familia $\{R_i:i\in I\}$ de dominios. Sea 
% 	$f\colon R\to \prod_{i\in I}R_i$, $f(x)=(x_i)_{i\in I}$, el morfismo inyectivo. 
% 	Si $x\in R$ es tal que $x^2=0$ entonces 
% 	\[
% 		(0)_{i\in I}=f(0)=f(x^2)=f(x)^2=(x_i^2)_{i\in I}
% 	\]
% 	y luego, como cada $R_i$ es un dominio, se concluye que $x_i=0$ para todo
% 	$i\in I$.
% \end{proof}


\begin{problem}
	\label{prob:reducido}
	Let $G$ be a torsion-free group. Is it true that 
	$K[G]$ is reduced? 
\end{problem}

Recall that if $R$ is a unitary ring, one proves that 
the Jacobson radical $J(R)$ is 
the set of elements $x$ such that
$1+\sum_{i=1}^n r_ixs_i$ is invertible 
for all $n$ and all $r_i,s_i\in R$.


\begin{problem}[Semisimplicity]
	\label{prob:J}
	Let $G$ be a torsion-free group. It is true that 
	$J(K[G])=\{0\}$ if $G$ is non-trivial?
\end{problem}

\index{Idempotent}
Recall that an element $e$ of a ring is said to be \emph{idempotent} 
if $e^2=e$. Examples of idempotents are $0$ and $1$ and 
these are known as the trivial idempotents. 

\begin{problem}[Idempotents]
	\label{pro:idempotente}
	Let $G$ be a torsion-free group and $\alpha\in K[G]$ be an idempotent. 
	Is it true that $\alpha\in\{0,1\}$?
\end{problem}

\begin{exercise}
	Prove that if $K[G]$ has no zero-divisors and $\alpha\in K[G]$ is an
	idempotent, then $\alpha\in\{0,1\}$.
\end{exercise}

\begin{exercise}
	Prove that $K[C_4]$ contains non-trivial zero divisors and every
	idempotent of $K[C_4]$ is trivial. 
\end{exercise}

The problems mentioned are all related. Our goal is the prove
the following implications:
\[
	\ref{prob:J}\Longleftarrow\ref{prob:units}\Longrightarrow\ref{prob:reducido}\Longleftrightarrow\ref{prob:dominio}
\]

We first prove that an affirmative solution to the Units
Problem~\ref{prob:units} yields a solution to Problem~\ref{prob:reducido}
about the reducibility of group algebras.

\begin{theorem}
    Let $K$ be a field of characteristic $\ne2$ 
	and $G$ be a non-trivial group. Assume that $K[G]$ has only trivial units.
	Then $K[G]$ is reduced. 
\end{theorem}

\begin{proof}
	Let $\alpha\in K[G]$ be such that $\alpha^2=0$. We claim that 
	$\alpha=0$. Since $\alpha^2=0$, 
	\[
		(1-\alpha)(1+\alpha)=1-\alpha^2=1, 
	\]
	it follows that $1-\alpha$ is a unit of $K[G]$. Since units of $K[G]$ are 
	trivial, there exist $\lambda\in K\setminus\{0\}$ and $g\in G$ such that 
	$1-\alpha=\lambda g$. We claim that $g=1$. If not, 
	\[
		0=\alpha^2=(1-\lambda g)^2=1-2\lambda g+\lambda^2g^2,
	\]
	a contradiction. Therefore $g=1$ and hence $\alpha=1-\lambda\in K$. Since
	$K$ is a field, one concludes that $\alpha=0$.
\end{proof}

\begin{exercise}
    What happens if $K$ is a field of characteristic two?
\end{exercise}

We now prove that an affirmative solution to the Units Problem
~\ref{prob:units} also yields a solution to the Jacobson Semisimplicity Problem
~\ref{prob:J}. 

\begin{theorem}
	Let $G$ be a non-trivial group. Assume that $K[G]$ has only trivial units. 
	If $|K|>2$ or $|G|>2$, then $J(K[G])=\{0\}$.
\end{theorem}

\begin{proof}
	Let $\alpha\in J(K[G])$. There exist $\lambda\in K\setminus\{0\}$ and $g\in
	G$ such that $1-\alpha=\lambda g$. We claim that $g=1$. Assume $g\ne 1$. 
	If $|K|\geq3$,
	then there exist $\mu\in K\setminus\{0,1\}$ such that
	\[
		1-\alpha\mu=1-\mu+\lambda\mu g 
	\]
	is a non-trivial unit, a contradiction.
	If $|G|\geq3$, there exists $h\in G\setminus\{1,g^{-1}\}$ such that
	$1-\alpha h=1-h+\lambda gh$ is a non-trivial unit, a contradiction.  Thus
	$g=1$ and hence $\alpha=1-\lambda\in K$. Therefore $1+\alpha h$ is a
	trivial unit for all $h\ne 1$ and hence 	$\alpha=0$.
\end{proof}

\begin{exercise}
	Prove that if $G=\langle g\rangle\simeq\Z/2$, then 
	$J(\F_2[G])=\{0,g-1\}\ne\{0\}$. 
\end{exercise}

\topic{The transfer map}

Now we prove that an affirmative solution 
to the Units Problem 
(Open Problem~\ref{prob:units}) 
yields a solution to 
Open Problem~\ref{prob:dominio} about zero divisors in group algebras.
The proof is hard and requires some preliminaries. We first need
to discuss a group theoretical tool known as the \emph{transfer map}. 

If $H$ is a subgroup of $G$, a \textbf{transversal} of $H$ in $G$ is a complete
set of coset representatives of $G/H$. 

\begin{lemma}
	\label{lem:d}
	Let $G$ be a group and $H$ be a subgroup of $G$ of finite index.  Let $R$
	and $S$ be transversals of $H$ in $G$ and let $\alpha\colon H\to H/[H,H]$
	be the canonical map. Then 
	\[
		d(R,S)=\prod \alpha(rs^{-1}),
	\]
	where the product is taken over all pairs 
	$(r,s)\in R\times S$ such that $Hr=Hs$,
	is well-defined and satisfies the following properties:
	\begin{enumerate}
		\item $d(R,S)^{-1}=d(S,R)$.
		\item $d(R,S)d(S,T)=d(R,T)$ for all transversal $T$ of $H$ in $G$.
		\item $d(Rg,Sg)=d(R,S)$ for all $g\in G$.
		\item $d(Rg,R)=d(Sg,S)$ for all $g\in G$.
	\end{enumerate}
\end{lemma}

\begin{proof}
	The product that defines $d(R,S)$ is well-defined since $H/[H,H]$ is 
	an abelian group. The first three claim are trivial. Let us prove
	4). By 2), 
	\[
		d(Rg,Sg)d(Sg,S)d(S,R)=d(Rg,S)d(S,R)=d(Rg,R).
	\]
	Since $H/[H,H]$ is abelian, 1) and 3) imply that 	
	\[
		d(Rg,Sg)d(Sg,S)d(S,R)=d(R,S)d(S,R)d(Sg,S)=d(Sg,S).\qedhere
	\]
\end{proof}

We are know ready to state and 
prove the theorem: 

\begin{theorem}
	\label{thm:transfer}
	Let $G$ be a group and $H$ be a finite-index subgroup of $G$. The map 	
	\[
		\nu\colon G\to H/[H,H],\quad
		g\mapsto d(Rg,R),
	\]
	does not depend on the transversal $R$ of $H$ in $G$ and it is a group
	homomorphism. 
\end{theorem}

\begin{proof}
	The lemma implies that the map does not depend on the transversal used. 
	Moreover, $\nu$ is a group homomorphism, as 
	\begin{align*}
		\nu(gh)&=d(R(gh),R)
		=d(R(gh),Rh)d(Rh,R)
		=d(Rg,R)d(Rh,R)=\nu(g)\nu(h).\qedhere
	\end{align*}
\end{proof}

The theorem justifies the following definition: 

\begin{definition}
	Let $G$ be a group and $H$ be a finite-index subgroup of $G$. The
	\textbf{transfer map} of $G$ in $H$ is the group homomorphism 
	\[
		\nu\colon G\to H/[H,H],
		\quad
		g\mapsto d(Rg,R),
	\]
	of Theorem~\ref{thm:transfer}, where $R$ is some transversal of $H$ in $G$.
\end{definition}

We need methods for computing the transfer map. If $H$ is a subgroup of 
$G$
and $(G:H)=n$, let $T=\{x_1,\dots,x_n\}$ be a transversal of $H$. For $g\in G$ let  
\[
	\nu(g)=\prod \alpha(xy^{-1}),
\]
where the product is taken over all pairs $(x,y)\in (Tg)\times T$ such that $Hx=Hy$
and $\alpha\colon H\to H/[H,H]$ is the canonical map. 
If we write 
$x=x_ig$ for some $i\in\{1,\dots,n\}$, then  
$Hx_ig=Hx_{\sigma(i)}$ for some permutation $\sigma\in\Sym_n$. Thus 
\[
	\nu(g)=\prod_{i=1}^n\alpha(x_igx_{\sigma(i)}^{-1}).
\]
The cycle structure of $\sigma$ turns out to be important. 
For example, if $\sigma=(12)(345)$ and $n=5$, then a direct calculation shows that 
\[
\prod_{i=1}^5\alpha\left(x_igx_{\sigma(i)}^{-1}\right)
=\alpha(x_1g^2x_1^{-1})\alpha(x_3g^3x_3^{-1}).
\]
This is precisely the content of the following lemma. 



% \begin{lemma}
% 	\label{lem:transfer}
% 	Let $G$ be a group and $H$ be a subgroup such that $(G:H)=n$. Let 
% 	$T$ be a transversal of $H$ in $G$. 
% 	For each $g\in G$ there exist $k$ and 
% 	positive integers 
% 	$n_1,\dots,n_k$ such that $n_1+\cdots+n_k=n$ and elements 
% 	$t_1,\dots,t_k\in T$ such that  
% 	\[
% 		\nu(g)=\prod_{i=1}^k \alpha(t_ig^{n_i}t_i^{-1}),
% 	\]
% 	where $\alpha\colon H\to H/[H,H]$ is the canonical map.
% \end{lemma}

\begin{lemma}
	\label{lem:transfer}
	Let $G$ be a group and $H$ be a subgroup of index $n$. Let 
	$T=\{t_1,\dots,t_n\}$ be a transversal of $H$ in $G$.  For each $g\in G$ there exist 
	$m\in\Z_{>0}$ and elements 
	$s_{1},\dots,s_{m}\in T$ and positive integers $n_1,\dots,n_m$
    such that 
	$s_i^{-1}g^{n_i}s_i\in H$,
	$n_1+\cdots+n_m=n$ and 
	\[
		\nu(g)=\prod_{i=1}^m \alpha(s_i^{-1}g^{n_i}s_i).
	\]
\end{lemma}

\begin{proof}
	For each $i$ there exist $h_1,\dots,h_n\in H$ and $\sigma\in\Sym_n$ such that 
	$gt_i=t_{\sigma(i)}h_i$. Write $\sigma$ as a product of disjoint cycles, say 
	\[
		\sigma=\alpha_1\cdots\alpha_m.
	\]

	Let $i\in\{1,\dots,n\}$ and write 
	$\alpha_i=(j_{1}\cdots j_{n_i})$. Since   
	\[
		g t_{j_k}=t_{\sigma(j_k)}h_{j_k}=\begin{cases}
			t_{j_1}h_{j_k} & \text{si $k=n_i$},\\
			t_{j_{k+1}}h_{j_k} & \text{otherwise},
		\end{cases}
	\]
	then 
	\begin{align*}
	t_{j_1}^{-1}g^{n_i}t_{j_1}
	&=t_{j_1}^{-1}g^{n_i-1}gt_{j_1}\\
	&=t_{j_1}^{-1}g^{n_i-1}t_{j_2}h_{j_1}\\
	&=t_{j_1}^{-1}g^{n_i-2}gt_{j_2}h_{j_1}\\
	&=t_{j_1}^{-1}g^{n_i-2}t_{j_3}h_{j_2}h_{j_1}\\
	&\phantom{=}\vdots\\
	&=t_{j_1}^{-1}gt_{j_{n_i}}h_{n_{i-1}}\cdots h_{j_2}h_{j_1}\\
	&=t_{j_1}^{-1}t_{j_1}h_{j_{n_i}}\cdots h_{j_2}h_{j_1}\in H. 	
	\end{align*}
	Thus $s_i=t_{j_1}\in T$. It only remains to note that $\nu(g)=h_1\cdots h_n$. 
\end{proof}

% \begin{proof}
% 	There exists $\sigma\in\Sym_n$ such that 
% 	\[
% 	\nu(g)=\prod_{i=1}^n \alpha( t_igt_{\sigma(i)}^{-1}). 
% 	\]
% 	Write $\sigma$ as a product of $k$ disjoint cycles
% 	$\sigma=\alpha_1\cdots\alpha_k$, where each $\alpha_j$ is a cycle of length 
% 	$n_j$. For every cycle of the form $(i_1\cdots i_{n_j})$
% 	we reorder the product in such a way that 
% 	\[
% 		\alpha(x_{i_1}gx_{i_2}^{-1})\alpha(x_{i_2}gx_{i_3}^{-1})\cdots \alpha(x_{i_{n_j}}gx_{i_1}^{-1})=\alpha(x_{i_1}g^{n_1}x_{i_1}^{-1}).
% 	\]
% 	There exist $t_1,\dots,t_k\in T$ such that 
% 	$\nu(g)=\prod_{j=1}^k \alpha(t_ig^{n_i}t_i^{-1}$). 
% \end{proof}

An application:

\begin{proposition}
	\label{pro:center}
	If $G$ is a group such that $Z(G)$ has finite index $n$, then
	$(gh)^n=g^nh^n$ for all $g,h\in G$.	
\end{proposition}

\begin{proof}
	Note that we may assume that $\alpha=\id$, as $Z(G)$ is
	abelian. Let $g\in G$. By Lemma~\ref{lem:transfer} there are positive integers 
    $n_1,\dots,n_k$ such that $n_1+\cdots+n_k=n$ and elements 
	$t_1,\dots,t_k$ of a transversal of $Z(G)$ in $G$ such that 
	\[
		\nu(g)=\prod_{i=1}^k t_ig^{n_1}t_i^{-1}.
	\]
	Since $g^{n_i}\in Z(G)$ for all $i\in\{1,\dots,k\}$ (as $t_ig^{n_i}t_i^{-1}\in Z(G)$), 
	it follows that 
	\[
	\nu(g)=g^{n_1+\cdots+n_k}=g^n.
	\]
	Now Theorem~\ref{thm:transfer} implies the claim.
\end{proof}

The same idea implies the following property:

\begin{exercise}
\label{xca:K_central}
	If $G$ is a group and $K$ is a central subgroup of finite index $n$, then
	$(gh)^n=g^nh^n$ for all $g,h\in G$.	
\end{exercise}

For a group $G$ we consider 
\[
	\Delta(G)=\{g\in G:(G:C_G(g))<\infty\}.
\]

\begin{exercise}
	Prove that $\Delta(\Delta(G))=\Delta(G)$.
\end{exercise}

\index{Subgroup!characteristic}
A subgroup $H$ of $G$ is a \textbf{characteristic} subgroup of $G$ 
if $f(H)\subseteq H$ for all $f\in\Aut(G)$. The center 
and the commutator subgroup of a group are characteristic subgroups. 
Every characteristic subgroup is a normal subgroup. 

\begin{exercise}
    Prove that if $H$ is characteristic in $K$ and $K$ is normal in $G$, then 
    $H$ is normal in $G$.
\end{exercise}

\begin{proposition}
	If $G$ is a group, then $\Delta(G)$ 
	is a characteristic subgroup of $G$.
\end{proposition}

\begin{proof}
	We first prove that $\Delta(G)$ is a subgroup of $G$. If $x,y\in\Delta(G)$
	and $g\in G$, then $g(xy^{-1})g^{-1}=(gxg^{-1})(gyg^{-1})^{-1}$. Moreover, 
	$1\in\Delta(G)$. Let us show now that $\Delta(G)$ is characteristic in $G$. If 
	$f\in\Aut(G)$ and $x\in G$, then, since 
	\[
	f(gxg^{-1})=f(g)f(x)f(g)^{-1},
	\]
	it follows that $f(x)\in\Delta(G)$.
\end{proof}

\begin{exercise}
	Prove that if $G=\langle r,s:s^2=1,srs=r^{-1}\rangle$ is the
	infinite dihedral group, then $\Delta(G)=\langle r\rangle$.
\end{exercise}

\begin{exercise}
	Let $H$ and $K$ be finite-index subgroups of $G$. Prove that
	\[
	(G:H\cap K)\leq (G:H)(G:K). 
	\]
\end{exercise}

