\section{}

\subsection{Passman's theorem}

\begin{proposition}
	\label{pro:FCabeliano}
	If $G$ is a torsion-free group such 
	that $\Delta(G)=G$, then $G$ is abelian.
\end{proposition}

\begin{proof}
	Let $x,y\in G=\Delta(G)$ and $S=\langle x,y\rangle$. The group $Z(S)=C_S(x)\cap C_S(y)$ has 
	finite index, say $n$, in $S$. By Proposition~\ref{pro:center}, 
	the map $S\to Z(S)$, $s\mapsto s^n$, is a group homomorphism. Thus  
	\[
		[x,y]^n=(xyx^{-1}y^{-1})^n=x^ny^nx^{-n}y^{-n}=1
	\]
	as $x^n\in Z(S)$. Since $G$ is torsion-free, $[x,y]=1$.
\end{proof}

\begin{lemma}[Neumann]
	\index{Lema!de Neumann}
	\label{lem:Neumann}
	Let $H_1,\dots,H_m$ be subgroups of $G$. 
	Assume there are finitely many elements
	$a_{ij}\in G$, $1\leq i\leq m$, $1\leq j\leq n$, such that 
	\[
		G=\bigcup_{i=1}^m\bigcup_{j=1}^n H_ia_{ij}.
	\]
	Then some $H_i$ has finite index in $G$.
\end{lemma}

\begin{proof}
	We proceed by induction on $m$. The case $m=1$ is trivial. 
	Let us assume that $m\geq2$. If $(G:H_1)=\infty$, there exists $b\in G$
	such that 
	\[
		H_1b\cap\left(
	\bigcup_{j=1}^nH_1a_{1j}\right)=\emptyset.
	\]
	Since $H_1b\subseteq\bigcup_{i=2}^m\bigcup_{j=1}^n H_ia_{ij}$, 
	it follows that 
	\[
		H_1a_{1k}\subseteq \bigcup_{i=2}^m\bigcup_{j=1}^n H_1a_{ij}b^{-1}a_{1k}.
	\]
	Hence $G$ can be covered by finitely many cosets of $H_2,\dots,H_m$. By the inductive hypothesis, 
	some of these $H_j$ has finite index in $G$.
\end{proof}

We now consider a projection operator of group algebras. If $G$ 
is a group and $H$ is a subgroup of $G$, let 
\[
	\pi_H\colon K[G]\to K[H],\quad
	\pi_H\left(\sum_{g\in G}\lambda_gg\right)=\sum_{g\in H}\lambda_gg.
\]

If $R$ and $S$ are rings, a $(R,S)$-bimodule is an abelian group
$M$ that is both a left $R$-module and a right $S$-module 
and the compatibility condition 
\[
(rm)s = r(ms)
\]
holds for all $r\in R$, $s\in S$ and $m\in M$.


\begin{exercise}
	Let $G$ be a group and $H$ be a subgroup of $G$. Prove that
	if $\alpha\in
	K[G]$, then $\pi_H$ is a $(K[H],K[H])$-bimodule homomorphism
	with usual left and right multiplications,
	\[
		\pi_H(\beta\alpha\gamma)=\beta\pi_H(\alpha)\gamma
	\]
	for all $\beta,\gamma\in K[H]$.
\end{exercise}

%\begin{proof}
%	Supongamos que $\alpha=\sum_{g\in G}\lambda_gg=\alpha_1+\alpha_2$, donde
%	$\alpha_1=\sum_{g\not\in H}\lambda_gg$ y $\alpha_2=\sum_{g\in
%	H}\lambda_gg=\pi_H(\alpha)$. Entonces
%	$\beta\alpha\gamma=\beta\alpha_1\gamma+\beta\alpha_2\gamma$, donde
%	$\beta\alpha_1\gamma\not\in K[H]$ y $\beta\alpha_2\gamma\in K[H]$.
%\end{proof}

\begin{lemma}
	\label{lem:escritura}
	Let $X$ be a left transversal of $H$ in $G$. Every $\alpha\in K[G]$ can be written
	uniquely as 
	\[
	\alpha=\sum_{x\in X}x\alpha_x,
	\]
	where $\alpha_x=\pi_H(x^{-1}\alpha)\in K[H]$.
\end{lemma}

\begin{proof}
	Let $\alpha\in K[G]$. Since $\supp\alpha$ is finite, $\supp\alpha$ is contained 
    in finitely many cosets of $H$, say $x_1H,\dots,x_nH$, where each 
	$x_j$ belongs to $X$. Write $\alpha=\alpha_1+\cdots+\alpha_n$,
	where $\alpha_i=\sum_{g\in x_iH}\lambda_gg$. If $g\in x_iH$, then 
	$x_i^{-1}g\in H$ and hence 
	\[
		\alpha=\sum_{i=1}^n x_i(x_i^{-1}\alpha_i)=\sum_{x\in X}x\alpha_x
	\]
	with $\alpha_x\in K[H]$ for all $x\in X$. For the uniqueness, note that 
	for each  $x\in X$ the previous exercise implies that  
	\[
		\pi_H(x^{-1}\alpha)
		=\pi_H\left(\sum_{y\in X}x^{-1}y\alpha_y\right)
		=\sum_{y\in X}\pi_H(x^{-1}y)\alpha_y=\alpha_x, 
	\]
	as  
	\[
		\pi_H(x^{-1}y)=\begin{cases}
		1 & \text{si $x=y$},\\
		0 & \text{si $x\ne y$}.
	\end{cases}\qedhere 
	\]
\end{proof}

%De la misma forma puede obtenerse un análogo al lema~\ref{lem:escritura} en el
%caso en que se tenga un transversal a derecha. 

\begin{lemma}
	\label{lem:ideal_pi}
	Let $G$ be a group and $H$ be a subgroup of $G$. If $I$ is a non-zero 
	left ideal
	of $K[G]$, then  $\pi_H(I)\ne\{0\}$.
\end{lemma}

\begin{proof}
	Let $X$ be a left transversal of $H$ in $G$ and $\alpha\in I\setminus\{0\}$. By Lemma
	\ref{lem:escritura} we can write $\alpha=\sum_{x\in
	X}x\alpha_x$ with $\alpha_x=\pi_H(x^{-1}\alpha)\in K[H]$ for all $x\in X$.
	Since $\alpha\ne0$, there exists $y\in X$ such that $0\ne
	\alpha_y=\pi_H(y^{-1}\alpha)\in\pi_H(I)$ ($y^{-1}\alpha\in I$ since $I$ is 
    a left ideal).
\end{proof}

Another application:

\begin{proposition}
	Let $G$ be a group, $H$ be a subgroup of $G$ and $\alpha\in K[H]$. The following statements hold:
	\begin{enumerate}
		\item $\alpha$ is invertible in $K[H]$ if and only if $\alpha$ is
			invertible in $K[G]$.
		\item $\alpha$ is a zero divisor of $K[H]$ if and only if $\alpha$ is  
			a zero divisor of $K[G]$.
	\end{enumerate}
\end{proposition}

\begin{proof}
	If $\alpha$ is invertible in $K[G]$, there exists $\beta\in K[G]$ such that 
	$\alpha\beta=\beta\alpha=1$. Apply $\pi_H$ and use that $\pi_H$ 
	is a $(K[H],K[H])$-bimodule homomorphism to obtain  
	\[
		\alpha\pi_H(\beta)=\pi_H(\alpha\beta)=\pi_H(1)=1=\pi_H(1)=\pi_H(\beta\alpha)=\pi_H(\beta)\alpha.
	\]
	
	Assume now that $\alpha\beta=0$ for some $\beta\in K[G]\setminus\{0\}$. Let $g\in G$
	be such that $1\in\supp(\beta g)$. Since $\alpha(\beta g)=0$, 
	\[
		0=\pi_H(0)=\pi_H(\alpha(\beta g))=\alpha\pi_H(\beta g),
	\]
	where $\pi_H(\beta g)\in K[H]\setminus\{0\}$, as $1\in\supp(\beta g)$. 
\end{proof}

\begin{lemma}[Passman]
	\index{Passman's lemma}
	\label{lem:Passman}
	Let $G$ be a group and 
	$\gamma_1,\gamma_2\in K[G]$ be such that $\gamma_1K[G]\gamma_2=\{0\}$.
	Then $\pi_{\Delta(G)}(\gamma_1)\pi_{\Delta(G)}(\gamma_2)=\{0\}$.
\end{lemma}

\begin{proof}
	It is enough to show that $\pi_{\Delta(G)}(\gamma_1)\gamma_2=\{0\}$, 
	as in this case
	\[
		\{0\}=\pi_{\Delta(G)}(\pi_{\Delta(G)}(\gamma_1)\gamma_2)=\pi_{\Delta}(\gamma_1)\pi_{\Delta(G)}(\gamma_2).
	\]
	Write $\gamma_1=\alpha_1+\beta_1$, where 
	\begin{align*}
		&\alpha_1=a_1u_1+\cdots+a_ru_r, && u_1,\dots,u_r\in\Delta(G),\\
		&\beta_1=b_1v_1+\cdots+b_sv_s, && v_1,\dots,v_s\not\in\Delta(G),\\
		&\gamma_2=c_1w_1+\cdots+c_tw_t,&& w_1,\dots,w_t\in G.
	\end{align*}
	The subgroup $C=\bigcap_{i=1}^rC_G(u_i)$ has finite index in $G$.
	Assume that 
	\[
		0\ne \pi_{\Delta}(\gamma_1)\gamma_2=\alpha_1\gamma_2. 
	\]
	Let $g\in\supp(\alpha_1\gamma_2)$. 
	If $v_i$ is a conjugate in $G$ of some 
	$gw_j^{-1}$, let $g_{ij}\in G$ be such that
	$g_{ij}^{-1}v_ig_{ij}=gw_j^{-1}$. If $v_i$ and $gw_j^{-1}$ 
	are not conjugate, 
	we take $g_{ij}=1$. 

	For every $x\in C$ it follows that
	$\alpha_1\gamma_2=(x^{-1}\alpha_1x)\gamma_2$. Since  
	\[
		x^{-1}\gamma_1x\gamma_2\in x^{-1}\gamma_1K[G]\gamma_2=0,
	\]
	it follows that
	\begin{align*}
		(a_1u_1+\cdots+a_ru_r)\gamma_2&=
		\alpha_1\gamma_2=x^{-1}\alpha_1x\gamma_2=-x^{-1}\beta_1x\gamma_2\\
		&=-x^{-1}(b_1v_1+\cdots+b_sv_r)x(c_1w_1+\cdots+c_tw_t).
	\end{align*}
	Now $g\in\supp(\alpha_1\gamma_2)$ implies that there exist $i,j$ such that
	$g=x^{-1}v_ixw_j$.
	Thus $v_i$ and $gw_j^{-1}$ are conjugate and hence
	$x^{-1}v_ix=gw_j^{-1}=g_{ij}^{-1}v_ig_{ij}$, that is
	$x\in C_G(v_i)g_{ij}$. This proves that 
	\[
		C\subseteq\bigcup_{i,j}C_G(v_i)g_{ij}. 
	\]
	Since $C$ has finite index in $G$, it follows that 
	$G$ can be covered by finitely many cosets of 
	the $C_G(v_i)$. Every $v_i\not\in\Delta(G)$, so 
	each $C_G(v_i)$ has infinite index in $G$, a contradiction 
	to Neumann's lemma.
\end{proof}

Before proving Passman's theorem, we need to mention 
that if $G$ is a torsion-free abelian group, then 
$K[G]$ has no non-zero divisors. We will prove this fact later, 
as an application of the theory of bi-ordered 
groups (see Corollary \ref{cor:domain_G_abelian}).


\begin{theorem}[Passman]
\index{Passman's theorem}
\label{thm:Passman}
	Let $G$ be a torsion-free group. If 
	$K[G]$ is reduced, then $K[G]$ is a domain.
\end{theorem}

\begin{proof}
	Assume that $K[G]$ is not a domain. Let $\gamma_1,\gamma_2\in K[G]\setminus\{0\}$
	be such that $\gamma_2\gamma_1=0$. If $\alpha\in K[G]$, then
	\[
		(\gamma_1\alpha\gamma_2)^2=\gamma_1\alpha\gamma_2\gamma_1\alpha\gamma_2=0
	\]
	and thus $\gamma_1\alpha\gamma_2=0$, as $K[G]$ is reduced. In particular, 
	$\gamma_1K[G]\gamma_2=\{0\}$. Let $I$ be the left ideal of $K[G]$ generated 
	by $\gamma_2$. Since $I\ne\{0\}$, it follows
	from Lemma~\ref{lem:ideal_pi} that 
	$\pi_{\Delta(G)}(I)\ne\{0\}$. Hence 
	$\pi_{\Delta(G)}(\beta\gamma_2)\ne\{ 0\}$ for some $\beta\in K[G]$. 
	Similarly, 
	$\pi_{\Delta(G)}(\gamma_1\alpha)\ne\{ 0\}$ for some $\alpha\in K[G]$. Since 
	\[
		\gamma_1\alpha K[G]\beta\gamma_2\subseteq \gamma_1 K[G]\gamma_2=\{0\},
	\]
    it follows that $\pi_{\Delta(G)}(\gamma_1\alpha)\pi_{\Delta(G)}(\beta\gamma_2)=\{0\}$
    by Passman's lemma. Hence $K[\Delta(G)]$ has zero divisors, a contradictions
    since $\Delta(G)$ is an abelian group.
\end{proof}

