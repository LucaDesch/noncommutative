\section{}

\subsection{More applications of the transfer}

Let us start with a group-theoretic application
of the transfer map. We start with some applications to the theory 
of finite groups. 

\begin{proposition}
	\label{prop:semidirecto}
	Let $G$ be a finite group and $H$ a central subgroup of index $n$, where 
	$n$ is coprime with $|H|$. Then
	$G\simeq N\rtimes H$.
\end{proposition}

\begin{proof}
	Since $H$ is abelian, $H=H/[H,H]$. Let  
	$\nu\colon G\to H$ be the transfer map and $h\in H$. 
	By Lemma~\ref{lem:transfer}, 
	\[
		\nu(h)
		=\prod_{i=1}^m s_i^{-1}h^{n_i}s_i,
	\]
	where each $s_i^{-1}h^{n_i}s_i\in H$. Since 
	$h^{n_i}\in H\subseteq Z(G)$ for all $i$, it follows that 
	$s_i^{-1}h^{n_i}s_i=h^{n_i}$ for all $i$. Thus 
	\[
		\nu(h)
		=\prod_{i=1}^m s_i^{-1}h^{n_i}s_i
		=\prod_{i=1}^mh^{n_i}
		=h^{\sum_{i=1}^m n_i}=h^n.
	\] 
	The composition $f\colon H\hookrightarrow G\xrightarrow{\nu} H$ is a group homomorphism. 
	We claim that it is an isomorphism. It is injective: If $h^n=1$, then 
	$|h|$ divides both $|H|$ and $n$. Since $n$ and $|H|$ are
	coprime, $h=1$. It is surjectice: Since $n$ and $|H|$ are coprime, there exists 
	$m\in\Z$ such that $nm\equiv 1\bmod |H|$. If $h\in H$, then $h^m\in
	H$ and $\nu(h^m)=h^{nm}=h$. 
	
	Let $N=\ker f$. We claim that $G=N\rtimes H$. 
	By definition, $N$ is normal in $G$ and $N\cap
	H=\{1\}$. To show that $G=NH$ note that 
	$|NH|=|N||H|$ and $G/N\simeq H$.
\end{proof}

\begin{exercise}
	Let $H$ be a central subgroup of a finite group $G$. If $|H|$
	and $|G/H|$ are coprime, then $G\simeq H\times G/H$.
\end{exercise}

%\begin{proof}
%	Es consecuencia inmediata del corolario~\ref{corollary:semidirecto} pues
%	$H$ es normal por ser un subgrupo central.
%\end{proof}

% TODO: Transitivity of the transfer

% serre, 7.12
An application to infinite groups taken from Serre's book 
\cite[7.12]{MR3469786}. 

\begin{theorem}
	Let $G$ be a torsion-free group that contains a finite-index subgroup isomorphic to  
	$\Z$. Then $G\simeq\Z$.
\end{theorem}

\begin{proof}
	We may assume that $G$ contains a finite-index normal subgroup isomorphic to $\Z$. Indeed, 
	if $H$ is a finite-index subgroup of $G$ such that $H\simeq\Z$, then 
	$K=\cap_{x\in G}xHx^{-1}$ is a non-trivial normal subgroup of $G$ (because $K=\Core_G(H)$ and 
	$G$ has no torsion) and hence $K\simeq\Z$ (because  
	$K\subseteq H$) and $(G:K)=(G:H)(H:K)$ is finite.
	The action of $G$ on $K$ by conjugation induces a group homomorphism  
	$\epsilon\colon G\to\Aut(K)$. Since $\Aut(K)\simeq\Aut(\Z)=\{-1,1\}$, 
	there are two cases to consider.
	
	Assume first that $\epsilon=\id$. Since $K\subseteq Z(G)$, let
	$\nu\colon G\to K$ be the transfer homomorphism. By
	Proposition~\ref{pro:center} (more precisely, 
	by Exercise \ref{xca:K_central}), $\nu(g)=g^n$, where $n=(G:K)$. Since
	$G$ has no torsion, $\nu$ is injective. Thus
	$G\simeq\Z$ because it is isomorphic to a subgroup of $K$.

	Assume now that $\epsilon\ne\id$. Let $N=\ker\epsilon\ne G$. Since
	$K\simeq\Z$ is abelian, $K\subseteq N$. The result proved in the previous paragraph 
	applied to $\epsilon|_N=1$ implies that $N\simeq\Z$, as 
	$N$ contains a finite-index subgroup isomorphic to $\Z$. Let $g\in G\setminus N$. 
	Since $N$ is normal in $G$, $G$ acts by conjugation on $N$ and hence 
	there exists a group homomorphism $c_g\in\Aut(N)\simeq\{-1,1\}$. Since
	$K\subseteq N$ y $g$ acts non-trivially on $K$, 
	\[
	c_g(n)=gng^{-1}=n^{-1}
	\]
	for all $n\in N$.  Since 
	$g^2\in N$, 
	\[
		g^2=gg^2g^{-1}=g^{-2}.
	\]
	Therefore $g^4=1$, a contradiction since $g\ne1$ and $G$ has no torsion.
\end{proof}

Before giving another application of the transfer map, we 
prove Dietzman's theorem: 

\begin{theorem}[Dietzmann]
	\index{Dietzmann's theorem}
	\label{theorem:Dietzmann} 
	Let $G$ be a group and $X\subseteq G$ be a finite subset of $G$ closed by
	conjugation. If there exists $n$ such that $x^n=1$ for all $x\in X$, then
	$\langle X\rangle$ is a finite subgroup of $G$.
\end{theorem}

\begin{proof}
	Let $S=\langle X\rangle$. Since $x^{-1}=x^{n-1}$, every element of $S$ can be 
	written as a finite product of elements of $X$. 
	Fix $x\in X$. We claim that if $x\in X$ appears $k\geq 1$ times 
	in the word $s$, then we can write $s$ as a product of $m$
	elements of $X$, where the first $k$ elements are equal to $x$. Suppose that 
	\[
	s=x_1x_2\cdots x_{t-1}xx_{t+1}\cdots x_m,
	\]
	where $x_j\ne x$ for all $j\in\{1,\dots,t-1\}$. Then 
	\[
		s=x(x^{-1}x_1x)(x^{-1}x_2x)\cdots (x^{-1}x_{t-1}x)x_{t+1}\cdots x_m
	\]
	is a product of $m$ elements of $X$ since $X$ is closed under conjugation and 
	the first element is $x$. The same argument implies that $s$
	can be written as 
	\[
		s=x^ky_{k+1}\cdots y_m,
	\]
	where each $y_j$ belongs to $X\setminus\{x\}$.

	Let $s\in S$ and write $s$ as a product of $m$ elements of 
	$X$, where $m$ is minimal. We need to show that 
	$m\leq (n-1)|X|$. 
	If $m>(n-1)|X|$, 
	then at least one $x\in X$ appears exactly $n$ 
	times in the representation of 
	$s$. Without loss of generality, we write 
	\[
		s=x^nx_{n+1}\cdots x_m=x_{n+1}\cdots x_m,
	\]
	a contradiction to the minimality of $m$. 
\end{proof}

The second result goes back to Schur:

\begin{theorem}[Schur]
\index{Schur's theorem}
\label{thm:Schur}
	Let $G$ be a group. 
	If $Z(G)$ has finite index in $G$, then $[G,G]$ is finite.
\end{theorem}

\begin{proof}
	Let $n=(G:Z(G))$ and  
	$X$ be the set of commutators of $G$. We claim that $X$ is finite, in fact
	$|X|\leq n^2$.
	A routine calculation shows that the map 
	\[
		\varphi\colon X\to G/Z(G)\times G/Z(G),\quad [x,y]\mapsto (xZ(G),yZ(G)),
	\]
	is well-defined. It is, moreover, 
	injective: if $(xZ(G),yZ(G))=(uZ(G),vZ(G))$, then $u^{-1}x\in Z(G)$, 
	$v^{-1}y\in Z(G)$. Thus 
	\begin{align*}
		[u,v]&=uvu^{-1}v^{-1}=uv(u^{-1}x)x^{-1}v^{-1}=xvx^{-1}(v^{-1}y)y^{-1}=xyx^{-1}y^{-1}=[x,y].
	\end{align*}
	Moreover, $X$ is closed under conjugation, as 
	\[
		g[x,y]g^{-1}=[gxg^{-1},gyg^{-1}]
	\]
	for all $g,x,y\in G$. Since $G\to Z(G)$, $g\mapsto g^n$ is a group
	homomorphism, Proposition~\ref{pro:center} implies that $[x,y]^n=[x^n,y^n]=1$ for
	all $[x,y]\in X$.  The theorem follows from applying Dietzmann's theorem. 
\end{proof}

\begin{exercise}
    Let $G$ be the group with generators $a,b,c$ and 
    relations $ab=ca$, $ac=ba$ and $bc=ab$. Prove the following statements:
    \begin{enumerate}
        \item $G$ is infinite and non-abelian.
        \item $Z(G)$ has finite index in $G$ and every conjugacy class of $G$ is finite.
        \item $[G,G]$ is finite. 
        \item The subgroup $N=\langle a^3\rangle$ of $G$ 
        generated by $a^3$ is central 
        and $G/N$ is finite.
    \end{enumerate}
\end{exercise}

We conclude the section with some results similar to that of Schur. 

\begin{theorem}[Niroomand]
\index{Niroomand´s theorem}
\label{thm:Niroomand}
	If the set of commutators of a group $G$ is finite, then 
	$[G,G]$ is finite.
\end{theorem}

\begin{proof}
 	Let $C=\{[x_1,y_1],\dots,[x_k,y_k]\}$ be the (finite) set of commutators of $G$ and  
	$H=\langle x_1,x_2,\dots,x_k,y_1,y_2,\dots,y_k\rangle$. Since $C$ is a set of commutators of $H$, 
	it follows that 
	$[G,G]=\langle C\rangle\subseteq [H,H]$. To simplify the notation we write 
	$H=\langle h_1,\dots,h_{2k}\rangle$. 	
 	Since $h\in Z(H)$ if and only if $h\in C_H(h_i)$ for all 
	$i\in\{1,\dots,2k\}$, we conclude that $Z(H)=C_H(h_1)\cap\cdots\cap C_H(h_{2k})$. Moreover, if 
	$h\in H$, then $hh_ih^{-1}=ch_i$ for some $c\in C$. Thus the conjugacy class of each 
	$h_i$ contains at most as many elements as $C$. This implies that 
	\[
		|H/Z(H)|=|H/\cap_{i=1}^{2k} C_H(h_i)|\leq\prod_{i=1}^{2k} (H:C_H(h_i))\leq |C|^{2k}.
	\]
	Since $H/Z(H)$ is finite, $[H,H]$ is finite. Hence  
	$[G,G]=\langle C\rangle\subseteq [H,H]$ 
	is a finite group. 
\end{proof}

\begin{theorem}[Hilton--Niroomand]
	\index{Hilton--Niroomand´s theorem}
	\label{thm:HiltonNiroomand}
	Let $G$ be a finitely generated group. If $[G,G]$ is finite and $G/Z(G)$ is generated by
	$n$ elements, then  
	\[
	|G/Z(G)|\leq |[G,G]|^n. 
	\]
\end{theorem}

\begin{proof}
	Assume that $G/Z(G)=\langle x_1Z(G),\dots,x_nZ(G)\rangle$. Let 
	\[
		f\colon G/Z(G)\to [G,G]\times\cdots\times [G,G],
		\quad
		y\mapsto ([x_1,y],\dots,[x_n,y]).
	\]
	Note that $f$ is well-defined: If $y\in G$ y $z\in Z(G)$, then $[x_i,y]=[x_i,yz]$ for all $i$. 
	Then $f(yz)=f(y)$.
	 
	The map $f$ is injective. Assume that $f(xZ(G))=f(yZ(G))$. Then 
	$[x_i,x]=[x_i,y]$ for all $i\in\{1,\dots,n\}$. For each $i$ we compute  
	\begin{align*}
		[x^{-1}y,x_i] &= x^{-1}[y,x_i]x[x^{-1},x_i]\\
		&=x^{-1}[y,x_i][x_i,x]x=x^{-1}[x_i,y]^{-1}[x_i,x]x=x^{-1}[x_i,y]^{-1}[x_i,y]x=1.
	\end{align*}
	This implies that $x^{-1}y\in Z(G)$. Indeed, since  
	every $g\in G$ can be written as $g=x_kz$ for some $k\in\{1,\dots,n\}$ and some $z\in Z(G)$, 
	it follows that 
    \[
    [x^{-1}y,g]=[x^{-1}y,x_kz]=[x^{-1}y,x_k]=1.
    \]
    Since $f$ is injective, 
	$|G/Z(G)|\leq |[G,G]|^n$. 
\end{proof}

\begin{exercise}
Prove Theorem~\ref{thm:HiltonNiroomand} from Theorem~\ref{thm:Niroomand}. 
\end{exercise}



