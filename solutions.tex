\chapter*{Some solutions}

\pagestyle{plain}
\fancyhf{}
\fancyhead[LE,RO]{Rings and modules}
\fancyhead[RE,LO]{Some solutions}
\fancyfoot[CE,CO]{\leftmark}
\fancyfoot[LE,RO]{\thepage}

\addcontentsline{toc}{chapter}{Some solutions}

\begin{sol}{xca:A1Bm}
    Let $A=\{a\}$ and $B=\{b_1,\dots,b_m\}$. 
    If $AB$ admits a non-unique product, say $x=ab_i=ab_j$ with $i\ne j$, 
    then $b_i=b_j$, a contradiction.
\end{sol}

\begin{sol}{xca:gABh}
    If $AB$ admits a non-unique product $x=ab=a_1b_1$, then so does $(gA)(Bh)$, as
    $(ga)(bh)=(ga_1)(b_1h)$ is a non-unique product of $(gA)(Bh)$. 
    The converse is trivial. 
\end{sol}

\begin{sol}{xca:A2Bm}
    Assume that $AB$ contains no unique products. 
    By Exercise \ref{xca:gABh} we may assume that $A=\{1,a\}$ and $B=\{1,b_2,\dots,b_m\}$. 
    We claim that $a^k\in B$ for all $k\geq0$. We proceed by induction on $k$. The case
    $k=0$ is easy, as $a^0=1\in B$. Now if $a^k\in B$, then $a^{k+1}=aa^{k}\in AB$. Since $G$ has no torsion and 
    $AB$ contains no unique products, 
    $a^{k+1}=b_j$ for some $j$. It follows that $\{a^k:k\geq0\}\subseteq B$, a contradiction. 
\end{sol}