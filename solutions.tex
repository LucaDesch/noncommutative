\section*{Some solutions}

\pagestyle{plain}
\fancyhf{}
\fancyhead[LE,RO]{Non-commutative algebra}
\fancyhead[RE,LO]{Some solutions}
\fancyfoot[CE,CO]{\leftmark}
\fancyfoot[LE,RO]{\thepage}

\addcontentsline{toc}{chapter}{Some solutions}


\begin{sol}{xca:derived_length}
    Let $d$ be the derived length of $G$. Thus $d$ is the smallest integer
    such that $G^{(d)}=\{1\}$. Since $G$ is nilpotent of class $m$, 
    \[
    G^{(m+1)}=\{1\}\ne G^{(d-1)}\subseteq G^{2^{d-1}}.
    \]
    Hence $2^{d-1}<m+1$. This implies $22^{d-1}\leq m$ and hence
    $d-1\leq\log_2m$. 
\end{sol}

\begin{sol}{xca:non_trivial:C2}
    Let $G=\langle g\rangle$ be the cyclic group of order two. 
    Then $1-\frac{\sqrt{2}}{2}g$ is a non-trivial unit, as 
    $\left(1-\frac{\sqrt{2}}{2}g\right)\left(2+\sqrt{2}g\right)=1$.
\end{sol}

\begin{sol}{xca:A1Bm}
    Let $A=\{a\}$ and $B=\{b_1,\dots,b_m\}$. 
    If $AB$ admits a non-unique product, say $x=ab_i=ab_j$ with $i\ne j$, 
    then $b_i=b_j$, a contradiction.
\end{sol}

\begin{sol}{xca:gABh}
    If $AB$ admits a non-unique product $x=ab=a_1b_1$, then so does $(gA)(Bh)$, as
    $(ga)(bh)=(ga_1)(b_1h)$ is a non-unique product of $(gA)(Bh)$. 
    The converse is trivial. 
\end{sol}

\begin{sol}{xca:A2Bm}
    Assume that $AB$ contains no unique products. 
    By Exercise \ref{xca:gABh} we may assume that $A=\{1,a\}$ and $B=\{1,b_2,\dots,b_m\}$. 
    We claim that $a^k\in B$ for all $k\geq0$. We proceed by induction on $k$. The case
    $k=0$ is easy, as $a^0=1\in B$. Now if $a^k\in B$, then $a^{k+1}=aa^{k}\in AB$. Since $G$ has no torsion and 
    $AB$ contains no unique products, 
    $a^{k+1}=b_j$ for some $j$. It follows that $\{a^k:k\geq0\}\subseteq B$, a contradiction. 
\end{sol}

\begin{sol}{xca:0=1}
The first claim follows from the compatibility condition~\eqref{eq:compatibility} with
$c=1$.  To prove the second claim let $d=b+c$.
Then~\eqref{eq:compatibility} becomes 
\[
	a\circ d =a\circ b-a+a\circ (-b+d)
\]
and the claim follows. The third claim is
proved similarly.
\end{sol}

\begin{sol}{xca:lambda}
The inverse of $\lambda_a$ is given by $\lambda^{-1}_a\colon A\to A$, $b\mapsto a'\circ (a+b)$. To prove
that $\lambda_a\in\Aut(A,+)$ we note that
\[
\lambda_a(b+c)=-a+a\circ(b+c)=-a+a\circ b-a+a\circ c=\lambda_a(b)+\lambda_a(c).
\]
Note that $\lambda_a(b)=-a+a\circ b=a\circ (a'+b)$, for all $a,b\in A$. Hence 
\begin{align*}
\lambda_a(\lambda_b(c))&=a\circ (a'+b\circ (b'+c))=-a+a\circ b\circ (b'+c)\\
&=-a+a-a\circ b+a\circ b\circ c=-a\circ b+a\circ b\circ c=\lambda_{a\circ b}(c).\qedhere    
\end{align*}
\end{sol}

\begin{sol}{xca:mu}
    Note that
    \[\mu_a(b)=\lambda_b(a)'\circ b\circ a=(b\circ (b'+a))'\circ b\circ a=(b'+a)'\circ a,\]
    for all $a,b\in A$. Hence $\mu_a$ is bijective and
    \[\mu_a^{-1}(b)=((b\circ a')'-a)'=(a\circ b'-a)'=(b'+a')'\circ a',\]
    for all $a,b\in A$. Now we have
    \begin{align*}
        \mu_b(\mu_a(c))&=\mu_b((c'+a)'\circ a)=(a'\circ (c'+a)+b)'\circ b\\
        &=(a'\circ c'-a'+b)'\circ b=(a'\circ (c'+a\circ b))'\circ b\\
        &=(c'+a\circ b)'\circ a\circ b=\mu_{a\circ b}(c),
    \end{align*}
    for all $a,b,c\in A$. Therefore the result follows.
\end{sol}

\begin{sol}{xca:socle}
    Let $b\in A$ and $a\in\Soc(A)$. Since   
    \[
    b'\circ (b\circ a+b)=a-b'
    \text{ and }
    b'\circ (b+b\circ a)=-b'+a,
    \]
    the first claim follows since
    $a\in Z(A,+)$.
    Now we prove the second claim:
    \[
    b\circ a\circ b'=b\circ (a\circ b')=b\circ (a+b')=b\circ a-b=-b+b\circ
    a=\lambda_b(a).\qedhere
    \]
\end{sol}

\begin{sol}{xca:bijective}
	We first prove that restriction restriction $\pi_1|_G$ of $\pi_1$ onto $G$ is injective. Let $(a,f)\in G$ and $(b,g)\in G$
	be such that 
	$\pi_1(a,f)=\pi_1(b,g)$. Then $a=b$. Since $G$ is a
	subgroup, 
	\[
		(-f^{-1}(a),f^{-1})=(a,f)^{-1}\in G,
	    \quad
		(-g^{-1}(a),g^{-1})=(a,g)^{-1}\in G,
	\]
	and hence $f=g$ since
	\[
	(-f^{-1}(a),f^{-1})\cdot a=0=(-g^{-1}(a),g^{-1})\cdot a 
	\]
	and $G$ is a regular subgroup.
	Now we prove that $\pi_1|_G$ is surjective. Let $a\in A$. 
	Since $G$ is regular, there exists $(x,f)\in G$ such that $x+f(a)=(x,f)\cdot a=0$, so $(-f(a),f)\in G$ for some $f\in\Aut(A)$. 
	Then $(a,f^{-1})=(-f(a),f)^{-1}\in G$ and $\pi_2|_G(a,f^{-1})=a$. 
\end{sol}


