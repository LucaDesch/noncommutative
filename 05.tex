\section{14/03/2024}

\subsection{Super solvable groups}

\begin{definition}
\index{Group!super solvable}
A group $G$ is said to be \textbf{super solvable} if there exists a sequence 
\[
G=G_0\supseteq G_1\supseteq\cdots\supseteq G_n=\{1\}
\]
of normal subgroups of $G$ such that every 
quotient $G_{i-1}/G_i$ is cyclic. 
\end{definition}

In the previous definition, we do not require the group to be finite. Hence the quotients 
could be finite cyclic groups or isomorphic to $\Z$. 

\begin{example}
The dihedral group $\D_{n}$ of order $2n$ is super solvable, as 
\[	
\D_{n}\supseteq \langle
r\rangle\supseteq \{1\}
\]
is a sequence of normal subgroups with cyclic factors. 
\end{example}

Every solvable group is super solvable. See Exercise~\ref{xca:solvable}.

\begin{example}
The alternating group $\Alt_4$ solvable but not super solvable. The only 
proper non-trivial normal subgroup of $\Alt_4$ is 
	\[
	\{\id,(12)(34),(13)(24),(14)(23)\}\simeq C_2\times C_2.
	\]
Thus $\Alt_4$ does not have a sequence of normal subgroups 
with cyclic factors. 
\end{example}

\begin{exercise}
\label{xca:Aff_supersolvable}
Prove that $\Aff(\Z)$ is super solvable. 
\end{exercise}
% aff(Z) es súper-resoluble

\begin{example}
The group $\SL_2(3)$ is solvable but not super solvable. Here is a computer verification: 
\begin{lstlisting}
gap> IsSolvable(SL(2,3));
true
gap> IsSupersolvable(SL(2,3));
false
\end{lstlisting}
\end{example}

\begin{exercise}
\label{xca:super}
Prove the following statements: 
\begin{enumerate}
\item Every subgroup of a super solvable group is super solvable. 
\item Quotients of super solvable groups are super solvable. 
\end{enumerate}
\end{exercise}

% \begin{svgraybox}
% 	Sea $G$ un grupo súper-resoluble y sea 			
% 	\[ 
% 	G=G_0\supseteq G_1\supseteq \cdots\supseteq G_n=1 
% 	\] 
% 	una sucesión de subgrupos normales
% 	donde cada cociente $G_{i-1}/G_{i}$ es cíclico. 
% 	\begin{enumerate}
% 		\item Sea $H$ un subgrupo de $G$. Como $G$ es
% 			súper-resoluble, Sea 
% 			\[
% 			H=H\cap G_0\supseteq H\cap G_1\supseteq\cdots\supseteq H\cap G_n=1
% 			\]
% 			una sucesión de subgrupos de $H$. Cada $H\cap G_i$ es normal en $H$
% 			pues $G_i$ es normal en $G$. Fijemos $i\in\{1,\dots,n\}$ y sea
% 			$\pi_{i-1}\colon G_{i-1}\to G_{i-1}/G_{i}$ el morfismo canónico. La
% 			restricción de $\pi_{i-1}$ al subgrupo $H\cap G_{i-1}$ es un morfismo con
% 			núcleo $G_{i}\cap H$.  Al usar el teorema de isomorfismos vemos que 
% 			\[
% 			\frac{H\cap G_{i-1}}{H\cap G_{i}}\simeq \pi_{i-1}(H\cap G_i)\subseteq G_{i-1}/G_i
% 			\]
% 			es un grupo cíclico por ser subgrupo de un grupo cíclico. 
% 		\item Sea $K$ un subgrupo normal de $G$ y sea $\pi\colon G\to G/K$ el
% 			morfismo canónico. Para cada $i$ sea $Q_i=\pi(G_i)$. Cada $Q_i$ es
% 			normal en $Q_n=\pi(G_n)=G/K$ pues $G_i$ es normal en $G$. Como
% 			$G_{i-1}K=G_{i-1}(G_iK)$ para todo $i$, 
% 			el grupo
% 			\begin{align*}
% 			Q_{i-1}/Q_i
% 			&\simeq\frac{G_{i-1}/G_{i-1}\cap K}{G_i/G_i\cap K}
% 			\simeq \frac{G_{i-1}K/K}{G_{i}K/K}\\
% 			&\simeq\frac{ G_{i-1}K}{G_iK}
% 			\simeq\frac{ G_{i-1}(G_iK)}{G_iK}
% 			\simeq\frac{ G_{i-1}}{G_iK\cap G_{i-1}}
% 			\simeq\frac{ G_{i-1}/G_i}{G_iK\cap G_{i-1}/G_i}
% 			\end{align*}
% 			es cíclico por ser un cociente de un grupo cíclico.
% 	\end{enumerate}
% \end{svgraybox}

\begin{exercise}
\label{xca:directosuper}
Prove that the direct product of super solvable groups is super solvable. 
\end{exercise}

% \begin{svgraybox}
% 	Supongamos que $G$ admite una sucesión $G=G_0\supseteq G_1\supseteq
% 	\cdots\supseteq G_n=1$ de de subgrupos normales tales que cada cociente
% 	$G_{i-1}/G_i$ es cíclico, y que $H$ admite una sucesión $H=H_0\supseteq
% 	H_1\supseteq \cdots\supseteq H_m=1$ de subgrupos normales donde cada
% 	$H_{i-1}/H_i$ es cíclico. Consideramos la sucesión 
% 	\[
% 		1=G_0\times H_0\supseteq G_1\times H_0\supseteq\cdots\supseteq G_n\times H_0\supseteq G_n\times H_1\supseteq \cdots\supseteq G_n\times H_m=G\times H
% 	\]
% 	tiene factores cíclicos pues 
% 	cada $G_{i-1}\times H_0/G_i\times H_0\simeq G_{i-1}/G_i$ es cíclico y cada 
% 	$G_n\times H_{j-1}/G_n\times H_j$ también pues
% 	\[
% 	G_n\times H_{j-1}/G_n\times H_j
% 	\simeq \frac{GH_{j-1}/G}{GH_j/G}
% 	\simeq \frac{H_{j-1}/H_{j-1}\cap G}{H_j/H_j\cap G}\simeq H_{j-1}/H_j.
% 	\]
% \end{svgraybox}

\begin{exercise}
\label{xca:super}
Let $H$ and $K$ be normal subgroups of a group $G$ such that $G/K$ and $G/H$
are super solvable. Prove that $G/H\cap K$ is super solvable. 
\end{exercise}

% \begin{svgraybox}
% 	El producto directo $G/H\times G/K$ es súper-resoluble. Sea $\partial\colon
% 	G\to G/H\times G/K$, $g\mapsto (gH,gK)$.  Como $\ker\partial=H\cap K$, se
% 	tiene que $G/H\cap K\simeq\partial(G)$, que es súper-resoluble por ser un
% 	subgrupo de un grupo súper-resoluble.
% \end{svgraybox}

\begin{exercise}
\label{xca:Nciclico}
Let $N$ be a cyclic normal subgroup of $G$. If $G/N$ is super solvable, then 
$G$ is super solvable. 
\end{exercise}

% todo: arreglar 

% \begin{proof}
% 	Sea $\pi\colon G\to G/N$ el morfismo canónico y sea $Q=G/N$. Como $Q$ es
% 	súper-resoluble, tenemos una sucesión
% 	\[
% 		Q=Q_0\supseteq Q_1\supseteq \cdots\supseteq Q_n=\{1\}
% 	\]
% 	de subgrupos normales de $Q$ tales que cada cociente $Q_{i-1}/Q_i$ es
% 	cíclico. Cada elemento de la sucesión
% 	\[
% 	G=\pi^{-1}(Q)\supseteq\pi^{-1}(Q_1)\supseteq\cdots\supseteq \pi^{-1}(Q_n)=N\supseteq \{1\}
% 	\]
% 	es normal en $G$ (por la correspondencia) y dejamos como 
% 	ejercicio demostrar que cada cociente es cíclico. 
% % 	cada cociente es cíclico $N$ es cíclico. 
% % 	Queda como ejercicio demostrar 
% % 	y cada 
% % 	\[
% % 	\frac{\pi^{-1}(Q_j)}{\pi^{-1}(Q_{j+1})}
% % 		=\frac{Q_jN}{Q_{j+1}N}
% % 		\simeq\frac{Q_jN/N}{Q_{j+1}N/N}
% % 		\simeq\frac{Q_j(Q_{j+1}N)}{Q_{j+1}N}
% % 		\simeq\frac{Q_j/Q_{j+1}}{Q_{j+1}N\cap Q_j}
% % 	\]
% % 	es cíclico por ser cociente de un grupo cíclico.
% \end{proof}

\begin{theorem}
\label{thm:ZorCp}
Let $G$ be a super solvable non-trivial group. Then $G$ admits a sequence 
\[
G=G_0\supseteq G_1\supseteq\cdots\supseteq G_n=\{1\}
\]
of normal subgroups 
such that every quotient $G_{i-1}/G_i$ is cyclic of prime order or isomorphic to 
$\Z$.
\end{theorem}

\begin{proof}
Let $G=G_0\supseteq G_1\supseteq\cdots\supseteq G_n=\{1\}$ be a sequence of normal subgroups
of $G$ such that every quotient $G_{i-1}/G_i$ is cyclic. Let 
$i\in\{1,\dots,n\}$ be such that $G_{i-1}/G_i\simeq C_n$ for some non-prime  
$n$ and let $\pi\colon G_{i-1}\to G_{i-1}/G_i$ be the canonical map. 
Let $p$ be a prime divisor of $n$ and $H$ be a subgroup of $G$ such that 
$\pi(H)$ is a subgroup of $G_{i-1}/G_i$ of order $p$. By the correspondence theorem, 
$G_{i}\subseteq H\subseteq G_{i-1}$. 

We claim that $H$ is normal in $G$. Let $g\in G$. Since $\pi(gHg^{-1})$ is a subgroup of order $p$ of 
the cyclic group $G_{i-1}/G_i$, $\pi(gHg^{-1})=\pi(H)$. Then 
$gHg^{-1}=G_{i}H\subseteq H$ and hence $gHg^{-1}=H$. 
% 	\[
% 	\frac{gHg^{-1}}{G_i}=\frac{G_{i}H}{G_{i}}\simeq \frac{H}{G_i\cap H}=\frac{H}{G_i}
% 	\]
% 	y entonces $gHg^{-1}\subseteq H$.  

Note that $H/G_i$ is cyclic of prime order, as 
\[
H/G_i=H/H\cap G_i\simeq \pi(H)\simeq C_p. 
\]
Moreover, $G_{i-1}/H$ is cyclic, as 
\[
G_{i-1}/H\simeq\frac{G_{i-1}/G_i}{H/G_i}
\]
is the quotient of a cyclic group. 
	
We have shown that by adding $H$ to our sequence of normal subgroups, 
we obtain a sequence with cyclic factors where 
$H/G_{i}$ is cylic of prime order. Repeating this procedure, we obtain the desired result. 
\end{proof}

Let us discuss an immediate application. 

\begin{corollary}
A finite super solvable group admits a sequence 
of normal subgroups where each quotient is cyclic of prime order. 
\end{corollary}

% \begin{proof}
% 	Es consecuencia inmediata del teorema~\ref{theorem:ZorCp}.
% \end{proof}

We now discuss other properties of super solvable groups. 

\begin{theorem}
\label{thm:super_structure}
Let $G$ be a super solvable group. The following statement hold:  
\begin{enumerate}
\item If $N$ is minimal normal in $G$, then $N\simeq C_p$ for some prime number $p$.
\item If $M$ is maximal in $G$, then $(G:M)=p$ for some prime number $p$.
\item The commutator subgroup $[G,G]$ is nilpotent. 
\item If $G$ is non-abelian, there exists a normal subgroup $N\ne G$ such that
	$Z(G)\subsetneq N$.
\end{enumerate}
\end{theorem}

\begin{proof}
Let us prove the first claim. Since $G$ is super solvable, there exists a sequence 
\[
G=G_0\supseteq G_1\supseteq
G_2\supseteq\cdots\supseteq G_n=\{1\}
\]
of normal subgroups with cyclic factors. Since 
each $G_i\cap N$ is a normal subgroup of $G$ contained in $N$, 
the minimality implies that 
each $G_i\cap N$ is either trivial or equal to $N$. Moreover, $N\cap G_0=N$ and $N\cap
G_n=\{1\}$. Let $j$ be the smallest positive integer such that $N\cap G_j=\{1\}$. 
Since $N\subseteq G_{j-1}$ (because $N\cap G_{j-1}=N$), the composition 
	\[
	N\hookrightarrow G_{j-1}\to G_{j-1}/G_j
	\]
is an injective group homomorphism, as its kernel is equal $N\cap G_{j}=\{1\}$. 
Thus $N$ is cyclic, as it is isomorphic to a subgroup of the cyclic group $G_{i-1}/G_i$. 
If $G_{i-1}/G_i\simeq\Z$, then $N\simeq\Z$, a contradiction to the fact that $N$ is minimal normal. (For example, 
$2\Z$ is characteristic subgroup of $\Z$ and hence it is normal in $G$. Thus $N$ is cyclic and finite. Hence $N\simeq C_p$.)

We now prove the second claim. Let $M$ be a maximal subgroup of $G$. If $M$ is normal in $G$, 
then $G/M$ does not contain non-trivial proper subgroups. Then 
$G/M\simeq C_p$ for some prime number $p$. Assume that $M$ is not normal in $G$. 
Let $H=\cap_{g\in G}gMg^{-1}$ and $\pi\colon G\to G/H$ be the canonical map.  
Since $\pi(M)$ is maximal in 
	$\pi(G)=G/H$ and 
	\[
		(G:M)=(G/H:M/H)=(G/H:M/H\cap M)=(\pi(G):\pi(M)),
	\]
we may assume that $M$ does not contain non-trivial normal subgroups of $G$ (if needed, 
we just replace $G$ by $G/H$). Since $G$ is super solvable, there exists a sequence 
$G=G_0\supseteq G_1\supseteq\cdots\supseteq G_n=\{1\}$ of normal subgroups of $G$ 
with factors either cyclic of prime order or isomorphic to $\Z$. Let 
$N=G_{n-1}$. Since $N$ is cyclic, every subgroup of $N$ is characteristic 
in $N$ and hence normal in $G$. In particular, $M\cap N$ is normal in 
$G$ and therefore $M\cap N=\{1\}$. Since $M\subseteq
NM\subseteq G$, the maximality of $M$ implies that either $M=NM$ or $G=NM$.
Since $N\subseteq NM=M$ yields a contradiction, we conclude that $G=NM$.

If $N\simeq C_p$ for some prime $p$, then $(G:M)=p$ and the proof is complete. 
Assume that $N\simeq\Z$. Let $H$ be a proper subgroup of $N$. Since 
$H$ is characteristic in $N$, $H$ is normal in $G$. Since 
$M\subseteq HM\subseteq NM=G$, the maximality of $M$ implies that either $HM=M$ or 
$HM=G$. Since $HM=M$ implies $H\subseteq M\cap N=\{1\}$,
we may assume that $HM=G$. If $n\in N\setminus H$, then $n=hm$ for some 
$h\in H$ and $m\in M$. Then $h=n$, as $h^{-1}n\in N\cap M=\{1\}$, a contradiction. 

	Demostremos ahora la tercera afirmación. Como $G$ es súper-resoluble, existe
	una sucesión
	\[
	G=G_0\supseteq G_1\supseteq\cdots\supseteq G_n=\{1\}
	\]
	de subgrupos normales tal que cada $G_i/G_{i+1}$ es cíclico. Para cada
	$i\in\{0,\dots,n\}$ sea $H_i=[G,G]\cap G_i$. Como $[G,G]$ y los $G_i$ son
	normales en $G$, se tiene una sucesión
	\[
	[G,G]=H_0\supseteq H_1\supseteq\cdots\supseteq H_n=\{1\}
	\]
	de subgrupos normales de $G$. Como $H_i$ y $H_{i+1}$ es normal en $G$, el
	grupo $G$ actúa por conjugación en $H_i/H_{i+1}$. Esto induce un morfismo
	$\gamma\colon G\to\Aut(H_i/H_{i+1})$. Como $H_i/H_{i+1}$ es cíclico, 
	$\Aut(H_i/H_{i+1})$ es abeliano y luego $[G,G]\subseteq\ker \gamma$. Luego
	$[G,G]$ actúa trivialmente por conjugación en $H_{i}/H_{i+1}$ y entonces
	\[
	H_i/H_{i+1}\subseteq Z([G,G]/H_{i+1}).
	\]
	%%% TODO: explicar mejor

	Por último demostremos la cuarta afirmación. Como $G$ es no abeliano,
	$Z(G)\ne G$. Sea $\pi\colon G\to G/Z(G)$ el morfismo canónico.  El cociente
	$G/Z(G)$ es súper-resoluble y la sucesión
	\[
	G/Z(G)=\pi(G)\supseteq \pi(G_1)\supseteq\cdots\supseteq \pi(1)=1
	\]
	es una sucesión de subgrupos normales de $G/Z(G)$ con cocientes cíclicos.
	En particular, $1\ne \pi(G_1)$ es propio y normal en $G/Z(G)$.  Por el
	teorema de la correspondencia, $\pi^{-1}(\pi(G_1))\ne G$ es un subgrupo normal
	de $G$ que contiene propiamente a $Z(G)$. 
\end{proof}

\begin{example}
	Si $G$ es un grupo resoluble, no necesariamente $[G,G]$ es un grupo nilpotente. El grupo
	$\Sym_4$ es resoluble pero $[\Sym_4,\Sym_4]=\Alt_4$ no es nilpotente.
\end{example}

\begin{proposition}
	\label{proposition:psuper}
	Sea $p$ un número primo.  Todo $p$-grupo finito es súper-resoluble.
\end{proposition}

\begin{proof}
	Sea $G$ un contraejemplo de orden minimal. Podemos suponer que $|G|=p^n$
	con $n>1$ (pues si $n=1$ el grupo $G$ es trivialmente súper-resoluble).
	Como $G$ es un $p$-grupo, es nilpotente  y existe un subgrupo normal $N$ de
	orden $p$. El cociente $G/N$ tiene orden $p^{n-1}$ entonces es
	súper-resoluble pues $|G/N|<|G|$. Como $N$ es cíclico y $G/N$ es
	súper-resoluble, $G$ es súper-resoluble por la
	proposición~\ref{proposition:Nciclico}.
\end{proof}

Como todo grupo finito nilpotente es producto directo de (finitos) subgrupos de
Sylow, cada $p$-grupo es súper-resoluble y el producto directo de súper-resolubles es súper-resoluble, 
se obtiene el siguiente resultado:

\begin{proposition}
	Todo grupo finito nilpotente es súper-resoluble.
\end{proposition}

% \begin{proof}
% 	Todo grupo finito nilpotente es producto directo (finito) de subgrupos de
% 	Sylow. Como cada $p$-grupo es súper-resoluble por la
% 	proposición~\ref{proposition:psuper}, el resultado se obtiene
% 	inmediatamente del ejercicio~\ref{exercise:directosuper}.
% \end{proof}

\begin{theorem}
	Todo grupo súper-resoluble tiene subgrupos maximales.	
\end{theorem}

\begin{proof}
	Procederemos por inducción en la longitud de la sucesión de
	superresolubilidad. Si la longitud es uno, el teorema es cierto pues en
	este caso el grupo es cíclico. Supongamos entonces que $G$ admite una
	sucesión
	\[
		G=G_0\supseteq\cdots\supseteq G_k=\{1\}
	\]
	y que la afirmación es cierta para grupos súper-resolubles con sucesiones 
	de longitud $<k$. Como $G_{k-1}$ es normal en $G$, sea $\pi\colon G\to
	G/G_{k-1}$ el morfismo canónico. Entonces la sucesión
	\[
		G/G_{k-1}=\pi(G)\supseteq \pi(G_1)\supseteq\cdots\supseteq\pi(G_{k-1})=\{1\}
	\]
	prueba la resolubilidad de $\pi(G)$ y tiene longitud $<k$. Por hipótesis
	inductiva, $G/G_{k-1}$ admite subgrupos maximales y luego, por el teorema
	de la correspondencia, $G$ también admite subgrupos maximales.
\end{proof}

Los grupos resolubles o nilpotentes no siempre admiten
subgrupos maximales, ver por ejemplo $\Q$.

\begin{definition}
	\index{Grupo!que satisface la condición maximal para subgrupos}
	Se dice que un grupo $G$ satisface la \textbf{condición maximal para
	subgrupos} si 
	todo subconjunto $\mathcal{S}$ no vacío de subgrupos tiene un subgrupo
	maximal (es decir, no contenido en ningún otro subgrupo de $\mathcal{S}$). 
	%toda sucesión creciente
	%$S_1\subseteq S_2\subseteq S_3\subseteq\cdots$
	%de subgrupos es finita. 
	%%si todo subconjunto $\mathcal{S}$ 
	%%no vacío de subgrupos tiene un elemento maximal, es decir: existe
	%$M\in\mathcal{S}$ tal que $S\subseteq M$ para todo $S\in\mathcal{S}$.
\end{definition}

%\begin{lemma}
%	Un grupo $G$ satisface la la condición maximal para subgrupos si y sólo si
%	todo subconjunto $\mathcal{S}$ no vacío de subgrupos tiene un subgrupo
%	maximal (es decir, no contenido en ningún otro subgrupo de $\mathcal{S}$). 
%\end{lemma}

\begin{proposition}
	\label{pro:MAX=fg}
	Sea $G$ un grupo. Entonces $G$ satisface la condición maximal para
	subgrupos si y sólo si todo subgrupo de $G$ es finitamente generado.
\end{proposition}

\begin{proof}
	Supongamos que $G$ satisface la condición maximal para subgrupos y sea $H$
	un subgrupo de $G$.  Sea $\mathcal{S}$ el conjunto de subgrupos de $H$
	finitamente generados. Como $\mathcal{S}$ es no vacío (pues
	$1\in\mathcal{S}$), existe un elemento maximal $M\in\mathcal{S}$.  Sea
	$x\in H$. Como $\langle M,x\rangle\in\mathcal{S}$, $M=\langle M,x\rangle$ y
	luego $x\in M$. Como entonces $H=M$, $H$ es finitamente generado.
	%Supongamos que $G$ no es finitamente generado y satisface la condición maximal para subgrupos. Sea $1\ne g\in G$
	%y sea $S_1=\langle g_1\rangle$. Como $S_1\ne G$, existe $g_2\in G\setminus S_1$, y entonces 
	%$S_1\subseteq S_2=\langle x_1,x_2\rangle$. 

	Supongamos ahora que todo subgrupo de $G$ es finitamente generado. Si
	$\mathcal{S}$ es un subconjunto no vacío de subgrupos de $G$ sin elemento
	maximal, podemos construir una sucesión de subgrupos $S_1\subseteq
	S_2\subseteq\cdots$ que no se estabiliza (acá necesitamos utilizar el
	axioma de elección). Como la unión 
	\[
		S=\bigcup_{j\geq1}S_j 
	\]
	es un subgrupo de $G$, es finitamente generado y luego $S\subseteq S_k$
	para algún $k$ suficientemente grande, una contradicción.
\end{proof}

Una consecuencia inmediata. 

\begin{proposition}
    Sean $G$ un grupo y $H$ un subgrupo de $G$.  Si $G$ satisface la
	condición maximal para subgrupos entonces $H$ también. 
\end{proposition}

\begin{proposition}
	\label{proposition:max:G/N}
	Sea $G$ un grupo y sea $N$ un subgrupo normal de $G$.  Si $G/N$ y $N$
	satisfacen la condición maximal para subgrupos entonces $G$ también. 
\end{proposition}

\begin{proof} 
	Sea $\pi\colon G\to G/N$ el morfismo canónico.  Sea $\mathcal{S}$ un
	subconjunto no vacío de subgrupos de $G$. El conjunto $\{S\cap
	N:S\in\mathcal{S}\}$ tiene un elemento maximal $A$ y el conjunto
	$\{\pi(S):S\in\mathcal{S},S\cap N=A\}$ tiene un elemento maximal $B$. Sea
	$S\in\mathcal{S}$ tal que $\pi(S)=B$ y $S\cap N=A$. Si $S$ no es maximal en
	$\mathcal{S}$, existe $T\in\mathcal{S}$ tal que $S\subseteq T$, $N\cap T=A$
	y $\pi(T)=B$. Sea $x\in T\setminus S$. Como $\pi(xN)=\pi(x)\in\pi(T)=B$,
	existe $y\in S$ tal que $xN=yN$. Luego $y^{-1}x\in N\cap T=A=N\cap S$, una
	contradicción pues $x\not\in S$. 
\end{proof}

% TODO: agregar teorema de Huppert (ver por ejemplo Robinson, p. 268)
% corolario: G super si y sólo G/\Phi(G) super
% teorema de Iwasawa, Hall 342-345, 19.3
% teorema de Zappa-Ore, Duke 5 (1939), 431-460, Duke 6 (1940), 511-512

%\begin{definition}
%	\index{Grupo!que satisface la condición minimal para subgrupos}
%	Se dice que un grupo $G$ satisface la \textbf{condición minimal para
%	subgrupos} si todo subconjunto no vacío de subgrupos tiene un elemento
%	minimal.
%\end{definition}
%
%\begin{example}
%	El grupo $\Z$ no satisface la condición minimal para subgrupos pues
%	el conjunto $\{2^n\Z:n\in\N\}$ no posee elemento minimal. 
%\end{example}
%
%\begin{proposition}
%	Sea $G$ un grupo que satisface la condición minimal sobre subgrupos.
%	Entonces todo elemento de $G$ tiene orden finito.
%\end{proposition}
%
%\begin{proof}
%	Si existe $x\in G$ de orden infinito, la sucesión $\mathcal{S}$ de subgrupos 
%	\[
%	\langle x\rangle\supsetneq\langle x^2\rangle\supsetneq\langle
%	x^4\rangle\supsetneq\cdots\supsetneq\langle x^{2^k}\rangle\supsetneq\cdots
%	\]
%	tiene infinitos elementos y luego no posee un elemento minimal. 
%\end{proof}
%
%\begin{exercise}
%	\label{exercise:min:N}
%	Sea $G$ un grupo y sea $H$ un subgrupo de $G$.  Si $G$ satisface la
%	condición minimal para subgrupos entonces $H$ también. 
%\end{exercise}
%
%\begin{svgraybox}
%	Si $\mathcal{S}$ es un subconjunto no vacío de subgrupos de $H$, entonces
%	$\mathcal{S}$ posee un elemento minimal por ser un subconjunto no vacío de
%	subgrupos de $G$.
%\end{svgraybox}
%
%\begin{proposition}
%	\label{proposition:min:G/N}
%	Sea $G$ un grupo y sea $N$ un subgrupo normal de $G$.  Si $G/N$ y $N$
%	satisfacen la condición minimal para subgrupos entonces $G$ también. 
%\end{proposition}
%
%\begin{proof}
%	
%\end{proof}

\begin{proposition}
	\label{proposition:superfg}
	Todo grupo súper-resoluble satisface la condición maximal para subgrupos. En
	particular, todo grupo súper-resoluble es finitamente generado.
\end{proposition}

\begin{proof}
	Procederemos por inducción en la longitud $n$ de la sucesión de
	súper-resolubilidad.  El caso $n=1$ es trivial pues entonces $G$ es cíclico.
	Supongamos entonces que el resultado vale para grupos súper-resolubles con
	serie de longitud $\leq n-1$.  Sea $G$ un grupo súper-resoluble no trivial y sea 
	\[
	G=G_0\supsetneq
	G_1\supsetneq\cdots\supsetneq G_n=\{1\}
	\]
	una sucesión de subgrupos normales de $G$ con factores cíclicos. Como
	$G_{1}$ es súper-resoluble por el ejercicio~\ref{xca:super},
	$G_{1}$ satisface la condición maximal para subgrupos por hipótesis
	inductiva.  Luego, por la proposición~\ref{proposition:max:G/N}, $G$ satisface la condición maximal para subgrupos porque
	$G/G_{1}$ es un grupo cíclico.
\end{proof}

%\begin{proposition}\
%	\begin{enumerate}
%		\item Si un grupo súper-resoluble admite una serie de composición,
%			entonces es finito. 
%		\item Si un grupo súper-resoluble satisface la condición de minimal en
%			subgrupos entonces es finito.
%	\end{enumerate}
%\end{proposition}
%
%\begin{proof}
%	%Para probar la segunda afirmación obsevemos que todo cociente de $G$ es súper-resoluble 
%	%y que por el teorema~\ref{theorem:ZorCp} todo factor de la serie debe ser finito pues
%	%$\Z$ no satisface la condición minimal para subgrupos.
%\end{proof}

\begin{example}
	El grupo abeliano $\Q$ es nilpotente pero no es súper-resoluble
	porque no es finitamente generado.
\end{example}

%\begin{example}
%	El grupo $\Sym_3$ es súper-resoluble pero no es nilpotente. 
%\end{example}

Si $G$ es un grupo y $x_1,\dots,x_{n+1}\in G$ se define 
\[
[x_1,\dots,x_{n+1}]=\left[ [x_1,\dots,x_n],x_{n+1} \right],\quad
n\geq1.
\]

\begin{lemma}
	\label{lemma:G_n}
	Sea $G$ un grupo finitamente generado, digamos $G=\langle X\rangle$ con $X$
	finito. Para cada $n\geq2$ se define
	\[
		G_n=\langle g[x_1,\dots,x_n]g^{-1}:x_1,\dots,x_n\in X,\,g\in G\rangle.
	\]
	Entonces $G_n=\gamma_n(G)$ para todo $n\geq2$. 
\end{lemma}

\begin{proof}
	Observemos que cada $G_n$ es normal en $G$.  Procederemos por inducción en
	$n$. El caso $n=2$ es trivial. Supongamos entonces que
	$\gamma_{n-1}(G)=G_{n-1}$. Sean $x_1,\dots,x_n\in X$. Como
	$[x_1,\dots,x_n]\in\gamma_{n}(G)$, $G_{n-1}\subseteq\gamma_n(G)$. Sea
	$N=G_n$ y sea $\pi\colon G\to G/N$ el morfismo canónico. El grupo $G/N$ es
	finitamente generado. Como
	\[
	[\pi([x_1,\dots,x_{n-1}]),\pi(x_n)]=\pi([x_1,\dots,x_n])=1,
	\]
	se tiene que $\pi([x_1,\dots,x_{n-1}])\in Z(G/N)$. Luego
	$\pi(g[x_1,\dots,x_n]g^{-1})=1$ para todo $g\in G$ y, por hipótesis
	inductiva, 
	se concluye que 
	\[
	\pi(\gamma_{n-1}(G))=\pi(G_{n-1})\subseteq Z(G/N).
	\]
	Como entonces 
	\[
	\pi(\gamma_{n}(G))=\pi([\gamma_{n-1}(G),G])=[\pi(\gamma_{n-1}(G)),\pi(G)]=\{1\},
	\]
	se concluye que $\gamma_n(G)\subseteq N=G_n$.
\end{proof}

\begin{lemma}
	\label{lemma:gamma_n/gamma_n+1}
	Sea $G$ un grupo finitamente generado.  Entonces
	$\gamma_n(G)/\gamma_{n+1}(G)$ es finitamente generado. 
\end{lemma}

\begin{proof}
	Supongamos que $G=\langle X\rangle$ con $X$ finito. 
	Al escribir 
	\[
	g[x_1,\dots,x_n]g^{-1}=[g,[x_1,\dots,x_n]][x_1,\dots,x_n]
	\]
	y usar el lema~\ref{lemma:G_n} para obtener 
	que $[g,[x_1,\dots,x_n]]\in \gamma_{n+1}(G)=G_{n+1}$, 
	\[
	g[x_1,\dots,x_n]g^{-1}\equiv [x_1,\dots,x_n]\bmod \gamma_{n+1}(G). 
	\]
	Luego $\gamma_{n}(G)/\gamma_{n+1}(G)$ está generado por 
	el conjunto finito 
	\[
	\{[x_1,\dots,x_n]\gamma_{n+1}(G):x_1,\dots,x_n\in X\}. \qedhere 
	\]
\end{proof}

\begin{theorem}
	\label{theorem:super=fg}
	Sea $G$ un grupo nilpotente. Entonces $G$ es súper-resoluble si y sólo si
	$G$ es finitamente generado.
\end{theorem}

\begin{proof}
	Si $G$ es súper-resoluble, es finitamente generado por la
	proposición~\ref{proposition:superfg}.  Supongamos que $G$ es finitamente
	generado y nilpotente. Como por el lema~\ref{lemma:gamma_n/gamma_n+1} cada
	$\gamma_{n}(G)/\gamma_{n+1}(G)$ es finitamente generado, digamos por
	$y_1,\dots,y_m$. Sea $\pi\colon G\to G/\gamma_{n+1}(G)$ el morfismo
	canónico.  Para cada $j\in\{1,\dots,m\}$ sea 
	\[
	K_j=\langle \gamma_{n+1}(G),y_1,\dots,y_j\rangle.
	\]
	Como
	$[K_j,G]\subseteq [\gamma_n(G),G]=\gamma_{n+1}(G)$, 
	se tiene que $\pi(K_j)$ es central en $\pi(G)$. Luego $\pi(K_j)$ es normal
	en $\pi(G)$ y por lo tanto $K_j$ es normal en $G$. Como cada $K_j/K_{j-1}$
	es cíclico generado por $y_jK_{j-1}$, entre $\gamma_n(G)$ y
	$\gamma_{n+1}(G)$ pudimos construir una sucesión de subgrupos normales de
	$G$ con factores cíclicos. Como $G$ es nilpotente, existe $c$ tal que
	$\gamma_{c+1}(G)=1$ y luego $G$ es súper-resoluble.
\end{proof}

\begin{corollary}
	\label{corollary:nilpotente=>max}
	Todo grupo nilpotente finitamente generado satisface la condición maximal
	en subgrupos.
\end{corollary}

\begin{proof}
	Es consecuencia del teorema~\ref{theorem:super=fg} y la
	proposición~\ref{proposition:superfg}.
\end{proof}

\begin{theorem}
	Sea $G$ un grupo nilpotente y finitamente generado. Entonces $T(G)$ es
	finito.
\end{theorem}

\begin{proof}
	Como $G$ es nilpotente, $G$ satisface la condición maximal para subgrupos
	por el corolario~\ref{corollary:nilpotente=>max} y entonces
	todo subgrupo de $G$ es finitamente generado. Como $T(G)$ es un subgrupo por el teorema~\ref{theorem:T(nilpotent)}, 
	es finitamente generado y de torsión. Luego $T(G)$ es finito por el
	teorema~\ref{theorem:T(G)finito}.
\end{proof}



