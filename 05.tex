\chapter{}

\topic{Bi-ordered groups}

Based on Example~\ref{example:k[Z]} we will study 
some properties of groups. 

Recall that a \textbf{total order} is a partial order in which any two elements are comparable. This
means that a total order is a binary relation $\leq$ on some set $X$ 
such that for all $x,y,z\in X$ one has 
\begin{enumerate}
    \item $x\leq x$.
    \item $x\leq y$ and $y\leq z$ imply $x\leq z$.
    \item $x\leq y$ and $y\leq x$ imply $x=y$.
    \item $x\leq y$ or $y\leq x$. 
\end{enumerate}
A set equipped with a total order is a \textbf{totally ordered set}. 

\begin{definition}
	\index{Group!bi-ordered}
	A group $G$ is \textbf{bi-ordered} if there exists a total order 
	$<$ in $G$
	such that $x<y$ implies that $xz<yz$ and $zx<zy$ for all $x,y,z\in G$.
\end{definition}

\begin{example}
	The group $\R_{>0}$ of positive real numbers is bi-ordered. 
\end{example}

The multiplicative group $\R\setminus\{0\}$ is not bi-ordered. Why?

\begin{exercise}
	Let $G$ be a bi-ordered group and $x,x_1,y,y_1\in G$. Prove that
	$x<y$ and $x_1<y_1$ imply $xx_1<yy_1$.
\end{exercise}

Clearly, bi-orderability is preserved under taking subgroups. 

\begin{exercise}
	Let $G$ be a bi-ordered group and $g,h\in G$. Prove that $g^n=h^n$
	for some $n>0$ implies $g=h$.
\end{exercise}

The following result goes back to Neumann.

\begin{exercise}
    Let $G$ be a bi-ordered group and $g,h\in G$. Prove that $g^n\in C_G(h)$ if 
    and only if $g\in C_G(h)$.
\end{exercise}

Bi-ordered groups do not behave nicely under extensions:

\begin{exercise}
\label{xca:BO_sequence}
    Let $1\to K\to G\to Q\to 1$ be an exact sequence of groups. Assume that $K$ and $Q$ 
    are bi-ordered. Prove that $G$ is bi-ordered if and only if 
    $x<y$ implies $gxg^{-1}<gyg^{-1}$ for all $x,y\in K$ and $g\in G$. 
\end{exercise}

\begin{definition}
	Let $G$ be a bi-ordered group. The \textbf{positive cone} of $G$  
	is the set $P(G)=\{x\in
	G:1<x\}$.
\end{definition}

Let us state some properties of positive cones. 

\begin{proposition}
	\label{pro:biordenableP1}
	Let $G$ be a bi-ordered group and let $P$ be its positive cone. 
	\begin{enumerate}
		\item $P$ is closed under multiplication, i.e. $PP\subseteq P$. 
		\item $G=P\cup P^{-1}\cup \{1\}$ (disjoint union).
		\item $xPx^{-1}=P$ for all $x\in G$.
	\end{enumerate}
\end{proposition}

\begin{proof}
	If $x,y\in P$ and $z\in G$, then, since $1<x$ and $1<y$, it follows that 
	$1<xy$.  Thus $1=z1z^{-1}<zxz^{-1}$. It remains to prove the second claim.  
	If $g\in G$, then $g=1$ or $g>1$ or $g<1$. Note that $g<1$ if and only if 
	$1<g^{-1}$, so the claim follows. 
\end{proof}

The previous proposition admits a converse statement. 

\begin{proposition}
	\label{pro:biordenableP2}
	Let $G$ be a group and $P$ be a subset of $G$ such that 
	$P$ is closed under multiplication, $G=P\cup P^{-1}\cup \{1\}$ (disjoint union) and
	$xPx^{-1}=P$ for all $x\in G$. Let $x<y$ whenever  
	$yx^{-1}\in P$. Then $G$ is bi-ordered with positive
	cone is $P$.
\end{proposition}

\begin{proof}
	Let $x,y\in G$. Since $yx^{-1}\in G$ and $G=P\cup
	P^{-1}\cup\{1\}$ (disjoint union), 
	either $yx^{-1}\in P$ or $xy^{-1}=(yx^{-1})^{-1}\in
	P$ or $yx^{-1}=1$. Thus either $x<y$ or $y<x$ or $x=y$. If $x<y$ and $z\in
	G$, then $zx<zy$, as $(zy)(zx)^{-1}=z(yx^{-1})z^{-1}\in P$ and  
	$zPz^{-1}=P$. Moreover, $xz<yz$ since $(yz)(xz)^{-1}=yx^{-1}\in P$. To prove
	that $P$ is the positive cone of $G$ note that 
	$x1^{-1}=x\in P$ if and only if $1<x$. 
\end{proof}

An important property:

\begin{proposition}
	\label{pro:BOsintorsion}
	Bi-ordered groups are torsion-free.
\end{proposition}

\begin{proof}
	Let $G$ be a bi-ordered group and $g\in G\setminus\{1\}$. 
	If $g>1$, then
	$1<g<g^2<\cdots$. If $g<1$, then $1>g>g^2>\cdots$. Hence $g^n\ne 1$ 
	for all $n\ne 0$. 
\end{proof}

The converse of the previous proposition does not hold. 

\begin{exercise}
Let $G=\langle x,y:yxy^{-1}=x^{-1}\rangle$. 
\begin{enumerate}
    \item Prove that $x$ and $y$ are torsion-free. 
    \item Prove that $G$ is torsion-free. 
    \item Prove that $G\simeq \langle a,b:a^2=b^2\rangle$.
\end{enumerate}
\end{exercise}

% To prove that $x$ is torsion-free, use the group homomorphism $G\to\Z$
% given by $x\mapsto 1$, $y\mapsto 0$. To prove that $y$ is torsion-free
% use the group homomorphism $x\mapsto\begin{pmatrix}-1&0\\0&1\end{pmatrix}$,
% $y\mapsto\begin{pmatrix}1&1\\0&1\end{pmatrix}$. 

% 	We first show that $G\simeq\langle a,b:a^2=b^2\rangle$. For that purpose, 
% 	it is enough to check that 
% 	the map $G\to \langle a,b:a^2=b^2\rangle$, $x\mapsto a$, $y\mapsto ab^{-1}$, 
% 	is a well-defined group homomorphism with inverse
% 	$\langle a,b:a^2=b^2\rangle\to G$, $a\mapsto x$, $b\mapsto y^{-1}x$. 

\begin{example}
	The torsion-free group $G=\langle x,y:yxy^{-1}=x^{-1}\rangle$ is not bi-ordered. 
	If not, let $P$ 
	be the positive cone. If $x\in P$, 
	then $yxy^{-1}=x^{-1}\in P$, a contradiction. Hence $x^{-1}\in P$
	and $x=y^{-1}x^{-1}y\in P$, a contradiction.
\end{example}

\begin{theorem}
	\label{thm:BO}
	Let $G$ be a bi-ordered group. Then $K[G]$ is a domain such that
	only has trivial units. Moreover, if $G$ is non-trivial, 
	then $J(K[G])=\{0\}$. 
\end{theorem}

\begin{proof}
	Let $\alpha,\beta\in K[G]$ be such that  
	\begin{align*}
		\alpha&=\sum_{i=1}^m a_ig_i, && g_1<g_2<\cdots<g_m,&& a_i\ne 0 && \forall i\in\{1,\dots,m\},\\
		\beta&=\sum_{j=1}^n b_jh_j, && h_1<h_2<\cdots<h_n, && b_j\ne 0 && \forall j\in\{1,\dots,n\}.
	\end{align*}
	Then 
	\[
		g_1h_1\leq g_ih_j\leq g_mh_n
	\]
	for all $i,j$. Moreover, $g_1h_1=g_ih_j$ if and only if $i=j=1$. The
	coefficient of $g_1h_1$ in $\alpha\beta$ is $a_1b_1\ne 0$. In particular, 
	$\alpha\beta\ne0$. If $\alpha\beta=\beta\alpha=1$, then the coefficient of
	$g_mh_n$ in $\alpha\beta$ is $a_mb_n$. Hence $m=n=1$ and therefore 
	$\alpha=a_1g_1$ and $\beta=b_1h_1$ with $a_1b_1=b_1a_1=1$ in $K$ and $g_1h_1=1$
	in $G$.
\end{proof}

\begin{theorem}[Levi]
	\label{thm:Levi}
	\index{Levi's theorem}
	Let $A$ be an abelian group. Then $A$ is bi-ordered if and only
	if $A$ is torsion-free.
\end{theorem}

\begin{proof}
	If $A$ is bi-ordered, then $A$ is torsion-free. Let us prove the non-trivial implication, 
	so assume that
	$A$ is torsion-free abelian. Let $\mathcal{S}$ be the class 
	of subsets $P$ of $A$ such that $0\in P$, are closed under 
	the addition of
	$A$ and satisfy the following property: if $x\in P$ and $-x\in P$,
	then $x=0$.
	Clearly, $\mathcal{S}\ne\emptyset$, as 
	$\{0\}\in\mathcal{S}$.  The inclusion turns $\mathcal{S}$ into a partially ordered set  
	and $\bigcup_{i\in I}P_i$ is an upper bound for the chain 
	$\{P_i:i\in I\}$. By Zorn's lemma, 
	$\mathcal{S}$ admits a maximal element $P\in\mathcal{S}$.

	\begin{claim}
		If $x\in A$ is such that $kx\in P$ for some $k>0$, then  $x\in P$.		
	\end{claim}

	Let $Q=\{x\in A:kx\in P\text{ for some 
	$k>0$}\}$. We will show that $Q\in\mathcal{S}$.  Clearly, $0\in Q$. Moreover, $Q$
	is closed under addition, as $k_1x_1\in P$ and $k_2x_2\in P$ imply 
	$k_1k_2(x_1+x_2)\in P$. Let $x\in A$ be such that $x\in Q$ and $-x\in Q$. Thus 
	$kx\in P$ and $l(-x)\in P$ for some $l>0$. Since $klx\in P$ and 
	$kl(-x)\in P$, it follows that $klx=0$, a contradiction since $A$ is torsion-free. 
	Hence $x\in Q\subseteq P$. 

	\begin{claim}
		If $x\in A$ is such that $x\not\in P$, then $-x\in P$. 	
	\end{claim}

	Assume that $-x\not\in P$ and let $P_1=\{y+nx:y\in P,\,n\geq0\}$. We will
	show that  $P_1\in\mathcal{S}$.  Clearly, $0\in P_1$ and $P_1$ is closed under
	addition. If $P_1\not\in S$, there exists 
	\[
		0\ne y_1+n_1x=-(y_2+n_2x),
	\]
	where $y_1,y_2\in P$ and $n_1,n_2\geq0$. Thus $y_1+y_2=-(n_1+n_2)x$. If 
	$n_1=n_2=0$, then $y_1=-y_2\in P$ and $y_1=y_2=0$, so it follows that
	$y_1+n_1x=0$, a contradiction. If $n_1+n_2>0$, then, since 
	\[
		(n_1+n_2)(-x)=y_1+y_2\in P,
	\]
	it follows from the first claim that $-x\in P$, a contradiction. 
	Let us show that $P_1\in\mathcal{S}$. 
	Since $P\subseteq P_1$, the maximality of $P$ implies that 
	$x\in P=P_1$.

	\medskip
	By Proposition~\ref{pro:biordenableP2}, 
	$P^*=P\setminus\{0\}$ is the positive cone of a bi-order in $A$. 
	In fact, $P^*$ is closed under addition, as $x,y\in
	P^*$ implies that $x+y\in P$ and $x+y=0$ implies $x=y=0$, as $x=-y\in P$. Moreover,
	$G=P^*\cup -P^*\cup\{0\}$ (disjoint union), as 
	the second claim states that $x\not\in P^*$ implies 
	$-x\in P$. 
\end{proof}

Our proof of Passman's theorem (Theorem \ref{thm:Passman}) 
used the fact that the group algebra $K[G]$ of
a torsion-free abelian group $G$ has no non-zero divisors. 
We now present a proof of this fact. 

\begin{corollary}
\label{cor:domain_G_abelian}
	Let $A$ be a non-trivial torsion-free abelian group. Then $K[A]$ 
	is a domain that only admits trivial units and $J(K[A])=\{0\}$. 
\end{corollary}

\begin{proof}
	Apply Levi's theorem and Theorem~\ref{thm:BO}.
\end{proof}

Some exercises. The first one is a variation on Exercise \ref{xca:BO_sequence}.

\begin{exercise}
    Let $N$ be a central subgroup of $G$. If $N$ and $G/N$ are bi-ordered, 
    then $G$ is bi-ordered. Prove with an example that $N$ needs to be central, normal 
    is not enough. 
\end{exercise}

\begin{exercise}
    Let $G$ be a group that admits 
    a sequence 
    \[
    \{1\}=G_0\subseteq G_1\subseteq\cdots\subseteq G_n=G
    \]
    such that
    each $G_k$ is normal in $G_{k+1}$ and each quotient $G_{k+1}/G_k$ is 
    torsion-free abelian. Prove that $G$ is bi-ordered.  
\end{exercise}

\begin{exercise}
    Prove that torsion-free nilpotent groups are bi-ordered. 
\end{exercise}


\topic{Left-ordered groups}

\begin{definition}
	\index{Group!left-ordered}
	A group $G$ is \textbf{left-ordered} if there is a total order 
	$<$ in $G$ such that $x<y$ implies $xz<yz$ for all $x,y,z\in G$.
\end{definition}

If $G$ is left-ordered, the positive cone of $G$ is defined as $P(G)=\{x\in G:1<x\}$. 

\begin{exercise}
	Let $G$ be left-ordered with positive cone $P$. Prove that 
	$P$ is closed under multiplication and that 
	$G=P\cup P^{-1}\cup \{1\}$ (disjoint union).
\end{exercise}

\begin{exercise}
\label{xca:LO_cone}
	Let $G$ be a group and $P$ be a subset closed under multiplication. Assume that 
	$G=P\cup P^{-1}\cup \{1\}$ (disjoint union). Prove that $x<y$ if and only if
	$x^{-1}y\in P$ turns $G$ into a left-ordered group with positive cone $P$.
\end{exercise}

Left-ordered groups behave nicely with respect to extensions. Let $G$ be a group
and $N$ be a left-ordered normal subgroup of $G$. If $\pi\colon G\to G/N$ is the 
canonical map and $G/N$ is left-ordered, then
$G$ is left-ordered with
$x<y$ if and only if either $\pi(x)<\pi(y)$ or $\pi(x)=\pi(y)$ and $1<x^{-1}y$. 


\begin{proposition}
	Let $G$ be a group and $N$ be a normal subgroup of $G$. 
	If $N$ and $G/N$ are left-ordered, then so is $G$.
\end{proposition}

\begin{proof}
	Since $N$ and $G/N$ are both left-ordered, there exist positive cones 
	$P(N)$ and $P(G/N)$. Let $\pi\colon G\to G/N$ be the canonical map and 
	\[
		P(G)=\{x\in G:\pi(x)\in P(G/N)\text{ or }x\in N\}.
	\]	
	A routine calculation shows that $P(G)$ is closed under multiplication 
	and that $G$ decomposes as $G=P(G)\cup P(G)^{-1}\cup \{1\}$ (disjoint union). It follows
	from Exercise \ref{xca:LO_cone} that 
	$G$ is left-ordered. 
\end{proof}


%
%\begin{theorem}
%	\label{theorem:}
%	Si $G$ tiene una serie finita subnormal $1=G_0\triangleleft
%	G_1\triangleleft\cdots\triangleleft G_n=G$ y cada cociente $G_{i+1}/G_i$ es
%	abeliano libre de torsión, entonces $G$ es ordenable a derecha. Si además
%	$G$ es libre de torsión y nilpotente, entonces $G$ es biordenable.
%\end{theorem}

We now present a criterion for detecting left-ordered groups. We shall need 
a lemma. 

\begin{lemma}
	\label{lem:fg}
	Let $G$ be a finitely generated group. If $H$ is a finite-index subgroup, 
	then $H$ is finitely generated. 
\end{lemma}

\begin{proof}
	Assume that $G$ is generated by $\{g_1,\dots,g_m\}$. Assume that
	for each $i$ there exists $k$ such that $g_i^{-1}=g_k$. Let $\{t_1,\dots,t_n\}$ be
	a transversal of $H$ in $G$. For $i\in\{1,\dots,n\}$ and 
	$j\in\{1,\dots,m\}$ write 
	\[
		t_ig_j=h(i,j)t_{k(i,j)}.
	\]
	We claim that $H$ is generated by the $h(i,j)$. For $x\in H$, write 
	\begin{align*}
	x &=g_{i_1}\cdots g_{i_s}\\
	&= (t_1g_{i_1})g_{i_2}\cdots g_{i_s}\\
	&= h(1,i_1)t_{k_1}g_{i_2}\cdots g_{i_s}\\
	&= h(1,i_1)h(k_1,i_2)t_{k_2}g_{i_3}\cdots g_{i_s}\\
	&= h(1,i_1)h(k_1,i_2)\cdots h(k_{s-1},g_{i_s})t_{k_s},
	\end{align*}
	where $k_1,\dots,k_{s-1}\in\{1,\dots,n\}$. Since $t_{k_s}\in H$, it follows that 
	$t_{k_s}=t_1\in H$ and therefore $x\in H$.
\end{proof}

Now the theorem.

\begin{theorem}
	Let $G$ be a finitely generated torsion-free group. If $A$ is an abelian normal
	subgroup such that $G/A$ is finite and cyclic, then $G$ is left-ordered. 
\end{theorem}

\begin{proof}
	Note that if $A$ is trivial, then so is $G$. Let us assume that $A\ne\{1\}$. 
    Since $(G:A)$ is finite, $A$ is finitely generated by the previous lemma. 
    We proceed by induction on the number of generators of $A$. Since 
    $G/A$ is cyclic, there exists $x\in G$ such that $G=\langle A,x\rangle$. Then
    $[x,A]=\langle [x,a]:a\in A\rangle$ is a normal subgroup of $G$ such that 
    $A/C_A(x)\simeq [x,A]$ (because $a\mapsto [x,a]$ is a group homomorphism $A\to A$
    with image $[x,A]$ and kernel $C_A(x)$). If $\pi\colon G\to G/[x,A]$ is the canonical map, then
    $G/[x,A]=\langle \pi(A),\pi(x)\rangle$ and thus $G/[x,A]$ is abelian, as 
    $[\pi(x),\pi(A)]=\pi[x,A]=1$. Moreover, $G/[x,A]$ is finitely generated, as $G$
    is finitely generated. Since $(G:A)=n$ and $G$ is torsion-free, it follows that 
    $1\ne x^n\in A$. Hence $x^n\in C_A(x)$ and therefore $1\leq \rank C_A(x)<\rank A$ (if $\rank
    C_A(x)=\rank A$, then $[x,A]$ would be a torsion subgroup of $A$, a contradiction
    since $x\not\in A$). So 
    \[
    \rank[x,A]=\rank (A/C_A(x))\leq\rank A-1
    \]
    and hence $\rank (A/[x,A])\geq 1$. We proved that $A/[x,A]$ is infinite and hence 
    $G/[x,A]$ is infinite. 

    Since $G/[x,A]$ is infinite, abelian and finitely generated, there exists a normal subgroup
    $H$ of $G$ such that $[x,A]\subseteq H$ and $G/H\simeq\Z$. The subgroup 
    $B=A\cap H$ is abelian, normal in $H$ and such that $H/B$ is cyclic
    (because it is isomorphic to a subgroup of $G/A$). Since $\rank B<\rank A$, the inductive hypothesis implies that $H$ is left-ordered. Hence $G$ is left-ordered. 
\end{proof}

\index{Lagrange--Rhemtulla's theorem} 
Lagrange and Rhemtulla proved that the integral isomorphism problem 
has an affirmative solution for left-ordered groups. More precisely,
if $G$ is left-ordered and $H$ is a group such that $\Z[G]\simeq\Z[H]$, then
$G\simeq H$, see \cite{MR240183}.

\begin{theorem}[Malcev--Neumann]
	\index{Malcev--Neumann's theorem}
	Let $G$ be left-ordered group. Then $K[G]$ has no zero divisors 
	and no non-trivial units. 
\end{theorem}

\begin{proof}
	If $\alpha=\sum_{i=1}^na_ig_i\in K[G]$ and
	$\beta=\sum_{j=1}^mb_jh_j\in K[G]$, then 
	\begin{equation}
		\label{eq:producto}
		\alpha\beta=\sum_{i=1}^n\sum_{j=1}^ma_ib_j(g_ih_j).
	\end{equation}
	Without loss of generality we may assume that $a_i\ne 0$ for
	all $i$ and $b_j\ne 0$ for all $j$. Moreover, we may assume that 
	$g_1<g_2<\cdots<g_n$. Let $i,j$ be such that 
	\[
		g_ih_j=\min\{g_ih_j:1\leq i\leq n,1\leq j\leq m\}.
	\]
	Then $i=1$, as $i>1$ implies
	$g_1h_j<g_ih_j$, a contradiction. Since $g_1h_j\ne g_1h_k$ whenever 
	$k\ne j$, there exists a unique minimal element in the left hand side of Equality~\eqref{eq:producto}. The same argument shows that there is a unique
	maximal element in~\eqref{eq:producto}. Thus 
	$\alpha\beta\ne 0$, as $a_1b_j\ne 0$, and therefore $K[G]$ has no zero divisors. 
	If, moreover, $n>1$ or $m>1$, then~\eqref{eq:producto} contains at least two
	terms than cancel out and thus  
	$\alpha\beta\ne1$. It follows that units of $K[G]$ are trivial. 
\end{proof}

\index{Formanek's theorem}
\index{Farkas--Snider's theorem}
\index{Brown's theorem}
Formanek proved that the zero divisors conjecture is true 
in the case of torsion-free super solvable. Brown and, independently, 
Farkas and Snider proved that the conjecture is true 
in the case of groups algebras (over fields of characteristic zero) of 
polycyclic-by-finite torsion-free groups. These results
can be found in Chapter 13 of
Passman's book \cite{MR798076}. 

\topic{The braid group}

\begin{definition}
    \index{Braid group}
    Let $n\geq1$. The \textbf{braid group} $\B_n$ is
    the group with generators $\sigma_1,\dots,\sigma_{n-1}$ and
    relations
    \begin{align*}
        &\sigma_i\sigma_{i+1}\sigma_i=\sigma_{i+1}\sigma_i\sigma_{i+1} && \text{if }1\leq i\leq n-2,\\
        &\sigma_i\sigma_j=\sigma_j\sigma_i && \text{if }|i-j|> 1.
    \end{align*}
\end{definition}

Note that $\B_1=\{1\}$ and $\B_2\simeq\Z$. The braid 
group $\B_3$ is generated by
$\sigma_1$ and $\sigma_2$ with relations
$\sigma_1\sigma_2\sigma_1=\sigma_2\sigma_1\sigma_2$.

\begin{exercise}
    Prove that there exists a group homomorphism $\B_n\to\Sym_n$ 
    given by $\sigma_i\mapsto (i\,i+1)$ for all $i\in\{1,\dots,n-1\}$. 
\end{exercise}

Note that if $n\geq3$, then 
$\B_n$ is a non-abelian group, as there exists a surjective
group homomorphism $\B_n\to\Sym_n$. 

\begin{exercise}
    Let $n\geq 2$. 
    Prove that the map $\deg\colon\B_n\to\Z$, $\sigma_i\mapsto 1$, 
    is a group homomorphism. Moreover, $\ker\deg=[\B_n,\B_n]$. 
\end{exercise}

The previous result implies, in particular, that $\B_n$ is an infinite 
group for all $n\geq2$. Moreover, $\sigma_i^m\ne1$ for all $m\in\Z\setminus\{0\}$ and all $i$.  

\begin{exercise}
    Prove that $\B_3\simeq\langle x,y:x^2=y^3\rangle$ and that 
    $\B_3/Z(\B_3)\simeq\PSL_2(\Z)$. 
\end{exercise}

\begin{exercise}
    Prove that the center $Z(\B_3)$ of $\B_3$ is
    the cyclic group generated by $(\sigma_1\sigma_2\sigma_1)^2$.
\end{exercise}

More generally, one can prove that
the center of $\B_n$ is generated by $\Delta_n^2$, where
\[
\Delta_n=(\sigma_1\cdots\sigma_{n-1})(\sigma_1\cdots\sigma_{n-2})\cdots(\sigma_1\sigma_2)\sigma_1, 
\]
see for example \cite[Theorem 1.24]{MR2435235}. 
As a corollary, $\B_n\simeq\B_m$ if and only if $n=m$. 

\begin{exercise}
    Let $n\geq3$. 
    Prove that $\B_n$ is not bi-ordered. 
\end{exercise}

One can prove that 
the natural map $\B_n\to\B_{n+1}$ is an injective group homomorphism, this is not an easy proof (see \cite[Corollary 1.14]{MR2435235}). Moreover,
the diagram
\[\begin{tikzcd}
	\B_n & \Sym_n \\
	\B_{n+1} & \Sym_{n+1}
	\arrow[two heads, from=1-1, to=1-2]
	\arrow[hook, from=1-2, to=2-2]
	\arrow[hook, from=1-1, to=2-1]
	\arrow[two heads, from=2-1, to=2-2]
\end{tikzcd}
\]
commutes. 

\begin{exercise}
\label{xca:derivedB3}
\index{Reidemeister--Schreier's method}
    Use the Reidemeister--Schreier's method to prove that 
    $[\B_3,\B_3]$ is isomorphic to the free group in two letters.
\end{exercise}

\index{Dehornoy's theorem}
A celebrated theorem of Dehornoy states that the braid group $\B_n$ 
is left-ordered (see for example \cite[Theorem 7.15]{MR2435235}). The proof of this fact is quite hard. However, 
there is a nice short proof of the fact that 
$\B_3$ is left-ordered, see 
\cite[\S7.2]{MR3560661}.

\begin{problem}[Burau's representation]
\index{Burau's representation}
    Let $\B_4\to\GL_4(\Z[t,t^{-1}])$ be the group 
    homomorphism given by
    \[
    \sigma_1\mapsto\begin{pmatrix}
    1-t&t&0&0\\
    1&0&0&0\\
    0&0&1&0\\
    0&0&0&1
    \end{pmatrix},
    \quad
    \sigma_2\mapsto\begin{pmatrix}
    1&0&0&0\\
    0&1-t&t&0\\
    0&1&0&0\\
    0&0&0&01
    \end{pmatrix},
    \quad
    \sigma_3\mapsto\begin{pmatrix}
    1&0&0&0\\
    0&1&0&0\\
    0&0&1-t&t\\
    0&0&1&0
    \end{pmatrix}.
    \]
    Is this homomorphism injective?
\end{problem}

In general, the Burau's representation
$\B_n\to\GL_{n}(\Z[t,t^{-1}])$ 
is defined
by
\[
\sigma_j\mapsto
I_{j-1}\oplus
\begin{pmatrix}
1-t & t\\
1 & 0
\end{pmatrix}
\oplus
I_{n-j-1},
\]
where $I_k$ denotes the $k\times k$ identity matrix. 

It is known that the Burau's representation of $\B_n$ is faithful
for $n\leq3$ and not faithful for $n\geq5$. Only the case
$n=4$ remains open.

Krammer \cite{MR1888796} and Bigelow \cite{MR1815219} 
independently proved that braid groups are linear. 

\topic{Locally indicable groups}

\begin{definition}
\index{Group!indicable}
    A group $G$ is \textbf{indicable} if
    there exists a non-trivial group homomorphism $G\to\Z$.
\end{definition}

We know that braid groups are indicable.
The free group $F_n$ in $n$ letters is indicable. 

\begin{definition}
\index{Group!locally indicable}
    A group $G$ is \textbf{locally indicable} if every 
    non-trivial finitely generated subgroup is indicable.
\end{definition}

\index{Burns--Hale's theorem}
Burns--Hale's theorem (see \cite[Theorem 1.50]{MR3560661}) states that a group $G$ is left-ordered if and only if
for every non-trivial finitely generated subgroup $H$ of $G$ there exists 
a left-ordered group $L$ and a non-trivial group homomorphism $H\to L$. As a consequence, 
locally indicable groups are left-ordered. 

\begin{example}
    Since subgroups of free groups are free, 
    it follows that $F_n$ is locally indicable. 
\end{example}

\begin{proposition}
\label{pro:LI_exact}
    Let $1\to K\to G\to Q\to 1$ be an exact sequence of groups. 
    If $K$ and $Q$ are
    locally indicable, then $G$ is locally indicable.
\end{proposition}

\begin{proof}
    Let $g_1,\dots,g_n\in G$ and $L=\langle g_1,\dots,g_n\rangle$. 
    Assume that $\beta(L)\ne\{1\}$. Since $Q$ is locally indicable, 
    there exists a non-trivial group homomorphism $\beta(L)\to\Z$. Then the 
    composition $L\to Q\to\Z$ is then a non-trivial group homomorphism. Assume now
    that $\beta(L)=\{1\}$. Then there exist $k_1,\dots,k_n\in K$ 
    such that $\alpha(k_i)=g_i$ for all $i\in\{1,\dots,n\}$. Note that
    $\alpha\colon \langle k_1,\dots,k_n\rangle\to L$ is a group isomorphism. Since
    $K$ is locally indicable, there exits a non-trivial group 
    homomorphism $\langle k_1,\dots,k_n\rangle\to\Z$. 
    Thus the composition $L\to\langle k_1,\dots,k_n\rangle\to\Z$ is a non-trivial
    group homomorphism 
    and hence $G$ is locally indicable. 
\end{proof}

As a consequence of the previous proposition, 
if $G$ and $H$ are locally indicable groups and 
$\sigma\colon G\to\Aut(H)$ is a group homomorphism, then 
$G\rtimes_\sigma H$ is locally indicable. In particular, the 
direct product of locally indicable groups is locally indicable.

\begin{example}
    The group $G=\langle x,y:x^{-1}yx=y^{-1}\rangle$ is
    locally indicable. We know that $G$ is torsion-free. Let 
    $K=\langle y\rangle\simeq\Z$. Then $G/K\simeq\Z$ and 
    then, since there is an exact sequence
    $1\to\Z\to G\to\Z\to1$ 
    it follows from Proposition \ref{pro:LI_exact} 
    that $G$ is locally indicable.
\end{example}

% $(x,y)\mapsto(x+1,y)$, $(x,y)\mapsto(-x,y+1)$
% A=2312, B=200-2
% Let $K=\langle y\rangle$. Then $G/K\simeq\Z$ and 
% the second result follows from the previous proposition. 

\begin{exercise}
\label{xca:B3_LI}
    Prove that $\B_3$ is locally indicable. 
\end{exercise}

The previous exercise uses the fact that $[\B_3,\B_3]$ is isomorphic to the free group in two letters, see
Exercise \ref{xca:derivedB3}.
An alternative solution to the previous fact goes as follows: $\B_3$ is the fundamental group
of the trefoil knot and fundamental groups of knots are locally indicable. 

\begin{exercise}
    Prove that $\B_4$ is locally indicable.
\end{exercise}

The previous exercise might be harder than Exercise \ref{xca:B3_LI}. One possible solution
is based on using the Reidemeister--Schreier method to prove that 
$[\B_4,\B_4]$ is a certain semidirect product 
between free groups in two generators. Another solution: Let 
$f\colon\B_4\to\B_3$ be the group homomorphism given by $f(\sigma_1)=f(\sigma_3)=\sigma_1$ 
and $f(\sigma_2)=\sigma_2$. Then $\ker f=\langle \sigma_1\sigma_3^{-1},\sigma_2\sigma_1\sigma_3^{-1}\sigma_2^{-1}\rangle$ 
is isomorphic to the free group in two letters. Now use the exact sequence
$1\to \ker f\to\B_4\to\B_3\to1$. 

\begin{exercise}
\label{xca:relations}
    Let $n\geq5$. Consider the elements of $\B_n$ given by 
    \begin{align*}
        &\beta_1=\sigma_1^{-1}\sigma_2,
        &&\beta_2=\sigma_2\sigma_1^{-1}, 
        &&\beta_3=\sigma_1\sigma_2\sigma_1^{-2},
        &&\beta_4=\sigma_3\sigma_1^{-1}, 
        &&\beta_5=\sigma_4\sigma_1^{-1}.
    \end{align*}
    Prove the following relations:
    \begin{enumerate}
        \item $\beta_1\beta_5=\beta_5\beta_2$.
        \item $\beta_2\beta_5=\beta_5\beta_3$.
        \item $\beta_1\beta_3=\beta_2$.
        \item $\beta_1\beta_4\beta_3=\beta_4\beta_2\beta_4$.
        \item $\beta_4\beta_5\beta_4=\beta_5\beta_4\beta_5$.
    \end{enumerate}
\end{exercise}

\begin{exercise}
    Let $n\geq 5$. 
    Prove that $\B_n$ is not locally indicable.
\end{exercise}

For the previous exercise one needs to show that
every group homomorphism $f\colon \langle\beta_1,\dots,\beta_5\rangle\to\Z$ is trivial. Hint: consider
the abelianization of $\langle\beta_1,\dots,\beta_5\rangle$. 

\topic{Unique product groups}

Let $G$ be a group and $A,B\subseteq G$ be non-empty subsets. 
An element $g\in G$ is a \textbf{unique product} in $AB$ if $g=ab=a_1b_1$ for some
$a,a_1\in
A$ and $b,b_1\in B$ implies that $a=a_1$ and $b=b_1$.

\begin{definition}
	\index{Group!unique product}
	A group $G$ has the \textbf{unique product property} if 
	for every finite non-empty subsets $A,B\subseteq G$ there exists at least one
	unique product in $AB$.
\end{definition}

\begin{proposition}
    Left-ordered groups have the unique product property.
\end{proposition}

\begin{proof}
    Let $G$ be a left-ordered group. 
	Let $A=\{a_1,\dots,a_n\}\subseteq G$ and $B\subseteq G$ non-empty and finite. 
	Assume that $a_1<a_2<\cdots<a_n$. Let $c\in B$ be such that $a_1c$ is the 
	minimum of $a_1B=\{a_1b:b\in B\}$. We claim that $a_1c$ admits a unique
	representation of the form $\alpha\beta$ with $\alpha\in A$ and 
	$\beta\in B$. If $a_1c=ab$, then, since $ab=a_1c\leq a_1b$, it follows that 
	$a\leq a_1$. Hence $a=a_1$ and $b=c$. 
\end{proof}

\begin{exercise}
	Prove that groups with the unique product property are
	torsion-free.
\end{exercise}

The converse does not hold. 
Promislow's group is a celebrated counterexample.

\begin{theorem}[Promislow]
\index{Promislow's theorem}
    The group $G=\langle a,b:a^{-1}b^2a=b^{-2},b^{-1}a^2b=a^{-2}\rangle$
    does not have the unique product property.
\end{theorem}

\begin{proof}
    Let 
    \begin{multline}
    \label{eq:Promislow}
    S=\{ a^2b,
    b^2a,
    aba^{-1},
    (b^2a)^{-1},
    (ab)^{-2},
    b,
    (ab)^2x,
    (ab)^2,
    (aba)^{-1},\\
    bab,
    b^{-1},
    a,
    aba,
    a^{-1}
    \}.
    \end{multline}
    We use \textsf{GAP} and the representation $G\to\GL(4,\Q)$ given by 
    \[
a\mapsto\begin{pmatrix}
1 & 0 & 0 & 1/2\\
0 & -1 & 0 & 1/2\\
0 & 0 & -1 & 0\\
0 & 0 & 0 & 1
\end{pmatrix},
\quad
b\mapsto\begin{pmatrix}
-1 & 0 & 0 & 0\\
0 & 1 & 0 & 1/2\\
0 & 0 & -1 & 1/2\\
0 & 0 & 0 & 1
\end{pmatrix}
\]
    to check that 
    $G$ does not have
    unique product property, as each 
    \[
    s\in S^2=\{s_1s_2:s_1,s_2\in S\}
    \]
    admits at least two different decompositions of the 
    form $s=xy=uv$ for $x,y,u,v\in S$. 
    We first create the matrix representations of $a$ and $b$.
\begin{lstlisting}
gap> a := [[1,0,0,1/2],[0,-1,0,1/2],[0,0,-1,0],[0,0,0,1]];;
gap> b := [[-1,0,0,0],[0,1,0,1/2],[0,0,-1,1/2],[0,0,0,1]];;
\end{lstlisting}
    Now we create
    a function that produces the set $S$.
\begin{lstlisting}
gap> Promislow := function(x, y)
> return Set([
> x^2*y,
> y^2*x,
> x*y*Inverse(x),
> (y^2*x)^(-1),
> (x*y)^(-2),
> y,
> (x*y)^2*x,
> (x*y)^2,
> (x*y*x)^(-1),
> y*x*y,
> y^(-1),
> x,
> x*y*x,
> x^(-1)
]);
end;;
\end{lstlisting}
So the set $S$ of \eqref{eq:Promislow} 
will be \lstinline{Promislow(a,b)}. We now
create a function that checks whether
every element of a Promislow subset 
admits more than one representation.
\begin{lstlisting}
gap> is_UPP := function(S)
> local l,x,y;
> l := [];
> for x in S do
> for y in S do
> Add(l,x*y);
> od;
> od;
> if ForAll(Collected(l), x->x[2] <> 1) then
> return false;
> else
> return fail;
> fi;
> end;;
\end{lstlisting}
Finally, we check whether every element of 
$S$ admits more than one representation.
\begin{lstlisting}
gap> S := Promislow(a,b);;
gap> is_UPP(S);
false
\end{lstlisting}
This completes the proof. 
\end{proof}

There are other examples. 

% Let $G=\langle x,y:x^{-1}y^2xy^2=x^{-2}yx^{-2}y^3=1\rangle$. 
% We first construct the group and a certain normal
% subgroup $N$ of index eight. 

% \begin{lstlisting}
% gap> f := FreeGroup(2);;
% gap> x := f.1;;
% gap> y := f.2;;
% gap> rels := [Inverse(x)*y^2*x*y^2, Inverse(x^2)*y*Inverse(x^2)*y^3];;
% gap> G := f/rels;;
% gap> x := G.1;;
% gap> y := G.2;;
% gap> N := Subgroup(G, [y*x*Inverse(x*y), y^2, x^4]);
% gap> IsNormal(N,G);
% true
% gap> StructureDescription(G/N);
% "C4 x C2"
% \end{lstlisting}
% The subgroup $N$ has a nice presentation. It can be presented
% as the group 
% \[
% N=\langle x_1,x_2,x_3:[x_1,x_2]=[x_1,x_3]=1,\,x_3x_2=x_2x_3x_1^8\rangle.
% \]
% \begin{lstlisting}
% gap> g := IsomorphismFpGroup(N);
% gap> RelatorsOfFpGroup(Image(g));
% [ F3*F1*F3^-1*F1^-1, F2*F1*F2^-1*F1^-1, F1*F3^-1*F1^5*F2*F1^2*F3*F2^-1 ]
% \end{lstlisting}
% From these relations
% one proves by induction that 
% \begin{align}
%     x_3^bx_2^a=x_2^ax_3^bx_1^{8ab}
% \end{align}
% for all $a,b\in\Z$. 
% % First induction in $n_1$, then induction in $n_2$. 
% It follows that every element of $N$ 
% is of the form $x_1^{n_1}x_2^{n_2}x_3^{n_3}$ 
% for $n_1,n_2,n_3\in\Z$. Moreover, 
% \[
% (x_1^{n_1}x_2^{n_2}x_3^{n_3})^k=x_1^{kn_1+(k-1)8n_2n_3}x_2^{kn_2}x_3^{kn_3}
% \]
% for all $k\in\Z$. Note that $x_1^8$ is a commutator. Moreover, 
% $N/[N,N]\simeq\Z\times\Z\times\Z/8$. This implies that
% $N$ is torsion-free. Let us prove that $G$ is torsion-free. 
% Let $\pi\colon G\to G/N$ be the canonical map. Let 
% $g\in G$ be a torsion element, in particular $g\not\in N$
% and hence $\pi(g)\ne 1$. So $\pi(g)$ has order two 
% or four. Without loss of generality we may assume that
% $\pi(g)$ has order two.  
% Then $\pi(g^2)=\pi(g)^2=y^2=1$ and hence 
% $g^2\in N$. Since $N$ is torsion free, it follows that $g=1$. 

\begin{definition}
	Se dice que un grupo $G$ tiene la \textbf{propiedad del doble producto
	único} si dados dos subconjuntos $A,B\subseteq G$ finitos y no vacíos tales
	que $|A|+|B|>2$ existen al menos dos productos únicos en $AB$.
\end{definition}

\begin{theorem}[Strojnowski]
	\label{theorem:Strojnowski}
	\index{Teorema!de Strojonowski}
	Sea $G$ un grupo. Las siguientes afirmaciones son equivalentes:
	\begin{enumerate}
		\item $G$ tiene la propiedad del doble producto único.
		\item Para todo subconjunto $A\subseteq G$ finito y no vacío, existe al
			menos un producto único en $AA=\{a_1a_2:a_1,a_2\in A\}$.
		\item $G$ tiene la propiedad del producto único.
	\end{enumerate}
\end{theorem}

\begin{proof}
	La implicación $(1)\implies(2)$ es trivial.  Demostremos que vale
	$(2)\implies(3)$. Si $G$ no tiene la propiedad del producto único, existen
	subconjuntos $A,B\subseteq G$ finitos y no vacíos tales que todo elemento
	de $AB$ admite al menos dos representaciones. Sea $C=AB$. Todo $c\in C$ es
	de la forma $c=(a_1b_1)(a_2b_2)$ con $a_1,a_2\in A$ y $b_1,b_2\in B$. Como
	$a_2^{-1}b_1^{-1}\in AB$, existen $a_3\in A\setminus\{a_2\}$ y $b_3\in B\setminus\{b_1\}$ tales que
	$a_2^{-1}b_1^{-1}=a_3^{-1}b_3^{-1}$. Luego $b_1a_2=b_3a_3$ y entonces
	\[
	c=(a_1b_1)(a_2b_2)=(a_1b_3)(a_3b_2)
	\]
	son dos representaciones distintas de $c$ en $AB$.
	pues $a_2\ne a_3$ y $b_1\ne b_3$.

	Demostremos ahora que $(3)\implies(1)$. Si $G$ tiene la propiedad del
	producto único pero no tiene la propiedad del doble producto único, existen
	subconjuntos $A,B\subseteq G$ finitos y no vacíos con $|A|+|B|>2$ tales que
	en $AB$ existe un único elemento $ab$ con una única representación en $AB$.
	Sean $C=a^{-1}A$ y $D=Bb^{-1}$. Entonces $1\in C\cap D$ y el elemento
	neutro $1$ admite una única representación en $CD$ (pues si $1=cd$ con
	$c=a^{-1}a_1\ne 1$ y $d=b_1b^{-1}\ne 1$, entonces $ab=a_1b_1$ con $a\ne
	a_1$ y $b\ne b_1)$. Sean $E=D^{-1}C$ y $F=DC^{-1}$. Todo elemento de $EF$
	se escribe como $(d_1^{-1}c_1)(d_2c_2^{-1})$. Si $c_1\ne 1$ o $d_2\ne 1$
	entonces $c_1d_2=c_3d_3$ para algún $c_3\in C\setminus\{c_1\}$ y algún
	$d_3\in D\setminus\{d_2\}$. Entonces
	$(d_1^{-1}c_1)(d_2c_2^{-1})=(d_1^{-1}c_3)(d_3c_2^{-1})$ son dos
	representaciones distintas para $(d_1^{-1}c_1)(d_2c_2^{-1})$. Si $c_2\ne 1$
	o $d_1\ne 1$ entonces $c_2d_1=c_4d_4$ para algún $d_4\in D\setminus\{d_1\}$
	y algún $c_4\in C\setminus\{c_2\}$ y entonces, como
	$d_1^{-1}c_2^{-1}=d_4^{-1}c_4^{-1}$,
	$(d_1^{-1}1)(1c_2^{-1})=(d_4^{-1}1)(1c_4^{-1})$.  Como $|C|+|D|>2$, $C$ o
	$D$ contienen algún $c\ne1$, y entonces $(1\cdot 1)(1\cdot 1)=(1\cdot
	c)(1\cdot c^{-1})$. Demostramos entonces que todo elemento de $EF$ tiene al
	menos dos representaciones. 
\end{proof}

% passman lema 1.9 pag 589
\begin{exercise}
	Demuestre que si $G$ es un grupo que satisface la propiedad del producto
	único, entonces $K[G]$ tiene solamente unidades triviales.
\end{exercise}

En general es muy difícil verificar si un grupo posee la propiedad del producto
único. Una propiedad similar es la de ser un grupo difuso. Si $G$ es un grupo
libre de torsión y $A\subseteq G$ es un subconjunto, diremos que $A$ es
antisimétrico si $A\cap A^{-1}\subseteq\{1\}$, donde $A^{-1}=\{a^{-1}:a\in
A\}$. El conjunto de \textbf{elementos extremales} de $A$ se define como
$\Delta(A)=\{a\in A:Aa^{-1}\text{ es antisimétrico}\}$. Luego
\[
	a\in A\setminus\Delta(A)
	\Longleftrightarrow
	\text{existe $g\in G\setminus\{1\}$ tal que $ga\in A$ y $g^{-1}a\in A$}.
\]

\begin{definition}
	\index{Grupo!difuso}
	Un grupo $G$ se dice \textbf{difuso} si para todo subconjunto $A\subseteq
	G$ tal que $2\leq |A|<\infty$ se tiene $|\Delta(A)|\geq2$.
\end{definition}

\begin{lemma}
	Si $G$ es ordenable a derecha, entonces $G$ es difuso.	
\end{lemma}

\begin{proof}
	Supongamos que $A=\{a_1,\dots,a_n\}$ y $a_1<a_2<\cdots<a_n$. Vamos a
	demostrar que $\{a_1,a_n\}\subseteq\Delta(A)$. Si $a_1\in
	A\setminus\Delta(A)$, existe $g\in G\setminus\{1\}$ tal que $ga_1\in A$ y
	$g^{-1}a_1\in A$. Esto implica que $a_1\leq ga_1$ y $a_1\leq g^{-1}a_1$, de
	donde se concluye que $1\leq g$ y $1\leq g^{-1}$, una contradicción. De la
	misma forma se demuestra que $a_n\in \Delta(A)$.
\end{proof}

\begin{lemma}
	\label{lemma:difuso=>2up}
	Si $G$ es difuso, entonces $G$ tiene la propiedad del doble producto único.	
\end{lemma}

\begin{proof}
	Supongamos que $G$ no tiene la propiedad del doble producto único. Existen
	entonces subconjuntos finitos $A,B\subseteq G$ con $|A|+|B|>2$ tales que
	$C=AB$ tiene a lo sumo un producto único. Luego $|C|\geq2$. Como $G$ es
	difuso, $|\Delta(C)|\geq2$. Si $c\in\Delta(C)$, entonces $c$ tiene una
	única expresión como $c=ab$ con $a\in A$ y $b\in B$ (de lo contrario, si
	$c=a_0b_0=a_1b_1$ con $a_0\ne a_1$ y $b_0\ne b_1$. Si $g=a_0a_1^{-1}$,
	entonces $g\ne 1$, $gc=a_0a_1^{-1}a_1b_1=a_0b_1\in C$ y además
	$g^{-1}c=a_1a_0^{-1}a_0b_0=a_1b_0\in C$. Luego $c\not\in\Delta(c)$, una
	contradicción.
\end{proof}

\begin{problem}
	Find a non-diffuse group with the unique product property.
\end{problem}

%Un grupo $G$ se dice \textbf{débilmente difuso} si para todo subconjunto
%finito $A\subseteq G$ no vacío se tiene $\Delta(A)\ne\emptyset$. La técnica
%usada para demostrar el lema~\ref{lemma:difuso=>2up} puede usarse para
%demostrar que un grupo débilmente difuso posee la propiedad del producto
%único. El teorema~\ref{theorem:Strojnowski} sugiere entonces la siguiente
%pregunta: 
%
%\begin{problem}
%	¿Existe un grupo débilmente difuso que no sea difuso?
%\end{problem}
%
%\section{El grupo de Promislow}
%
%Veremos un ejemplo concreto de un grupo sin torsión que no es ordenable, no es
%difuso y no tiene la propiedad del producto único.
%
%\begin{exercise}
%	\label{exercise:Dinfty}
%	Demuestre que $G=\langle x,y:x^2=y^2=1\rangle$ es isomorfo al grupo diedral infinito.
%\end{exercise}
%
%\begin{definition}
%	Se define el grupo de Promislow como 
%	\[
%		G=\langle x,y:x^{-1}y^2x=y^{-2},\,y^{-1}x^2y=x^{-2}\rangle.
%	\]
%\end{definition}
%
%\begin{proposition}
%	\label{proposition:Promislow}
%	El grupo de Promislow es libre de torsión y no satisface la propiedad del
%	producto único. 
%\end{proposition}
%
%\begin{proof}
%	
%\end{proof}

\topic{Connel's theorem}

When $K[G]$ is prime? Connel's theorem gives a full answer to this natural
question in the case where $K$ is of characteristic zero. 

%\begin{lemma}
%	\label{lemma:Dfg}
%	Sea $H$ un subgrupo finitamente generado de $\Delta(G)$.
%	\begin{enumerate}
%		\item $(G:C_G(H))$ es finito.
%		\item $(H:Z(H))$ es finito.
%		\item $[H,H]$ es finito.
%		\item Si $H_0$ es el conjunto de elementos de torsión de $H$, $H_0$ es
%			un subgrupo normal finito de $H$ y $H/H_0$ es finitamente generado,
%			abeliano y libre de torsión.
%	\end{enumerate}
%\end{lemma}
%
%\begin{proof}
%	Veamos la primera afirmación: Si $H=\langle
%	h_1,\dots,h_n\rangle\subseteq\Delta(G)$, entonces $(G:C_G(h_i))$ es finito
%	para todo $i\in\{1,\dots,n\}$. Como $C_G(H)=\cap_{i=1}^nC_G(h_i)$, se
%	concluye que $(G:C_G(H))$ es finito.
%
%	Para demostrar la segunda afirmación basta observar que $Z(H)=H\cap C_G(H)$
%	y luego $(H:Z(H))\leq(G:C_G(H)<\infty$. % necesito dos lemas
%
%	La tercera afirmación es consecuencia de la segunda gracias a un teorema de
%	Schur.
%
%	Por último, demostremos la cuarta afirmación.  El grupo $H/[H,H]$ es
%	abeliano y finitamente generado y luego, sus elementos de torsión forman un
%	grupo finito. Como $[H,H]$ es finito, $[H,H]$ es un subgrupo normal de
%	$H_0$. Vamos a demostrar que la torsión de $H/[H,H]$ es igual a
%	$H_0/[H,H]$. La inclusión $\supseteq$ es trivial. Veamos entonces que vale
%	$\subseteq$: so $(x[H,H])^k=1$, entonces $x^k\in[H,H]$. Luego $(x^k)^m=1$ y
%	luego $x\in H_0$. Tenemos entonces que 
%	\[
%		H/[H,H]\simeq\Z^r\times\operatorname{tor}(H/[H,H])\simeq\Z^r\times H_0/[H,H]
%	\]
%	y luego $H/H_0$ es finitamente generado, abeliano y libre de torsión.
%
%\end{proof}
%
%\begin{lemma}
%	\label{lemma:K[abelian]}
%	Si $G$ un grupo abeliano finitamente generado y sin torsión, entonces
%	$K[G]$ es un dominio. 
%\end{lemma}
%
%\begin{proof}
%	Por el teorema
%	de estructura de grupos abelianos finitamente generados podemos escribir
%	$G=\langle x_1\rangle\times\cdots\langle x_n\rangle$, donde
%	$\langle x_j\rangle\simeq\Z$ para todo $j\in\{1,\dots,n\}$. Todo elemento
%	de $G$ se escribe unívocamente como $x_1^{m_1}\cdots x_n^{m_n}$ y
%	luego la función 
%	\[
%		\iota\colon K[X_1,\dots,X_n]\to K[G],\quad
%		X_j\mapsto x_j,
%	\]
%	es un
%	morfismo de anillos inyectivo. Si $\alpha\in K[G]$, entonces existe
%	$m\in\N$ suficientemente grande tal que $\iota((X_1\cdots X_n)^m)\alpha\in
%	\iota(K[X_1,\dots,X_n])\simeq K[X_1,\dots,X_n]$. Luego $K[G]\subseteq
%	K(X_1,\dots,X_n)$ y $K[G]$ es un dominio.
%\end{proof}

%\begin{lemma}
%	Si $G$ es un grupo, entonces
%	$\Delta(G)/\Delta^+(G)$ es abeliano y libre de torsión.
%%	Valen las siguientes afirmaciones:
%%	\begin{enumerate}
%%		%\item $\Delta^+(G)$ está generado por los subgrupos normales finitos de $G$.
%%		\item 
%%		\item Si $\Delta^+(G)=1$, entonces $K[\Delta(G)]$ es un dominio.
%%	\end{enumerate}
%\end{lemma}
%
%\begin{proof}
%%	Demostremos la primera afirmación. 
%	Sean $y_1,\dots,y_n\in\Delta(G)$ y sea $L=\langle y_1,\dots,y_n\rangle$.
%	Como $[L,L]$ es finito por el lema~\ref{lemma:Dfg}, $[L,L]\subseteq\Delta^+(G)$. Luego
%	$\Delta(G)/\Delta^+(G)$ es abeliano y libre de torsión.
%%
%%	Para demostrar la segunda afirmación basta observar que si $\Delta^+(G)=1$
%%	entonces, por el primer ítem, $\Delta(G)$ es abeliano, finitamente generado
%%	y libre de torsión. Luego $K[\Delta(G)]$ es un dominio por el
%%	lema~\ref{lemma:K[abelian]}. 
%\end{proof}

If $S$ is a finite subset of a group $G$, then we define 
$\widehat{S}=\sum_{x\in S}x$. 

\begin{lemma}
	\label{lemma:sumN}
	Let $N$ be a finite normal subgroup of $G$. Then $\widehat{N}=\sum_{x\in N}x$ is central
	in $K[G]$ and $\widehat{N}(\widehat{N}-|N|1)=0$.
\end{lemma}

\begin{proof}
	Assume that $N=\{n_1,\dots,n_k\}$. Let 
	$g\in G$. Since $N\to N$, $n\mapsto gng^{-1}$, is bijective, 
	\[
		g\widehat{N}g^{-1}=g(n_1+\cdots+n_k)g^{-1}=gn_1g^{-1}+\cdots+gn_kg^{-1}=\widehat{N}.
	\]
	Since $nN=N$ if $n\in N$, it follows that $n\widehat{N}=\widehat{N}$. Thus 
	$\widehat{N}\widehat{N}=\sum_{j=1}^k n_j\widehat{N}=|N|\widehat{N}$.
\end{proof}

%Before proving Connel's theorem we need to prove two group theoretical results.



Si $G$ es un grupo, consideramos el subconjunto %los siguientes subconjuntos:
\begin{align*}
%	&\Delta(G)=\{x\in G:(G:C_G(x))<\infty\},\\
	&\Delta^+(G)=\{x\in \Delta(G):\text{$x$ tiene orden finito}\}.
\end{align*}

\begin{lemma}
	\label{lem:DcharG}
	Si $G$ es un grupo, entonces $\Delta^+(G)$ es un subgrupo
	característico de $G$.
\end{lemma}

\begin{proof}
	Claramente $1\in\Delta^+(G)$. 
	Sean $x,y\in\Delta^+(G)$ y sea $H$ el subgrupo de $G$ generado por el
	conjunto $C$ formado por los finitos conjugados de $x$ e $y$. Si $|x|=n$ y
	$|y|=m$, entonces $c^{nm}=1$ para todo $c\in C$. Como $C$ es 
	finito y cerrado por conjugación, el teorema de Dietzmann implica que $H$ es
	finito. Luego $H\subseteq\Delta^+(G)$ y en particular $xy^{-1}\in\Delta^+(G)$.  Es
	evidente que $\Delta^+(G)$ es un subgrupo característico pues para todo
	$f\in\Aut(G)$ se tiene que $f(x)\in\Delta^+(G)$ si $x\in\Delta^+(G)$.
%	Primero veamos que $\Delta(G)$ es un subgrupo de $G$. Si $x,y\in\Delta(G)$
%	y $g\in G$, entonces $g(xy^{-1})g^{-1}=(gxg^{-1})(gyg^{-1})^{-1}$. Además
%	$1\in\Delta(G)$. Veamos ahora que $\Delta(G)$ es característico en $G$. Si
%	$f\in\Aut(G)$ y $x\in G$, entonces, como $f(gxg^{-1})=f(g)f(x)f(g)^{-1}$,
%	se concluye que $f(x)\in\Delta(G)$.
%	Para ver que $\Delta^+(G)$ es un subgrupo, 
%	Sean
%	$x_1,\dots,x_n\in\Delta^+(G)$ y $H=\langle x_1,\dots,x_n\rangle$. Como
%	$H$ es finito, $H\subseteq\Delta^+(G)$ y luego $\Delta^+(G)$ es un
%	subgrupo. Es evidente que es un subgrupo característico pues para todo
%	$f\in\Aut(G)$ se tiene que $f(x)\in\Delta^+(G)$ si $x\in\Delta^+(G)$.
\end{proof}

La segunda aplicación del teorema de Dietzmann es el siguiente resultado:

\begin{lemma}
	\label{lem:Connel}
	Sea $G$ un grupo y sea  $x\in\Delta^+(G)$.  Existe entonces un subgrupo
	finito $H$ normal en $G$ tal que $x\in H$.
\end{lemma}

Dejamos la demostración como ejercicio ya que el muy similar a lo que hicimos
en la demostración del lema~\ref{lem:DcharG}.

%\begin{proof}
%	Sea $H$ el subgrupo generado por los conjugados de $x$. Como $x$ tiene
%	finitos conjugados, $H$ es finitamente generado. Además $H$ es claramente
%	normal en $G$ y está generado por elementos de torsión. Todos los finitos
%	generadores de $H$ tienen el mismo orden, digamos $n$. Por el teorema de
%	Dietzmann, $H$ resulta ser un grupo finito.
%\end{proof}

\begin{theorem}[Connell]
	\label{thm:Connel}
	\index{Teorema!de Connel}
	Supongamos que el cuerpo $K$ es de característica cero. 
	Sea $G$ un grupo. Las siguientes afirmaciones son equivalentes:
	\begin{enumerate}
		\item $K[G]$ es primo.
		\item $Z(K[G])$ es primo.
		\item $G$ no tiene subgrupos finitos normales no triviales.
		\item $\Delta^+(G)=1$.
	\end{enumerate}
\end{theorem}

\begin{proof}
	Demostremos que $(1)\implies(2)$. Como $Z(K[G])$ es un anillo conmutativo,
	probar que es primo es equivalente a probar que no existen divisores de
	cero no triviales. Sean $\alpha,\beta\in Z(K[G])$ tales que
	$\alpha\beta=0$. Sean $A=\alpha K[G]$ y $B=\beta K[G]$. Como $\alpha$ y
	$\beta$ son centrales, $A$ y $B$ son ideales de $K[G]$. Como $AB=0$,
	entonces $A=\{0\}$ o $B=\{0\}$ pues $K[G]$ es primo.  Luego $\alpha=0$ o
	$\beta=0$.

	Demostremos ahora que $(2)\implies(3)$. Sea $N$ un subgrupo normal finito.
	Por el lema~\ref{lemma:sumN}, $\widehat{N}=\sum_{x\in N}x$ es central en
	$K[G]$ y $\widehat{N}(\widehat{N}-|N|1)=0$. Como $\widehat{N}\ne 0$ (pues
	$K$ tiene característica cero) y $Z(K[G])$ es un dominio,
	$\widehat{N}=|N|1$, es decir: $N=\{1\}$.

	Demostremos que $(3)\implies(4)$. Sea $x\in\Delta^+(G)$. Por el
	lema~\ref{lem:Connel} sabemos que existe un subgrupo finito $H$ normal en
	$G$ que contiene a $x$. Como por hipótesis $H$ es trivial, se concluye que
	$x=1$.

	Finalmente demostramos que $(4)\implies(1)$. Sean $A$ y $B$ ideales de
	$K[G]$ tales que $AB=0$. Supongamos que $B\ne 0$ y sea $\beta\in
	B\setminus\{0\}$.  Si $\alpha\in A$, entonces, como $\alpha
	K[G]\beta\subseteq \alpha B\subseteq AB=0$, el lema~\ref{lem:Passman} de
	Passman implica que $\pi_{\Delta(G)}(\alpha)\pi_{\Delta(G)}(\beta)=0$.
	Como por hipótesis $\Delta^+(G)$ es trivial, sabemos que $\Delta(G)$ es 
	libre de torsión y luego $\Delta(G)$ es abeliano por el
	lema~\ref{lem:FCabeliano}. Esto nos dice que $K[\Delta(G)]$ no tiene
	divisores de cero y luego $\alpha=0$. Demostramos entonces que $B\ne0$
	implica que $A=0$.
\end{proof}

% necesito: 
% pag 376 del Hungerford: un módulo no nulo admite una serie de composición
% si y sólo si es noetheriano y artiniano
% agregar además el teorema de Hopkins--Levitzky que dice


\begin{theorem}[Connel]
	Sea $K$ un cuerpo de característica cero y sea $G$ un grupo. Entonces
	$K[G]$ es artiniano a izquierda si y sólo si $G$ es finito.
\end{theorem}

\begin{proof}
	Si $G$ es finito, $K[G]$ es un álgebra de dimensión finita y luego
	es artiniano a izquierda. Supongamos entonces que $K[G]$ es artiniano
	a izquierda. 
	
	Primero observemos que si $K[G]$ es un álgebra prima, entonces por el
	teorema de Wedderburn $K[G]$ es simple y luego
	$G$ es el grupo trivial (pues si $G$ no es trivial, $K[G]$ no es simple ya
	que el ideal de aumentación es un ideal no nulo de $K[G]$).

	Como $K[G]$ es artiniano a izquierda, es noetheriano a izquierda por
	Hopkins--Levitzky y entonces, $K[G]$ admite una serie de composición por el
	teorema~\ref{thm:serie_de_composicion}.  Para demostrar el teorema
	procederemos por inducción en la longitud de la serie de composición de
	$K[G]$. Si la longitud es uno, $\{0\}$ es el único ideal de $K[G]$ y luego
	$K[G]$ es prima y el resultado está demostrado. Si suponemos que el
	resultado vale para longitud $n$ y además $K[G]$ no es prima, entonces, por
	el teorema de Connel, $G$ posee un subgrupo normal $H$ finito y no trivial. Al
	considerar el morfismo canónico $K[G]\to K[G/H]$ vemos que $K[G/H]$ es
	artiniano a izquierda y tiene longitud $<n$. Por hipótesis inductiva, $G/H$
	es un grupo finito y luego, como $H$ también es finito, $G$ es finito.
\end{proof}
