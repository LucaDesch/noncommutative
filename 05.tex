\chapter{}

\section{Unique product groups}

Sea $G$ un grupo y sean $A,B\subseteq G$ subconjuntos no vacíos. Diremos que un
elemento $g\in G$ es un producto único en $AB$ si $g=ab=a_1b_1$ con $a,a_1\in
A$ y $b,b_1\in B$ implica que $a=a_1$ y $b=b_1$.

\begin{definition}
	\index{Grupo!con la propiedad del producto único}
	Se dice que un grupo $G$ tiene la \textbf{propiedad del producto único} si
	dados dos subconjuntos $A,B\subseteq G$ finitos y no vacíos existe al menos
	un producto único en $AB$.
\end{definition}

\begin{proposition}
	Si un grupo $G$ es ordenable a derecha, entonces $G$ tiene la propiedad del
	producto único.
\end{proposition}

\begin{proof}
	Sean $A=\{a_1,\dots,a_n\}\subseteq G$ y $B\subseteq G$ ambos finitos y no
	vacíos. Supongamos que $a_1<a_2<\cdots<a_n$. Sea $c\in B$ tal que $a_1c$ es
	el mínimo del conjunto $a_1B=\{a_1b:b\in B\}$. Veamos que $a_1c$ admite una
	única representación de la forma $\alpha\beta$ con $\alpha\in A$ y
	$\beta\in B$. Si $a_1c=ab$, entonces, como $ab=a_1c\leq a_1b$, se tiene que
	$a\leq a_1$ y luego $a=a_1$ y $b=c$. 
\end{proof}

\begin{exercise}
	Demuestre que un grupo que satisface la propiedad del producto único es
	libre de torsión.
\end{exercise}

The converse does not hold. 
Promislow's group is a celebrated counterexample.

\begin{theorem}[Promislow]
    The 
    group $G=\langle a,b:a^{-1}b^2a=b^{-2},b^{-1}a^2b=a^{-2}\rangle$
    does not have the unique product property.
\end{theorem}

\begin{proof}
    Let 
    \begin{multline}
    \label{eq:Promislow}
    S=\{ a^2b,
    b^2a,
    aba^{-1},
    (b^2a)^{-1},
    (ab)^{-2},
    b,
    (ab)^2x,
    (ab)^2,
    (aba)^{-1},\\
    bab,
    b^{-1},
    a,
    aba,
    a^{-1}
    \}.
    \end{multline}
    We use \textsf{GAP} and the representation $G\to\GL(4,\Q)$ given by 
    \[
a\mapsto\begin{pmatrix}
1 & 0 & 0 & 1/2\\
0 & -1 & 0 & 1/2\\
0 & 0 & -1 & 0\\
0 & 0 & 0 & 1
\end{pmatrix},
\quad
b\mapsto\begin{pmatrix}
-1 & 0 & 0 & 0\\
0 & 1 & 0 & 1/2\\
0 & 0 & -1 & 1/2\\
0 & 0 & 0 & 1
\end{pmatrix}
\]
    to check that 
    $G$ does not have
    unique product property, as each 
    \[
    s\in S^2=\{s_1s_2:s_1,s_2\in S\}
    \]
    admits at least two different decompositions of the 
    form $s=xy=uv$ for $x,y,u,v\in S$. 
    We first create the matrix representations of $a$ and $b$.
\begin{lstlisting}
gap> a := [[1,0,0,1/2],[0,-1,0,1/2],[0,0,-1,0],[0,0,0,1]];;
gap> b := [[-1,0,0,0],[0,1,0,1/2],[0,0,-1,1/2],[0,0,0,1]];;
\end{lstlisting}
    Now we create
    a function that produces the set $S$.
\begin{lstlisting}
gap> Promislow := function(x, y)
> return Set([
> x^2*y,
> y^2*x,
> x*y*Inverse(x),
> (y^2*x)^(-1),
> (x*y)^(-2),
> y,
> (x*y)^2*x,
> (x*y)^2,
> (x*y*x)^(-1),
> y*x*y,
> y^(-1),
> x,
> x*y*x,
> x^(-1)
]);
end;;
\end{lstlisting}
So the set $S$ of \eqref{eq:Promislow} 
will be \lstinline{Promislow(a,b)}. We now
create a function that checks whether
every element of a Promislow subset 
admits more than one representation.
\begin{lstlisting}
gap> is_UPP := function(S)
> local l,x,y;
> l := [];
> for x in S do
> for y in S do
> Add(l,x*y);
> od;
> od;
> if ForAll(Collected(l), x->x[2] <> 1) then
> return false;
> else
> return fail;
> fi;
> end;;
\end{lstlisting}
Finally, we check whether every element of 
$S$ admits more than one representation.
\begin{lstlisting}
gap> S := Promislow(a,b);;
gap> is_UPP(S);
false
\end{lstlisting}
This completes the proof. 
\end{proof}

There are other examples. 

% Let $G=\langle x,y:x^{-1}y^2xy^2=x^{-2}yx^{-2}y^3=1\rangle$. 
% We first construct the group and a certain normal
% subgroup $N$ of index eight. 

% \begin{lstlisting}
% gap> f := FreeGroup(2);;
% gap> x := f.1;;
% gap> y := f.2;;
% gap> rels := [Inverse(x)*y^2*x*y^2, Inverse(x^2)*y*Inverse(x^2)*y^3];;
% gap> G := f/rels;;
% gap> x := G.1;;
% gap> y := G.2;;
% gap> N := Subgroup(G, [y*x*Inverse(x*y), y^2, x^4]);
% gap> IsNormal(N,G);
% true
% gap> StructureDescription(G/N);
% "C4 x C2"
% \end{lstlisting}
% The subgroup $N$ has a nice presentation. It can be presented
% as the group 
% \[
% N=\langle x_1,x_2,x_3:[x_1,x_2]=[x_1,x_3]=1,\,x_3x_2=x_2x_3x_1^8\rangle.
% \]
% \begin{lstlisting}
% gap> g := IsomorphismFpGroup(N);
% gap> RelatorsOfFpGroup(Image(g));
% [ F3*F1*F3^-1*F1^-1, F2*F1*F2^-1*F1^-1, F1*F3^-1*F1^5*F2*F1^2*F3*F2^-1 ]
% \end{lstlisting}
% From these relations
% one proves by induction that 
% \begin{align}
%     x_3^bx_2^a=x_2^ax_3^bx_1^{8ab}
% \end{align}
% for all $a,b\in\Z$. 
% % First induction in $n_1$, then induction in $n_2$. 
% It follows that every element of $N$ 
% is of the form $x_1^{n_1}x_2^{n_2}x_3^{n_3}$ 
% for $n_1,n_2,n_3\in\Z$. Moreover, 
% \[
% (x_1^{n_1}x_2^{n_2}x_3^{n_3})^k=x_1^{kn_1+(k-1)8n_2n_3}x_2^{kn_2}x_3^{kn_3}
% \]
% for all $k\in\Z$. Note that $x_1^8$ is a commutator. Moreover, 
% $N/[N,N]\simeq\Z\times\Z\times\Z/8$. This implies that
% $N$ is torsion-free. Let us prove that $G$ is torsion-free. 
% Let $\pi\colon G\to G/N$ be the canonical map. Let 
% $g\in G$ be a torsion element, in particular $g\not\in N$
% and hence $\pi(g)\ne 1$. So $\pi(g)$ has order two 
% or four. Without loss of generality we may assume that
% $\pi(g)$ has order two.  
% Then $\pi(g^2)=\pi(g)^2=y^2=1$ and hence 
% $g^2\in N$. Since $N$ is torsion free, it follows that $g=1$. 

\begin{definition}
	Se dice que un grupo $G$ tiene la \textbf{propiedad del doble producto
	único} si dados dos subconjuntos $A,B\subseteq G$ finitos y no vacíos tales
	que $|A|+|B|>2$ existen al menos dos productos únicos en $AB$.
\end{definition}

\begin{theorem}[Strojnowski]
	\label{theorem:Strojnowski}
	\index{Teorema!de Strojonowski}
	Sea $G$ un grupo. Las siguientes afirmaciones son equivalentes:
	\begin{enumerate}
		\item $G$ tiene la propiedad del doble producto único.
		\item Para todo subconjunto $A\subseteq G$ finito y no vacío, existe al
			menos un producto único en $AA=\{a_1a_2:a_1,a_2\in A\}$.
		\item $G$ tiene la propiedad del producto único.
	\end{enumerate}
\end{theorem}

\begin{proof}
	La implicación $(1)\implies(2)$ es trivial.  Demostremos que vale
	$(2)\implies(3)$. Si $G$ no tiene la propiedad del producto único, existen
	subconjuntos $A,B\subseteq G$ finitos y no vacíos tales que todo elemento
	de $AB$ admite al menos dos representaciones. Sea $C=AB$. Todo $c\in C$ es
	de la forma $c=(a_1b_1)(a_2b_2)$ con $a_1,a_2\in A$ y $b_1,b_2\in B$. Como
	$a_2^{-1}b_1^{-1}\in AB$, existen $a_3\in A\setminus\{a_2\}$ y $b_3\in B\setminus\{b_1\}$ tales que
	$a_2^{-1}b_1^{-1}=a_3^{-1}b_3^{-1}$. Luego $b_1a_2=b_3a_3$ y entonces
	\[
	c=(a_1b_1)(a_2b_2)=(a_1b_3)(a_3b_2)
	\]
	son dos representaciones distintas de $c$ en $AB$.
	pues $a_2\ne a_3$ y $b_1\ne b_3$.

	Demostremos ahora que $(3)\implies(1)$. Si $G$ tiene la propiedad del
	producto único pero no tiene la propiedad del doble producto único, existen
	subconjuntos $A,B\subseteq G$ finitos y no vacíos con $|A|+|B|>2$ tales que
	en $AB$ existe un único elemento $ab$ con una única representación en $AB$.
	Sean $C=a^{-1}A$ y $D=Bb^{-1}$. Entonces $1\in C\cap D$ y el elemento
	neutro $1$ admite una única representación en $CD$ (pues si $1=cd$ con
	$c=a^{-1}a_1\ne 1$ y $d=b_1b^{-1}\ne 1$, entonces $ab=a_1b_1$ con $a\ne
	a_1$ y $b\ne b_1)$. Sean $E=D^{-1}C$ y $F=DC^{-1}$. Todo elemento de $EF$
	se escribe como $(d_1^{-1}c_1)(d_2c_2^{-1})$. Si $c_1\ne 1$ o $d_2\ne 1$
	entonces $c_1d_2=c_3d_3$ para algún $c_3\in C\setminus\{c_1\}$ y algún
	$d_3\in D\setminus\{d_2\}$. Entonces
	$(d_1^{-1}c_1)(d_2c_2^{-1})=(d_1^{-1}c_3)(d_3c_2^{-1})$ son dos
	representaciones distintas para $(d_1^{-1}c_1)(d_2c_2^{-1})$. Si $c_2\ne 1$
	o $d_1\ne 1$ entonces $c_2d_1=c_4d_4$ para algún $d_4\in D\setminus\{d_1\}$
	y algún $c_4\in C\setminus\{c_2\}$ y entonces, como
	$d_1^{-1}c_2^{-1}=d_4^{-1}c_4^{-1}$,
	$(d_1^{-1}1)(1c_2^{-1})=(d_4^{-1}1)(1c_4^{-1})$.  Como $|C|+|D|>2$, $C$ o
	$D$ contienen algún $c\ne1$, y entonces $(1\cdot 1)(1\cdot 1)=(1\cdot
	c)(1\cdot c^{-1})$. Demostramos entonces que todo elemento de $EF$ tiene al
	menos dos representaciones. 
\end{proof}

% passman lema 1.9 pag 589
\begin{exercise}
	Demuestre que si $G$ es un grupo que satisface la propiedad del producto
	único, entonces $K[G]$ tiene solamente unidades triviales.
\end{exercise}

En general es muy difícil verificar si un grupo posee la propiedad del producto
único. Una propiedad similar es la de ser un grupo difuso. Si $G$ es un grupo
libre de torsión y $A\subseteq G$ es un subconjunto, diremos que $A$ es
antisimétrico si $A\cap A^{-1}\subseteq\{1\}$, donde $A^{-1}=\{a^{-1}:a\in
A\}$. El conjunto de \textbf{elementos extremales} de $A$ se define como
$\Delta(A)=\{a\in A:Aa^{-1}\text{ es antisimétrico}\}$. Luego
\[
	a\in A\setminus\Delta(A)
	\Longleftrightarrow
	\text{existe $g\in G\setminus\{1\}$ tal que $ga\in A$ y $g^{-1}a\in A$}.
\]

\begin{definition}
	\index{Grupo!difuso}
	Un grupo $G$ se dice \textbf{difuso} si para todo subconjunto $A\subseteq
	G$ tal que $2\leq |A|<\infty$ se tiene $|\Delta(A)|\geq2$.
\end{definition}

\begin{lemma}
	Si $G$ es ordenable a derecha, entonces $G$ es difuso.	
\end{lemma}

\begin{proof}
	Supongamos que $A=\{a_1,\dots,a_n\}$ y $a_1<a_2<\cdots<a_n$. Vamos a
	demostrar que $\{a_1,a_n\}\subseteq\Delta(A)$. Si $a_1\in
	A\setminus\Delta(A)$, existe $g\in G\setminus\{1\}$ tal que $ga_1\in A$ y
	$g^{-1}a_1\in A$. Esto implica que $a_1\leq ga_1$ y $a_1\leq g^{-1}a_1$, de
	donde se concluye que $1\leq g$ y $1\leq g^{-1}$, una contradicción. De la
	misma forma se demuestra que $a_n\in \Delta(A)$.
\end{proof}

\begin{lemma}
	\label{lemma:difuso=>2up}
	Si $G$ es difuso, entonces $G$ tiene la propiedad del doble producto único.	
\end{lemma}

\begin{proof}
	Supongamos que $G$ no tiene la propiedad del doble producto único. Existen
	entonces subconjuntos finitos $A,B\subseteq G$ con $|A|+|B|>2$ tales que
	$C=AB$ tiene a lo sumo un producto único. Luego $|C|\geq2$. Como $G$ es
	difuso, $|\Delta(C)|\geq2$. Si $c\in\Delta(C)$, entonces $c$ tiene una
	única expresión como $c=ab$ con $a\in A$ y $b\in B$ (de lo contrario, si
	$c=a_0b_0=a_1b_1$ con $a_0\ne a_1$ y $b_0\ne b_1$. Si $g=a_0a_1^{-1}$,
	entonces $g\ne 1$, $gc=a_0a_1^{-1}a_1b_1=a_0b_1\in C$ y además
	$g^{-1}c=a_1a_0^{-1}a_0b_0=a_1b_0\in C$. Luego $c\not\in\Delta(c)$, una
	contradicción.
\end{proof}

\begin{problem}
	Find a non-diffuse group with the unique product property.
\end{problem}

%Un grupo $G$ se dice \textbf{débilmente difuso} si para todo subconjunto
%finito $A\subseteq G$ no vacío se tiene $\Delta(A)\ne\emptyset$. La técnica
%usada para demostrar el lema~\ref{lemma:difuso=>2up} puede usarse para
%demostrar que un grupo débilmente difuso posee la propiedad del producto
%único. El teorema~\ref{theorem:Strojnowski} sugiere entonces la siguiente
%pregunta: 
%
%\begin{problem}
%	¿Existe un grupo débilmente difuso que no sea difuso?
%\end{problem}
%
%\section{El grupo de Promislow}
%
%Veremos un ejemplo concreto de un grupo sin torsión que no es ordenable, no es
%difuso y no tiene la propiedad del producto único.
%
%\begin{exercise}
%	\label{exercise:Dinfty}
%	Demuestre que $G=\langle x,y:x^2=y^2=1\rangle$ es isomorfo al grupo diedral infinito.
%\end{exercise}
%
%\begin{definition}
%	Se define el grupo de Promislow como 
%	\[
%		G=\langle x,y:x^{-1}y^2x=y^{-2},\,y^{-1}x^2y=x^{-2}\rangle.
%	\]
%\end{definition}
%
%\begin{proposition}
%	\label{proposition:Promislow}
%	El grupo de Promislow es libre de torsión y no satisface la propiedad del
%	producto único. 
%\end{proposition}
%
%\begin{proof}
%	
%\end{proof}

\topic{Connel's theorem}

When $K[G]$ is prime? Connel's theorem gives a full answer to this natural
question in the case where $K$ is of characteristic zero. 

%\begin{lemma}
%	\label{lemma:Dfg}
%	Sea $H$ un subgrupo finitamente generado de $\Delta(G)$.
%	\begin{enumerate}
%		\item $(G:C_G(H))$ es finito.
%		\item $(H:Z(H))$ es finito.
%		\item $[H,H]$ es finito.
%		\item Si $H_0$ es el conjunto de elementos de torsión de $H$, $H_0$ es
%			un subgrupo normal finito de $H$ y $H/H_0$ es finitamente generado,
%			abeliano y libre de torsión.
%	\end{enumerate}
%\end{lemma}
%
%\begin{proof}
%	Veamos la primera afirmación: Si $H=\langle
%	h_1,\dots,h_n\rangle\subseteq\Delta(G)$, entonces $(G:C_G(h_i))$ es finito
%	para todo $i\in\{1,\dots,n\}$. Como $C_G(H)=\cap_{i=1}^nC_G(h_i)$, se
%	concluye que $(G:C_G(H))$ es finito.
%
%	Para demostrar la segunda afirmación basta observar que $Z(H)=H\cap C_G(H)$
%	y luego $(H:Z(H))\leq(G:C_G(H)<\infty$. % necesito dos lemas
%
%	La tercera afirmación es consecuencia de la segunda gracias a un teorema de
%	Schur.
%
%	Por último, demostremos la cuarta afirmación.  El grupo $H/[H,H]$ es
%	abeliano y finitamente generado y luego, sus elementos de torsión forman un
%	grupo finito. Como $[H,H]$ es finito, $[H,H]$ es un subgrupo normal de
%	$H_0$. Vamos a demostrar que la torsión de $H/[H,H]$ es igual a
%	$H_0/[H,H]$. La inclusión $\supseteq$ es trivial. Veamos entonces que vale
%	$\subseteq$: so $(x[H,H])^k=1$, entonces $x^k\in[H,H]$. Luego $(x^k)^m=1$ y
%	luego $x\in H_0$. Tenemos entonces que 
%	\[
%		H/[H,H]\simeq\Z^r\times\operatorname{tor}(H/[H,H])\simeq\Z^r\times H_0/[H,H]
%	\]
%	y luego $H/H_0$ es finitamente generado, abeliano y libre de torsión.
%
%\end{proof}
%
%\begin{lemma}
%	\label{lemma:K[abelian]}
%	Si $G$ un grupo abeliano finitamente generado y sin torsión, entonces
%	$K[G]$ es un dominio. 
%\end{lemma}
%
%\begin{proof}
%	Por el teorema
%	de estructura de grupos abelianos finitamente generados podemos escribir
%	$G=\langle x_1\rangle\times\cdots\langle x_n\rangle$, donde
%	$\langle x_j\rangle\simeq\Z$ para todo $j\in\{1,\dots,n\}$. Todo elemento
%	de $G$ se escribe unívocamente como $x_1^{m_1}\cdots x_n^{m_n}$ y
%	luego la función 
%	\[
%		\iota\colon K[X_1,\dots,X_n]\to K[G],\quad
%		X_j\mapsto x_j,
%	\]
%	es un
%	morfismo de anillos inyectivo. Si $\alpha\in K[G]$, entonces existe
%	$m\in\N$ suficientemente grande tal que $\iota((X_1\cdots X_n)^m)\alpha\in
%	\iota(K[X_1,\dots,X_n])\simeq K[X_1,\dots,X_n]$. Luego $K[G]\subseteq
%	K(X_1,\dots,X_n)$ y $K[G]$ es un dominio.
%\end{proof}

%\begin{lemma}
%	Si $G$ es un grupo, entonces
%	$\Delta(G)/\Delta^+(G)$ es abeliano y libre de torsión.
%%	Valen las siguientes afirmaciones:
%%	\begin{enumerate}
%%		%\item $\Delta^+(G)$ está generado por los subgrupos normales finitos de $G$.
%%		\item 
%%		\item Si $\Delta^+(G)=1$, entonces $K[\Delta(G)]$ es un dominio.
%%	\end{enumerate}
%\end{lemma}
%
%\begin{proof}
%%	Demostremos la primera afirmación. 
%	Sean $y_1,\dots,y_n\in\Delta(G)$ y sea $L=\langle y_1,\dots,y_n\rangle$.
%	Como $[L,L]$ es finito por el lema~\ref{lemma:Dfg}, $[L,L]\subseteq\Delta^+(G)$. Luego
%	$\Delta(G)/\Delta^+(G)$ es abeliano y libre de torsión.
%%
%%	Para demostrar la segunda afirmación basta observar que si $\Delta^+(G)=1$
%%	entonces, por el primer ítem, $\Delta(G)$ es abeliano, finitamente generado
%%	y libre de torsión. Luego $K[\Delta(G)]$ es un dominio por el
%%	lema~\ref{lemma:K[abelian]}. 
%\end{proof}

If $S$ is a finite subset of a group $G$, then we define 
$\widehat{S}=\sum_{x\in S}x$. 

\begin{lemma}
	\label{lemma:sumN}
	Let $N$ be a finite normal subgroup of $G$. Then $\widehat{N}=\sum_{x\in N}x$ is central
	in $K[G]$ and $\widehat{N}(\widehat{N}-|N|1)=0$.
\end{lemma}

\begin{proof}
	Assume that $N=\{n_1,\dots,n_k\}$. Let 
	$g\in G$. Since $N\to N$, $n\mapsto gng^{-1}$, is bijective, 
	\[
		g\widehat{N}g^{-1}=g(n_1+\cdots+n_k)g^{-1}=gn_1g^{-1}+\cdots+gn_kg^{-1}=\widehat{N}.
	\]
	Since $nN=N$ if $n\in N$, it follows that $n\widehat{N}=\widehat{N}$. Thus 
	$\widehat{N}\widehat{N}=\sum_{j=1}^k n_j\widehat{N}=|N|\widehat{N}$.
\end{proof}

%Before proving Connel's theorem we need to prove two group theoretical results.



Si $G$ es un grupo, consideramos el subconjunto %los siguientes subconjuntos:
\begin{align*}
%	&\Delta(G)=\{x\in G:(G:C_G(x))<\infty\},\\
	&\Delta^+(G)=\{x\in \Delta(G):\text{$x$ tiene orden finito}\}.
\end{align*}

\begin{lemma}
	\label{lem:DcharG}
	Si $G$ es un grupo, entonces $\Delta^+(G)$ es un subgrupo
	característico de $G$.
\end{lemma}

\begin{proof}
	Claramente $1\in\Delta^+(G)$. 
	Sean $x,y\in\Delta^+(G)$ y sea $H$ el subgrupo de $G$ generado por el
	conjunto $C$ formado por los finitos conjugados de $x$ e $y$. Si $|x|=n$ y
	$|y|=m$, entonces $c^{nm}=1$ para todo $c\in C$. Como $C$ es 
	finito y cerrado por conjugación, el teorema de Dietzmann implica que $H$ es
	finito. Luego $H\subseteq\Delta^+(G)$ y en particular $xy^{-1}\in\Delta^+(G)$.  Es
	evidente que $\Delta^+(G)$ es un subgrupo característico pues para todo
	$f\in\Aut(G)$ se tiene que $f(x)\in\Delta^+(G)$ si $x\in\Delta^+(G)$.
%	Primero veamos que $\Delta(G)$ es un subgrupo de $G$. Si $x,y\in\Delta(G)$
%	y $g\in G$, entonces $g(xy^{-1})g^{-1}=(gxg^{-1})(gyg^{-1})^{-1}$. Además
%	$1\in\Delta(G)$. Veamos ahora que $\Delta(G)$ es característico en $G$. Si
%	$f\in\Aut(G)$ y $x\in G$, entonces, como $f(gxg^{-1})=f(g)f(x)f(g)^{-1}$,
%	se concluye que $f(x)\in\Delta(G)$.
%	Para ver que $\Delta^+(G)$ es un subgrupo, 
%	Sean
%	$x_1,\dots,x_n\in\Delta^+(G)$ y $H=\langle x_1,\dots,x_n\rangle$. Como
%	$H$ es finito, $H\subseteq\Delta^+(G)$ y luego $\Delta^+(G)$ es un
%	subgrupo. Es evidente que es un subgrupo característico pues para todo
%	$f\in\Aut(G)$ se tiene que $f(x)\in\Delta^+(G)$ si $x\in\Delta^+(G)$.
\end{proof}

La segunda aplicación del teorema de Dietzmann es el siguiente resultado:

\begin{lemma}
	\label{lem:Connel}
	Sea $G$ un grupo y sea  $x\in\Delta^+(G)$.  Existe entonces un subgrupo
	finito $H$ normal en $G$ tal que $x\in H$.
\end{lemma}

Dejamos la demostración como ejercicio ya que el muy similar a lo que hicimos
en la demostración del lema~\ref{lem:DcharG}.

%\begin{proof}
%	Sea $H$ el subgrupo generado por los conjugados de $x$. Como $x$ tiene
%	finitos conjugados, $H$ es finitamente generado. Además $H$ es claramente
%	normal en $G$ y está generado por elementos de torsión. Todos los finitos
%	generadores de $H$ tienen el mismo orden, digamos $n$. Por el teorema de
%	Dietzmann, $H$ resulta ser un grupo finito.
%\end{proof}

\begin{theorem}[Connell]
	\label{thm:Connel}
	\index{Teorema!de Connel}
	Supongamos que el cuerpo $K$ es de característica cero. 
	Sea $G$ un grupo. Las siguientes afirmaciones son equivalentes:
	\begin{enumerate}
		\item $K[G]$ es primo.
		\item $Z(K[G])$ es primo.
		\item $G$ no tiene subgrupos finitos normales no triviales.
		\item $\Delta^+(G)=1$.
	\end{enumerate}
\end{theorem}

\begin{proof}
	Demostremos que $(1)\implies(2)$. Como $Z(K[G])$ es un anillo conmutativo,
	probar que es primo es equivalente a probar que no existen divisores de
	cero no triviales. Sean $\alpha,\beta\in Z(K[G])$ tales que
	$\alpha\beta=0$. Sean $A=\alpha K[G]$ y $B=\beta K[G]$. Como $\alpha$ y
	$\beta$ son centrales, $A$ y $B$ son ideales de $K[G]$. Como $AB=0$,
	entonces $A=\{0\}$ o $B=\{0\}$ pues $K[G]$ es primo.  Luego $\alpha=0$ o
	$\beta=0$.

	Demostremos ahora que $(2)\implies(3)$. Sea $N$ un subgrupo normal finito.
	Por el lema~\ref{lemma:sumN}, $\widehat{N}=\sum_{x\in N}x$ es central en
	$K[G]$ y $\widehat{N}(\widehat{N}-|N|1)=0$. Como $\widehat{N}\ne 0$ (pues
	$K$ tiene característica cero) y $Z(K[G])$ es un dominio,
	$\widehat{N}=|N|1$, es decir: $N=\{1\}$.

	Demostremos que $(3)\implies(4)$. Sea $x\in\Delta^+(G)$. Por el
	lema~\ref{lem:Connel} sabemos que existe un subgrupo finito $H$ normal en
	$G$ que contiene a $x$. Como por hipótesis $H$ es trivial, se concluye que
	$x=1$.

	Finalmente demostramos que $(4)\implies(1)$. Sean $A$ y $B$ ideales de
	$K[G]$ tales que $AB=0$. Supongamos que $B\ne 0$ y sea $\beta\in
	B\setminus\{0\}$.  Si $\alpha\in A$, entonces, como $\alpha
	K[G]\beta\subseteq \alpha B\subseteq AB=0$, el lema~\ref{lem:Passman} de
	Passman implica que $\pi_{\Delta(G)}(\alpha)\pi_{\Delta(G)}(\beta)=0$.
	Como por hipótesis $\Delta^+(G)$ es trivial, sabemos que $\Delta(G)$ es 
	libre de torsión y luego $\Delta(G)$ es abeliano por el
	lema~\ref{lem:FCabeliano}. Esto nos dice que $K[\Delta(G)]$ no tiene
	divisores de cero y luego $\alpha=0$. Demostramos entonces que $B\ne0$
	implica que $A=0$.
\end{proof}

% necesito: 
% pag 376 del Hungerford: un módulo no nulo admite una serie de composición
% si y sólo si es noetheriano y artiniano
% agregar además el teorema de Hopkins--Levitzky que dice


\begin{theorem}[Connel]
	Sea $K$ un cuerpo de característica cero y sea $G$ un grupo. Entonces
	$K[G]$ es artiniano a izquierda si y sólo si $G$ es finito.
\end{theorem}

\begin{proof}
	Si $G$ es finito, $K[G]$ es un álgebra de dimensión finita y luego
	es artiniano a izquierda. Supongamos entonces que $K[G]$ es artiniano
	a izquierda. 
	
	Primero observemos que si $K[G]$ es un álgebra prima, entonces por el
	teorema de Wedderburn $K[G]$ es simple y luego
	$G$ es el grupo trivial (pues si $G$ no es trivial, $K[G]$ no es simple ya
	que el ideal de aumentación es un ideal no nulo de $K[G]$).

	Como $K[G]$ es artiniano a izquierda, es noetheriano a izquierda por
	Hopkins--Levitzky y entonces, $K[G]$ admite una serie de composición por el
	teorema~\ref{thm:serie_de_composicion}.  Para demostrar el teorema
	procederemos por inducción en la longitud de la serie de composición de
	$K[G]$. Si la longitud es uno, $\{0\}$ es el único ideal de $K[G]$ y luego
	$K[G]$ es prima y el resultado está demostrado. Si suponemos que el
	resultado vale para longitud $n$ y además $K[G]$ no es prima, entonces, por
	el teorema de Connel, $G$ posee un subgrupo normal $H$ finito y no trivial. Al
	considerar el morfismo canónico $K[G]\to K[G/H]$ vemos que $K[G/H]$ es
	artiniano a izquierda y tiene longitud $<n$. Por hipótesis inductiva, $G/H$
	es un grupo finito y luego, como $H$ también es finito, $G$ es finito.
\end{proof}
