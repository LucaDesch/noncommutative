\section{29/02/2024}



\index{Normalizer}
Let $G$ be a group and $K$ be a subgroup of $G$. We say that $K$ \textbf{normalizes} 
$H$ if $K\subseteq N_G(H)$.
\index{Centralizer}
We say that $K$ \textbf{centralizes} 
$H$ if $K\subseteq C_G(H)$, that is if and only if $[H,K]=\{1\}$.

\begin{exercise}
Let $K$ and $H$ be subgroups of $G$ such that $K\subseteq H$ and $K$ is normal in $G$.
Prove that $[H,G]\subseteq K$ if and only if $H/K\subseteq Z(G/K)$. 
\end{exercise}

\begin{lemma}
\label{lem:gamma_zeta}
Let $G$ be a group. There exists an integer $c$ such that 
$\zeta_c(G)=G$ if and only if 
$\gamma_{c+1}(G)=\{1\}$. In this case, 
\[
\gamma_{i+1}(G)\subseteq\zeta_{c-i}(G)
\]
for all $i\in\{0,1,\dots,c\}$. 
\end{lemma}

\begin{proof}
Assume first that $\zeta_c(G)=G$. To prove that 
$\gamma_{i+1}(G)\subseteq\zeta_{c-i}(G)$ holds for all $i$, we proceed by induction. 
The case $i=0$ is trivial. So assume that the result holds for some $i\geq0$. If
$g\in\gamma_{i+2}(G)=[G,\gamma_{i+1}(G)]$, then     
\[
g=\prod_{k=1}^N [x_k,g_k],
\]
for some $g_1,\dots,g_N\in\gamma_{i+1}(G)$ and $x_1,\dots,x_N\in G$. By the inductive 
hypothesis, 
	\[
	g_k\in\gamma_{i+1}(G)\subseteq\zeta_{c-i}(G)
	\]
for all $k$. Hence $[x_k,g_k]\in\zeta_{c-i-1}(G)$ for all $k$. Therefore   
$g\in\zeta_{c-(i+1)}(G)$. 
	
We now assume that $\gamma_{c+1}(G)=\{1\}$. We aim to prove that 
$\gamma_{i+1}(G)\subseteq\zeta_{c-i}(G)$ holds for all $i$. We proceed by backward induction on $i$. 
The case $i=c$ is trivial. So assume the result holds for some $i+1\leq c$. 
Let $g\in\gamma_{i}(G)$. By the inductive hypothesis, 
	\[
	[x,g]\in [G,\gamma_i(G)]=\gamma_{i+1}(G)\subseteq\zeta_{c-i}(G).
	\]
Thus $g\in\zeta_{c-i+1}(G)$ by definition. 
\end{proof}

\begin{example}
Let $G=\Sym_3$. Then $\zeta_j(G)=\{1\}$ for all $j\geq 0$: 
\begin{lstlisting}
gap> UpperCentralSeries(SymmetricGroup(3));
[ Group(()) ]
\end{lstlisting}
\end{example}

\begin{definition}
\index{Central series}
Let $G$ be a group. A \textbf{central series} for $G$ 
is a sequence 
	\[
		G=G_0\supseteq G_1\supseteq\cdots\supseteq G_n=\{1\}
	\]
of normal subgroups of $G$ such that 
for each $i\in\{1,\dots,n\}$, 
$\pi_i(G_{i-1})$ is a subgroup of $Z(G/G_i)$, where  $\pi_i\colon G\to
G/G_i$ is the canonical map.
\end{definition}

\begin{lemma}
\label{lem:central_series}
Let $G=G_0\supseteq G_1\supseteq\cdots\supseteq G_n=\{1\}$ be 
a central series of a group $G$. Then $\gamma_{i+1}(G)\subseteq G_i$ for all $i$.
\end{lemma}

\begin{proof}
We proceed by induction on $i$. The case $i=0$ is trivial. So assume the result holds for some
$i\geq0$. Let  
	$\pi_i\colon G\to
	G/G_i$ be the canonical map. 
	Then  
	\[
	\gamma_{i+1}(G)=[G,\gamma_i(G)]\subseteq [G,G_{i-1}].
	\]
	Since $\pi_i(G_{i-1})\subseteq Z(G/G_{i})$, 
	\[
        \pi_i([G,G_{i-1}])=[\pi_i(G),\pi_i(G_{i-1})]=\{1\}.
        \]
        Hence $\gamma_{i+1}(G)\subseteq [G,G_{i-1}]\subseteq G_i$. 
\end{proof}

% extender el lema para ver qué pasa con zeta_i

\begin{theorem}
A group is nilpotent if and only if it admits a central series. 
\end{theorem}

\begin{proof}
Let $G$ be a group. If $G$ is nilpotent, then the $\gamma_j(G)$ form a central series of 
$G$. Conversely, if $G=G_0\supseteq
G_1\supseteq\cdots\supseteq G_n=\{1\}$ is a central series of $G$, 
then, by the previous lemma,  
	\[
	\gamma_{n+1}(G)\subseteq G_n=\{1\}.
	\]
Hence $G$ is nilpotent. 
\end{proof}

\begin{exercise}
\label{xca:nilpotent_central}
Let $G$ be a group. Prove that if $K$ is a subgroup of $Z(G)$ such that 
$G/K$ is nilpotent, then $G$ is nilpotent. 
\end{exercise}

\subsection{Hirsch's theorem}

\begin{theorem}[Hirsch]
\label{thm:Hirsch}
\index{Hirsch's theorem}
Let $G$ be a non-trivial nilpotent group. If $H$ is a non-trivial normal subgroup of $G$, 
then $H\cap Z(G)\ne\{1\}$. In particular, $Z(G)\ne\{1\}$. 
\end{theorem}

\begin{proof}
Since $\zeta_0(G)=\{1\}$ and there exists an integer $c$ such that $\zeta_c(G)=G$, 
there exists 
\[
m=\min\{k:H\cap\zeta_k(G)\ne\{1\}\}.
\]
Since $H$ is normal in $G$, 
\[
[G,H\cap\zeta_m(G)]\subseteq H\cap[G,\zeta_m(G)]\subseteq H\cap\zeta_{m-1}(G)=\{1\}.
\]
Therefore $\{1\}\ne H\cap\zeta_m(G)\subseteq H\cap Z(G)$. If $H=G$, then $Z(G)\ne\{1\}$. 
\end{proof}

\begin{exercise}
\label{xca:nilpotent_minimalnormal}
Let $G$ be a nilpotent group and $M$ be a minimal normal subgroup of $G$. Prove 
that $M\subseteq Z(G)$.
\end{exercise}

% \begin{svgraybox}
% 	Como $M\cap Z(G)$ es normal en $G$, la minimalidad de $M$ implica que hay
% 	dos posibilidades: $M\cap Z(G)$ es trivial o bien $M=M\cap Z(G)\subseteq Z(G)$.
% 	Por el teorema~\ref{theorem:Z(nilpotent)}, $M\cap Z(G)\ne 1$.
% \end{svgraybox}

\begin{definition}
\index{Subgroup!Maximal normal}
    Let $G$ be a group. A subgroup $M$ is said to be \textbf{maximal normal} in $G$
    if $M\ne G$ and $M$ is the only proper normal subgroup of $G$ containing $M$. 
\end{definition}

\begin{corollary}
Let $G$ be a non-abelian nilpotent group, and $A$ be a maximal normal and abelian 
subgroup of $G$. Then $A=C_G(A)$.
\end{corollary}

\begin{proof}
Since $A$ is abelian, $A\subseteq C_G(A)$. Assume that $A\ne C_G(A)$.
The centralizer $C_G(A)$ is normal in $G$, since $A$ is normal in $G$ and  
\[
gC_G(A)g^{-1}=C_G(gAg^{-1})=C_G(A)
\]
for all $g\in G$. Let $\pi\colon G\to G/A$ be the canonical map. 
Then $\pi(C_G(A))$ is a non-trivial normal subgroup of $\pi(G)$. Since 
$G$ is nilpotent, $\pi(G)$ is nilpotent. By Hirsch's theorem, 
\[
\pi(C_G(A))\cap Z(\pi(G))\ne\{1\}.
\]
Let $x\in C_G(A)\setminus A$ be such that $\pi(x)$ is central in $\pi(G)$.  Then 
$\langle A,x\rangle$ is abelian, as $x\in C_G(A)$. Moreover,  $\langle
A,x\rangle$ is normal in $G$, as $A$ is normal in $G$ and 
$gxg^{-1}x^{-1}\in A$ for all  $g\in G$ (because $\pi(x)$ is central). Hence
	$gxg^{-1}\in \langle A,x\rangle$ and therefore $A\subsetneq \langle
	A,x\rangle\subsetneq G$, a contradiction.
\end{proof}

\begin{theorem}
Let $G$ be a nilpotent group. The following statements hold: 
\begin{enumerate}
\item Every minimal normal subgroup of $G$ has prime order and is central. 
\item Every maximal subgroup of $G$ is normal of prime index and contains $[G,G]$. 
\end{enumerate}
\end{theorem}

\begin{proof}\
\begin{enumerate}
    \item Let $N$ be a minimal normal subgroup of $G$. Since 
        $N\cap Z(G)\ne\{1\}$ by Hirsch's theorem, $N\cap Z(G)$ is a normal subgroup of 
        $G$ contained in $N$. Then $N=N\cap Z(G)\subseteq
	Z(G)$ by the minimality of $N$. In particular, $N$ is abelian. Since every subgroup of 
         $N$ is normal in $G$, $N$ is simple. Hence $N\simeq
	C_p$ for some prime number $p$.
 \item  If $M$ is a maximal subgroup, then $M$
is normal in $G$ by the normalizer condition (Lemma \ref{lem:normalizadora}). By the maximality
of $M$, the quotient $G/M$ contains no proper non-trivial subgroups. Thus 
$G/M\simeq C_p$ for some prime $p$. Since 
	$G/M$ is abelian, $[G,G]\subseteq M$. \qedhere 
\end{enumerate}
\end{proof}

The previous theorem does not prove the existence of maximal subgroups. For example, 
$\Q$ is a nilpotent group (as it is abelian) 
that contains no maximal subgroups. 

\begin{proposition}
\label{pro:g^n}
Let $G$ be a nilpotent group and $H$ be a subgroup with $(G:H)=n$. If 
$g\in G$, then $g^n\in H$.
\end{proposition}

\begin{proof}
We proceed by induction on $n$. The case $n=1$ is trivial. The case  
$n=2$ follows from the normality of  $H$. So assume the result 
holds for all groups of index $<n$. Let $H$ be a subgroup of $G$ such that $(G:H)=n$. 
Let $H_0=H$ and $H_{i+1}=N_G(H_i)$ for all $i\geq0$. By definition, $H_{i}$ is normal in 
$H_{i+1}$. Since $G$ is nilpotent, $H_i\ne G$
implies that $H_i\subsetneq H_{i+1}$ by the normalizer condition. 
Since $(G:H)$ is finite, there exists $k$ such that  $H_k=G$. Since
$(H_j:H_{j-1})<n$ for all $j$, the inductive hypothesis implies that 
	$x^{(H_j:H_{j-1})}\in H_{j-1}$ for all $x\in H_j$ and all $j$. Hence 
	\[
		g^{(G:H)}=g^{(H_k:H_{k-1})(H_{k-1}:H_{k-2})\cdots (H_1:H_0)}\in H.\qedhere 
	\]
% 	El resultado es obvio en el caso en que $H$ sea un subgrupo normal.  Sea
% 	$H_0=H$ y $H_{i+1}=N_G(H_i)$ para $i\geq0$. Por definición, $H_{i}$ es
% 	normal en $H_{i+1}$ y además, como $G$ es nilpotent, si $H_i\ne G$
% 	entonces $H_i\subsetneq H_{i+1}$ por la condición normalizadora. 
% 	Como $(G:H)$ es finito, existe $k$ tal que $H_k=G$. Veamos que 
% 	\[
% 		g^{(G:H)}=g^{(H_k:H_{k-1})(H_{k-1}:H_{k-2})\cdots (H_1:H_0)}\in H.
% 	\]
% 	Observemos que $g^{(H_k:H_{k-1})}\in H_{k-1}$ pues $H_{k-1}$ es normal en $H_k=G$, y que, como 
% 	$g^{(H_k:H_{k-1})}\in H_k$, entonces 
% 	\[
% 	g^{(H_k:H_{k-2})}=g^{(H_k:H_{k-1})(H_{k-1}:H_{k-2})}=\left(g^{(H_k:H_{k-1})}\right)^{(H_{k-1}:H_{k-2})}\in H_{k-2}
% 	\]
% 	pues $H_{k-2}$ es normal en $H_{k-1}$. Al repetir este argumento, $g^{(G:H)}\in H$. 
\end{proof}

\begin{exercise}
\label{xca:g^n}
    Does the previous proposition hold for non-nilpotent groups? 
\end{exercise}

% \begin{sol}{xca:g^n}
% No. Take for example 
% $G=\Sym_3$ and $H=\{\id,(12)\}$. Then $H$ is index three in $G$ and 
% for  
% $g=(13)$ one has $g^{3}=(13)\not\in H$.
% \end{sol}

The following lemma is useful for performing induction 
on the nilpotency index of nilpotent groups. 

\begin{lemma}
\label{lem:a[GG]}
Let $G$ be a nilpotent group of class $c\geq2$. If $x\in G$, then 
the subgroup 
$\langle x,[G,G]\rangle$ is nilpotent of class $<c$.
\end{lemma}

\begin{proof}
Let $H=\langle x,[G,G]\rangle$.  If $x\in [G,G]$, then there is nothing to prove. 
So assume that $x\not\in [G,G]$. Note that 
	\[
		H=\{x^nc:n\in\Z,c\in [G,G]\},
	\]
as $[G,G]$ is normal in $G$. We need to show that
$[H,H]\subseteq\gamma_3(G)$. Let $h=x^nc,k=x^md\in H$
be such that $c,d\in [G,G]$. 
Since 
	\[
	[h,x^m]=[x^n,[c,x^m]][c,x^m]\in\gamma_4(G)\gamma_3(G)\subseteq\gamma_3(G),
	\]
then  
	\begin{align*}
		[h,k]&=[h,x^m][x^m,[h,d]][h,d]\\
			&=[x^n,[c,x^m]][c,x^m][x^m,[h,d]][h,d]\in\gamma_3(G).\qedhere
	\end{align*}
%	Como
%	$[G,G]=\gamma_2(G)\subseteq\zeta_{c-1}(G)$, al usar la definición de $H$ se
%	obtiene que $[G,G]\subseteq \zeta_{c-1}(G)\cap H\subseteq\zeta_{c-1}(H)$.
%	Luego $H/\zeta_{c-1}(H)$ es cíclico generado por $a\zeta_{c-1}(H)$. Si
%	$\zeta_{c-1}(H)=H$, no hay nada para demostrar. En caso contrario,
%	$Z(H)\subseteq\zeta_{c-1}(H)$ implicaría que $H$ es abelian y luego $H$ es
%	nilpotent de clase uno.
\end{proof}

\begin{example}
Let $G=\D_{8}=\langle r,s:r^{8}=s^2=1,srs=r^{-1}\rangle$ the dihedral group of order 16.
Then $G$ is nilpotent of class three and 
$[G,G]=\{1,r^2,r^4,r^6\}\simeq C_4$. The subgroup $\langle
s,[G,G]\rangle\simeq\D_4$ is nilpotent of class two. 
\begin{lstlisting}
gap> G := DihedralGroup(IsPermGroup,16);;
gap> gens := GeneratorsOfGroup(G);;
gap> r := gens[1];;
gap> s := gens[2];;
gap> D := DerivedSubgroup(G);;
gap> S := Subgroup(G, Concatenation(Elements(D), [s]));;
gap> StructureDescription(S);
"D8"
gap> NilpotencyClassOfGroup(G);
3
gap> NilpotencyClassOfGroup(S);
2
	\end{lstlisting}
\end{example}

Let us discuss a concrete application of Lemma \ref{lem:a[GG]}.

\begin{theorem}
\label{thm:T(nilpotent)}
If $G$ is a nilpotent group, then 
\[
T(G)=\{g\in G:g^n=1\text{ for some $n\geq1$}\}
\]
is a subgroup of $G$. 
\end{theorem}

\begin{proof}
We proceed by induction on the nilpotency class of $G$. 
Let $a,b\in T(G)$ and 
\[
	A=\langle a,[G,G]\rangle,\quad
	B=\langle b,[G,G]\rangle.
\]
Since  $A$ and $B$ are nilpotent of class $<c$ by the previous lemma, 
the inductive hypothesis 
implies that $T(A)$ is a subgroup of $A$ and $T(B)$ is a subgroup of $B$.
Since $T(A)$ is characteristic in $A$ and $A$ is normal in $G$, $T(A)$ is normal in $G$. 
Similarly, $T(B)$ is normal in $G$. 

We claim that every element of 
$T(A)T(B)$ has finite order. If 
$x\in T(A)T(B)$, say $x=a_1b_1$ with
$a_1$ of order $m$, then $x$ has finite order, as 
\begin{align*}
x^m=(a_1b_1)^m&=%(a_1b_1a_1^{-1})(a_1^2b_1a_1^{-2})\cdots (a_1^{m-1} b_1 a_1^{-m+1}b_1)\\
(a_1b_1a_1^{-1})(a_1^2b_1a_1^{-2})\cdots (a_1^{m-1} b_1 a_1^{-m+1})b_1\in T(B).
\end{align*}

To see clearly what is what we did, let us work out a concrete 
example, say $m=3$. In this case, we obtain the following formula:
\begin{align*}
(a_1b_1)^3&=(a_1b_1)(a_1b_1)(a_1b_1)\\
&=(a_1b_1a_1^{-1})(a_1^2b_1a_1^{-2})a_1^3b_1
=(a_1b_1a_1^{-1})(a_1^2b_1a_1^{-2})b_1,
\end{align*}
as $a_1^3=1$.

With this trick, we prove that $ab$ has finite order. Hence $T(G)$ 
is a subgroup of $G$. 
\end{proof}

Another application:

\begin{theorem}
\label{thm:a=b}
Let $G$ be a torsion-free nilpotent group and $a,b\in G$. If there exists 
$n\ne 0$ such that $a^n=b^n$, then $a=b$.
\end{theorem}

\begin{proof}
We proceed by induction on the nilpotency order $c$ of $G$. 
The result clearly holds for abelian groups. Assume that $G$ is nilpotent of class $c\geq2$. 
Since $\langle a,[G,G]\rangle$ is a nilpotent subgroup of $G$ of class $<c$
and $bab^{-1}=[b,a]a\in \langle
a,[G,G]\rangle$, the inductive hypothesis implies that 
$ba=ab$, as 
\[
	a^n=(bab^{-1})^n=b^n.
\]
Thus $(ab^{-1})^n=a^nb^{-n}=1$. Since $G$ has no torsion, we conclude that 
$a=b$.
\end{proof}

\begin{corollary}
Let $G$ be a torsion-free nilpotent group. If $x,y\in G$ are such that 
$x^ny^m=y^mx^n$ for some $n,m\ne 0$, then $xy=yx$.
\end{corollary}

\begin{proof}
Let $a=x$ and $b=y^nxy^{-n}$. Since $a^m=b^m$, the previous theorem implies that 
$a=b$. Thus $xy^n=y^nx$. Apply the previous theorem again, this time with 
$a=y$ and $b=xyx^{-1}$. Then we conclude that 
$xy=yx$. 
\end{proof}

Before proving another theorem, we recall a basic 
lemma about finitely generated groups. 

\begin{lemma}
\label{lem:fg}
Let $G$ be a finitely generated group and 
$H$ a finite-index subgroup. 
Then $H$ is finitely generated. 
\end{lemma}

\begin{proof}
Assume that $G$ is generated by $\{g_1,\dots,g_m\}$. Without loss of generality, we may assume that 
for each $i$ there exists $k$ such that $g_i^{-1}=g_k$. 
	
Let $\{1=t_1,\dots,t_n\}$ be a right transversal of $H$ in $G$, so $G$ decomposes as 
\[
G=\bigcup_{i=1}^n Ht_i\quad\text{(disjoint union).}
\]
For $i\in\{1,\dots,n\}$ 
and $j\in\{1,\dots,m\}$, write 
\[
t_ig_j=h(i,j)t_{k(i,j)}.
\]
We claim that $H$ is generated by the $h(i,j)$. Let $x\in H$. Then  
	\begin{align*}
	x &=g_{i_1}\cdots g_{i_s}\\
	&= (t_1g_{i_1})g_{i_2}\cdots g_{i_s}\\
	&= h(1,i_1)t_{k_1}g_{i_2}\cdots g_{i_s}\\
	&= h(1,i_1)h(k_1,i_2)t_{k_2}g_{i_3}\cdots g_{i_s}\\
	&= h(1,i_1)h(k_1,i_2)\cdots h(k_{s-1},i_s)t_{k_s},
	\end{align*}
where $k_1,\dots,k_{s-1}\in\{1,\dots,n\}$. Since $t_{k_s}\in H$ (because $x\in H$), 
$t_{k_s}=1\in H$. Hence  $x$ is generated by the $h(i,j)$.
\end{proof}

Now the theorem:

\begin{theorem}
\label{thm:T(G)finito}
Let $G$ be a finitely generated torsion group that is nilpotent. 
Then $G$ is finite 
\end{theorem}

\begin{proof}
We proceed by induction on the nilpotency class $c$ of $G$. The case
$c=1$ is true, as $G$ is abelian. So assume the result holds for 
groups of class $c\geq1$. Since $G/[G,G]$ is a finitely generated abelian torsion group, 
it is finite. Thus $[G,G]$ has finite index in $G$. By Lemma~\ref{lem:fg}, 
$[G,G]$ is finitely generated. Moreover, $[G,G]$ is a finitely generated 
nilpotent torsion group of class  $<c$.  
By the inductive hypothesis, 
$[G,G]$ is a finite group. Since $[G,G]$ and $G/[G,G]$ are finite, 
$G$ is finite. % of order $|[G,G]|(G:[G,G])$.
\end{proof}

%\begin{lemma}
%	\label{lemma:kgenerators}
%	Sea $G$ un grupo y sea $G=G_0\supseteq G_1\supseteq\cdots\supseteq G_k=1$
%	una sucesión de subgrupos de $G$ tal que cada $G_{i+1}$ es normal en $G_i$
%	y cada $G_{i}/G_{i+1}$ es cíclico. Todo subgrupo de $G$ es finitamente
%	generado por $k$ elementos.
%\end{lemma}
%
%\begin{proof}
%	Procedemos por inducción en $k$. Supongamos primero que $k=1$. Entonces
%	$G\simeq G_0/G_1$ es cíclico y luego todo subgrupo de $G$ está generado por
%	un elemento. Supongamos ahora que el resultado es válido para $k\geq1$. Sea
%	$H$ un subgrupo de $G$, sea $N=G_{1}$ y sea $\pi\colon G\to G/N$ el
%	morfismo canónico. El grupo 
%	\[
%		\pi(H)\simeq H/H\cap N
%	\]
%	es cíclico pues un un subgrupo del grupo cíclico $G_k/G_{k-1}=G/N$. Como
%	existe $h\in H$ tal que $\pi(H)$ está generado por $\pi(h)$, se concluye que 
%	$H=\langle \pi(h),H\cap N\rangle$. Por hipótesis
%	inductiva, $H\cap N$ está generado por $k-1$ elementos y luego $H$ está
%	generado por $k$ elementos.
%\end{proof}
%
%\begin{theorem}
%	Sea $G$ un grupo nilpotent y finitamente generado. Entonces $T(G)$ es
%	finito.
%\end{theorem}
%
%%%% aca hay que hacer producto tensorial para construir una serie con factores cíclicos
%%%% ver libro de Khukhro
%%%% Nilpotent Groups and Their Automorphisms
%\begin{proof}
%	Sabemos por el teorema~\ref{theorem:} que existe una sucesión
%	$G=G_0\supseteq G_1\supseteq\cdots\supseteq G_k=G$ de subgrupos normales de
%	$G$ con factores cíclicos. 
%\end{proof}



