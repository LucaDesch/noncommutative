\section{28/02/2024}

\subsection{Frattini subgroup}

\begin{definition}
\index{Frattini subgroup}
Let $G$ be a group. If $G$ has maximal subgroups, 
the \textbf{Frattini subgroup} is the intersection 
$\Phi(G)$ of all the maximal subgroups of $G$. 
Otherwise, 
$\Phi(G)=G$. 
\end{definition}

\begin{exercise}
\label{xca:Phi(G)char}
Prove that $\Phi(G)$ is a characteristic subgroup of $G$. 
\end{exercise}

\begin{example}
Sea $G=\Sym_3$. The maximal subgroups of $G$ are 
\[
M_1=\langle (123)\rangle,
\quad
M_2=\langle (12)\rangle,
\quad
M_3=\langle (23)\rangle,
\quad
M_4=\langle (13)\rangle.
\]
Hence $\Phi(G)=\{1\}$. 
\end{example}

\begin{example}
Let $G=\langle g\rangle\simeq C_{12}$. The subgroups of $G$ are 
\[
\{1\},\quad
\langle g^6\rangle\simeq C_2,\quad
\langle g^4\rangle\simeq C_3,\quad
\langle g^3\rangle\simeq C_4,\quad
\langle g^2\rangle\simeq C_6,\quad
G.
\]
Let us draw a picture:
\[\begin{tikzcd}
	& {C_{12}} \\
	{C_4} && {C_6} \\
	& {C_2} && {C_3} \\
	&& {\{1\}}
	\arrow[no head, from=4-3, to=3-2]
	\arrow[no head, from=4-3, to=3-4]
	\arrow[no head, from=3-4, to=2-3]
	\arrow[no head, from=2-3, to=1-2]
	\arrow[no head, from=3-2, to=2-1]
	\arrow[no head, from=2-1, to=1-2]
	\arrow[no head, from=3-2, to=2-3]
\end{tikzcd}\]
The maximal subgroups of $G$ are 
$\langle g^3\rangle\simeq C_4$ and $\langle
g^2\rangle\simeq C_6$. Hence $\Phi(G)=\langle g^3\rangle\cap \langle
g^2\rangle=\langle g^6\rangle\simeq C_2$. 
Let us see how to do this calculation with the computer:
\begin{lstlisting}
gap> G = CyclicGroup(12);;
gap> StructureDescription(FrattiniSubgroup(G));
"C2"
\end{lstlisting} 
\end{example}

\begin{lemma}[Dedekind]
\label{lem:Dedekind}
\index{Dedekind's!lemma}
Let $H$, $K$ and $L$ be subgroups of $G$ 
such that $H\subseteq L\subseteq G$. Then 
$HK\cap L=H(K\cap L)$.
\end{lemma}

\begin{proof}
One inclusion is trivial. Let us prove then that 
$HK\cap L\subseteq H(K\cap L)$. If 
$x=hk\in HK\cap L$ with $x\in L$, $h\in H$ and $k\in K$, then 
$k=h^{-1}x\in L\cap K$, as $H\subseteq L$. Thus $x=hk\in H(L\cap
	K)$.
\end{proof}

\begin{lemma}
\label{lem:G=HPhi(G)}
Let $G$ be a finite group and $H$ be a subgroup of $G$ such that 
$G=H\Phi(G)$. Then $H=G$.
\end{lemma}

\begin{proof}
If $H\ne G$, let $M$ be a maximal subgroup of $G$ such that 
$H\subseteq M$. Since $\Phi(G)\subseteq M$, $G=H\Phi(G)\subseteq M$, a 
contradiction. 
\end{proof}

\begin{proposition}
\label{pro:phi(N)phi(G)}
Let $N$ be a normal subgroup of a finite group $G$. Then 
$\Phi(N)\subseteq\Phi(G)$.
\end{proposition}

\begin{proof}
Since $\Phi(N)$ is characteristic in $N$ and $N$ 
is normal in $G$, $\Phi(N)$ is normal in $G$. 
If there exists a maximal subgroup $M$ such that 
$\Phi(N)\not\subseteq M$, then $\Phi(N)M=G$. (This happens
because, otherwise, $M=\Phi(N)M\supseteq\Phi(N)$.) By Dedekind's lemma (with  $H=\Phi(N)$, $K=M$ and $L=N$), 
\[
N=G\cap N=(\Phi(N)M)\cap N=\Phi(N)(M\cap N).
\]
By Lemma \ref{lem:G=HPhi(G)} (with $G=N$ and $H=M\cap N$), 
$\Phi(N)\subseteq N\subseteq M$, a contradiction. 
Hence every maximal subgroup of $G$ contains $\Phi(N)$ and therefore 
$\Phi(G)\supseteq\Phi(N)$. 
\end{proof}

The following proposition states that the 
elements of the Frattini subgroup are the \textbf{non-generators} of 
the group. 

\begin{proposition}
	\label{pro:nongenerators}
	Let $G$ be a finite group. Then 
 	\[
	\Phi(G)=\{x\in G:\text{if $G=\langle x,Y\rangle$ for some $Y\subseteq G$, then $G=\langle Y\rangle$}\}.
	\]
\end{proposition}

\begin{proof}
We first prove $\supseteq$. Let $x\in G$. If $M$ is a maximal subgroup of $G$ such that $x\not\in M$, then, since $G=\langle
	x,M\rangle$, we obtain that $G=\langle M\rangle=M$, a contradiction. Thus $x\in M$ for all maximal subgroup $M$ of $G$. Hence 
 $x\in \Phi(G)$. 

We now prove $\subseteq$. Let $x\in\Phi(G)$ be such that $G=\langle
	x,Y\rangle$ for some subset $Y$ of $G$. If $G\ne \langle Y\rangle$,
	there exists a maximal subgroup $M$ such that $\langle Y\rangle\subseteq M$. Since
	$x\in M$, $G=\langle x,Y\rangle\subseteq M$, a contradiction. 
\end{proof}

\begin{example}
For a prime number $p$, let $G$ be an elementary $p$-group, that is 
$G\simeq C_p^m$ for some $m\geq1$. Assume that 
	$G=\langle x_1\rangle\times\cdots\times\langle x_m\rangle$ with $\langle x_j\rangle\simeq C_p$.  
	We claim that $\Phi(G)$ is trivial. 
	For $j\in\{1,\dots,m\}$, let $n_j\in\{1,\dots,p-1\}$. Since 
	\[
	\{x_1,\dots,x_{j-1},x_j^{n_j},x_{j+1},\dots,x_m\}
	\]
	generates $G$ and $\{x_1,\dots,x_{j-1},x_{j+1},\dots,x_m\}$ does not, 
	$x_j^{n_j}\not\in\Phi(G)$ by Proposition \ref{pro:nongenerators}. 
	Hence $\Phi(G)=\{1\}$.
\end{example}

\begin{theorem}[Frattini]
\label{thm:Frattini}
\index{Frattini's!theorem}
Let $G$ be a finite group. Then $\Phi(G)$ is nilpotent.
\end{theorem}

\begin{proof}
Let $P\in\Syl_p(\Phi(G))$ for some prime number $p$. Since $\Phi(G)$ is normal in 
$G$, Lemma~\ref{lem:Frattini_argument} (Frattini's argument) implies that 
$G=\Phi(G)N_G(P)$. By Lemma~\ref{lem:G=HPhi(G)},
$G=N_G(P)$. Since every Sylow subgroup of $\Phi(G)$ is normal in $G$,
$\Phi(G)$ is nilpotent. 
\end{proof}

\begin{exercise}
\label{xca:G/M}
Let $G$ be a group and $M$ be a normal subgroup of $G$. Prove that if  
$M$ is maximal, then 
$G/M$ is cyclic of prime order. 
\end{exercise}

% \begin{svgraybox}
% 	Por el teorema de la correspondencia, $G/M$ no tiene subgrupos no trivales.
% 	Luego $G/M\simeq C_p$ para algún primo $p$.
% \end{svgraybox}

\begin{theorem}[Gasch\"utz]
	\label{thm:Gaschutz}
	\index{Gasch\"utz'!theorem}
	If $G$ is a finite group, then 
	\[
	[G,G]\cap Z(G)\subseteq\Phi(G).
	\]
\end{theorem}

\begin{proof}
Let $D=[G,G]\cap Z(G)$. Assume that $D$ is not contained in $\Phi(G)$.
Since $\Phi(G)$ is contained in every maximal subgroup of $G$, 
there is a maximal subgroup $M$ of $G$ not containing $D$. Then
$G=MD$. Since $D\subseteq Z(G)$, $M$ is normal in $G$, as 
	$g=md\in G=MD$ implies 
	\[
		gMg^{-1}=(md)Md^{-1}m^{-1}=mMm^{-1}=M.
	\]
	Since $G/M$ is cyclic of prime order, 
	$G/M$ is, in particular, abelian and hence $[G,G]\subseteq M$. Therefore 
	$D\subseteq [G,G]\subseteq M$, a contradiction.
\end{proof}

\begin{lemma}
\label{lem:N_G(H)=H}
Let $G$ be a finite group and $P\in\Syl_p(G)$. If $H$ is a subgroup of $G$ such that
$N_G(P)\subseteq H$, then $N_G(H)=H$.
\end{lemma}

\begin{proof}
Let $x\in N_G(H)$. Since $P\in\Syl_p(H)$ and $Q=xPx^{-1}\in\Syl_p(H)$, the second Sylow's theorem 
implies that there exists 
$h\in H$ such that $hQh^{-1}=(hx)P(hx)^{-1}=P$. Then $hx\in
N_G(P)\subseteq H$ and hence $x\in H$. 
\end{proof}

\begin{theorem}[Wielandt]
\label{thm:Wielandt}
\index{Wielandt's!theorem}
A finite group $G$ is nilpotent if and only if 
$[G,G]\subseteq\Phi(G)$.
\end{theorem}

\begin{proof}
Assume that $[G,G]\subseteq\Phi(G)$. Let $P\in\Syl_p(G)$. If $N_G(P)\ne
G$, then $N_G(P)\subseteq M$ for some maximal subgroup $M$ of $G$. If 
$g\in G$ and $m\in M$, then, since 
\[
	gmg^{-1}m^{-1}=[g,m]\in [G,G]\subseteq\Phi(G)\subseteq M,
\]
$M$ is normal in $G$. Furthermore $N_G(P)\subseteq M$. 
By Lemma~\ref{lem:N_G(H)=H},
\[
G=N_G(M)=M,
\]
a contradiction.
Thus $N_G(P)=G$ and every Sylow subgroup of $G$ si normal in $G$. Therefore 
$G$ is nilpotent. 

Conversely, assume that $G$ is nilpotent. Let $M$ be a maximal subgroup of $G$.
Since $M$ is normal in $G$ and maximal, $G/M$ has no proper non-trivial subgroups. 
Then $G/M\simeq C_p$ for some prime number $p$. In particular, $G/M$ is abelian
and $[G,G]\subseteq M$. Since $[G,G]$ is contained in every maximal subgroup of $G$, 
$[G,G]\subseteq\Phi(G)$.
\end{proof}

\begin{theorem}
\label{the:G/phi(G)}
A finite group $G$ is nilpotent if and only if 
$G/\Phi(G)$ is nilpotent. 
\end{theorem}

%%% TODO: la demostración no está bien explicada!
\begin{proof}
If $G$ is nilpotent, then $G/\Phi(G)$ is nilpotent. Conversely, assume that 
$G/\Phi(G)$ is nilpotent. Let $P\in\Syl_p(G)$. Since 
$\Phi(G)P/\Phi(G)\in\Syl_p(G/\Phi(G))$ and $G/\Phi(G)$ is nilpotent,
$\Phi(G)P/\Phi(G)$ is a normal subgroup of $G/\Phi(G)$. By the correspondence theorem, 
$\Phi(G)P$ is a normal subgroup of $G$.
Since $P\in\Syl_p(\Phi(G)P)$, Frattini's argument 
(Lemma~\ref{lem:Frattini_argument}) implies that 
\[
G=\Phi(G)PN_G(P)=\Phi(G)N_G(P), 
\]
as $P\subseteq N_G(P)$. Thus $G=N_G(P)$ by Lemma~\ref{lem:G=HPhi(G)}). Hence 
$P$ is normal in $G$ and therefore $G$ is nilpotent. 
\end{proof}

\begin{theorem}[Hall]
\index{Hall's!theorem}
\label{thm:Hall_nilpotente}
Let $G$ be a finite group with a normal subgroup $N$. If both $N$ and 
$G/[N,N]$ are nilpotent, then $G$ is nilpotent.
\end{theorem}

\begin{proof}
Since $N$ is nilpotent, $[N,N]\subseteq\Phi(N)$ by 
Wielandt's theorem~\ref{thm:Wielandt}. 
By Proposition~\ref{pro:phi(N)phi(G)},
$[N,N]\subseteq\Phi(N)\subseteq\Phi(G)$. 
By the universal property, there exists a surjective group homomorphism 
$G/[N,N]\to G/\Phi(G)$ such that the diagram 
    \[
    \begin{tikzcd}
	G & {G/\Phi(G)} \\
	{G/[N,N]}
	\arrow[from=1-1, to=1-2]
	\arrow[from=1-1, to=2-1]
	\arrow[dashed, from=2-1, to=1-2]
    \end{tikzcd}
    \]
%     \[
% 	\xymatrix{
% 	G
% 	\ar[d]
% 	\ar[r]
% 	& G/\Phi(G)
% 	\\
% 	G/[N,N]\ar@{-->}[ur]
% 	}
%    \xymatrix{ & P\ar[d]^f\ar@{-->}[ld]_h\\ M\ar[r]^g & N\ar[r] & 0 }
%    \]
is commutative. Since $G/[N,N]$ is nilpotent, $G/\Phi(G)$ is nilpotent by 
Exercise~\ref{xca:nilpotente}. Thus $G$ is nilpotent by the previous theorem. 
\end{proof}

\begin{definition}
A \textbf{minimal generating set} of a group $G$ is a set 
$X$ of generators of $G$ such that no proper subset of $X$ generates $G$. 
\end{definition}

Note that a minimal generating set does not necessarily have minimal size. 
	
\begin{example}
Let $G=\langle g\rangle\simeq C_6$.  If $a=g^2$ and 
$b=g^3$, then $\{a,b\}$ is a minimal generating set of $G$ that does not have
minimal size, as $G=\langle ab\rangle$.
\end{example}
	
For a prime number $p$, we write $\F_p$ to denote the field of $p$ elements. 

\begin{lemma}
\label{lem:Burnside:minimal}
Let $p$ be a prime number and 
$G$ be a finite $p$-group. Then $G/\Phi(G)$ is a vector space over $\F_p$.
\end{lemma}

\begin{proof}
Let $K$ be a maximal subgroup of $G$. Since $G$ is nilpotent 
(see Proposition~\ref{pro:nilpotent_pgroups}), 
$K$ is normal in $G$ (Exercise~\ref{xca:normalizadora}). 
Thus $G/K\simeq C_p$ because it is a simple $p$-group. 
	
It is enough to prove that $G/\Phi(G)$ is an elementary abelian $p$-group. It is a 
$p$-group because $G$ is a $p$-group. Let $K_1,\dots,K_m$ be the maximal subgroups 
of $G$. If $x\in G$, then $x^p\in K_j$ for all $j\in\{1,\dots,m\}$. Hence 
$x^p\in\Phi(G)=\cap_{j=1}^m K_j$. Moreover,  $G/\Phi(G)$ is abelian, as 
$[G,G]\subseteq \Phi(G)$ because $G$ is nilpotent (Wielandt's theorem~\ref{thm:Wielandt}). 
\end{proof}

\begin{theorem}[Burnside]
\label{thm:Burnside:basis}
Let $p$ be a prime number and $G$ a finite $p$-group. If $X$ is a minimal 
generating set of $G$, then $|X|=\dim G/\Phi(G)$. 
\end{theorem}

\begin{proof}
By Lemma \ref{lem:Burnside:minimal}, $G/\Phi(G)$ is a vector space over $\F_p$. 
Let $\pi\colon G\to G/\Phi(G)$ be the canonical map and 
$\{x_1,\dots,x_n\}$ be a minimal generating set of $G$.
We claim that $\{\pi(x_1),\dots,\pi(x_n)\}$ is a linearly independent subset of $G/\Phi(G)$. 
Assume this is not the case. Without loss of generality, let us assume that 
$\pi(x_1)\in\langle \pi(x_2),\dots,\pi(x_n)\rangle$. There exists $y\in
\langle x_2,\dots,x_n\rangle$ such that $x_1y^{-1}\in\Phi(G)$. Since $G$ is generated by 
$\{x_1y^{-1},x_2,\dots,x_n\}$ and $x_1y^{-1}\in\Phi(G)$, Proposition~\ref{pro:nongenerators}
implies that $G$ is generated by 
$\{x_2,\dots,x_n\}$, a contradiction to the minimality of $\{x_1,\dots,x_n\}$. 
Therefore $n=\dim G/\Phi(G)$.
\end{proof}

% TODO: agregar una aplicación (teorema de Hall). Ver Passman permutation groups, 11.7, pag 47

\begin{exercise}
Let $p$ be a prime number and $G$ a finite $p$-group. Prove that if $x\not\in\Phi(G)$, then 
$x$ belongs to some minimal set of generators of $G$. 
\end{exercise}
