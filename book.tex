\RequirePackage{amsmath} 

\documentclass[graybox,envcountsect]{svmono}
%\usepackage{marginnote}
\usepackage[T1]{fontenc}
\usepackage[utf8]{inputenc}
\usepackage{amsmath}
\usepackage[notref,notcite]{showkeys}
\usepackage{anyfontsize}
\usepackage{fancyhdr}
\usepackage{float}
\usepackage{amssymb}
\usepackage{amstext}
\usepackage{mathtools}
\usepackage{xcolor} 
\usepackage{centernot}
\usepackage{listings}
\usepackage{multicol}
\usepackage{mathptmx}
%\let\openbox\relax
\usepackage{newtxtext,newtxmath}
%\usepackage{txfonts}
\usepackage{datetime}
\usepackage{stmaryrd}
\usepackage{tikz-cd}

\usepackage{helvet}
\usepackage{courier}
\usepackage{type1cm}         
\usepackage{makeidx}        
\usepackage{graphicx}        
\usepackage{multicol}        
\usepackage{hyperref} 
\usepackage{colortbl}
\usepackage{chngcntr}





% Table of contents for lectures and topics
\makeatletter
\newcommand\listtopicsname{List of topics}
\newcommand\listoftopics{
    \chapter*{\listtopicsname}\@starttoc{top}}
\makeatother

\makeatletter
\newcommand\listlecturesname{Contents}
\newcommand\listoflectures{
    \chapter*{\listlecturesname}\@starttoc{lec}}
\makeatother

\newcommand{\lecture}[1]{
    \chapter{#1}
    \addcontentsline{lec}{chapter}{Lecture \thechapter}
}

\newcommand{\topic}[1]{
    \section{#1}
    \addcontentsline{top}{chapter}{\S\thesection\quad #1}
}


%\usepackage[small,bf]{caption}

\usepackage{tikz}
%\usetikzlibrary{braids}
	
\usepackage[bottom]{footmisc}

% for QED
\let\proof\relax\let\endproof\relax
\let\openbox\relax
\usepackage{amsthm}

\overfullrule=1mm

%%% for Spanish
% \def\abstractname{Resumen}%
% \def\ackname{Agradecimientos}%
% \def\andname{y}%
% \def\bibname{Referencias}%
% \def\lastandname{, y}%
% \def\appendixname{Apéndice}%
\def\chaptername{Lecture}%
% \def\claimname{Afirmación}%
% \def\conjecturename{Conjetura}%
% \def\contentsname{Contenidos}%
% \def\corollaryname{Corolario}%
% \def\definitionname{Definici\'on}%
% \def\emailname{e-mail}%
% \def\examplename{Ejemplo}%
\def\examplesname{Examples}%
% \def\exercisename{Ejercicio}%
\def\figurename{Figure}%
% \def\forewordname{Foreword}%
% \def\keywordname{{\bf Palabras clave:}}%
% \def\indexname{Índice}%
% \def\lemmaname{Lema}%
% \def\listfigurename{Figuras}%
% \def\listtablename{Tablas}%
% \def\notename{Nota}%
% \def\partname{Parte}%
% \def\prefacename{Prefacio}%
\def\problemname{Open problem}%
% \def\proofname{Demostración}%
% \def\propertyname{Propiedad}%
% \def\propositionname{Proposici\'on}%
% \def\questionname{Pregunta}%
% \def\refname{Referencias}%
% \def\remarkname{Observación}%
% \def\seename{see}%
% \def\solutionname{Solución}%
% \def\tablename{Tabla}%
% \def\theoremname{Teorema}
\def\notationname{Notation}
\def\stepsname{Algorithm}
% \def\conventionname{Convención}

% change numbers 
\let\remark\relax
\let\theorem\relax
\let\lemma\relax
\let\definition\relax
\let\proposition\relax
\let\corollary\relax
\let\exercise\relax
\let\example\relax
\let\conjecture\relax

% Numerar con sección y no resetear al cambiar de capítulo
\counterwithout{section}{chapter}
\counterwithout{theorem}{chapter}
\spnewtheorem{theorem}{\theoremname}[section]{\bfseries}{\itshape}

\renewcommand\thetheorem{\thesection.\arabic{theorem}}
\spnewtheorem{lemma}[theorem]{\lemmaname}{\bfseries}{\itshape}
\spnewtheorem{definition}[theorem]{\definitionname}{\bfseries}{\upshape}
\spnewtheorem{proposition}[theorem]{\propositionname}{\bfseries}{\itshape}
\spnewtheorem{corollary}[theorem]{\corollaryname}{\bfseries}{\itshape}
\spnewtheorem{exercise}[theorem]{\exercisename}{\bfseries}{\upshape}
\spnewtheorem{example}[theorem]{\examplename}{\bfseries}{\upshape}
\spnewtheorem{examples}[theorem]{\examplesname}{\bfseries}{\upshape}
\spnewtheorem{remark}[theorem]{\remarkname}{}{\upshape}
\spnewtheorem{conjecture}[theorem]{\conjecturename}{\bfseries}{\upshape}
\spnewtheorem{notation}[theorem]{\notationname}{\bfseries}{\upshape}
\spnewtheorem{steps}[theorem]{\stepsname}{\bfseries}{\upshape}
\spnewtheorem{convention}[theorem]{\conventionname}{\bfseries}{\upshape}

% Numerar con sección y no resetear al cambiar de capítulo
\counterwithout{section}{chapter}

% No sections in TOC
\setcounter{secnumdepth}{1}
\setcounter{tocdepth}{0}

 \usepackage{titlesec}
 \titleformat{\section}
   {\secsize\secstyle}{\S\thesection.}{1em}{}

% para enumerar
\renewcommand{\labelenumi}{\textbf{\arabic{enumi})}}

\makeindex             

\renewcommand{\I}{\operatorname{I}}
\newcommand{\II}{\operatorname{II}}

\newcommand{\GAP}{\textsf{GAP}}
\newcommand{\FK}{\mathcal{E}}
\newcommand{\ad}[1]{\operatorname{ad}\,#1}

%\newcommand{\N}{\mathbb{N}}
\newcommand{\Q}{\mathbb{Q}}
\newcommand{\Z}{\mathbb{Z}}
\newcommand{\F}{\mathbb{F}}
\newcommand{\R}{\mathbb{R}}
\newcommand{\C}{\mathbb{C}}
\renewcommand{\H}{\mathbb{H}}
\newcommand{\A}{\mathbb{A}}
\newcommand{\K}{\mathbb{K}}
\newcommand{\T}{\mathbb{T}}
\renewcommand{\D}{\mathbb{D}}
\newcommand{\B}{\mathbb{B}}
\newcommand{\Fun}{\operatorname{Fun}}
\newcommand{\mpl}{\operatorname{mpl}}
\newcommand{\cL}{\mathcal{L}}
\newcommand{\cE}{\mathcal{E}}
\newcommand{\cH}{\mathcal{H}}

\newcommand{\GF}{\mathsf{GF}}
\newcommand{\MAX}{\operatorname{MAX}}
\newcommand{\MIN}{\operatorname{MIN}}
\newcommand{\cf}{\operatorname{cf}}
\newcommand{\cl}{\operatorname{cl}}
\newcommand{\cd}{\operatorname{cd}}
\newcommand{\bL}{\mathbf{L}}
\newcommand{\bP}{\mathbf{P}}

\newcommand{\Nil}{\operatorname{Nil}}
\newcommand{\rad}{\operatorname{rad}}
\newcommand{\rank}{\operatorname{rank}}

\newcommand{\Aff}{\mathrm{Aff}}
\newcommand{\Ann}{\operatorname{Ann}}
\newcommand{\Der}{\operatorname{Der}}
\newcommand{\Core}{\operatorname{Core}}
\newcommand{\Soc}{\operatorname{Soc}}
\newcommand{\Fix}{\operatorname{Fix}}
\newcommand{\Rad}{\mathrm{rad}}
\newcommand{\Inn}{\mathrm{Inn}}
\newcommand{\dist}{\mathrm{dist}}
\newcommand{\Out}{\mathrm{Out}}
\newcommand{\Ext}{\mathrm{Ext}}
\newcommand{\Img}{\mathrm{im}}
\newcommand{\Hol}{\operatorname{Hol}}
\newcommand{\Hom}{\operatorname{Hom}}
\newcommand{\Alg}{\operatorname{Alg}}
\newcommand{\Bil}{\operatorname{Bil}}
\newcommand{\op}{\operatorname{op}}
\newcommand{\gr}{\operatorname{gr}}
\newcommand{\Syl}{\mathrm{Syl}}
\newcommand{\id}{\operatorname{id}}
\newcommand{\Aut}{\operatorname{Aut}}
\newcommand{\End}{\operatorname{End}}
\newcommand{\Irr}{\operatorname{Irr}}
\newcommand{\Alt}{\mathbb{A}}
\newcommand{\Sym}{\mathbb{S}}
\newcommand{\lcm}{\mathrm{mcm}}
\newcommand{\diag}{\operatorname{diag}}
\newcommand{\spec}{\operatorname{Spec}}
\newcommand{\supp}{\operatorname{supp}}
\newcommand{\trace}{\operatorname{trace}}
\newcommand{\sgn}{\operatorname{sign}}
\newcommand{\ch}{\operatorname{char}}

\newcommand{\inner}{\operatorname{inn}}
\newcommand{\ext}{\operatorname{ext}}
\newcommand{\im}{\operatorname{im}}
\newcommand{\Ret}{\operatorname{Ret}}

\newcommand{\GL}{\mathbf{GL}}
\newcommand{\SL}{\mathbf{SL}}
\newcommand{\PSL}{\mathbf{PSL}}
\newcommand{\PGL}{\mathbf{PGL}}

\newcommand{\legendre}[2]{\left(\frac{#1}{#2}\right)}

\newcommand{\Char}{\operatorname{Char}}

% multiset
\def\multiset#1#2{\ensuremath{\left(\kern-.3em\left(\genfrac{}{}{0pt}{}{#1}{#2}\right)\kern-.3em\right)}}

% column vector
\newcount\colveccount
\newcommand*\colvec[1]{
\global\colveccount#1
\begin{pmatrix}
	\colvecnext
	}
	\def\colvecnext#1{
	#1
	\global\advance\colveccount-1
	\ifnum\colveccount>0
	\\
	\expandafter\colvecnext
	\else
\end{pmatrix}
\fi
}


% numero como secciones
\renewcommand{\thesection}{\arabic{section}}
%\renewcommand{\thesubsection}{\Alph{section}}

% To remove Springer from the title page
\usepackage{etoolbox}
\makeatletter
\patchcmd{\@maketitle}{{\Large Springer\par}}{}{}{}
\def\ps@bchap{%
  \let\@oddhead\@empty\let\@evenhead\@empty
  \def\@oddfoot{\reset@font\small\hfil\thepage\hfil}%
  \let\@evenfoot\@oddfoot
}

% Heading 
\def\ps@headings{%
  \let\@mkboth\markboth
  \def\@oddfoot{\reset@font\small\hfil\thepage\hfil}%
  \let\@evenfoot\@oddfoot
  \def\@evenhead{\runheadsize\runheadstyle\hfil\leftmark}%
  \def\@oddhead{\runheadsize\runheadstyle\rightmark\hfil}%
  \def\chaptermark##1{%
    \markboth{%
      {\if@chapnum Lecture \thechapter\thechapterend\fi ##1}%
    }{%
      {\if@chapnum Lecture \thechapter\thechapterend\fi ##1}}%
    }%
    \def\sectionmark##1{\markright{{\ifnum\c@secnumdepth>\z@
     \S\thesection\seccounterend\hskip\betweenumberspace\fi ##1}}}
}
\makeatother
\pagestyle{headings}

\begin{document}
 
\lstset{language=GAP,
  showstringspaces=false,
  xleftmargin=0.0cm,
  xrightmargin=0.0cm,
  basicstyle=\small\ttfamily,
  frame=single,
  framerule=0pt,
}

\author{Leandro Vendramin}
\title{Non-commutative algebra}
\subtitle{Notes}
\maketitle

\frontmatter

%\include{dedic}
\preface

The notes correspond to the master  
course \emph{Non-commutative Algebra} of the 
Vrije Universiteit Brussel, 
Faculty of Sciences, 
Department of Mathematics and Data Sciences. The course
is divided into thirteen two-hours lectures. 

Most of the material is based on standard 
results on group algebras covered in the VUB course \emph{Associative Algebras}. Lecture  
notes for this course are freely available at 
\url{https://github.com/vendramin/associative}. 
Basic texts on group algebras are Lam's book \cite{MR1125071}
and Passman's book \cite{MR798076}.
% are for example \cite{MR1645586}\dots

% As usual, we also mention a set of great expository papers by 
% Keith Conrad available at 
% \url{https://kconrad.math.uconn.edu/blurbs/}. 
% The notes are extremely well-written and are useful at  
% at every stage of a mathematical career. 
 
Thanks go to Arjen Elbert Dujardin and Robynn Corveleyn. 
%Arne van Antwerpen, Luca Descheemaeker, Lucas Simons
% %and Geoffrey Jassens. 

This version 
was compiled on \today~at~\currenttime.

\bigskip
\begin{flushright}
Leandro Vendramin\\Brussels, Belgium\par
\end{flushright}


\tableofcontents 
\listoftopics

\mainmatter

\section{15/02/2024}

\subsection{Solvable groups}

\index{Subgroup!characteristic}
A subgroup $H$ of $G$ is said to be \textbf{characteristic} if 
$f(H)\subseteq H$ for all $f\in\Aut(G)$. 
The center and the commutator subgroup are
characteristic subgroups. 
Every characteristic subgroup is normal, 
as the maps $x\mapsto gxg^{-1}$ are automorphisms. 

\begin{exercise}
    Prove that if $H$ is characteristic in $K$ and $K$ is normal in $G$, then
    $H$ is normal in $G$. 
\end{exercise}

For a group $G$ and $x,y,z\in G$, conjugation will be considered as a left action of $G$ on $G$ 
and we will use the following notation: $\prescript{x}{}y=xyx^{-1}$. The commutator between $x$ and $y$ 
will be written as 
\[
[x,y]=xyx^{-1}y^{-1}=(\prescript{x}{}y)y^{-1}.
\]
We will also use the following notation:  
\[
[x,y,z]=[x,[y,z]].
\]

For subgroups $H$ and $K$ of $G$, let 
\[
[H,K]=\langle [h,k]:h\in H,\,k\in K\rangle.
\]
Note that $[X,Y]=[Y,X]$. 
For subgroups $X$, $Y$ and $Z$ of $G$, we write 
\[
[X,Y,Z]=\left[ X,[Y,Z] \right].
\]

\index{Derived series}
For a group $G$, let 
$G^{(0)}=G$ and 
$G^{(i+1)}=[G^{(i)},G^{(i)}]$ for $i\geq0$.
The \textbf{derived series} of $G$ is the sequence
\[
G=G^{(0)}\supseteq G^{(1)}\supseteq G^{(2)}\supseteq\cdots
\]
Each $G^{(i)}$ is a characteristic subgroup of $G$.

\begin{definition}
\index{Group!solvable}
We say that a group 
$G$ is \textbf{solvable} if $G^{(n)}=\{1\}$ for some $n$.  
\end{definition}

\begin{example}
	Abelian groups are solvable. 
\end{example}

\begin{example}
	The group $\SL_2(3)$ is solvable, as the derived series is 
	\[
	\SL_2(3)\supseteq Q_8\supseteq C_4\supseteq \{1\}.
	\]
\end{example}

\begin{example}
	Non-abelian simple groups cannot be solvable. 
\end{example}

\begin{exercise}
	\label{xca:solvable}
	Let $G$ be a group. Prove the following statements:
	\begin{enumerate}
		\item A subgroup $H$ of $G$ is solvable, when $G$ is solvable.
		\item Let $K$ be a normal subgroup of $G$. 
		    Then $G$ is solvable if and only if $K$ and $G/K$ are solvable.
	\end{enumerate}
\end{exercise}

\begin{example}
    For $n\geq5$ the group $\Alt_n$ is simple and non-abelian. Hence it 
    is not solvable. It follows that 
    $\Sym_n$ is not solvable for $n\geq5$. 
\end{example}

\begin{exercise}
\label{xca:pgroups_solvable}
Let $p$ be a prime number. Prove that finite $p$-groups are solvable.
\end{exercise}

\begin{exercise}
\label{xca:Robinson:5.4.1}
    Let $p$, $q$ and $r$ be prime numbers. Prove that groups
    of order $p^\alpha q$, $p^2q^2$ and $pqr$ are solvable. 
\end{exercise}

\begin{exercise}
\label{xca:less60}
    Prove that groups of order $<60$ are solvable. 
\end{exercise}


\begin{theorem}[Burnside]
	\index{Burnside's theorem}
        \label{thm:Burnside}
	Let $p$ be a prime number. If $G$ is a finite group that has 
        a conjugacy class with $p^k>1$ elements, then $G$ 
	is not simple.
\end{theorem}

The easiest way to prove Theorem \ref{thm:Burnside}
is using character theory. 

\begin{theorem}[Burnside]
  \index{Burnside's theorem}
  Let $p$ and $q$ be prime numbers. If $G$ has order $p^aq^b$, then $G$ is solvable.
\end{theorem}

\begin{proof}
	If $G$ is abelian, then it is solvable.
	Suppose now $G$ is non-abelian.
	Let us assume that the theorem is not true. Let $G$ be a group
	of minimal order $p^aq^b$
	that is not solvable. Since $|G|$ is minimal, $G$ is a non-abelian simple group.
	By the previous theorem, 
	$G$ has no conjugacy classes of size $p^k$ nor 
	conjugacy classes of size $q^l$ with $k,l\geq1$. The size
	of every conjugacy class of $G$ is one or divisible by $pq$. 
	Note that, since $G$ is a non-abelian simple group,
	the center of $G$ is trivial.
	Thus there is only one conjugacy class of size one.
	By the class
	equation,
	\[
		|G|=1+\sum_{C:|C|>1}|C|\equiv 1 \bmod pq,
	\]
	where the sum is taken over all conjugacy classes 
	with more than one element, a contradiction.
\end{proof}

A recent generalization of Burnside's theorem
is based on \emph{word maps}. A word map
of a group $G$ is a map 
\[
G^k\to G,\quad 
(x_1,\dots,x_k)\mapsto w(x_1,\dots,x_k)
\]
for some word $w(x_1,\dots,x_k)$ of the free group $F_k$ of rank $k$. 
Some word maps are surjective in certain families of groups. For example, 
Ore's conjecture is precisely the surjectivity of the word map
$(x,y)\mapsto [x,y]=xyx^{-1}y^{-1}$ in every finite non-abelian simple 
group. 

\begin{theorem}[Guralnick--Liebeck--O'Brien--Shalev--Tiep]
    Let $a,b\geq0$, $p$ and $q$ be prime numbers and $N=p^aq^b$. The map 
    $(x,y)\mapsto x^Ny^N$ is surjective in every non-abelian 
    finite simple group. 
\end{theorem}

The proof appears in~\cite{MR3827208}. 

The theorem implies Burnside's theorem. Let $G$ be a group of order
$N=p^aq^b$. Assume that $G$ is not solvable. 
Fix a composition series of $G$. There is a non-abelian factor $S$ 
of order dividing $N$. Since 
$S$ is simple non-abelian and $s^N=1$ for all $s\in S$, 
it follows that the word map
$(x,y)\mapsto x^Ny^N$ has trivial image in $S$, a contradiction 
to the theorem. 

\begin{theorem}[Feit--Thompson]
    \index{Feit--Thompson theorem}
    Groups of odd order are solvable. 
\end{theorem}

The proof of Feit--Thompson theorem is extremely hard. 
It occupies a full volume of the 
\emph{Pacific Journal of Mathematics}~\cite{MR166261}. 
A formal verification of the proof 
(based on the computer software Coq) 
was announced in~\cite{MR3111271}.  This motivates a natural problem: To formally verify 
the classification of finite simple groups.  
Will mathematics move away from depending on just humans to verify proofs? Formal verification with computer-proof assistants 
could become the new standard for rigor in mathematics. 

Back in the day, it was believed that if a certain divisibility 
conjecture is true, 
the proof of Feit--Thompson theorem 
could be simplified. 

\begin{conjecture}[Feit--Thompson]
\index{Feit--Thompson conjecture}
    There are no prime numbers $p$ and $q$ such that
    $\frac {p^{q}-1}{p-1}$ divides $\frac{q^{p} - 1}{q - 1}$. 
\end{conjecture}

The conjecture remains open. However, now we know that 
proving the conjecture will not simplify further
the proof of Feit--Thompson theorem. 

In 2012, Le proved that the conjecture is true for $q=3$, see 
\cite{MR2900154}. 

In~\cite{MR297686} 
Stephens proved that a certain stronger version of the conjecture 
does not hold, as the integers 
$\frac {p^{q}-1}{p-1}$ and $\frac{q^{p} - 1}{q - 1}$ 
could have common factors. In fact, if $p=17$ and $q=3313$, 
then 
\[
\gcd\left(\frac {p^{q}-1}{p-1},\frac{q^{p} - 1}{q - 1}\right)=112643.
\]
Nowadays we can check this easily on almost every desktop computer:
\begin{lstlisting}
gap> Gcd((17^3313-1)/16,(3313^17-1)/3312);
112643
\end{lstlisting}
No other counterexamples have been found of Stephen’s 
stronger version of the conjecture.

\begin{definition}
\index{Subgroup!elementary abelian}
Let $p$ be a prime number. A $p$-group $P$ is said to be 
\textbf{elementary abelian} if $x^p=1$ for all $x\in P$.
\end{definition}

\begin{definition}
\index{Subgroup!minimal normal}
A subgroup $M$ of $G$ is said to be \textbf{minimal normal} if $M\ne\{1\}$
(or $G = \{1\}$),
$M$ is normal in $G$ and the only normal 
subgroup of $G$ properly contained in $M$ is $\{1\}$. 
\end{definition}

\begin{example}
    If a normal subgroup $M$ is minimal (with respect to the inclusion), 
    then it is minimal and normal. However, 
    the converse statement does not hold. For example, the subgroup
    of $\Alt_4$ generated by $(12)(34)$, $(13)(24)$ and $(14)(23)$ is minimal normal in $\Alt_4$ 
    but it is not minimal. 
\end{example}

\begin{exercise}
Prove that every finite group contains a minimal normal subgroup. 
\end{exercise}

\begin{example}
    Let $G=\D_{6}=\langle r,s:r^6=s^2=1,\,srs=r^{-1}\rangle$ the dihedral group of twelve elements. The subgroups
    $S=\langle r^2\rangle$ 
    and $T=\langle r^3\rangle$ are (the only) minimal normal in $G$.
\end{example}

\begin{example}
    Let $G=\SL_2(3)$. The only minimal normal subgroup of $G$ is its center
    $Z(\SL_2(3))\simeq C_2$.
\end{example}


The following lemma will be very useful later. 

\begin{lemma}
\label{lem:minimal_normal}
Let $M$ be a minimal normal subgroup of $G$. If $M$ is solvable and finite, 
then $M$ is an elementary abelian $p$-group for some prime number $p$. 
\end{lemma}

\begin{proof}
Since $M$ is solvable, $[M,M]\subsetneq M$. Moreover, $[M,M]$ is normal in $G$, as
$[M,M]$ is characterisitic in $M$ and $M$ is normal in $G$. Since $M$ is minimal normal, 
$[M,M]=\{1\}$. Hence $M$ is abelian. 
	
Since $M$ is finite, there is a prime number $p$ such that $\{1\}\ne P=\{x\in
M:x^p=1\}\subseteq M$.  Since $P$ is characteristic in $M$, $P$ is normal in 
$G$. By minimality, $P=M$.
\end{proof}

\begin{theorem}
	Let $G$ be a finite non-trivial solvable group. Then 
		every maximal subgroup of $G$ has index $p^\alpha$ for some prime number $p$. 
\end{theorem}

\begin{proof}
	We proceed by induction on $|G|$.
	If $|G|$ is a prime power, the claim is clear. Assume that $|G|\geq6$ and let $M$ be a maximal subgroup of $G$. 
        Let $N$ be a minimal normal subgroup of $G$ and $\pi\colon G\to G/N$ the canonical map. 
	If $N=G$, then $N=G$ is a $p$-group and we are done. Assume then that 
	$N\ne G$. Since $M\subseteq NM\subseteq G$,
	either $M=NM$ or $NM=G$ (by the maximality of $M$).  If 
	$M=NM\supseteq N$, then $\pi(M)$ is a maximal subgroup of $\pi(G)=G/N$. Hence 
	\[
	(G:M)=(\pi(G):\pi(M))
	\]
	is a prime power by the inductive hypothesis. If 
	$NM=G$, then 
	\[
	(G:M)=\frac{|G|}{|M|}=\frac{|NM|}{|M|}=\frac{|N|}{|N\cap M|}
	\]
	is a prime power, because $N$ is a $p$-group by the previous lemma. 
\end{proof}

\begin{exercise}
    Let $G$ be a finite non-trivial solvable group. Prove that 
    there exists a prime number $p$ such that $G$ 
    contains a minimal normal $p$-subgroup. 
\end{exercise}

\begin{example}
	Let $G=\Sym_4$. The $2$-subgroup 
	\[
	K=\{\id,(12)(34),(13)(24),(14)(23)\}\simeq
	C_2\times C_2
	\]
	is minimal normal. Note that $G$ does not have minimal normal $3$-subgroups. 
\end{example}

\begin{theorem}
Let $G$ be a finite non-trivial group. Then $G$ is solvable if and only if 
every non-trivial quotient of $G$ contains an abelian non-trivial normal subgroup. 
\end{theorem}

\begin{proof}
Every quotient of $G$ is solvable and therefore contains an abelian 
minimal normal subgroup. To prove the converse we proceed by induction on $|G|$. Let 
$N$ be a normal abelian subgroup of $G$. If $N=G$, then $G$ is solvable (because it is abelian). If 
$N\ne G$, then $|G/N|<|G|$. Since every quotient of $G/N$ is a quotient of $G$, the group 
$G/N$ satisfies the assumptions of the theorem. Hence $G/N$ is solvable by the inductive hypothesis. 
Now $N$ and $G/N$ are solvable, so is $G$. 
\end{proof}

% Since aplicación del teorema~\ref{theorem:SchurZassenhaus} de
% Schur--Zassenhaus, in el capítulo~\ref{extensiones},
% teorema~\ref{theorem:solvable_maximal}, demostraremos que si $G$ is un
% grupo finite solvable no trivial and $p$ is un primo que divide al orden de
% $G$, existe un subgrupo maximal de índice una potencia de $p$. 
% Otra aplicación del teorema de Schur--Zassenhaus: la teoría de Hall, 
% una generalización de
% la teoría de Sylow para grupos solvables. 

\begin{exercise}
	\label{exercise:solvable}
	Let $G$ be a group. Prove that $G$ is solvable if and only if there is a sequence
	\[
		\{1\}=N_0\subseteq N_1\subseteq\cdots\subseteq N_k=G
	\]
        of normal subgroups such that every quotient 
	$N_i/N_{i-1}$ is abelian.
\end{exercise}

% \begin{svgraybox}
% 	Si existe una sucesión de subgrupos normales $1=N_0\subseteq
% 	N_1\subseteq\cdots\subseteq N_k=G$ tales que cada cociente $N_i/N_{i-1}$ es
% 	abeliano, entonces $[N_i,N_i]\subseteq N_{i-1}$ para cada $i$. Demostremos
% 	por inducción que $G^{(m)}\subseteq N_{n-m}$ para todo $m\leq n$. El caso
% 	$m=0$ is trivial pues $G^{(0)}=G\subseteq N_{n}=G$; si suponemos que el
% 	resultado vale para $m$ entonces, por hipótesis inductiva, 
% 	\[
% 	G^{(m+1)}=[G^{(m)},G^{(m)}]\subseteq [N_{n-m},N_{n-m}]\subseteq N_{n-m-1}.
% 	\]
% 	Luego $G^{(n)}\subseteq N_{0}=1$ and $G$ is solvable.

% 	Supongamos ahora que $G$ is solvable. Entonces existe $n\in\N$ tal que
% 	$G^{(n)}=1$. Since cada $G^{(i)}$ is característico in $G$, in particular
% 	cada $G^{(i)}$ is normal in $G$. Además 
% 	\[
% 		1=G^{(n)}\subseteq G^{(n-1)}\subseteq\cdots\subseteq G^{(0)}=G.
% 	\]
% \end{svgraybox}

\subsection{Hall's theorem}

We start with an extremely simple and useful tool. 

\begin{lemma}[Frattini's argument]
    \label{lem:Frattini_argument}
    \index{Frattini's argument}
    Let $G$ be a finite group and $K$ be a normal subgroup of $G$. If 
    $P\in\Syl_p(K)$ for some prime number $p$, then $G=KN_G(P)$.
\end{lemma}

\begin{proof}
	Let $g\in G$. Since $gPg^{-1}\subseteq gKg^{-1}=K$, because $K$ is normal in $G$, 
	and $gPg^{-1}\in\Syl_p(K)$, there exists $k\in K$ such that 
	$kPk^{-1}=gPg^{-1}$. Hence $k^{-1}g\in N_G(P)$, as
	$P=(k^{-1}g)P(k^{-1}g)^{-1}$. Therefore $g=k(k^{-1}g)\in
	KN_G(P)$.
\end{proof}

\begin{theorem}[Hall]
	\label{theorem:Hall}
        \index{Hall's theorem}
	Let $G$ be a finite group such that every maximal subgroup of $G$ 
        has a prime or a prime-square index. Then $G$ is solvable. 
\end{theorem}

\begin{proof}
We proceed by induction on $|G|$. Let $p$ be the largest prime divisor of $|G|$. Let 
$S\in\Syl_p(G)$ and $N=N_G(S)$. 

If $N=G$, then $S$ is normal in $G$. Since every 
maximal subgroup of $G/S$ has prime or a prime-square index, 
$G/S$ is solvable by the inductive hypothesis. Since $S$ is a $p$-group, it is solvable. 
Therefore $G$ is solvable. 

Assume now that $N\ne G$. Let $H$ be a maximal subgroup of $G$ containing $N$. Then 
\[
N=N_H(S)=N_G(S),
\]
the number of Sylow $p$-subgroups of $G$ is $(G:N)\equiv1\bmod p$  
and the number of Sylow $p$-subgroups of $H$ is $(H:N)\equiv1\bmod p$ 
by the third Sylow's theorem. So,
\[
\underbrace{(G:N)}_{\equiv1\bmod p} = (G:H)\underbrace{(H:N)}_{\equiv1\bmod p},
\]
implies that also $(G:H)\equiv1\bmod p$.
By assumption, there exists a prime number $q$ 
such that 
$(G:H)\in\{q,q^2\}$. Since $q$ divides $|G|$, it follows that $q<p$. If $(G:H)=q$, then 
$p$ divides $q-1$ and therefore $p\leq q-1<p$, a contradiction. Thus $q^2=(G:H)\equiv 1\bmod p$.
From this it follows that $q\equiv -1\bmod p$ and hence $q=2$ and 
$p=q+1=3$. 

Therefore $G$ has order $2^\alpha 3^ \beta$. 
If we apply Burnside's theorem, we are done. Instead, we will finish the proof with an elementary argument. Let $K$ be a minimal normal subgroup of $G$.
By Frattini's argument (Lemma~\ref{lem:Frattini_argument}),  $G=KN=KH$. Since $H$ is maximal, 
\[
(K:K\cap H) = (G:H)=4, 
\]
as $(G:H)=|G|/|H|=|KH|/|H|=|K|/|K\cap H|$. 
Since $(K:K\cap H)=4$, letting $K$ act on $K/K\cap H$ by left multiplication, there exists 
a non-trivial group homomorphism $\rho\colon K\to\Sym_4$. Since $[K,K]$ is characteristic in $K$ and $K$ is normal in $G$, $[K,K]\subseteq K$ is normal in $G$. Since $K$ is minimal normal in $G$, 
there are two possible cases: either $[K,K]=\{1\}$ or 
$[K,K]=K$. If $[K,K]=K$, 
since $\Sym_4$ is solvable, $\rho(K)$ is solvable. Then
\[
\rho(K)=\rho([K,K])=[\rho(K),\rho(K)],
\]
a contradiction. Therefore $[K,K]=\{1\}$ and $K$ is solvable (as it is abelian).
\end{proof}

% \begin{proof}
% 	We proceed by induction on $|G|$. Let $N$ be a minimal normal subgroup of 
% 	$G$ and $p$ be the largest prime divisor of $|N|$. Let $P\in\Syl_p(N)$ and 
% 	$L=N_G(P)$. If $L=G$, then $P$ is normal in $G$. Since $P$ and 
% 	$G/P$ are solvable by the inductive hypothesis, $G$ is solvable. Assume now that 
% 	$L\ne G$. Let $M$ be a maximal subgroup of $G$ containing $L$. 
% 	    By Frattini's argument (Lemma~\ref{lem:Frattini_argument}),  
% 	$G=NL=NM$. Since $M$ is maximal, 
% 	there exists a prime number $q$ such that  
% 	\[
% 	(N:N\cap M)=(G:M)\in\{q,q^2\}, 
% 	\]
% 	as $(G:M)=|G|/|M|=|NM|/|M|=|N|/|N\cap M|$.  
% 	Thus $q$ divides $|N|$ and hence $q\leq p$. In particular, $q\not\equiv
% 	1\bmod p$ (otherwise, if $q\equiv1\bmod p$, 
%         then $p\mid q-1$ and thus $p\leq q-1<q\leq p$, a contradiction). 
% 	If $g\in G$, then  
% 	\[
% 	gPg^{-1}\subseteq gNg^{-1}=N
% 	\]
% 	and hence $gPg^{-1}\in\Syl_p(N)$. Let $G$ act by conjugation on 
% 	$\Syl_p(N)$. The number of $p$-subgroups of $N$ is then 
% 	\[
% 		(G:N_G(P))=(G:L)\equiv 1\bmod p.	
% 	\]
% 	Since $L\subseteq M$, $L=N_M(P)$. 
	
% 	%Supongamos que $|P|=p^{\alpha}$.  
% 	Since $P\subseteq M$, $P$ acts 
% 	on the set $X=\{mPm^{-1}:m\in M\}$. We claim that $\{P\}$ is the only one-element orbit. 
%         If $\{P_1\}$ is an orbit, then $P_1=mPm^{-1}$ for some $m\in M$ and 
%  	$P=m^{-1}P_1m=P_1$. 
% % 	es un subgrupo normal de $\langle P,P_1\rangle$ y
% % 	luego $P_1P$ is un subgrupo de $M$ de orden $p^{\beta}$ con
% % 	$\beta\leq\alpha$. Since $P\subseteq P_1P$, se concluye que $P_1P=P$ and luego
% % 	$P=P_1$. 

%     Decompose $X$ as a disjoint union of orbits: 
% 	\[
% 	X=\{P\}\cup O(P_1)\cup\cdots\cup O(P_k),
% 	\]
% 	where $\{P\}$ is the only one-element orbit and 
%         each $O(P_j)$ has size divisible by $p$. 
% 	Then 
% 	\[
% 		(M:N_M(P))=(M:L)=|X|\equiv1\bmod p.
% 	\]
% 	From $(G:L)=(G:M)(M:L)$ we conclude that 
%         $(G:M)\equiv1\bmod p$.
% 	Therefore 
%  \[
%  (G:M)=q^2\equiv 1\bmod p.
%  \]
%  This implies that $q\equiv -1\bmod
% 	p$. Hence $q=2$ and $p=q+1=3$.

% 	Since $(N:N\cap M)=4$, letting $N$ act on $N/N\cap M$ by left multiplication, there exists 
%         a non-trivial group homomorphism $\rho\colon
% 	N\to\Sym_4$. Since $[N,N]$ is characteristic in $N$ and $N$ is normal in $G$,
% 	there are two cases: either $[N,N]=\{1\}$ or 
% 	$[N,N]=N$. If $[N,N]=N$, then
% 	\[
% 	\rho(N)=\rho([N,N])=[\rho(N),\rho(N)].
% 	\]
% 	Since $\Sym_4$ is solvable, $\rho(N)$ is solvable. Hence $\rho(N)=\{1\}$, a contradiction. 
% 	Therefore $[N,N]=\{1\}$. Since $N$ is solvable (it is abelian) 
% 	and $G/N$ is solvable by the inductive hypothesis, $G$ is solvable.
% %	El grupo $N$ tiene orden $2^a3^b$, and entonces $N$ is solvable por el
% %	teorema de Burnside.  Since $N$ is solvable and $G/N$ is solvable por
% %	hipótesis inductiva, $G$ is solvable por el
% %	teorema~\ref{theorem:solvable}. 
% \end{proof}


% \chapter{}

We now describe some very-well known open problems
in the theory of group rings and the connection between 
them. 

\begin{definition}
\index{Ring!reduced}
A ring $R$ is \textbf{reduced} if for all $r\in R$ such that 
$r^2=0$ one has $r=0$.
\end{definition}

% \topic{Andrunakevic--Rjabuhin's theorem}
%https://ysharifi.wordpress.com/2010/06/04/about-reduced-rings-1/
% \begin{exercise}
% \label{xca:reduced}
%     Let $R$ be a ring and $I$ be an ideal of $R$.
%     Prove that $I$ is prime if and only if $xRy\subseteq I$ implies
%     either $x\in I$ or $y\in I$. 
% \end{exercise}

% \begin{sol}{xca:reduced}
%     Let $A$ and $B$ be ideals such that $AB\subseteq I$. If 
%     $A\not\subseteq I$ and $B\not\subseteq J$, let 
%     $x\in A\setminus P$ and $y\in B\setminus P$. Then
%     $xRy\subseteq AB\subseteq I$, a contradiction. 
%     Conversely, if $xRy\subseteq I$ and $x\not I$ and $y\not\in I$, 
%     then $A=(x)\not\subseteq I$ and $B=(y)\not\subseteq P$.
% \end{sol}

Integral domains and boolean rings are reduced. $\Z/8$ 
and $M_2(\R)$ are not reduced. 

\begin{example}
    $\Z^n$ with $(a_1,\dots,a_n)(b_1,\dots,b_n)=(a_1b_1,\dots,a_nb_n)$
    is reduced. 
\end{example}

The structure of 
reduced rings is described by
Andrunakevic--Rjabuhin's theorem. It states
that a ring is reduced if and only if
it is a subdirect products of domains. See
\cite[3.20.5]{MR2015465} for a proof. 

% \begin{theorem}[Andrunakevic--Rjabuhin]
% \index{Andrunakevic--Rjabuhin's theorem}
% 	Let $R$ be a non-zero ring. Then $R$ is reduced if and only 
% 	if $R$ is a subdirect product of domains.
% \end{theorem}

% We shall need some lemmas. 

% \begin{lemma}
%     Let $P$ be a minimal prime ideal of $R$. Then 
%     $S=R\setminus P$ is multiplicatively closed if and only if
%     $s_1\cdots s_k\ne 0$ for all $k\geq 1$ and $s_1,\dots,s_k\in S$. 
% \end{lemma}

% \begin{proof}
%     Let $T=\{s_1\cdots s_k:k\geq 1,s_1,\dots,s_k\in S\}$. Clearly $T$ 
%     is multiplicatively closed and $S\subseteq T$. We claim that 
%     $T\subseteq S$. Let $X=\{I:\text{$I$ is an ideal of $R$ such that $I\cap T=\emptyset$}\}$. 
%     Then $X\ne\emptyset$, as $\{0\}\in X$. If $C$ is a chain in $X$, then
%     $\cup_{I\in C}I$ is an upper bound, so there exists a maximal element $Q\in X$ 
%     by Zorn's lemma. It is an exercise to show that $Q$ is prime. Since 
%     $Q\cap T=\emptyset$, it follows that $Q\cap S=\emptyset$, that is $Q\subseteq P$. 
%     Since $P$ is minimal, $P=Q\in X$. In particular, $T\subseteq S$. 
% \end{proof}

% \begin{lemma}
%     Let $P$ be a minimal prime ideal of $R$. 
%     Then $R/P$ is a domain.
% \end{lemma}

% \begin{proof}
    
% \end{proof}


% \begin{proof}
% 	If $R$ is reduced, then every prime ideal contains a minimal prime ideal. 
% 	Thus $\cap_{i\in I}P'_i=\{0\}$, where $\{P'_i:\in I\}$ is the collection
% 	of minimal prime ideals. Then each $R/P'_i$ is a domain and 
% 	there is an injective map $R\to\prod_{i\in I}R/P'_i$. 
	
% 	Supongamos ahora que $R$ es producto subdirecto de la familia $\{R_i:i\in I\}$ de dominios. Sea 
% 	$f\colon R\to \prod_{i\in I}R_i$, $f(x)=(x_i)_{i\in I}$, el morfismo inyectivo. 
% 	Si $x\in R$ es tal que $x^2=0$ entonces 
% 	\[
% 		(0)_{i\in I}=f(0)=f(x^2)=f(x)^2=(x_i^2)_{i\in I}
% 	\]
% 	y luego, como cada $R_i$ es un dominio, se concluye que $x_i=0$ para todo
% 	$i\in I$.
% \end{proof}


\begin{problem}
	\label{prob:reducido}
	Let $G$ be a torsion-free group. Is it true that 
	$K[G]$ is reduced? 
\end{problem}

\begin{problem}[Semisimplicity]
	\label{prob:J}
	Let $G$ be a torsion-free group. It is true that 
	$J(K[G])=\{0\}$ if $G$ is non-trivial?
\end{problem}

\index{Idempotent}
Recall that an element $e$ of a ring is said to be \emph{idempotent} 
if $e^2=e$. Examples of idempotents are $0$ and $1$ and 
these are known as the trivial idempotents. 

\begin{problem}[Idempotents]
	\label{pro:idempotente}
	Let $G$ be a torsion-free group and $\alpha\in K[G]$ be an idempotent. 
	Is it true that $\alpha\in\{0,1\}$?
\end{problem}

\begin{exercise}
	Prove that if $K[G]$ has no zero-divisors and $\alpha\in K[G]$ is an
	idempotent, then $\alpha\in\{0,1\}$.
\end{exercise}

\begin{exercise}
	Prove that $K[C_4]$ contains non-trivial zero divisors and every
	idempotent of $K[C_4]$ is trivial. 
\end{exercise}

The problems mentioned are all related. Our goal is the prove
the following implications:
\[
	\ref{prob:J}\Longleftarrow\ref{prob:units}\Longrightarrow\ref{prob:reducido}\Longleftrightarrow\ref{prob:dominio}
\]

We first prove that an affirmative solution to the Units
Problem~\ref{prob:units} yields a solution to Problem~\ref{prob:reducido}
about the reducibility of group algebras.

\begin{theorem}
    Let $K$ be a field of characteristic $\ne2$ 
	and $G$ be a non-trivial group. Assume that $K[G]$ has only trivial units.
	Then $K[G]$ is reduced. 
\end{theorem}

\begin{proof}
	Let $\alpha\in K[G]$ be such that $\alpha^2=0$. We claim that 
	$\alpha=0$. Since $\alpha^2=0$, 
	\[
		(1-\alpha)(1+\alpha)=1-\alpha^2=1, 
	\]
	it follows that $1-\alpha$ is a unit of $K[G]$. Since units of $K[G]$ are 
	trivial, there exist $\lambda\in K\setminus\{0\}$ and $g\in G$ such that 
	$1-\alpha=\lambda g$. If $g\ne 1$, then 
	\[
		0=\alpha^2=(1-\lambda g)^2=1-2\lambda g+\lambda^2g^2,
	\]
	a contradiction. Therefore $g=1$ and hence $\alpha=1-\lambda\in K$. Since
	$K$ is a field, one concludes that $\alpha=0$.
\end{proof}


We now prove that an affirmative solution to the Units Problem
~\ref{prob:units} also yields a solution to the Jacobson Semisimplicity Problem
~\ref{prob:J}. 

\begin{theorem}
	Let $G$ be a non-trivial group. Assume that $K[G]$ has only trivial units. 
	If $|K|>2$ or $|G|>2$, then $J(K[G])=\{0\}$.
\end{theorem}

\begin{proof}
	Let $\alpha\in J(K[G])$. There exist $\lambda\in K\setminus\{0\}$ and $g\in
	G$ such that $1-\alpha=\lambda g$. Assume that $g\ne 1$.  If $|K|\geq3$,
	then there exist $\mu\in K\setminus\{0,1\}$ such that
	\[
		1-\alpha\mu=1-\mu+\lambda\mu g 
	\]
	is a non-trivial unit, a contradiction.
	If $|G|\geq3$, there exists $h\in G\setminus\{1,g^{-1}\}$ such that
	$1-\alpha h=1-h+\lambda gh$ is a non-trivial unit, a contradiction.  Thus
	$g=1$ and hence $\alpha=1-\lambda\in K$. Therefore $1+\alpha h$ is a
	trivial unit for all $h\ne 1$ and hence 	$\alpha=0$.
\end{proof}

\begin{exercise}
	Prove that if $G=\langle g\rangle\simeq\Z/2$, then 
	$J(\F_2[G])=\{0,g-1\}\ne\{0\}$. 
\end{exercise}

\topic{Passman's theorem}

Now we prove that an affirmative solution 
to the Units Problem 
(Open Problem~\ref{prob:units}) 
yields a solution to 
Open Problem~\ref{prob:dominio} about zero divisors in group algebras.
The proof is hard and requires some preliminaries. We first need
to discuss a group theoretical tool known as the \emph{transfer map}. 

If $H$ is a subgroup of $G$, a \textbf{transversal} of $H$ in $G$ is a complete
set of coset representatives of $G/H$. 

\begin{lemma}
	\label{lem:d}
	Let $G$ be a group and $H$ be a subgroup of $G$ of finite index.  Let $R$
	and $S$ be transversals of $H$ in $G$ and let $\alpha\colon H\to H/[H,H]$
	be the canonical map. Then 
	\[
		d(R,S)=\prod \alpha(rs^{-1}),
	\]
	where the product is taken over all pairs 
	$(r,s)\in R\times S$ such that $Hr=Hs$,
	is well-defined and satisfies the following properties:
	\begin{enumerate}
		\item $d(R,S)^{-1}=d(S,R)$.
		\item $d(R,S)d(S,T)=d(R,T)$ for all transversal $T$ of $H$ in $G$.
		\item $d(Rg,Sg)=d(R,S)$ for all $g\in G$.
		\item $d(Rg,R)=d(Sg,S)$ for all $g\in G$.
	\end{enumerate}
\end{lemma}

\begin{proof}
	The product that defines $d(R,S)$ is well-defined since $H/[H,H]$ is 
	an abelian group. The first three claim are trivial. Let us prove
	4). By 2), 
	\[
		d(Rg,Sg)d(Sg,S)d(S,R)=d(Rg,S)d(S,R)=d(Rg,R).
	\]
	Since $H/[H,H]$ is abelian, 1) and 3) imply that 	
	\[
		d(Rg,Sg)d(Sg,S)d(S,R)=d(R,S)d(S,R)d(Sg,S)=d(Sg,S).\qedhere
	\]
\end{proof}

We are know ready to state and 
prove the theorem: 

\begin{theorem}
	\label{thm:transfer}
	Let $G$ be a group and $H$ be a finite-index subgroup of $G$. The map 	
	\[
		\nu\colon G\to H/[H,H],\quad
		g\mapsto d(Rg,R),
	\]
	does not depend on the transversal $R$ of $H$ in $G$ and it is a group
	homomorphism. 
\end{theorem}

\begin{proof}
	The lemma implies that the map does not depend on the transversal used. 
	Moreover, $\nu$ is a group homomorphism, as 
	\begin{align*}
		\nu(gh)&=d(R(gh),R)
		=d(R(gh),Rh)d(Rh,R)
		=d(Rg,R)d(Rh,R)=\nu(g)\nu(h).\qedhere
	\end{align*}
\end{proof}

The theorem justifies the following definition: 

\begin{definition}
	Let $G$ be a group and $H$ be a finite-index subgroup of $G$. The
	\textbf{transfer map} of $G$ in $H$ is the group homomorphism 
	\[
		\nu\colon G\to H/[H,H],
		\quad
		g\mapsto d(Rg,R),
	\]
	of Theorem~\ref{thm:transfer}, where $R$ is some transversal of $H$ in $G$.
\end{definition}

We need methods for computing the transfer map. If $H$ is a subgroup of 
$G$
and $(G:H)=n$, let $T=\{x_1,\dots,x_n\}$ be a transversal of $H$. For $g\in G$ let  
\[
	\nu(g)=\prod \alpha(xy^{-1}),
\]
where the product is taken over all pairs $(x,y)\in (Tg,T)$ such that $Hx=Hy$
and $\alpha\colon H\to H/[H,H]$ is the canonical map. 
If we write 
$x=x_ig$ for some $i\in\{1,\dots,n\}$, then  
$Hx_ig=Hx_{\sigma(i)}$ for some permutation $\sigma\in\Sym_n$. Thus 
\[
	\nu(g)=\prod_{i=1}^n\alpha(x_igx_{\sigma(i)}^{-1}).
\]

\begin{lemma}
	\label{lem:transfer}
	Let $G$ be a group and $H$ be a subgroup such that $(G:H)=n$. Let 
	$T$ be a transversal of $H$ in $G$. 
	For each $g\in G$ there exist $k$ and 
	positive integers 
	$n_1,\dots,n_k$ such that $n_1+\cdots+n_k=n$ and elements 
	$t_1,\dots,t_k\in T$ such that  
	\[
		\nu(g)=\prod_{i=1}^k \alpha(t_ig^{n_i}t_i^{-1}),
	\]
	where $\alpha\colon H\to H/[H,H]$ is the canonical map.
\end{lemma}

\begin{proof}
	There exists $\sigma\in\Sym_n$ such that 
	\[
	\nu(g)=\prod_{i=1}^n \alpha( t_igt_{\sigma(i)}^{-1}). 
	\]
	Write $\sigma$ as a product of $k$ disjoint cycles
	$\sigma=\alpha_1\cdots\alpha_k$, where each $\alpha_j$ is a cycle of length 
	$n_j$. For every cycle of the form $(i_1\cdots i_{n_j})$
	we reorder the product in such a way that 
	\[
		\alpha(x_{i_1}gx_{i_2}^{-1})\alpha(x_{i_2}gx_{i_3}^{-1})\cdots \alpha(x_{i_{n_j}}gx_{i_1}^{-1})=\alpha(x_{i_1}g^{n_1}x_{i_1}^{-1}).
	\]
	There exist $t_1,\dots,t_k\in T$ such that 
	$\nu(g)=\prod_{j=1}^k \alpha(t_ig^{n_i}t_i^{-1}$). 
\end{proof}

An application:

\begin{proposition}
	\label{pro:center}
	If $G$ is a group such that $Z(G)$ has finite index $n$, then
	$(gh)^n=g^nh^n$ for all $g,h\in G$.	
\end{proposition}

\begin{proof}
	Let $g\in G$. By Lemma~\ref{lem:transfer} there are positive integers 
    $n_1,\dots,n_k$ such that $n_1+\cdots+n_k=n$ and elements 
	$t_1,\dots,t_k$ of a transversal of $Z(G)$ in $G$ such that 
	\[
		\nu(g)=\prod_{i=1}^k \alpha(t_ig^{n_1}t_i^{-1}),
	\]
	where $\alpha\colon G\to H/[H,H]$ is the canonical map. Since
	$g^{n_i}\in Z(G)$ for all $i\in\{1,\dots,k\}$ (as $t_ig^{n_i}t_i^{-1}\in Z(G)$), 
	it follows that 
	$\nu(g)=g^{n_1+\cdots+n_k}=g^n$.  Since $\nu$ is a group homomorphism by 
	Theorem~\ref{thm:transfer}, we conclude that 
	\[
		(gh)^n=\nu(gh)=\nu(g)\nu(h)=g^nh^n. \qedhere 
	\]
\end{proof}

For a group $G$ we consider 
\[
	\Delta(G)=\{g\in G:(G:C_G(g))<\infty\}.
\]

\begin{exercise}
	Prove that $\Delta(\Delta(G))=\Delta(G)$.
\end{exercise}

\begin{lemma}
	If $G$ is a group, then $\Delta(G)$ 
	is a characteristic subgroup of $G$.
\end{lemma}

\begin{proof}
	We first prove that $\Delta(G)$ is a subgroup of $G$. If $x,y\in\Delta(G)$
	and $g\in G$, then $g(xy^{-1})g^{-1}=(gxg^{-1})(gyg^{-1})^{-1}$. Moreover, 
	$1\in\Delta(G)$. Let us show now that $\Delta(G)$ is characteristic in $G$. If 
	$f\in\Aut(G)$ and $x\in G$, then, since 
	\[
	f(gxg^{-1})=f(g)f(x)f(g)^{-1},
	\]
	it follows that $f(x)\in\Delta(G)$.
\end{proof}

\begin{exercise}
	Prove that if $G=\langle r,s:s^2=1,srs=r^{-1}\rangle$ is the
	infinite dihedral group, then $\Delta(G)=\langle r\rangle$.
\end{exercise}

\begin{exercise}
	Let $H$ and $K$ be finite-index subgroups of $G$. Prove that
	\[
	(G:H\cap K)\leq (G:H)(G:K). 
	\]
\end{exercise}


% \chapter{}

\begin{proposition}
	\label{pro:FCabeliano}
	If $G$ is a torsion-free group such 
	that $\Delta(G)=G$, then $G$ is abelian.
\end{proposition}

\begin{proof}
	Let $x,y\in G$ and $S=\langle x,y\rangle$. The group $Z(S)=C_S(x)\cap C_S(y)$ has 
	finite index, say $n$, in $S$. By Proposition~\ref{pro:center}, 
	the map $S\to Z(S)$, $s\mapsto s^n$, is a group homomorphism. Thus  
	\[
		[x,y]^n=(xyx^{-1}y^{-1})^n=x^ny^nx^{-n}y^{-n}=1
	\]
	as $x^n\in Z(S)$. Since $G$ is torsion-free, $[x,y]=1$.
\end{proof}

\begin{lemma}[Neumann]
	\index{Lema!de Neumann}
	\label{lem:Neumann}
	Let $H_1,\dots,H_m$ be subgroups of $G$. 
	Assume there are finitely many elements
	$a_{ij}\in G$, $1\leq i\leq m$, $1\leq j\leq n$, such that 
	\[
		G=\bigcup_{i=1}^m\bigcup_{j=1}^n H_ia_{ij}.
	\]
	Then some $H_i$ has finite index in $G$.
\end{lemma}

\begin{proof}
	We proceed by induction on $m$. The case $m=1$ is trivial. 
	Let us assume that $m\geq2$. If $(G:H_1)=\infty$, there exists $b\in G$
	such that 
	\[
		Hb\cap\left(
	\bigcup_{j=1}^nH_1a_{1j}\right)=\emptyset.
	\]
	Since $H_1b\subseteq\bigcup_{i=2}^m\bigcup_{j=1}^n H_ia_{ij}$, 
	it follows that 
	\[
		H_1a_{1k}\subseteq \bigcup_{i=2}^m\bigcup_{j=1}^n Ha_{ij}b^{-1}a_{1k}.
	\]
	Hence $G$ can be covered by finitely many cosets of $H_2,\dots,H_m$. By the inductive hypothesis, 
	some of these $H_j$ has finite index in $G$.
\end{proof}

We now consider a projection operator of group algebras. If $G$ 
is a group and $H$ is a subgroup of $G$, let 
\[
	\pi_H\colon K[G]\to K[H],\quad
	\pi_H\left(\sum_{g\in G}\lambda_gg\right)=\sum_{g\in H}\lambda_gg.
\]

If $R$ and $S$ are rings, a $(R,S)$-bimodule is an abelian group
$M$ that is both a left $R$-module and a right $S$-module 
and the compatibility condition 
\[
(rm)s = r(ms)
\]
holds for all $r\in R$, $s\in S$ and $m\in M$.


\begin{exercise}
	Let $G$ be a group and $H$ be a subgroup of $G$. Prove that
	if $\alpha\in
	K[G]$, then $\pi_H$ is a $(K[H],K[H])$-bimodule homomorphism
	with usual left and right multiplications,
	\[
		\pi_H(\beta\alpha\gamma)=\beta\pi_H(\alpha)\gamma
	\]
	for all $\beta,\gamma\in K[H]$.
\end{exercise}

%\begin{proof}
%	Supongamos que $\alpha=\sum_{g\in G}\lambda_gg=\alpha_1+\alpha_2$, donde
%	$\alpha_1=\sum_{g\not\in H}\lambda_gg$ y $\alpha_2=\sum_{g\in
%	H}\lambda_gg=\pi_H(\alpha)$. Entonces
%	$\beta\alpha\gamma=\beta\alpha_1\gamma+\beta\alpha_2\gamma$, donde
%	$\beta\alpha_1\gamma\not\in K[H]$ y $\beta\alpha_2\gamma\in K[H]$.
%\end{proof}

\begin{lemma}
	\label{lem:escritura}
	Let $X$ be a left transversal of $H$ in $G$. Every $\alpha\in K[G]$ can be written
	uniquely as 
	\[
	\alpha=\sum_{x\in X}x\alpha_x,
	\]
	where $\alpha_x=\pi_H(x^{-1}\alpha)\in K[H]$.
\end{lemma}

\begin{proof}
	Let $\alpha\in K[G]$. Since $\supp\alpha$ is finite, $\supp\alpha$ is contained 
    in finitely many cosets of $H$, say $x_1H,\dots,x_nH$, where each 
	$x_j$ belongs to $X$. Write $\alpha=\alpha_1+\cdots+\alpha_n$,
	where $\alpha_i=\sum_{g\in x_iH}\lambda_gg$. If $g\in x_iH$, then 
	$x_i^{-1}g\in H$ and hence 
	\[
		\alpha=\sum_{i=1}^n x_i(x_i^{-1}\alpha_i)=\sum_{x\in X}x\alpha_x
	\]
	with $\alpha_x\in K[H]$ for all $x\in X$. For the uniqueness, note that 
	for each  $x\in X$ the previous exercise implies that  
	\[
		\pi_H(x^{-1}\alpha)
		=\pi_H\left(\sum_{y\in X}x^{-1}y\alpha_y\right)
		=\sum_{y\in X}\pi_H(x^{-1}y)\alpha_y=\alpha_x, 
	\]
	as  
	\[
		\pi_H(x^{-1}y)=\begin{cases}
		1 & \text{si $x=y$},\\
		0 & \text{si $x\ne y$}.
	\end{cases}\qedhere 
	\]
\end{proof}

%De la misma forma puede obtenerse un análogo al lema~\ref{lem:escritura} en el
%caso en que se tenga un transversal a derecha. 

\begin{lemma}
	\label{lem:ideal_pi}
	Let $G$ be a group and $H$ be a subgroup of $G$. If $I$ is a non-zero 
	left ideal
	of $K[G]$, then  $\pi_H(I)\ne 0$.
\end{lemma}

\begin{proof}
	Let $X$ be a left transversal of $H$ in $G$ and $\alpha\in I\setminus\{0\}$. By Lemma
	\ref{lem:escritura} we can write $\alpha=\sum_{x\in
	X}x\alpha_x$ with $\alpha_x=\pi_H(x^{-1}\alpha)\in K[H]$ for all $x\in X$.
	Since $\alpha\ne0$, there exists $y\in X$ such that $0\ne
	\alpha_y=\pi_H(y^{-1}\alpha)\in\pi_H(I)$ ($y^{-1}\alpha\in I$ since $I$ is 
    a left ideal).
\end{proof}

Another application:

\begin{proposition}
	Let $G$ be a group, $H$ be a subgroup of $G$ and $\alpha\in K[H]$. The following statements hold:
	\begin{enumerate}
		\item $\alpha$ is invertible in $K[H]$ if and only if $\alpha$ is
			invertible in $K[G]$.
		\item $\alpha$ is a zero divisor of $K[H]$ if and only if $\alpha$ is  
			a zero divisor of $K[G]$.
	\end{enumerate}
\end{proposition}

\begin{proof}
	If $\alpha$ is invertible in $K[G]$, there exists $\beta\in K[G]$ such that 
	$\alpha\beta=\beta\alpha=1$. Apply $\pi_H$ and use that $\pi_H$ 
	is a $(K[H],K[H])$-bimodule homomorphism to obtain  
	\[
		\alpha\pi_H(\beta)=\pi_H(\alpha\beta)=\pi_H(1)=1=\pi_H(1)=\pi_H(\beta\alpha)=\pi_H(\beta)\alpha.
	\]
	
	Assume now that $\alpha\beta=0$ for some $\beta\in K[G]\setminus\{0\}$. Let $g\in G$
	be such that $1\in\supp(\beta g)$. Since $\alpha(\beta g)=0$, 
	\[
		0=\pi_H(0)=\pi_H(\alpha(\beta g))=\alpha\pi_H(\beta g),
	\]
	where $\pi_H(\beta g)\in K[H]\setminus\{0\}$, as $1\in\supp(\beta g)$. 
\end{proof}

\begin{lemma}[Passman]
	\index{Passman's lemma}
	\label{lem:Passman}
	Let $G$ be a group and 
	$\gamma_1,\gamma_2\in K[G]$ be such that $\gamma_1K[G]\gamma_2=0$.
	Then $\pi_{\Delta(G)}(\gamma_1)\pi_{\Delta(G)}(\gamma_2)=0$.
\end{lemma}

\begin{proof}
	It is enough to show that $\pi_{\Delta(G)}(\gamma_1)\gamma_2=0$, 
	as in this case
	\[
		0=\pi_{\Delta(G)}(\pi_{\Delta(G)}(\gamma_1)\gamma_2)=\pi_{\Delta}(\gamma_1)\pi_{\Delta(G)}(\gamma_2).
	\]
	Write $\gamma_1=\alpha_1+\beta_1$, where 
	\begin{align*}
		&\alpha_1=a_1u_1+\cdots+a_ru_r, && u_1,\dots,u_r\in\Delta(G),\\
		&\beta_1=b_1v_1+\cdots+b_sv_s, && v_1,\dots,v_s\not\in\Delta(G),\\
		&\gamma_2=c_1w_1+\cdots+c_tw_t,&& w_1,\dots,w_t\in G.
	\end{align*}
	The subgroup $C=\bigcap_{i=1}^rC_G(u_i)$ has finite index in $G$.
	Assume that 
	\[
		0\ne \pi_{\Delta}(\gamma_1)\gamma_2=\alpha_1\gamma_2. 
	\]
	Let $g\in\supp(\alpha_1\gamma_2)$. 
	If $v_i$ is a conjugate in $G$ of some 
	$gw_j^{-1}$, let $g_{ij}\in G$ be such that
	$g_{ij}^{-1}v_ig_{ij}=gw_j^{-1}$. If $v_i$ and $gw_j^{-1}$ 
	are not conjugate, 
	we take $g_{ij}=1$. 

	For every $x\in C$ it follows that
	$\alpha_1\gamma_2=(x^{-1}\alpha_1x)\gamma_2$. Since  
	\[
		x^{-1}\gamma_1x\gamma_2\in x^{-1}\gamma_1K[G]\gamma_2=0,
	\]
	it follows that
	\begin{align*}
		(a_1u_1+\cdots+a_ru_r)\gamma_2&=
		\alpha_1\gamma_2=x^{-1}\alpha_1x\gamma_2=-x^{-1}\beta_1x\gamma_2\\
		&=-x^{-1}(b_1v_1+\cdots+b_sv_r)x(c_1w_1+\cdots+c_tw_t).
	\end{align*}
	Now $g\in\supp(\alpha_1\gamma_2)$ implies that there exist $i,j$ such that
	$g=x^{-1}v_ixw_j$.
	Thus $v_i$ and $gw_j^{-1}$ are conjugate and hence
	$x^{-1}v_ix=gw_j^{-1}=g_{ij}^{-1}v_ig_{ij}$, that is
	$x\in C_G(v_i)g_{ij}$. This proves that 
	\[
		C\subseteq\bigcup_{i,j}C_G(v_i)g_{ij}. 
	\]
	Since $C$ has finite index in $G$, it follows that 
	$G$ can be covered by finitely many cosets of 
	the $C_G(v_i)$. Every $v_i\not\in\Delta(G)$, so 
	each $C_G(v_i)$ has infinite index in $G$, a contradiction 
	to Neumann's lemma.
\end{proof}

\begin{theorem}[Passman]
\index{Passman's theorem}
	Let $G$ be a torsion-free group. If 
	$K[G]$ is reduced, then $K[G]$ is a domain.
\end{theorem}

\begin{proof}
	Assume that $K[G]$ is not a domain. Let $\gamma_1,\gamma_2\in K[G]\setminus\{0\}$
	be such that $\gamma_2\gamma_1=0$. If $\alpha\in K[G]$, then
	\[
		(\gamma_1\alpha\gamma_2)^2=\gamma_1\alpha\gamma_2\gamma_1\alpha\gamma_2=0
	\]
	and thus $\gamma_1\alpha\gamma_2=0$, as $K[G]$ is reduced. In particular, 
	$\gamma_1K[G]\gamma_2=0$. Let $I$ be the left ideal of $K[G]$ generated 
	by $\gamma_2$. Since $I\ne 0$, $\pi_{\Delta(G)}(I)\ne 0$ by Lemma~\ref{lem:ideal_pi}
	and hence  $\pi_{\Delta(G)}(\beta\gamma_2)\ne 0$ for some $\beta\in K[G]$. 
	Similarly, 
	$\pi_{\Delta(G)}(\gamma_1\alpha)\ne 0$ for some $\alpha\in K[G]$. Since 
	\[
		\gamma_1\alpha K[G]\beta\gamma_2\subseteq \gamma_1 K[G]\gamma_2=0,
	\]
    it follows that $\pi_{\Delta(G)}(\gamma_1\alpha)\pi_{\Delta(G)}(\beta\gamma_2)=0$
    by Passman's lemma. Hence $K[\Delta(G)]$ has zero divisors, a contradictions
    since $\Delta(G)$ is an abelian group.
\end{proof}


% \chapter{}

\topic{More applications of the transfer}

Let us start with a group-theoretic application
of the transfer map. We start with some applications to the theory 
of finite groups. 

\begin{proposition}
	\label{prop:semidirecto}
	Let $G$ be a finite group and $H$ a central subgroup of index $n$, where 
	$n$ is coprime with $|H|$. Then
	$G\simeq N\rtimes H$.
\end{proposition}

\begin{proof}
	Since $H$ is abelian, $H=H/[H,H]$. Let  
	$\nu\colon G\to H$ be the transfer map and $h\in H$. 
	By Lemma~\ref{lem:transfer}, 
	\[
		\nu(h)
		=\prod_{i=1}^m s_i^{-1}h^{n_i}s_i,
	\]
	where each $s_i^{-1}h^{n_i}s_i\in H$. Since 
	$h^{n_i}\in H\subseteq Z(G)$ for all $i$, it follows that 
	$s_i^{-1}h^{n_i}s_i=h^{n_i}$ for all $i$. Thus 
	\[
		\nu(h)
		=\prod_{i=1}^m s_i^{-1}h^{n_i}s_i
		=\prod_{i=1}^mh^{n_i}
		=h^{\sum_{i=1}^m n_i}=h^n.
	\] 
	The composition $f\colon H\hookrightarrow G\xrightarrow{\nu} H$ is a group homomorphism. 
	We claim that it is an isomorphism. It is injective: If $h^n=1$, then 
	$|h|$ divides both $|H|$ and $n$. Since $n$ and $|H|$ are
	coprime, $h=1$. It is surjectice: Since $n$ and $|H|$ are coprime, there exists 
	$m\in\Z$ such that $nm\equiv 1\bmod |H|$. If $h\in H$, then $h^m\in
	H$ and $\nu(h^m)=h^{nm}=h$. 
	
	Let $N=\ker f$. We claim that $G=N\rtimes H$. 
	By definition, $N$ is normal in $G$ and $N\cap
	H=\{1\}$. To show that $G=NH$ note that 
	$|NH|=|N||H|$ and $G/N\simeq H$.
\end{proof}

\begin{exercise}
	Let $H$ be a central subgroup of a finite group $G$. If $|H|$
	and $|G/H|$ are coprime, then $G\simeq H\times G/H$.
\end{exercise}

%\begin{proof}
%	Es consecuencia inmediata del corolario~\ref{corollary:semidirecto} pues
%	$H$ es normal por ser un subgrupo central.
%\end{proof}

% TODO: Transitivity of the transfer

% serre, 7.12
An application to infinite groups taken from Serre's book 
\cite[7.12]{MR3469786}. 

\begin{theorem}
	Let $G$ be a torsion-free group that contains a finite-index subgroup isomorphic to  
	$\Z$. Then $G\simeq\Z$.
\end{theorem}

\begin{proof}
	We may assume that $G$ contains a finite-index normal subgroup isomorphic to $\Z$. Indeed, 
	if $H$ is a finite-index subgroup of $G$ such that $H\simeq\Z$, then 
	$K=\cap_{x\in G}xHx^{-1}$ is a non-trivial normal subgroup of $G$ (because $K=\Core_G(H)$ and 
	$G$ has no torsion) and hence $K\simeq\Z$ (because  
	$K\subseteq H$) and $(G:K)=(G:H)(H:K)$ is finite.
	The action of $G$ on $K$ by conjugation induces a group homomorphism  
	$\epsilon\colon G\to\Aut(K)$. Since $\Aut(K)\simeq\Aut(\Z)=\{-1,1\}$, 
	there are two cases to consider.
	
	Assume first that $\epsilon=\id$. Since $K\subseteq Z(G)$, let
	$\nu\colon G\to K$ be the transfer homomorphism. By
	Proposition~\ref{pro:center} (more precisely, 
	by Exercise \ref{xca:K_central}), $\nu(g)=g^n$, where $n=(G:K)$. Since
	$G$ has no torsion, $\nu$ is injective. Thus
	$G\simeq\Z$ because it is isomorphic to a subgroup of $K$.

	Assume now that $\epsilon\ne\id$. Let $N=\ker\epsilon\ne G$. Since
	$K\simeq\Z$ is abelian, $K\subseteq N$. The result proved in the previous paragraph 
	applied to $\epsilon|_N=1$ implies that $N\simeq\Z$, as 
	$N$ contains a finite-index subgroup isomorphic to $\Z$. Let $g\in G\setminus N$. 
	Since $N$ is normal in $G$, $G$ acts by conjugation on $N$ and hence 
	there exists a group homomorphism $c_g\in\Aut(N)\simeq\{-1,1\}$. Since
	$K\subseteq N$ y $g$ acts non-trivially on $K$, 
	\[
	c_g(n)=gng^{-1}=n^{-1}
	\]
	for all $n\in N$.  Since 
	$g^2\in N$, 
	\[
		g^2=gg^2g^{-1}=g^{-2}.
	\]
	Therefore $g^4=1$, a contradiction since $g\ne1$ and $G$ has no torsion.
\end{proof}

Before giving another application of the transfer map, we 
prove Dietzman's theorem: 

\begin{theorem}[Dietzmann]
	\index{Dietzmann's theorem}
	\label{theorem:Dietzmann} 
	Let $G$ be a group and $X\subseteq G$ be a finite subset of $G$ closed by
	conjugation. If there exists $n$ such that $x^n=1$ for all $x\in X$, then
	$\langle X\rangle$ is a finite subgroup of $G$.
\end{theorem}

\begin{proof}
	Let $S=\langle X\rangle$. Since $x^{-1}=x^{n-1}$, every element of $S$ can be 
	written as a finite product of elements of $X$. 
	Fix $x\in X$. We claim that if $x\in X$ appears $k\geq 1$ times 
	in the word $s$, then we can write $s$ as a product of $m$
	elements of $X$, where the first $k$ elements are equal to $x$. Suppose that 
	\[
	s=x_1x_2\cdots x_{t-1}xx_{t+1}\cdots x_m,
	\]
	where $x_j\ne x$ for all $j\in\{1,\dots,t-1\}$. Then 
	\[
		s=x(x^{-1}x_1x)(x^{-1}x_2x)\cdots (x^{-1}x_{t-1}x)x_{t+1}\cdots x_m
	\]
	is a product of $m$ elements of $X$ since $X$ is closed under conjugation and 
	the first element is $x$. The same argument implies that $s$
	can be written as 
	\[
		s=x^ky_{k+1}\cdots y_m,
	\]
	where each $y_j$ belongs to $X\setminus\{x\}$.

	Let $s\in S$ and write $s$ as a product of $m$ elements of 
	$X$, where $m$ is minimal. We need to show that 
	$m\leq (n-1)|X|$. 
	If $m>(n-1)|X|$, 
	then at least one $x\in X$ appears exactly $n$ 
	times in the representation of 
	$s$. Without loss of generality, we write 
	\[
		s=x^nx_{n+1}\cdots x_m=x_{n+1}\cdots x_m,
	\]
	a contradiction to the minimality of $m$. 
\end{proof}

The second result goes back to Schur:

\begin{theorem}[Schur]
\index{Schur's theorem}
\label{thm:Schur}
	Let $G$ be a group. 
	If $Z(G)$ has finite index in $G$, then $[G,G]$ is finite.
\end{theorem}

\begin{proof}
	Let $n=(G:Z(G))$ and  
	$X$ be the set of commutators of $G$. We claim that $X$ is finite, in fact
	$|X|\leq n^2$.
	A routine calculation shows that the map 
	\[
		\varphi\colon X\to G/Z(G)\times G/Z(G),\quad [x,y]\mapsto (xZ(G),yZ(G)),
	\]
	is well-defined. It is, moreover, 
	injective: if $(xZ(G),yZ(G))=(uZ(G),vZ(G))$, then $u^{-1}x\in Z(G)$, 
	$v^{-1}y\in Z(G)$. Thus 
	\begin{align*}
		[u,v]&=uvu^{-1}v^{-1}=uv(u^{-1}x)x^{-1}v^{-1}=xvx^{-1}(v^{-1}y)y^{-1}=xyx^{-1}y^{-1}=[x,y].
	\end{align*}
	Moreover, $X$ is closed under conjugation, as 
	\[
		g[x,y]g^{-1}=[gxg^{-1},gyg^{-1}]
	\]
	for all $g,x,y\in G$. Since $G\to Z(G)$, $g\mapsto g^n$ is a group
	homomorphism, Proposition~\ref{pro:center} implies that $[x,y]^n=[x^n,y^n]=1$ for
	all $[x,y]\in X$.  The theorem follows from applying Dietzmann's theorem. 
\end{proof}

\begin{exercise}
    Let $G$ be the group with generators $a,b,c$ and 
    relations $ab=ca$, $ac=ba$ and $bc=ab$. Prove the following statements:
    \begin{enumerate}
        \item $G$ is infinite and non-abelian.
        \item $Z(G)$ has finite index in $G$ and every conjugacy class of $G$ is finite.
        \item $[G,G]$ is finite. 
        \item The subgroup $N=\langle a^3\rangle$ of $G$ 
        generated by $a^3$ is central 
        and $G/N$ is finite.
    \end{enumerate}
\end{exercise}

We conclude the section with some results similar to that of Schur. 

\begin{theorem}[Niroomand]
\index{Niroomand´s theorem}
\label{thm:Niroomand}
	If the set of commutators of a group $G$ is finite, then 
	$[G,G]$ is finite.
\end{theorem}

\begin{proof}
 	Let $C=\{[x_1,y_1],\dots,[x_k,y_k]\}$ be the (finite) set of commutators of $G$ and  
	$H=\langle x_1,x_2,\dots,x_k,y_1,y_2,\dots,y_k\rangle$. Since $C$ is a set of commutators of $H$, 
	it follows that 
	$[G,G]=\langle C\rangle\subseteq [H,H]$. To simplify the notation we write 
	$H=\langle h_1,\dots,h_{2k}\rangle$. 	
 	Since $h\in Z(H)$ if and only if $h\in C_H(h_i)$ for all 
	$i\in\{1,\dots,2k\}$, we conclude that $Z(H)=C_H(h_1)\cap\cdots\cap C_H(h_{2k})$. Moreover, if 
	$h\in H$, then $hh_ih^{-1}=ch_i$ for some $c\in C$. Thus the conjugacy class of each 
	$h_i$ contains at most as many elements as $C$. This implies that 
	\[
		|H/Z(H)|=|H/\cap_{i=1}^{2k} C_H(h_i)|\leq\prod_{i=1}^{2k} (H:C_H(h_i))\leq |C|^{2k}.
	\]
	Since $H/Z(H)$ is finite, $[H,H]$ is finite. Hence  
	$[G,G]=\langle C\rangle\subseteq [H,H]$ 
	is a finite group. 
\end{proof}

\begin{theorem}[Hilton--Niroomand]
	\index{Hilton--Niroomand´s theorem}
	\label{thm:HiltonNiroomand}
	Let $G$ be a finitely generated group. If $[G,G]$ is finite and $G/Z(G)$ is generated by
	$n$ elements, then  
	\[
	|G/Z(G)|\leq |[G,G]|^n. 
	\]
\end{theorem}

\begin{proof}
	Assume that $G/Z(G)=\langle x_1Z(G),\dots,x_nZ(G)\rangle$. Let 
	\[
		f\colon G/Z(G)\to [G,G]\times\cdots\times [G,G],
		\quad
		y\mapsto ([x_1,y],\dots,[x_n,y]).
	\]
	Note that $f$ is well-defined: If $y\in G$ y $z\in Z(G)$, then $[x_i,y]=[x_i,yz]$ for all $i$. 
	Then $f(yz)=f(y)$.
	 
	The map $f$ is injective. Assume that $f(xZ(G))=f(yZ(G))$. Then 
	$[x_i,x]=[x_i,y]$ for all $i\in\{1,\dots,n\}$. For each $i$ we compute  
	\begin{align*}
		[x^{-1}y,x_i] &= x^{-1}[y,x_i]x[x^{-1},x_i]\\
		&=x^{-1}[y,x_i][x_i,x]x=x^{-1}[x_i,y]^{-1}[x_i,x]x=x^{-1}[x_i,y]^{-1}[x_i,y]x=1.
	\end{align*}
	This implies that $x^{-1}y\in Z(G)$. Indeed, since  
	every $g\in G$ can be written as $g=x_kz$ for some $k\in\{1,\dots,n\}$ and some $z\in Z(G)$, 
	it follows that 
    \[
    [x^{-1}y,g]=[x^{-1}y,x_kz]=[x^{-1}y,x_k]=1.
    \]
    Since $f$ is injective, 
	$|G/Z(G)|\leq |[G,G]|^n$. 
\end{proof}

\begin{exercise}
Prove Theorem~\ref{thm:HiltonNiroomand} from Theorem~\ref{thm:Niroomand}. 
\end{exercise}




% \section{14/03/2024}

\subsection{Super-solvable groups}

\begin{definition}
\index{Group!super-solvable}
A group $G$ is said to be \textbf{super-solvable} if there exists a sequence 
\[
G=G_0\supseteq G_1\supseteq\cdots\supseteq G_n=\{1\}
\]
of normal subgroups of $G$ such that every 
quotient $G_{i-1}/G_i$ is cyclic. 
\end{definition}

In the previous definition, we do not require the group to be finite. Hence the quotients 
could be finite cyclic groups or isomorphic to $\Z$. 

\begin{example}
The dihedral group $\D_{n}$ of order $2n$ is super-solvable, as 
\[	
\D_{n}\supseteq \langle
r\rangle\supseteq \{1\}
\]
is a sequence of normal subgroups with cyclic factors. 
\end{example}

Every solvable group is super-solvable. See Exercise~\ref{xca:solvable}.

\begin{example}
The alternating group $\Alt_4$ solvable but not super-solvable. The only 
proper non-trivial normal subgroup of $\Alt_4$ is 
	\[
	\{\id,(12)(34),(13)(24),(14)(23)\}\simeq C_2\times C_2.
	\]
Thus $\Alt_4$ does not have a sequence of normal subgroups 
with cyclic factors. 
\end{example}

\begin{exercise}
\label{xca:Aff_supersolvable}
Prove that $\Aff(\Z)$ is super-solvable. 
\end{exercise}
% aff(Z) es súper-resoluble

\begin{example}
The group $\SL_2(3)$ is solvable but not super-solvable. Here is a computer verification: 
\begin{lstlisting}
gap> IsSolvable(SL(2,3));
true
gap> IsSupersolvable(SL(2,3));
false
\end{lstlisting}
\end{example}

\begin{exercise}
\label{xca:super}
Prove the following statements: 
\begin{enumerate}
\item Every subgroup of a super-solvable group is super-solvable. 
\item Quotients of super-solvable groups are super-solvable. 
\end{enumerate}
\end{exercise}

% \begin{svgraybox}
% 	Sea $G$ un grupo súper-resoluble y sea 			
% 	\[ 
% 	G=G_0\supseteq G_1\supseteq \cdots\supseteq G_n=1 
% 	\] 
% 	una sucesión de subgrupos normales
% 	donde cada cociente $G_{i-1}/G_{i}$ es cíclico. 
% 	\begin{enumerate}
% 		\item Sea $H$ un subgrupo de $G$. Como $G$ es
% 			súper-resoluble, Sea 
% 			\[
% 			H=H\cap G_0\supseteq H\cap G_1\supseteq\cdots\supseteq H\cap G_n=1
% 			\]
% 			una sucesión de subgrupos de $H$. Cada $H\cap G_i$ es normal en $H$
% 			pues $G_i$ es normal en $G$. Fijemos $i\in\{1,\dots,n\}$ y sea
% 			$\pi_{i-1}\colon G_{i-1}\to G_{i-1}/G_{i}$ el morfismo canónico. La
% 			restricción de $\pi_{i-1}$ al subgrupo $H\cap G_{i-1}$ es un morfismo con
% 			núcleo $G_{i}\cap H$.  Al usar el teorema de isomorfismos vemos que 
% 			\[
% 			\frac{H\cap G_{i-1}}{H\cap G_{i}}\simeq \pi_{i-1}(H\cap G_i)\subseteq G_{i-1}/G_i
% 			\]
% 			es un grupo cíclico por ser subgrupo de un grupo cíclico. 
% 		\item Sea $K$ un subgrupo normal de $G$ y sea $\pi\colon G\to G/K$ el
% 			morfismo canónico. Para cada $i$ sea $Q_i=\pi(G_i)$. Cada $Q_i$ es
% 			normal en $Q_n=\pi(G_n)=G/K$ pues $G_i$ es normal en $G$. Como
% 			$G_{i-1}K=G_{i-1}(G_iK)$ para todo $i$, 
% 			el grupo
% 			\begin{align*}
% 			Q_{i-1}/Q_i
% 			&\simeq\frac{G_{i-1}/G_{i-1}\cap K}{G_i/G_i\cap K}
% 			\simeq \frac{G_{i-1}K/K}{G_{i}K/K}\\
% 			&\simeq\frac{ G_{i-1}K}{G_iK}
% 			\simeq\frac{ G_{i-1}(G_iK)}{G_iK}
% 			\simeq\frac{ G_{i-1}}{G_iK\cap G_{i-1}}
% 			\simeq\frac{ G_{i-1}/G_i}{G_iK\cap G_{i-1}/G_i}
% 			\end{align*}
% 			es cíclico por ser un cociente de un grupo cíclico.
% 	\end{enumerate}
% \end{svgraybox}

\begin{exercise}
\label{xca:directosuper}
Prove that the direct product of super-solvable groups is super-solvable. 
\end{exercise}

% \begin{svgraybox}
% 	Supongamos que $G$ admite una sucesión $G=G_0\supseteq G_1\supseteq
% 	\cdots\supseteq G_n=1$ de de subgrupos normales tales que cada cociente
% 	$G_{i-1}/G_i$ es cíclico, y que $H$ admite una sucesión $H=H_0\supseteq
% 	H_1\supseteq \cdots\supseteq H_m=1$ de subgrupos normales donde cada
% 	$H_{i-1}/H_i$ es cíclico. Consideramos la sucesión 
% 	\[
% 		1=G_0\times H_0\supseteq G_1\times H_0\supseteq\cdots\supseteq G_n\times H_0\supseteq G_n\times H_1\supseteq \cdots\supseteq G_n\times H_m=G\times H
% 	\]
% 	tiene factores cíclicos pues 
% 	cada $G_{i-1}\times H_0/G_i\times H_0\simeq G_{i-1}/G_i$ es cíclico y cada 
% 	$G_n\times H_{j-1}/G_n\times H_j$ también pues
% 	\[
% 	G_n\times H_{j-1}/G_n\times H_j
% 	\simeq \frac{GH_{j-1}/G}{GH_j/G}
% 	\simeq \frac{H_{j-1}/H_{j-1}\cap G}{H_j/H_j\cap G}\simeq H_{j-1}/H_j.
% 	\]
% \end{svgraybox}

\begin{exercise}
\label{xca:super}
Let $H$ and $K$ be normal subgroups of a group $G$ such that $G/K$ and $G/H$
are super-solvable. Prove that $G/H\cap K$ is super-solvable. 
\end{exercise}

% \begin{svgraybox}
% 	El producto directo $G/H\times G/K$ es súper-resoluble. Sea $\partial\colon
% 	G\to G/H\times G/K$, $g\mapsto (gH,gK)$.  Como $\ker\partial=H\cap K$, se
% 	tiene que $G/H\cap K\simeq\partial(G)$, que es súper-resoluble por ser un
% 	subgrupo de un grupo súper-resoluble.
% \end{svgraybox}

\begin{exercise}
\label{xca:Nciclico}
Let $N$ be a cyclic normal subgroup of $G$. If $G/N$ is super-solvable, then 
$G$ is super-solvable. 
\end{exercise}

% todo: arreglar 

% \begin{proof}
% 	Sea $\pi\colon G\to G/N$ el morfismo canónico y sea $Q=G/N$. Como $Q$ es
% 	súper-resoluble, tenemos una sucesión
% 	\[
% 		Q=Q_0\supseteq Q_1\supseteq \cdots\supseteq Q_n=\{1\}
% 	\]
% 	de subgrupos normales de $Q$ tales que cada cociente $Q_{i-1}/Q_i$ es
% 	cíclico. Cada elemento de la sucesión
% 	\[
% 	G=\pi^{-1}(Q)\supseteq\pi^{-1}(Q_1)\supseteq\cdots\supseteq \pi^{-1}(Q_n)=N\supseteq \{1\}
% 	\]
% 	es normal en $G$ (por la correspondencia) y dejamos como 
% 	ejercicio demostrar que cada cociente es cíclico. 
% % 	cada cociente es cíclico $N$ es cíclico. 
% % 	Queda como ejercicio demostrar 
% % 	y cada 
% % 	\[
% % 	\frac{\pi^{-1}(Q_j)}{\pi^{-1}(Q_{j+1})}
% % 		=\frac{Q_jN}{Q_{j+1}N}
% % 		\simeq\frac{Q_jN/N}{Q_{j+1}N/N}
% % 		\simeq\frac{Q_j(Q_{j+1}N)}{Q_{j+1}N}
% % 		\simeq\frac{Q_j/Q_{j+1}}{Q_{j+1}N\cap Q_j}
% % 	\]
% % 	es cíclico por ser cociente de un grupo cíclico.
% \end{proof}

\begin{theorem}
\label{thm:ZorCp}
Let $G$ be a super-solvable non-trivial group. Then $G$ admits a sequence 
\[
G=G_0\supseteq G_1\supseteq\cdots\supseteq G_n=\{1\}
\]
of normal subgroups 
such that every quotient $G_{i-1}/G_i$ is cyclic of prime order or isomorphic to 
$\Z$.
\end{theorem}

\begin{proof}
Let $G=G_0\supseteq G_1\supseteq\cdots\supseteq G_n=\{1\}$ be a sequence of normal subgroups
of $G$ such that every quotient $G_{i-1}/G_i$ is cyclic. Let 
$i\in\{1,\dots,n\}$ be such that $G_{i-1}/G_i\simeq C_n$ for some non-prime  
$n$ and let $\pi\colon G_{i-1}\to G_{i-1}/G_i$ be the canonical map. 
Let $p$ be a prime divisor of $n$ and $H$ be a subgroup of $G$ such that 
$\pi(H)$ is a subgroup of $G_{i-1}/G_i$ of order $p$. By the correspondence theorem, 
$G_{i}\subseteq H\subseteq G_{i-1}$. 

We claim that $H$ is normal in $G$. Let $g\in G$. Since $\pi(gHg^{-1})$ is a subgroup of order $p$ of 
the cyclic group $G_{i-1}/G_i$, $\pi(gHg^{-1})=\pi(H)$. Then 
$gHg^{-1}=G_{i}H\subseteq H$ and hence $gHg^{-1}=H$. 
% 	\[
% 	\frac{gHg^{-1}}{G_i}=\frac{G_{i}H}{G_{i}}\simeq \frac{H}{G_i\cap H}=\frac{H}{G_i}
% 	\]
% 	y entonces $gHg^{-1}\subseteq H$.  

Note that $H/G_i$ is cyclic of prime order, as 
\[
H/G_i=H/H\cap G_i\simeq \pi(H)\simeq C_p. 
\]
Moreover, $G_{i-1}/H$ is cyclic, as 
\[
G_{i-1}/H\simeq\frac{G_{i-1}/G_i}{H/G_i}
\]
is the quotient of a cyclic group. 
	
We have shown that by adding $H$ to our sequence of normal subgroups, 
we obtain a sequence with cyclic factors where 
$H/G_{i}$ is cylic of prime order. Repeating this procedure, we obtain the desired result. 
\end{proof}

Let us discuss an immediate application. 

\begin{corollary}
A finite super-solvable group admits a sequence 
of normal subgroups where each quotient is cyclic of prime order. 
\end{corollary}

% \begin{proof}
% 	Es consecuencia inmediata del teorema~\ref{theorem:ZorCp}.
% \end{proof}

We now discuss other properties of super-solvable groups. 

\begin{theorem}
\label{thm:super_structure}
Let $G$ be a super-solvable group. The following statement hold:  
\begin{enumerate}
\item If $N$ is minimal normal in $G$, then $N\simeq C_p$ for some prime number $p$.
\item If $M$ is maximal in $G$, then $(G:M)=p$ for some prime number $p$.
\item The commutator subgroup $[G,G]$ is nilpotent. 
\item If $G$ is non-abelian, there exists a normal subgroup $N\ne G$ such that
	$Z(G)\subsetneq N$.
\end{enumerate}
\end{theorem}

\begin{proof}
Let us prove the first claim. Since $G$ is super-solvable, there exists a sequence 
\[
G=G_0\supseteq G_1\supseteq
G_2\supseteq\cdots\supseteq G_n=\{1\}
\]
of normal subgroups with cyclic factors. Since 
each $G_i\cap N$ is a normal subgroup of $G$ contained in $N$, 
the minimality implies that 
each $G_i\cap N$ is either trivial or equal to $N$. Moreover, $N\cap G_0=N$ and $N\cap
G_n=\{1\}$. Let $j$ be the smallest positive integer such that $N\cap G_j=\{1\}$. 
Since $N\subseteq G_{j-1}$ (because $N\cap G_{j-1}=N$), the composition 
	\[
	N\hookrightarrow G_{j-1}\to G_{j-1}/G_j
	\]
is an injective group homomorphism, as its kernel is equal $N\cap G_{j}=\{1\}$. 
Thus $N$ is cyclic, as it is isomorphic to a subgroup of the cyclic group $G_{i-1}/G_i$. 
If $G_{i-1}/G_i\simeq\Z$, then $N\simeq\Z$, a contradiction to the fact that $N$ is minimal normal. (For example, 
$2\Z$ is characteristic subgroup of $\Z$ and hence it is normal in $G$. Thus $N$ is cyclic and finite. Hence $N\simeq C_p$.)

We now prove the second claim. Let $M$ be a maximal subgroup of $G$. If $M$ is normal in $G$, 
then $G/M$ does not contain non-trivial proper subgroups. Then 
$G/M\simeq C_p$ for some prime number $p$. Assume that $M$ is not normal in $G$. 
Let $H=\cap_{g\in G}gMg^{-1}$ and $\pi\colon G\to G/H$ be the canonical map.  
Since $\pi(M)$ is maximal in 
	$\pi(G)=G/H$ and 
	\[
		(G:M)=(G/H:M/H)=(G/H:M/H\cap M)=(\pi(G):\pi(M)),
	\]
we may assume that $M$ does not contain non-trivial normal subgroups of $G$ (if needed, 
we just replace $G$ by $G/H$). Since $G$ is super-solvable, there exists a sequence 
$G=G_0\supseteq G_1\supseteq\cdots\supseteq G_n=\{1\}$ of normal subgroups of $G$ 
with factors either cyclic of prime order or isomorphic to $\Z$. Let 
$N=G_{n-1}$. Since $N$ is cyclic, every subgroup of $N$ is characteristic 
in $N$ and hence normal in $G$. In particular, $M\cap N$ is normal in 
$G$ and therefore $M\cap N=\{1\}$. Since $M\subseteq
NM\subseteq G$, the maximality of $M$ implies that either $M=NM$ or $G=NM$.
Since $N\subseteq NM=M$ yields a contradiction, we conclude that $G=NM$.

If $N\simeq C_p$ for some prime $p$, then $(G:M)=p$ and the proof is complete. 
Assume that $N\simeq\Z$. Let $H$ be a proper subgroup of $N$. Since 
$H$ is characteristic in $N$, $H$ is normal in $G$. Since 
$M\subseteq HM\subseteq NM=G$, the maximality of $M$ implies that either $HM=M$ or 
$HM=G$. Since $HM=M$ implies $H\subseteq M\cap N=\{1\}$,
we may assume that $HM=G$. If $n\in N\setminus H$, then $n=hm$ for some 
$h\in H$ and $m\in M$. Then $h=n$, as $h^{-1}n\in N\cap M=\{1\}$, a contradiction. 

We now prove the third claim. Since $G$ is super-solvable, 
there exists a sequence
	\[
	G=G_0\supseteq G_1\supseteq\cdots\supseteq G_n=\{1\}
	\]
of normal subgroups of $G$ such that each 
$G_i/G_{i+1}$ is cyclic. For  
	$i\in\{0,\dots,n\}$, let $H_i=[G,G]\cap G_i$. Since $[G,G]$ the each 
 $G_i$ are normal in $G$, one obtains a sequence 
	\[
	[G,G]=H_0\supseteq H_1\supseteq\cdots\supseteq H_n=\{1\}
	\]
of normal subgroups of $G$. Since $H_i$ and $H_{i+1}$ are normal in $G$, 
the group $G$ acts by conjugation on $H_i/H_{i+1}$. Thus there exists a group
homomorphism 
	$\gamma\colon G\to\Aut(H_i/H_{i+1})$. Since $H_i/H_{i+1}$ is cyclic, 
	$\Aut(H_i/H_{i+1})$ is abelian. Thus $[G,G]\subseteq\ker \gamma$. Therefore 
	$[G,G]$ acts trivially by conjugation on $H_{i}/H_{i+1}$. Hence 
 	\[
	H_i/H_{i+1}\subseteq Z([G,G]/H_{i+1}).
	\]

 Finally, we prove the fourth claim. Since $G$ is non-abelian,
	$Z(G)\ne G$. Let $\pi\colon G\to G/Z(G)$ be the canonical map. The group 
	$G/Z(G)$ is super-solvable and the sequence 
	\[
	G/Z(G)=\pi(G)\supseteq \pi(G_1)\supseteq\cdots\supseteq \pi(1)=\{1\}
	\]
is a sequence of normal subgroups of $G/Z(G)$ with cyclic quotients. 
In particular, $1\ne \pi(G_1)$ is a proper normal subgroup of $G/Z(G)$. By the correspondence theorem, $\pi^{-1}(\pi(G_1))\ne G$ is a normal subgroup of 
$G$ properly containing $Z(G)$. 
\end{proof}

There are solvable groups with a non-nilpotent derived subgroup. 

\begin{example}
The group $\Sym_4$ is solvable and $[\Sym_4,\Sym_4]=\Alt_4$ is not nilpotent.
\end{example}

\begin{proposition}
\label{pro:psuper}
Let $p$ be a prime number. Every finite $p$-group is super-solvable.
\end{proposition}

\begin{proof}
Let $G$ be a minimal counterexample. We may assume that $|G|=p^n$ for some 
$n>1$ (otherwise, if $n=1$, then $G$ is trivially super-solvable). 
The group $G$ is nilpotent and contains a normal subgroup $N$ of order $p$. 
Moreover, since $|G/N|=p^{n-1}$, the group $G/N$ is super-solvable. 
Since $N$ is cyclic and $G/N$ is super-solvable, 
$G$ is super-solvable by Exercise~\ref{xca:Nciclico}.
\end{proof}

% Como todo grupo finito nilpotente es producto directo de (finitos) subgrupos de
% Sylow, cada $p$-grupo es súper-resoluble y el producto directo de súper-resolubles es súper-resoluble, 
% se obtiene el siguiente resultado:

\begin{exercise}
\label{xca:nilpotent=>supersolvable}
Prove that finite nilpotent groups are super-solvable.
\end{exercise}

% \begin{proof}
% 	Todo grupo finito nilpotente es producto directo (finito) de subgrupos de
% 	Sylow. Como cada $p$-grupo es súper-resoluble por la
% 	proposición~\ref{proposition:psuper}, el resultado se obtiene
% 	inmediatamente del ejercicio~\ref{exercise:directosuper}.
% \end{proof}

\begin{theorem}
Super-solvable groups have maximal subgroups. 	
\end{theorem}

\begin{proof} 
We proceed by induction on the length of the super-solvable series. The claim holds for groups with a super-solvable series of length one, as in this case we are dealing with cyclic groups. So let 
$G$ be a group admitting a sequence
	\[
		G=G_0\supseteq\cdots\supseteq G_k=\{1\}
	\]
and suppose the theorem holds for super-solvable groups
with super-solvable series of length $<k$. Each  
$G_{k-1}$ is normal in $G$. Let $\pi\colon G\to
	G/G_{k-1}$ be the canonical map. 
The sequence 
 	\[
		G/G_{k-1}=\pi(G)\supseteq \pi(G_1)\supseteq\cdots\supseteq\pi(G_{k-1})=\{1\}
	\]
has length 
$<k$ and proves the super-solvability of $\pi(G)$. By the inductive hypothesis, 
$G/G_{k-1}$ admits maximal subgroups. By the correspondence theorem, 
$G$ admits maximal subgroups. 
\end{proof}

Solvable or nilpotent groups do not always admit maximal subgroups. Can you give an example?

\begin{definition}
	\index{Group!que satisface la condición maximal para subgrupos}
A group $G$ satisfies the \textbf{maximal condition on subgroups} if
for every non-empty subset $\mathcal{S}$ of subgroups contains a maximal 
subgroup (i.e. a subgroup not contained in any other subgroup of $\mathcal{S}$). 
	%toda sucesión creciente
	%$S_1\subseteq S_2\subseteq S_3\subseteq\cdots$
	%de subgrupos es finita. 
	%%si todo subconjunto $\mathcal{S}$ 
	%%no vacío de subgrupos tiene un elemento maximal, es decir: existe
	%$M\in\mathcal{S}$ tal que $S\subseteq M$ para todo $S\in\mathcal{S}$.
\end{definition}

%\begin{lemma}
%	Un grupo $G$ satisface la la condición maximal para subgrupos si y sólo si
%	todo subconjunto $\mathcal{S}$ no vacío de subgrupos tiene un subgrupo
%	maximal (es decir, no contenido en ningún otro subgrupo de $\mathcal{S}$). 
%\end{lemma}

\begin{exercise}
\label{xca:MAX=fg}
A group satisfies the maximal condition on subgroups if and only if
every subgroup of $G$ is finitely generated. 
\end{exercise}

% \begin{proof}
% 	Supongamos que $G$ satisface la condición maximal para subgrupos y sea $H$
% 	un subgrupo de $G$.  Sea $\mathcal{S}$ el conjunto de subgrupos de $H$
% 	finitamente generados. Como $\mathcal{S}$ es no vacío (pues
% 	$1\in\mathcal{S}$), existe un elemento maximal $M\in\mathcal{S}$.  Sea
% 	$x\in H$. Como $\langle M,x\rangle\in\mathcal{S}$, $M=\langle M,x\rangle$ y
% 	luego $x\in M$. Como entonces $H=M$, $H$ es finitamente generado.
% 	%Supongamos que $G$ no es finitamente generado y satisface la condición maximal para subgrupos. Sea $1\ne g\in G$
% 	%y sea $S_1=\langle g_1\rangle$. Como $S_1\ne G$, existe $g_2\in G\setminus S_1$, y entonces 
% 	%$S_1\subseteq S_2=\langle x_1,x_2\rangle$. 

% 	Supongamos ahora que todo subgrupo de $G$ es finitamente generado. Si
% 	$\mathcal{S}$ es un subconjunto no vacío de subgrupos de $G$ sin elemento
% 	maximal, podemos construir una sucesión de subgrupos $S_1\subseteq
% 	S_2\subseteq\cdots$ que no se estabiliza (acá necesitamos utilizar el
% 	axioma de elección). Como la unión 
% 	\[
% 		S=\bigcup_{j\geq1}S_j 
% 	\]
% 	es un subgrupo de $G$, es finitamente generado y luego $S\subseteq S_k$
% 	para algún $k$ suficientemente grande, una contradicción.
% \end{proof}

% Una consecuencia inmediata. 

\begin{exercise}
Let $H$ be a subgroup of a group $G$. If $G$ satisfies 
the maximal condition on subgroups, then so does $H$. 
\end{exercise}

\begin{exercise}
\label{xca:max:G/N}
Let $G$ be a group and $N$ be a normal subgroup of $G$. If $G/N$ and $N$
satisfy the maximal condition on subgroups, then so does $G$.
\end{exercise}

% \begin{proof} 
% 	Sea $\pi\colon G\to G/N$ el morfismo canónico.  Sea $\mathcal{S}$ un
% 	subconjunto no vacío de subgrupos de $G$. El conjunto $\{S\cap
% 	N:S\in\mathcal{S}\}$ tiene un elemento maximal $A$ y el conjunto
% 	$\{\pi(S):S\in\mathcal{S},S\cap N=A\}$ tiene un elemento maximal $B$. Sea
% 	$S\in\mathcal{S}$ tal que $\pi(S)=B$ y $S\cap N=A$. Si $S$ no es maximal en
% 	$\mathcal{S}$, existe $T\in\mathcal{S}$ tal que $S\subseteq T$, $N\cap T=A$
% 	y $\pi(T)=B$. Sea $x\in T\setminus S$. Como $\pi(xN)=\pi(x)\in\pi(T)=B$,
% 	existe $y\in S$ tal que $xN=yN$. Luego $y^{-1}x\in N\cap T=A=N\cap S$, una
% 	contradicción pues $x\not\in S$. 
% \end{proof}

% TODO: agregar teorema de Huppert (ver por ejemplo Robinson, p. 268)
% corolario: G super si y sólo G/\Phi(G) super
% teorema de Iwasawa, Hall 342-345, 19.3
% teorema de Zappa-Ore, Duke 5 (1939), 431-460, Duke 6 (1940), 511-512

%\begin{definition}
%	\index{Grupo!que satisface la condición minimal para subgrupos}
%	Se dice que un grupo $G$ satisface la \textbf{condición minimal para
%	subgrupos} si todo subconjunto no vacío de subgrupos tiene un elemento
%	minimal.
%\end{definition}
%
%\begin{example}
%	El grupo $\Z$ no satisface la condición minimal para subgrupos pues
%	el conjunto $\{2^n\Z:n\in\N\}$ no posee elemento minimal. 
%\end{example}
%
%\begin{proposition}
%	Sea $G$ un grupo que satisface la condición minimal sobre subgrupos.
%	Entonces todo elemento de $G$ tiene orden finito.
%\end{proposition}
%
%\begin{proof}
%	Si existe $x\in G$ de orden infinito, la sucesión $\mathcal{S}$ de subgrupos 
%	\[
%	\langle x\rangle\supsetneq\langle x^2\rangle\supsetneq\langle
%	x^4\rangle\supsetneq\cdots\supsetneq\langle x^{2^k}\rangle\supsetneq\cdots
%	\]
%	tiene infinitos elementos y luego no posee un elemento minimal. 
%\end{proof}
%
%\begin{exercise}
%	\label{exercise:min:N}
%	Sea $G$ un grupo y sea $H$ un subgrupo de $G$.  Si $G$ satisface la
%	condición minimal para subgrupos entonces $H$ también. 
%\end{exercise}
%
%\begin{svgraybox}
%	Si $\mathcal{S}$ es un subconjunto no vacío de subgrupos de $H$, entonces
%	$\mathcal{S}$ posee un elemento minimal por ser un subconjunto no vacío de
%	subgrupos de $G$.
%\end{svgraybox}
%
%\begin{proposition}
%	\label{proposition:min:G/N}
%	Sea $G$ un grupo y sea $N$ un subgrupo normal de $G$.  Si $G/N$ y $N$
%	satisfacen la condición minimal para subgrupos entonces $G$ también. 
%\end{proposition}
%
%\begin{proof}
%	
%\end{proof}

\begin{proposition}
\label{pro:superfg}
Super-solvable groups satisfy the maximal condition on subgroups. In particular, 
every super-solvable group is finitely generated. 
\end{proposition}

\begin{proof}
We proceed by induction on the length of the super-solvable sequence. If the length
is one, the result holds as the group is cyclic. 
So assume the result holds for super-solvable groups with 
super-solvable series of length $\leq n-1$.  Let $G$
be a non-trivial super-solvable group and 
	\[
	G=G_0\supsetneq
	G_1\supsetneq\cdots\supsetneq G_n=\{1\}
	\]
a sequence of normal subgroups of $G$ with cyclic factors. Since 
$G_{1}$ is super-solvable (Exercise~\ref{xca:super}),
	$G_{1}$ satisfies the maximal condition on subgroups by the inductive
 hypothesis. By Exercise~\ref{xca:max:G/N}, $G$ satisfies the maximal
 condition on subgroups, as 
 $G/G_{1}$ is cyclic. 
\end{proof}

%\begin{proposition}\
%	\begin{enumerate}
%		\item Si un grupo súper-resoluble admite una serie de composición,
%			entonces es finito. 
%		\item Si un grupo súper-resoluble satisface la condición de minimal en
%			subgrupos entonces es finito.
%	\end{enumerate}
%\end{proposition}
%
%\begin{proof}
%	%Para probar la segunda afirmación obsevemos que todo cociente de $G$ es súper-resoluble 
%	%y que por el teorema~\ref{theorem:ZorCp} todo factor de la serie debe ser finito pues
%	%$\Z$ no satisface la condición minimal para subgrupos.
%\end{proof}

\begin{example}
The abelian group $\Q$ is nilpotent but not super-solvable, 
as it is not finitely generated. 
\end{example}

%	El grupo $\Sym_3$ es súper-resoluble pero no es nilpotente. 

If $G$ is a group and $x_1,\dots,x_{n+1}\in G$, 
let 
\[
[x_1,x_2\dots,x_{n+1}]=\left[ x_1,[x_2,\dots,x_{n+1}]\right],\quad
n\geq1.
\]

We will prove in Theorem~\ref{thm:super=fg} that nilpotent groups are super-solvable 
if and only if they are finitely generated. For this, we need two lemmas. 

\begin{lemma}
	\label{lem:G_n}
	Let $G$ be a finite generated group, say $G=\langle X\rangle$ for some finite set $X$.  
	For $n\geq2$, let 
 	\[
		G_n=\langle g[x_1,\dots,x_n]g^{-1}:x_1,\dots,x_n\in X,\,g\in G\rangle.
	\]
	Then $G_n=\gamma_n(G)$ for all $n\geq2$. 
\end{lemma}

\begin{proof}
	Note that each $G_n$ is normal in $G$. We proceed by induction on $n$. The case 
    $n=2$ is trivial. So let us assume that 
    $\gamma_{n-1}(G)=G_{n-1}$ for some $n\geq2$. Let $x_1,\dots,x_n\in X$. Since 
	$[x_1,\dots,x_n]\in\gamma_{n}(G)$, $G_{n-1}\subseteq\gamma_n(G)$. Let 
	$N=G_n$ and $\pi\colon G\to G/N$ be the canonical map. The group $G/N$ is finitely generated. Since 
	\[
	[\pi(x_1),[\pi(x_2),\dots,\pi(x_{n})]]=\pi([x_1,\dots,x_n])=1,
	\]
	we obtain that $\pi([x_2,\dots,x_{n}])\in Z(G/N)$. Hence 
	$\pi(g[x_2,\dots,x_n]g^{-1})=1$ for all $g\in G$. By the inductive hypothesis, 
 	\[
	\pi(\gamma_{n-1}(G))=\pi(G_{n-1})\subseteq Z(G/N).
	\]
	Since  
	\[
	\pi(\gamma_{n}(G))=\pi([G,\gamma_{n-1}(G)])=[\pi(G),\pi(\gamma_{n-1}(G))]=\{1\},
	\]
	we conclude that $\gamma_n(G)\subseteq N=G_n$.
\end{proof}

\begin{lemma}
	\label{lem:gamma_n/gamma_n+1}
	Let $G$ be a finitely generated group. Then 
 	$\gamma_n(G)/\gamma_{n+1}(G)$ is finitely generated. 
\end{lemma}

\begin{proof}
	Assume that $G=\langle X\rangle$ for some finite set $X$. Write 
	\[
	g[x_1,\dots,x_n]g^{-1}=[g,[x_1,\dots,x_n]][x_1,\dots,x_n]. 
	\]
	By Lemma~\ref{lem:G_n}, 
    $[g,[x_1,\dots,x_n]]\in \gamma_{n+1}(G)=G_{n+1}$. Then 
	\[
	g[x_1,\dots,x_n]g^{-1}\equiv [x_1,\dots,x_n]\bmod \gamma_{n+1}(G). 
	\]
	Hence $\gamma_{n}(G)/\gamma_{n+1}(G)$ is generated by the finite set 
	\[
	\{[x_1,\dots,x_n]\gamma_{n+1}(G):x_1,\dots,x_n\in X\}. \qedhere 
	\]
\end{proof}

\begin{theorem}
    \label{thm:super=fg}
    Let $G$ be a nilpotent group. Then $G$ is super-solvable if and only if 
    $G$ is finitely generated. 
\end{theorem}

\begin{proof}
    If $G$ is super-solvable, it is then finitely generated by 
    Proposition~\ref{pro:superfg}.  
    
    Now assume that the nilpotent group $G$ is finitely generated. By Lemma~\ref{lem:gamma_n/gamma_n+1}, 
    each quotient $\gamma_{n}(G)/\gamma_{n+1}(G)$ is finitely generated, say by 
    the elements $y_1,\dots,y_m$. Let $\pi\colon G\to G/\gamma_{n+1}(G)$ the canonical map. 
    For $j\in\{1,\dots,m\}$, let 
    \[
    K_j=\langle \gamma_{n+1}(G),y_1,\dots,y_j\rangle.
    \]
    Since $[G,K_j]\subseteq [G,\gamma_n(G)]=\gamma_{n+1}(G)$, 
    we obtain that $\pi(K_j)$ is central in $\pi(G)$. Thus $\pi(K_j)$ is normal
    in $\pi(G)$. Hence $K_j$ is normal in $G$. Each quotient $K_j/K_{j-1}$
    is cyclic and generated by $y_jK_{j-1}$. Therefore, in between $\gamma_n(G)$ and 
    $\gamma_{n+1}(G)$, we have constructed a sequence of normal subgroups of $G$ 
    with cyclic factors. Since $G$ is nilpotent, there exists an integer $c$ such that 
    $\gamma_{c+1}(G)=\{1\}$. Hence $G$ is super-solvable. 
\end{proof}

\begin{corollary}
	\label{cor:nilpotente=>max}
    Every finitely generated nilpotent group satisfies the maximal condition on subgroups. 
\end{corollary}

\begin{proof}
    This is an immediate consequence of Proposition~\ref{pro:superfg} and 
    Theorem~\ref{thm:super=fg}.  
\end{proof}

\begin{theorem}
    Let $G$ be a nilpotent finitely generated group. Then $T(G)$ is finite. 
\end{theorem}

\begin{proof}
    Since $G$ is nilpotent, $G$ satisfies the maximal condition on subgroups 
    (Corollary~\ref{cor:nilpotente=>max}). Thus 
	every subgroup of $G$ is finitely generated. Since 
    $T(G)$ is a subgroup (Theorem~\ref{thm:T(nilpotent)}), it is a torsion finitely generated group. 
	Hence $T(G)$ is finite by Theorem~\ref{thm:T(G)finito}.
\end{proof}




% \section{21/03/2024}

\subsection{The Schur--Zassenhaus theorem}

% Para leer este capítulo es conveniente haber entendido el capítulo \ref{derivaciones}, ya
% que demostraremos el teorema de Schur--Zassenhaus gracias al uso de algunos trucos que
% involucran derivaciones. También es necesario utilizar el subgrupo de Frattini, capítulo \ref{Frattini}. 
% Daremos una aplicación 
% del teorema de Schur--Zassenhaus a grupos súper-resolubles, estudiados en el capítulo \ref{super_resoluble}. 

Recall that a group $Q$ \textbf{acts by automorphisms} on a group $K$ if 
there exists a map $Q\times K\to K$, $(q,k)\mapsto q\cdot k$, 
such that 
\begin{enumerate}
    \item $1\cdot a=a$ for all $a\in K$, 
    \item $x\cdot (y\cdot a)=(xy)\cdot a$ for all $x,y\in Q$ and $a\in K$, 
    \item $x\cdot 1=1$ for all $x\in Q$, and 
    \item $x\cdot (ab)=(x\cdot a)(x\cdot b)$ for all $x\in Q$ and $a,b\in K$, 
\end{enumerate}
For example, if $K$ is a normal subgroup of $G$, 
then $G$ acts by automorphisms on $K$ by conjugation. 

\begin{definition}
\index{1-cocycle}
Let $Q$ and $K$ be groups, where $Q$ acts by automorphisms on $K$. 
A map 
$\varphi\colon Q\to K$ is said to be a \textbf{1-cocycle} if 
\[
	\varphi(xy)=\varphi(x)(x\cdot\varphi(y))
\]
for all $x,y\in Q$.  
\end{definition}

Let $Q$ and $K$ be groups, where $Q$ acts by automorphisms on $K$. 
The set of 1-cocycles $Q\to K$ will be denoted by 
\[
Z^1(Q,K)=\{\delta\colon Q\to K:\text{$\delta$ is a 1-cocycle}\}.
\]

\begin{example}
Let $Q$ be a group acting by automorphisms on $K$. 
The semidirect product $K\rtimes Q$ 
is a group $G$ that contains a normal subgroup isomorphic to $K$ 
and a subgroup isomorphic to such that 
$G=KQ$ and $K\cap Q=\{1\}$. Under the obvious identifications, 
$Q$ acts on $K$ by conjugation. For each $k\in K$, the map 
$Q\to K$, $x\mapsto [k,x]=kxk^{-1}x^{-1}$, is a 1-cocycle. 
\end{example}

\begin{exercise}
\label{xca:1cocycle}
Let $\varphi\colon Q\to K$ be a 1-cocycle. Prove the following statements:
\begin{enumerate}
	\item $\varphi(1)=1$.
	\item $\varphi(y^{-1})=(y^{-1}\cdot\phi(y))^{-1}=y^{-1}\cdot\phi(y)^{-1}$.
	\item The set $\ker\varphi=\{x\in Q:\varphi(x)=1\}$ is a subgroup of $Q$. 
\end{enumerate}
\end{exercise}

\begin{lemma}
\label{lem:1cocycle}
Let $G$ be a group with a normal subgroup $N$. 
If $\varphi\colon G\to N$ is a 1-cocycle (where $G$ acts on $N$ by conjugation)
with kernel 
\[
K=\ker\varphi=\{g\in G:\varphi(g)=1\}, 
\]
then 
$\varphi(x)=\varphi(y)$ if and only if $xK=yK$. In particular,
$(G:K)=|\varphi(G)|$. 
\end{lemma}

\begin{proof}
If $\varphi(x)=\varphi(y)$, then, since  
\[
\varphi(x^{-1}y)
=\varphi(x^{-1})(x^{-1}\cdot\varphi(y))
=\varphi(x^{-1})(x^{-1}\cdot\varphi(x))
=\varphi(x^{-1}x)=\varphi(1)
=1,
\]
we obtain that $xK=yK$. Conversely, if $x^{-1}y\in K$, then, since 
\[
1=\varphi(x^{-1}y)=\varphi(x^{-1})(x^{-1}\cdot \varphi(y)),
\]
we obtain that $\varphi(y)=x\cdot\varphi(x^{-1})^{-1}$. We conclude that 
$\varphi(x)=\varphi(y)$.

The second claim now is clear, as $\varphi$ is constant in each coclass of $K$ 
and takes $(G:K)$ different values. 
\end{proof}

\begin{lemma}
	\label{lem:d}
	Let $G$ be a finite group, $N$ be an abelian normal subgroup of $G$ and $S$, $T$ and $U$
    be transversals of $N$ in $G$. Let 
	\[
	d(S,T)=\prod st^{-1}\in N,
	\]
	where the product runs over all elements $s\in S$ and $t\in T$ such that 
	$sN=tN$. The following statements hold: 
	\begin{enumerate}
		\item $d(S,T)d(T,U)=d(S,U)$.
		\item $d(gS,gT)=gd(S,T)g^{-1}$ for all $g\in G$.
		\item $d(nS,S)=n^{(G:N)}$ for all $n\in N$.
	\end{enumerate}
\end{lemma}

\begin{proof}
	If $s\in S$, $t\in T$ and $u\in U$ are such that $sN=tN=uN$, then, since $N$ is 
	abelian and $(st^{-1})(tu^{-1})=su^{-1}$, we obtain that 
	\[
		d(S,T)d(T,U)=\prod (st^{-1})(tu^{-1})=\prod su^{-1}=d(S,U).
	\]

	Since $sN=tN$ if and only if $gsN=gtN$ for all $g\in G$, 
	\[
	g\left(\prod st^{-1}\right)g^{-1}=\prod gst^{-1}g^{-1}=\prod (gs)(gt)^{-1}=d(gS,gT).
	\]

	Finally, since $N$ is normal in $G$, $nsN=sN$ for all $n\in N$. Thus 
	\[
		d(nS,S)=\prod (ns)s^{-1}=n^{(G:N)}.\qedhere
	\]
\end{proof}

Recall that a subgroup $K$ of $G$ admits a \textbf{complement} $Q$ 
if $G$ factorizes as 
$G=KQ$ with $K\cap Q=\{1\}$. 
A typical example is the semidirect product $G=K\rtimes Q$, where $K$ is a normal subgroup of 
$G$ and $Q$ is a subgroup of $G$ such that $K\cap Q=\{1\}$. 

\begin{exercise}
\label{xca:complementos}
Let $Q$ act by automorphisms on $K$. Prove that there is a bijection 
between the set of complements of $K$ in $K\rtimes Q$ and the set 
$Z^1(Q,K)$.
\end{exercise}

% \begin{sol}{xca:complementos}
% 	El grupo $Q$ actúa en $K$ por conjugación, entonces $\delta\in\Der(Q,K)$ si
% 	y sólo si $\delta(xy)=\delta(x)x\delta(y)x^{-1}$, $x,y\in Q$. En este caso,
% 	las fórmulas del ejercicio anterior quedan así:
% 	$\delta(1)=1$, $\delta(x^{-1})=x^{-1}\delta(x)^{-1}x$.
	
% 	Sea $\mathcal{C}$ el conjunto de complementos de $K$ en $K\rtimes Q$.  Sea
% 	$C\in\mathcal{C}$. Si $x\in Q$, sabemos que 
% 	existen únicos $k\in K$ y $c\in C$ tales que $x=k^{-1}c$. Queda bien
% 	definida entonces la función $\delta_C\colon Q\to K$, $x\mapsto k$. Vale
% 	que $\delta(x)x=c\in C$. 
	
% 	Veamos que $\delta_C\in\Der(Q,K)$. Si $x,x_1\in Q$, escribimos $x=k^{-1}c$
% 	y $x_1=k_1^{-1}c_1$, donde $k,k_1\in K$ y $c,c_1\in C$. Como $K$ es normal
% 	en $K\rtimes Q$, podemos escribir a $xx_1$ como $xx_1=k_2c_2$, donde
% 	$k_2=k^{-1}(ck_1^{-1}c^{-1})\in K$, $c_2=cc_1\in C$. Luego 
% 	\[
% 		\delta(xx_1)xx_1=cc_1=\delta(x)x\delta(x_1)x_1
% 	\]
% 	implica que $\delta(xx_1)=\delta(x)x\delta(x_1)x^{-1}$. 
% 	Tenemos así una función $F\colon\mathcal{C}\to\Der(Q,K)$, $F(C)=\delta_C$.

% 	Vamos a construir ahora $G\colon\Der(Q,K)\to\mathcal{C}$. 
% 	Para
% 	cada $\delta\in\Der(Q,K)$ vamos a definir un complemento $\Delta$ de $K$ en $K\rtimes Q$: 
% 	\[
% 	\Delta=\{\delta(x)x:x\in Q\}.
% 	\]

% 	Veamos que $\Delta$ es un subgrupo de $K\rtimes Q$. Como $\delta(1)=1$,
% 	$1\in X$. Si $x,y\in Q$ entonces
% 	$\delta(x)x\delta(y)y=\delta(x)x\delta(y)x^{-1}xy=\delta(xy)xy\in \Delta$.
% 	Por último si $x\in Q$ entonces
% 	$(\delta(x)x)^{-1}=x^{-1}\delta(x)^{-1}xx^{-1}=\delta(x^{-1})x^{-1}$.
	
	
% 	Veamos que $\Delta\cap K=\{1\}$. Si $x\in Q$ es tal que $\delta(x)x\in K$
% 	entonces, como $\delta(x)\in K$, $x\in K\cap Q=\{1\}$. Si $g\in G$ entonces
% 	existen únicos $k\in K$, $x\in Q$ tales que $g=kx$. Escribimos
% 	$g=k\delta(x)^{-1}\delta(x)x$. Como $k\delta(x)^{-1}\in K$ y $\delta(x)x\in
% 	\Delta$, se concluye que $G=K\Delta$. Queda bien definida entonces la
% 	función $G\colon\Der(Q,K)\to\mathcal{C}$, $G(\delta)=\Delta$.

% 	Veamos ahora que $G\circ F=\id_{\mathcal{C}}$. 
% 	Sea $C\in\mathcal{C}$. Entonces 
% 	\[
% 	G(F(C))=G(\delta_C)=\{\delta_C(x)x:x\in
% 	Q\}=C,
% 	\]
% 	por construcción. (Vimos que $\delta_C(x)x\in C$. Recíprocamente,  si $c\in
% 	C$, escribimos $c=kx$ para únicos $k\in K$, $x\in Q$ y luego $x=k^{-1}c$
% 	que implica $c=\delta_c(x)x$.)

% 	Por último veamos que $F\circ G=\id_{\Der(Q,K)}$. Sea $\delta\in\Der(Q,K)$.
% 	Entonces 
% 	\[
% 	F(G(\delta))=F(\Delta)=\delta_{\Delta}.
% 	\]
% 	Queremos demostrar que $\delta_\Delta=\delta$.  Sea $x\in Q$. Existe
% 	$\delta(y)y\in\Delta$ para algún $y\in Q$ tal que $x=k^{-1}\delta(y)y$.
% 	Luego $\delta_{\Delta}(x)x=\delta(y)y$ y luego $\delta(x)=\delta(y)$ por la
% 	unicidad de la escritura.
% \end{sol}

We are now ready to prove the first version of the
Schur--Zassenhaus theorem. 

\begin{theorem}[Schur--Zassenhaus]
	\index{Schur--Zassenhaus!theorem}
	\label{thm:SchurZassenhaus:abeliano}
	Let $G$ be a finite group and $N$ be an abelian normal subgroup of $G$. If 
 	$|N|$ and $(G:N)$ are coprime, then $N$ admits a complement in $G$. Moreover, 
    all complements of $N$ are conjugate. 
\end{theorem}

\begin{proof}
	Let $T$ be a transversal of $N$ in $G$ and $\theta\colon G\to N$,
	$\theta(g)=d(gT,T)$. Since $N$ is abelian, Lemma~\ref{lem:d} implies that 
	$\theta$ is a 1-cocycle, where $G$ acts on $N$ by conjugation: 
	\begin{align*}
		\theta(xy)&=d(xyT,T)
		=d(xyT,xT)d(xT,T)\\
		&=(xd(yT,T)x^{-1})d(xT,T)=(x\cdot\theta(y))\theta(x).
	\end{align*}

	\begin{claim}
		$\theta|_N\colon N\to N$ is surjective. 
	\end{claim}

	If $n\in N$, Lemma~\ref{lem:d} implies that 
	$\theta(n)=d(nT,T)=n^{(G:N)}$. Since $|N|$ and $(G:N)$ are coprime, 
	there exist $r,s\in\Z$ such that $r|N|+s(G:N)=1$. Thus 
	\[
		n=n^{r|N|+s(G:N)}=(n^s)^{(G:N)}=\theta(n^s).
	\]

	Let $H=\ker\theta$. We prove that $H$ is a complement of $N$. 
	By Exercise~\ref{xca:1cocycle}, $H$ is a subgroup of $G$. By Lemma~\ref{lem:1cocycle}, 
	\[
		|N|=|\theta(G)|=(G:H)=\frac{|G|}{|H|}. 
	\]
	
	Since $N\cap H$ is a subgroup of $N$ and a subgroup of $H$, $N\cap H=\{1\}$, as the numbers 
	$|N|$ and $(G:N)=|H|$ are coprime. Since $|NH|=|N||H|=|G|$, we conclude that 
	$G=NH$. Hence $H$ is a complement of~$N$. 

	We now prove that two complements of $N$ are conjugate. 
	Let $K$ be a complement of $N$ in $G$. Since $NK=G$ and $N\cap K=\{1\}$, $K$ is a transversal of $N$. 
 Let $m=d(T,K)\in N$. Since the restriction map $\theta|_N$ is surjective, 
	there exists $n\in N$ such that $\theta(n)=m$. By Lemma~\ref{lem:d}, 
	\[
	kmk^{-1}=kd(T,K)k^{-1}=d(kT,kK)=d(kT,K)=d(kT,T)d(T,K)=\theta(k)m
	\]
    for all $k\in K$. 
	Since $N$ is abelian,
	$\theta(n^{-1})=m^{-1}$. Thus 
	\begin{align*}
		\theta(nkn^{-1})&=\theta(n)n\theta(kn^{-1})n^{-1}
		=m\theta(kn^{-1})\\
		&=m\theta(k)k\theta(n^{-1})k^{-1}
		=m\theta(k)km^{-1}k^{-1}=1.
	\end{align*}
	Therefore $nKn^{-1}\subseteq H=\ker\theta$. Since 
	$|K|=(G:N)=|H|$, we conclude that $nKn^{-1}=H$.
\end{proof}


\begin{theorem}[Schur--Zassenhaus]
	\index{Schur--Zassenhaus!theorem}
	\label{thm:SchurZassenhaus}
	Let $G$ be a finite group and $N$ be a normal subgroup of $G$. If $|N|$ and 
	$(G:N)$ are coprime, then $N$ admits a complement in $G$. 
\end{theorem}

\begin{proof}
	We proceed by induction on $|G|$. If there is a proper subgroup $K$ of 
	$G$ such that $NK=G$, then, since $(K:K\cap N)=(G:N)$ and $|N|$ are coprime,
	$(K:K\cap N)=(G:N)$ is coprime with $|K\cap N|$. Since $K\cap N$ is normal in $K$,
	the inductive hypothesis implies that $K\cap N$ admits a complement in $K$. Thus there exists 
    a subgroup $H$ of $K$ such that $|H|=(K:K\cap N)=(G:N)$. 

    Assume that there is no proper subgroup $K$ of $G$ such that 
	$NK=G$. We may assume that $N\ne\{1\}$ (otherwise, $G$ would be a complement of $N$ in $G$). Since $N$ is contained in 
    every maximal subgroup of $G$ (because, if there is a maximal subgroup $M\subsetneq G$ such that 
	$N\not\subseteq M$, then $NM=G$), it follows that $N\subseteq\Phi(G)$. By Frattini's theorem~\ref{thm:Frattini}, 
    $\Phi(G)$ is nilpotent. Thus $N$ is nilpotent and then $Z(N)\ne\{1\}$. Let $\pi\colon G\to
	G/Z(N)$ be the canonical map. Since $N$ is normal in $G$ and $Z(N)$ is characteristic in $N$, 
    $Z(N)$ is normal in $G$.  Moreover, 
	\[
	(\pi(G):\pi(N))=\frac{|\pi(G)|}{|\pi(N)|}=\frac{|G/Z(N)|}{|N/N\cap Z(N)|}=(G:N)
	\]
	is coprime with $|N|$. Then $(\pi(G):\pi(N))$ is coprime with $|\pi(N)|$. By the inductive hypothesis, 
	$\pi(N)$ admits a complement in $G/Z(N)$, say $\pi(K)$
	for some subgroup $K$ of $G$. Hence $G=NK$, as 
	$\pi(G)=\pi(N)\pi(K)=\pi(NK)$. 
	Since $K=G$ (because there is no $K$ such that $G=NK$), 
	$\pi(N)$ is abelian, as 
	\[
		\pi(Z(N)=\pi(N)\cap\pi(K)=\pi(N)\cap\pi(G)=\pi(N).
	\]
	Thus $N\subseteq Z(N)$ is abelian. By Theorem~\ref{thm:SchurZassenhaus:abeliano}, the subgroup $N$ 
    admits a complement. 
\end{proof}

\begin{theorem}[Schur--Zassenhaus conjugation theorem]
	\label{thm:SchurZassenhaus:conjugation}
    Let $G$ be a finite group and $N$ be a normal subgroup of $G$ such that 
    $|N|$ and 
	$(G:N)$ are coprime. If either $N$ or $G/N$ is solvable, then 
    all complements of $N$ in $G$ are conjugate. 
\end{theorem}

%\begin{proof}
%	Sea $G$ un contraejemplo minimal, es decir: existen complementos $K_1$ y
%	$K_2$ a $N$ en $G$ que no son conjugados.
%
%	\begin{claim}
%		$N$ es minimal en $G$.
%	\end{claim}
%
%	Si $M\subseteq N$ es minimal normal en $G$, $M\ne1$ pues $N\ne1$. Sea
%	$\pi\colon G\to G/M$ el morfismo canónico. El grupo $\pi(G)$ contiene un
%	subgrupo normal $\pi(N)$ de índice coprimo con $|\pi(N)|$. Además
%	$\pi(K_1)$ y $\pi(K_2)$ complementan a $\pi(N)$. Como $|G|$ es minimal,
%	$\pi(K_1)$ y $\pi(K_2)$ son conjugados en $\pi(G)$, es decir: existe $x\in G$ tal que 
%	$\pi(K_1)=\pi(xK_2x^{-1})$.
%
%\end{proof}

\begin{proof}
	Let $G$ be a minimal counterexample to the theorem, that is there are complements $K_1$ and 
	$K_2$ of $N$ in $G$ such that $K_1$ and $K_2$ are not conjugate. 

	\begin{claim}
		Every subgroup $U$ of $G$ satisfies the assumptions of the theorem with respect to the normal subgroup 
        $U\cap N$.
%		Sea $U$ un subgrupo de $G$. Entonces $U$ satisface las hipótesis del
%		teorema con respecto al subgrupo normal $U\cap N$. Si $U$ contiene un
%		complemento $H$ para $N$ en $G$, entonces $H$ complementa a $U\cap N$
%		en $U$.
	\end{claim}
	
	Since $N$ is normal in $G$, $U\cap N$ is normal in $U$. Moreover, $|U\cap N|$ and 
	$(U:U\cap N)$ are coprime, as $|U\cap N|$ divides $|N|$ and $(U:U\cap
	N)=(UN:N)$ divides $(G:N)$.  If $G/N$ is solvable, then $U/U\cap N$
	is solvable, as $U/U\cap N$ is isomorphic to a subgroup of $G/N$. If $N$ is 
	solvable, then so is $U\cap N$.
%
%	Como $|H|$ divide a $|U|$ y $|H|$ es coprimo con $|U\cap N|$, se tiene que
%	$|H|$ divide a $(U:U\cap N)$. Como además $(U:U\cap N)$ divide a
%	$(G:N)=|H|$, se concluye que $|H|=|U:U\cap N|$. Luego $H$ complementa a
%	$U\cap N$ en $U$.

	\begin{claim}
		If there is a normal subgroup $L$ of $G$ such that $\pi(N)$ is normal in 
		$\pi(G)$, where $\pi\colon G\to G/L$ is the canonical map, then 
    	$\pi(G)$ satisfies the theorem's assumptions with respect to $\pi(N)$.
		In this case, if $H$  is a complement of $N$ in $G$, then $\pi(H)$ 
		is a complement of $\pi(N)$ in $\pi(G)$.
	\end{claim}

	If $N$ is solvable, then so is $\pi(N)$. If $G/N$ is solvable, then so is 
	$\pi(G)/\pi(N)\simeq G/NL$. Moreover, 
	$(\pi(G):\pi(N))=\frac{|G/L|}{|N/N\cap L|}$ divides $(G:N)$. 
	
	If $H$ is a complement of $N$ in $G$, $|\pi(H)|$ and $|\pi(N)|$ are 
	coprime. Then $\pi(H)$ is a complement of $\pi(N)$, as 
	$\pi(G)=\pi(N)\pi(H)=\pi(NH)$ and 
	$\pi(N)\cap\pi(H)=\{1\}$. 

	\begin{claim}
		$N$ is minimal normal in $G$.
	\end{claim}

	Let $M\ne\{1\}$ be a normal subgroup of $G$ such that $M\subseteq N$. Let $\pi\colon G\to G/M$ be the canonical map. 
	Then $\pi(G)$ satisfies the theorem's assumptions with respect to the normal subgroup 
	$\pi(N)$. By the minimality of $|G|$, there exists 
	$x\in G$ such that $\pi(xK_1x^{-1})=\pi(K_2)$. The subgroup 
	$U=MK_2$ satisfies the theorem's assumptions with respect to the normal subgroup 
	$U\cap N$. Since $xK_1x^{-1}\cup K_2\subseteq U$,
	we conclude that both $xK_1x^{-1}$ and $K_2$ complement $U\cap N$ in $U$.
	Hence $MK_2=G$, as $xK_1x^{-1}$ and $K_2$ are not conjugate and $G$ is a minimal counterexample. 
	minimal. Therefore $M=N$, as 
	\[
		\frac{|K_2|}{|M\cap K_2|}=(MK_2:M)=(G:M)=\frac{|NK_2|}{|M|}=(N:M)|K_2|.
	\]

	\begin{claim}
		$N$ is not solvable and $G/N$ is solvable. 
	\end{claim}
	
	Otherwise, by Lemma~\ref{lem:minimal_normal}, $N$ is abelian (because it is minimal normal). This contradicts
    Theorem~\ref{thm:SchurZassenhaus:abeliano}, as it states that 
	$K_1$ and $K_2$ are conjugate. 
 
	\medskip
	Let $p\colon G\to G/N$ be the canonical map and $S$ be a subgroup such that $p(S)$
	is minimal normal in $p(G)=G/N$.  By Lemma~\ref{lem:minimal_normal},
	$p(S)$ is a $p$-group for some prime number $p$. Since $G=NK_1=NK_2$ and $N\subseteq
	S$, Dedekind's lemma~\ref{lem:Dedekind} implies that 
	\[
	S=N(S\cap K_1)=N(S\cap K_2).
	\]
	Hence both $S\cap K_1$ and $S\cap K_2$
	complement $N$ in $S$. Since $p(S)=p(S\cap K_1)=p(S\cap K_2)$  is a $p$-group, 
 	$p$ divides $|S|$. The theorem's assumptions hold for $S$ with respect to the normal subgroup $N$, 
    so $|N|$ and $(S:N)$ are coprime. If $p\mid |N|$, then 
	$p\nmid (S:N)=|S\cap K_1|=|S\cap K_2|$, a contradiction. Thus $p\nmid |N|$ and 
	hence $p\nmid |S|$. This implies that both $S\cap K_1$ and $S\cap K_2$ are Sylow 
	$p$-subgroups of $S$, as 
	\[
		|S\cap K_1|=(S:N)=|S\cap K_2|.
	\]
	By Sylow's theorem, there exists $s\in
	S$ such that 
    \[
	S\cap sK_1s^{-1}=S\cap K_2.
	\]
	In particular, $S\ne G$ by the minimality of $G$.
	Let 
	\[
		L=S\cap K_2=S\cap sK_1s^{-1}\ne\{1\}.
	\]
	Since $S$ is normal in $G$, $sK_1s^{-1}\cup K_2\subseteq N_G(L)$ (because $L$
	is both normal in $sK_1s^{-1}$ and in $K_2$). The subgroups $sK_1s^{-1}\subseteq
	N_G(L)$ and $K_2\subseteq N_G(L)$ complement $N\cap N_G(L)$ in $N_G(L)$. Hence 
	$N_G(L)=G$ by the minimality of $G$ (if $N_G(L)\ne G$, then both 
	$sK_1s^{-1}$ and $K_2$ are conjugate in $G$ because they are conjugate in  $N_G(L)$). Therefore 
	$L$ is normal in $G$. 
	
	Let $\pi_L\colon G\to G/L$ be the canonical map. Since both 
	$\pi_L(K_1)$ and $\pi_L(K_2)$ complement $\pi_L(N)$ in $\pi_L(G)$, the minimality of 
	$|G|$ implies that there exists $g\in G$ such that $\pi_L(gK_1g^{-1})=\pi_L(K_2)$, that is 
	there exists $g\in G$ such that $(gK_1g^{-1})L=K_2L$.  Hence $gK_1g^{-1}\cup
	K_2\subseteq \langle K_2,L\rangle=K_2$, because $L\subseteq K_2$. In conclusion, 
	$gK_1g^{-1}=K_2$, a contradiction to the minimality of $|G|$. 
%	Sea $L$ un subgrupo maximal normal de $G$ tal que $N\subseteq L$. Por
%	definición $L\ne G$. Como $L\cap K_1$ y $L\cap K_2$ complementan a $N$ en
%	$L$, la minimalidad de $G$ implica que existe $x\in G$ tal que 
%	\[ 
%	L\cap K_2=x(L\cap K_1)x^{-1}=L\cap xK_1x^{-1}.
%	\]
%	Sea $D=L\cap K_2$. Como $L$ es normal en $G$, $D$ es normal en $K_2$ y en
%	$xK_1x^{-1}$. Como $K_2$ y $xK_1x^{-1}$ son complementos para 
%	$N$ en $G$ y además
%	$xK_1x^{-1}\cup K_2\subseteq N_G(D)$, la minimalidad de $G$ implica que $N_G(D)=G$.
%	Si $N$ es resoluble, $N\ne1$ (pues de lo contrario $G=H=K$ y no hay nada
%	para demostrar). Sea $L\subseteq N$ un subgrupo minimal normal de $G$. Por
%	el lema~\ref{lemma:minimal_normal}, $L$ es abeliano\dots
%
%	Si $G/N$ es resoluble,\dots 
\end{proof}


By the Feit--Thompson theorem, in the previous theorem 
we do not need to assume that either $N$ or $G/N$ is solvable. Since every group of odd order is solvable 
and $|N|$ and $(G:N)$ are coprime, one of these groups should have odd order. 

\begin{theorem}
	\label{thm:solvable_maximal}
	Let $G$ be a finite solvable group and $p$ a prime number dividing $|G|$. There exists a maximal 
    subgroup $M$ of $G$ of index a power of $p$. 
\end{theorem}

\begin{proof}
	We proceed by induction on $|G|$. If $G$ is a $p$-group, the result clearly holds. So we may assume that $|G|$ is divisible by at least two different prime numbers. 
    Let $p$ be a prime dividing $|G|$, $N$ be a minimal normal subgroup of $G$ and 
    $\pi\colon G\to G/N$ be the canonical map. Since $G$ is solvable, by Lemma~\ref{lem:minimal_normal}, 
    $N$ is a $q$-group for some prime $q$. Since $G/N$ is solvable, if $p$ divides 
	$(G:N)$, then, by the inductive hypothesis, $G/N$ has a maximal subgroup 
 	$M_1$ of index a power of $p$. By the correspondence theorem, 
  $M=\pi^{-1}(M_1)$ is a maximal subgroup of $G$ of index a power of $p$. 
  $p$. If $p$ does not divide $(G:N)$, then $p$ divides $|N|$. Thus 
	$N\in\Syl_p(G)$. Since $N$ is normal in $G$ and $|N|$ and $|G/N|$ are coprime, by 
	Schur--Zassenhaus theorem~\ref{thm:SchurZassenhaus}, 
	there exists a complement $K$ of $N$ in $G$, that is $G=NK$ and $N\cap K=\{1\}$. Let 
	$M$ be a maximal subgroup containing $K$. Then $(G:M)$ is a power of $p$. 
\end{proof}

We now discuss an application to finite super-solvable groups. 

\begin{definition}
	\index{Group!lagrangian}
	A finite group $G$ is said to be \textbf{lagrangian} if for each $d$ dividing $|G|$ 
	there exists a subgroup of $G$ of order $d$.
\end{definition}

The group $\Alt_4$ is not lagrangian, as it has no subgroups of order six. 

\begin{theorem}
	Every finite super-solvable group is lagrangian. 
\end{theorem}

\begin{proof}
	Let $p$ be a prime number dividing $|G|$. Since subgroups of super-solvable groups are super-solvable, it is enough to 
    show that there exists a subgroup of index $p$. 
	Since $G$ is solvable, there exists a maxima subgroup $M$ of index 
	$p^{\alpha}$ by Theorem~\ref{thm:solvable_maximal}. Since maximal subgroups of super-solvable groups have prime index 
    by Theorem~\ref{thm:super_structure}, we conclude that $\alpha=1$.
\end{proof}

See \cite{MR294497} for an elementary proof. 

\subsection{*Hall's theory for solvable groups}

As an application of the Schur--Zassenhaus theorem, 
we present Hall's theory of solvable groups. 
For an elementary presentation, see \cite{MR600654}. 

\begin{definition}
\index{$\pi$-number}
\index{$\pi$-group}
\index{$\pi$-subgroup}
Let $G$ be a finite group and $\pi$ be a set of prime numbers. We say that 
$G$ is a \textbf{$\pi$-group} if every prime dividing $|G|$ belongs to $\pi$. 
Similarly, a $\pi$-subgroup of $G$ is a subgroup of $G$ that is also a $\pi$-group.  
\end{definition}

For a set $\pi$ of prime numbers, 
we define a $\pi$-number as an integer whose prime divisors 
belong to $\pi$. The set of prime numbers not belonging to $\pi$ will be denoted 
as $\pi'$. Thus a $\pi'$-number is an integers not divisible by 
the prime numbers of $\pi$. 

\begin{definition}
	\index{Hall!subgroup}
	Let $G$ be a group and $\pi$ be a set of prime numbers. A subgroup $H$ of $G$ 
    is a \textbf{Hall $\pi$-subgroup} if $H$ is a $\pi$-subgroup of $G$ and 
    $(G:H)$ is a $\pi'$-number.
\end{definition}

We now prove that a finite solvable group of order $nm$ with $\gcd(n,m)=1$ 
always admits a subgroup of order $m$. 

\begin{theorem}[Hall's existence theorem]
	\index{Hall's existence!theorem}
 	\label{theorem:HallE}
	Let $\pi$ be a set of prime numbers and $G$ be a finite solvable group. 
    Then $G$ has a Hall $\pi$-subgroup. 
\end{theorem}

\begin{proof}
	Assume that $|G|=nm>1$ and $\gcd(n,m)=1$. We want to show that $G$ admits a
	subgroup of order $m$. We proceed by induction on $|G|$. Let $K$ be a minimal
	normal subgroup of $G$ and $\pi\colon G\to G/N$ be the canonical map. Since $G$
	is solvable, $K$ is a $p$-group (Lemma~\ref{lem:minimal_normal}).
	
	There are two cases to consider. Assume first that $p$ divides $m$. Since
	$|G/K|<|G|$, the inductive hypothesis and the correspondence theorem imply that
	there exists a subgroup $J$ of $G$ containing $K$ such that $\pi(J)$ is a
	subgroup of 
    $\pi(G)=G/K$ of order $m/|K|$. Then $J$ has order $m$, as 
    \[
	m/|K|=|\pi(J)|=\frac{|J|}{|K\cap J|}=(J:K).
	\]

	Assume now that $p$ does not divide $m$. By the inductive hypothesis and the
	correspondence theorem, there exists a subgroup $H$ of $G$ containing $K$ such
	that $\pi(H)$ is a subgroup of $G/K$ of order $m$. Since $|H|=m|K|$, $K$ is
	normal in $H$ and $|K|$ is coprime with $|H:K|$, the Schur--Zassenhaus theorem
	(Theorem~\ref{thm:SchurZassenhaus}) implies that there exists a complement $J$
	of $K$ in $H$. Hence $J$ is a subgroup of $G$ such that $|J|=m$.
\end{proof}

\begin{example}
	The group $\Alt_5$ contains a Hall $\{2,3\}$-subgroups isomorphic to 
 	$\Alt_4$.
\end{example}

\begin{example}
	The simple group $\PSL_3(2)$ of order $168$ does not contain Hall $\{2,7\}$-subgroups.
\end{example}

\begin{theorem}[Hall's conjugation theorem]
	\index{Hall's conjugation theorem}
	\label{theorem:HallC}
	Let $G$ be a finite solvable group and $\pi$ be a set of prime numbers. 
    Then all two Hall $\pi$-subgroups of $G$ are conjugate. 
\end{theorem}

\begin{proof}
	We may assume that $G\ne\{1\}$. We proceed by induction on $|G|$.  Let $H$
	and $K$ be Hall $\pi$-subgroups of $G$. Let $M$ be a minimal normal subgroup of 
    $G$ and $\pi\colon G\to G/M$ be the canonical map. Since $G$ is solvable, 
	$M$ is a $p$-group for some prime number $p$ (Lemma~\ref{lem:minimal_normal}). 
    Since $\pi(H)$ and $\pi(K)$ are both Hall 
	$\pi$-subgroups of $G/M$, by the inductive hypothesis, 
    the subgroups $\pi(H)$ and $\pi(K)$ are 
	conjugate in $G/M$. Thus there exists $g\in G$ such that $gHMg^{-1}=KM$. 

	There are two cases to consider. Assume first that $p\in\pi$. Since $|HM|$ and 
	$|KM|$ are $\pi$-numbers and $|H|=|K|$ is the largest $\pi$-number dividing $|G|$, 
    we conclude that $H=HM$ and $K=KM$. In particular, $gHg^{-1}=K$. 

	Assume now that $p\not\in\pi$. Then $K$ admits a complement $M$ in 
 	$KM$, as $K\cap M=\{1\}$. We claim that $gHg^{-1}$ complements $M$ in $KM$. Since 
	$M$ is normal in $G$, 
 	\[
	(gHg^{-1})M=gHMg^{-1}=KM,
	\]
	and $gHg^{-1}\cap M=\{1\}$, as $p\not\in\pi$. These complements are conjugate 
    by the Schur--Zassenhaus theorem~\ref{thm:SchurZassenhaus:conjugation}.
\end{proof}

\begin{corollary}
	Let $G$ be a finite group, $N$ a normal subgroup of $G$ and $n=|N|$. 
 	Assume that either $N$ of $G/N$ is solvable. 
    If $|G:N|=m$ is coprime with $n$ and 
    $m_1$ divide a $m$, then every subgroup of $G$ of order $m_1$ 
    is contained in some subgroup of order $m$. 
\end{corollary}

\begin{proof}
	Let $H$ be a complement of $N$ in $G$. Then $|H|=m$. Let $H_1$
	be a subgroup of $G$ such that $|H_1|=m_1$. 
	Since $\gcd(n,m)=1$, $m_1=|H_1|=|H\cap NH_1|$, as 
	\[
	\frac{|H||N||H_1|}{|H\cap NH_1|}=
	\frac{|H||NH_1|}{|H\cap NH_1|}=|H(NH_1)|=|G|=|NH|=|N||H|.
	\]
	Since both $H_1$ and $H\cap NH_1$ are complements of $N$ in $NH_1$, and both 
    groups have orders coprime with $n$, there exists 
	$g\in G$ such that $H_1=g(H\cap NH_1)g^{-1}$. Thus  
	$H_1\subseteq gHg^{-1}$ and hence $|gHg^{-1}|=m$. 
\end{proof}

% \chapter{}

\topic{Locally indicable groups}

\begin{definition}
\index{Group!indicable}
    A group $G$ is \textbf{indicable} if
    there exists a non-trivial group homomorphism $G\to\Z$.
\end{definition}

We know that braid groups are indicable.
The free group $F_n$ in $n$ letters is indicable. 

\begin{definition}
\index{Group!locally indicable}
    A group $G$ is \textbf{locally indicable} if every 
    non-trivial finitely generated subgroup is indicable.
\end{definition}

\index{Burns--Hale's theorem}
Burns--Hale's theorem (see \cite[Theorem 1.50]{MR3560661}) states that a group $G$ is left-ordered if and only if
for every non-trivial finitely generated subgroup $H$ of $G$ there exists 
a left-ordered group $L$ and a non-trivial group homomorphism $H\to L$. As a consequence, 
locally indicable groups are left-ordered. 

\begin{example}
    Since subgroups of free groups are free, 
    it follows that $F_n$ is locally indicable. 
\end{example}

There are groups that are left-ordered and not locally indicable, see
for example \cite{MR1084707}. The braid group
$\B_n$ for $n\geq5$ is another example of a left-ordered group that is not locally indicable.

\begin{proposition}
\label{pro:LI_exact}
    Let
    \[\begin{tikzcd}
	1 & K & G & Q & 0
	\arrow[from=1-1, to=1-2]
	\arrow["\alpha", from=1-2, to=1-3]
	\arrow["\beta", from=1-3, to=1-4]
	\arrow[from=1-4, to=1-5]
\end{tikzcd}
\]
    be an exact sequence of groups and group homomorphisms. 
    If $K$ and $Q$ are
    locally indicable, then $G$ is locally indicable.
\end{proposition}

\begin{proof}
    Let $g_1,\dots,g_n\in G$ and $L=\langle g_1,\dots,g_n\rangle$. 
    Assume first that $\beta(L)\ne\{1\}$. Since $Q$ is locally indicable, 
    there exists a non-trivial group homomorphism $\beta(L)\to\Z$. Then the 
    composition $L\to\beta(Q)\to\Z$ is then a non-trivial group homomorphism. Assume now
    that $\beta(L)=\{1\}$. Then there exist $k_1,\dots,k_n\in K$ 
    such that $\alpha(k_i)=g_i$ for all $i\in\{1,\dots,n\}$. Note that
    $\alpha\colon \langle k_1,\dots,k_n\rangle\to L$ is a group isomorphism. Since
    $K$ is locally indicable, there exits a non-trivial group 
    homomorphism $\langle k_1,\dots,k_n\rangle\to\Z$. 
    Thus the composition $L\to\langle k_1,\dots,k_n\rangle\to\Z$ is a non-trivial
    group homomorphism 
    and hence $G$ is locally indicable. 
\end{proof}

As a consequence of the previous proposition, 
if $G$ and $H$ are locally indicable groups and 
$\sigma\colon G\to\Aut(H)$ is a group homomorphism, then 
$G\rtimes_\sigma H$ is locally indicable. In particular, the 
direct product of locally indicable groups is locally indicable.

\begin{example}
    The group $G=\langle x,y:x^{-1}yx=y^{-1}\rangle$ is
    locally indicable. We know that $G$ is torsion-free. Let 
    $K=\langle y\rangle\simeq\Z$. Then $G/K\simeq\Z$ and 
    then, since there is an exact sequence
    $1\to\Z\to G\to\Z\to1$ 
    it follows from Proposition \ref{pro:LI_exact} 
    that $G$ is locally indicable.
\end{example}

% $(x,y)\mapsto(x+1,y)$, $(x,y)\mapsto(-x,y+1)$
% A=2312, B=200-2
% Let $K=\langle y\rangle$. Then $G/K\simeq\Z$ and 
% the second result follows from the previous proposition. 

\begin{exercise}
\label{xca:B3_LI}
    Prove that $\B_3$ is locally indicable. 
\end{exercise}

The previous exercise uses the fact that $[\B_3,\B_3]$ is isomorphic to the free group in two letters, see
Exercise \ref{xca:derivedB3}.
An alternative solution to the previous fact goes as follows: $\B_3$ is the fundamental group
of the trefoil knot and fundamental groups of knots are locally indicable. 

\begin{exercise}
    Prove that $\B_4$ is locally indicable.
\end{exercise}

The previous exercise might be harder than Exercise \ref{xca:B3_LI}. One possible solution
is based on using the Reidemeister--Schreier method to prove that 
$[\B_4,\B_4]$ is a certain semidirect product 
between free groups in two generators. Another solution: Let 
$f\colon\B_4\to\B_3$ be the group homomorphism given by $f(\sigma_1)=f(\sigma_3)=\sigma_1$ 
and $f(\sigma_2)=\sigma_2$. Then $\ker f=\langle \sigma_1\sigma_3^{-1},\sigma_2\sigma_1\sigma_3^{-1}\sigma_2^{-1}\rangle$ 
is isomorphic to the free group in two letters. Now use the exact sequence
$1\to \ker f\to\B_4\to\B_3\to1$. 

\begin{exercise}
\label{xca:relations}
    Let $n\geq5$. Consider the elements of $\B_n$ given by 
    \begin{align*}
        &\beta_1=\sigma_1^{-1}\sigma_2,
        &&\beta_2=\sigma_2\sigma_1^{-1}, 
        &&\beta_3=\sigma_1\sigma_2\sigma_1^{-2},
        &&\beta_4=\sigma_3\sigma_1^{-1}, 
        &&\beta_5=\sigma_4\sigma_1^{-1}.
    \end{align*}
    Prove the following relations:
    \begin{enumerate}
        \item $\beta_1\beta_5=\beta_5\beta_2$.
        \item $\beta_2\beta_5=\beta_5\beta_3$.
        \item $\beta_1\beta_3=\beta_2$.
        \item $\beta_1\beta_4\beta_3=\beta_4\beta_2\beta_4$.
        \item $\beta_4\beta_5\beta_4=\beta_5\beta_4\beta_5$.
    \end{enumerate}
\end{exercise}

\begin{exercise}
    Let $n\geq 5$. 
    Prove that $\B_n$ is not locally indicable.
\end{exercise}

For the previous exercise one needs to show that
every group homomorphism $f\colon \langle\beta_1,\dots,\beta_5\rangle\to\Z$ is trivial. Hint: consider
the abelianization of $\langle\beta_1,\dots,\beta_5\rangle$. 

\topic{Unique product groups}

Let $G$ be a group and $A,B\subseteq G$ be non-empty subsets. 
An element $g\in G$ is a \textbf{unique product} in $AB$ if $g=ab=a_1b_1$ for some
$a,a_1\in
A$ and $b,b_1\in B$ implies that $a=a_1$ and $b=b_1$.

\begin{definition}
	\index{Group!unique product}
	A group $G$ has the \textbf{unique product property} if 
	for every finite non-empty subsets $A,B\subseteq G$ there exists at least one
	unique product in $AB$.
\end{definition}

\begin{proposition}
    Left-ordered groups have the unique product property.
\end{proposition}

\begin{proof}
    Let $G$ be a left-ordered group. 
	Let $A$ be a non-empty finite subset of $G$ and $B=\{b_1,\dots,b_n\}\subseteq G$. 
	Assume that $b_1<b_2<\cdots<b_n$. Let $c\in A$ be such that $cb_1$ is the 
	minimum of $Ab_1=\{ab_1:a\in A\}$. We claim that $cb_1$ admits a unique
	representation of the form $\alpha\beta$ with $\alpha\in A$ and 
	$\beta\in B$. If $cb_1=ab$, then, since $ab=cb_1\leq ab_1$, it follows that 
	$b\leq b_1$. Hence $b=b_1$ and $a=c$. 
\end{proof}

\begin{exercise}
	Prove that groups with the unique product property are
	torsion-free.
\end{exercise}

The converse does not hold. 
Promislow's group is a celebrated counterexample.

\begin{theorem}[Promislow]
\index{Promislow's theorem}
    The group $G=\langle a,b:a^{-1}b^2a=b^{-2},b^{-1}a^2b=a^{-2}\rangle$
    does not have the unique product property.
\end{theorem}

\begin{proof}
    Let 
    \begin{multline}
    \label{eq:Promislow}
    S=\{ a^2b,
    b^2a,
    aba^{-1},
    (b^2a)^{-1},
    (ab)^{-2},
    b,
    (ab)^2a,
    (ab)^2,
    (aba)^{-1},\\
    bab,
    b^{-1},
    a,
    aba,
    a^{-1}
    \}.
    \end{multline}
    We use \textsf{GAP} and the representation $G\to\GL(4,\Q)$ given by 
    \[
a\mapsto\begin{pmatrix}
1 & 0 & 0 & 1/2\\
0 & -1 & 0 & 1/2\\
0 & 0 & -1 & 0\\
0 & 0 & 0 & 1
\end{pmatrix},
\quad
b\mapsto\begin{pmatrix}
-1 & 0 & 0 & 0\\
0 & 1 & 0 & 1/2\\
0 & 0 & -1 & 1/2\\
0 & 0 & 0 & 1
\end{pmatrix}
\]
    to check that 
    $G$ does not have
    unique product property, as each 
    \[
    s\in S^2=\{s_1s_2:s_1,s_2\in S\}
    \]
    admits at least two different decompositions of the 
    form $s=xy=uv$ for $x,y,u,v\in S$. 
    We first create the matrix representations of $a$ and $b$.
\begin{lstlisting}
gap> a := [[1,0,0,1/2],[0,-1,0,1/2],[0,0,-1,0],[0,0,0,1]];;
gap> b := [[-1,0,0,0],[0,1,0,1/2],[0,0,-1,1/2],[0,0,0,1]];;
\end{lstlisting}
    Now we create
    a function that produces the set $S$.
\begin{lstlisting}
gap> Promislow := function(x, y)
> return Set([
> x^2*y,
> y^2*x,
> x*y*Inverse(x),
> (y^2*x)^(-1),
> (x*y)^(-2),
> y,
> (x*y)^2*x,
> (x*y)^2,
> (x*y*x)^(-1),
> y*x*y,
> y^(-1),
> x,
> x*y*x,
> x^(-1)
]);
end;;
\end{lstlisting}
So the set $S$ of \eqref{eq:Promislow} 
will be \lstinline{Promislow(a,b)}. We now
create a function that checks whether
every element of a Promislow subset 
admits more than one representation.
\begin{lstlisting}
gap> is_UPP := function(S)
> local l,x,y;
> l := [];
> for x in S do
> for y in S do
> Add(l,x*y);
> od;
> od;
> if ForAll(Collected(l), x->x[2] <> 1) then
> return false;
> else
> return fail;
> fi;
> end;;
\end{lstlisting}
Finally, we check whether every element of 
$S^2$ admits more than one representation.
\begin{lstlisting}
gap> S := Promislow(a,b);;
gap> is_UPP(S);
false
\end{lstlisting}
This completes the proof. 
\end{proof}

\begin{exercise}
    Let $G$ be a group and $A,B\subseteq G$ be finite non-empty subsets. Prove that 
    if $|A|=1$, then $AB$ contains a unique product. 
\end{exercise}

The size of the set $A$ can be extended. 

\begin{exercise}
    Let $G$ be a torsion-free group and $A,B\subseteq G$ be finite non-empty subsets. Prove that
    $AB$ has no unique products if and only if $(gA)(Bh)$ has no unique product for all $g,h\in G$. 
\end{exercise}

\begin{exercise}
    Let $G$ be a group and $A,B\subseteq G$ be finite non-empty subsets. Prove that 
    if $|A|=2$, then $AB$ contains a unique product. 
\end{exercise}

The case where the set $A$ has size three is still open. One can prove, for example, 
that if $|A|=3$, then $|B|\geq7$. 

% There are other examples. 

% \begin{exercise}
%     Let $G=\langle x,y:x^{-1}y^2xy^2=x^{-2}yx^{-2}y^3=1\rangle$. 
%     \begin{enumerate}
%         \item Prove that the subgroup $N$ generated by 
%         \[
%         [y*x*Inverse(x*y), y^2, x^4]
%         \]
%         is a normal subgroup of index eight. 
%         \item Prove that 
%         \[
%         N=\langle x_1,x_2,x_3:[x_1,x_2]=[x_1,x_3]=1,\,x_3x_2=x_2x_3x_1^8\rangle.
%         \]

%     \end{enumerate}
% \end{exercise}

% Let $G=\langle x,y:x^{-1}y^2xy^2=x^{-2}yx^{-2}y^3=1\rangle$. 
% We first construct the group and a certain normal
% subgroup $N$ of index eight. 

% \begin{lstlisting}
% gap> f := FreeGroup(2);;
% gap> x := f.1;;
% gap> y := f.2;;
% gap> rels := [Inverse(x)*y^2*x*y^2, Inverse(x^2)*y*Inverse(x^2)*y^3];;
% gap> G := f/rels;;
% gap> x := G.1;;
% gap> y := G.2;;
% gap> N := Subgroup(G, [y*x*Inverse(x*y), y^2, x^4]);
% gap> IsNormal(N,G);
% true
% gap> StructureDescription(G/N);
% "C4 x C2"
% \end{lstlisting}
% The subgroup $N$ has a nice presentation. It can be presented
% as the group 
% \[
% N=\langle x_1,x_2,x_3:[x_1,x_2]=[x_1,x_3]=1,\,x_3x_2=x_2x_3x_1^8\rangle.
% \]
% \begin{lstlisting}
% gap> g := IsomorphismFpGroup(N);
% gap> RelatorsOfFpGroup(Image(g));
% [ F3*F1*F3^-1*F1^-1, F2*F1*F2^-1*F1^-1, F1*F3^-1*F1^5*F2*F1^2*F3*F2^-1 ]
% \end{lstlisting}
% From these relations
% one proves by induction that 
% \begin{align}
%     x_3^bx_2^a=x_2^ax_3^bx_1^{8ab}
% \end{align}
% for all $a,b\in\Z$. 
% % First induction in $n_1$, then induction in $n_2$. 
% It follows that every element of $N$ 
% is of the form $x_1^{n_1}x_2^{n_2}x_3^{n_3}$ 
% for $n_1,n_2,n_3\in\Z$. Moreover, 
% \[
% (x_1^{n_1}x_2^{n_2}x_3^{n_3})^k=x_1^{kn_1+(k-1)8n_2n_3}x_2^{kn_2}x_3^{kn_3}
% \]
% for all $k\in\Z$. Note that $x_1^8$ is a commutator. Moreover, 
% $N/[N,N]\simeq\Z\times\Z\times\Z/8$. This implies that
% $N$ is torsion-free. Let us prove that $G$ is torsion-free. 
% Let $\pi\colon G\to G/N$ be the canonical map. Let 
% $g\in G$ be a torsion element, in particular $g\not\in N$
% and hence $\pi(g)\ne 1$. So $\pi(g)$ has order two 
% or four. Without loss of generality we may assume that
% $\pi(g)$ has order two.  
% Then $\pi(g^2)=\pi(g)^2=y^2=1$ and hence 
% $g^2\in N$. Since $N$ is torsion free, it follows that $g=1$. 

\begin{definition}
\index{Group!double unique product}
	A group $G$ has the \textbf{double property of unique products} 
	if for every finite non-empty subsets $A,B\subseteq G$ such that 
	$|A|+|B|>2$ there are at least two unique products in $AB$.
\end{definition}

\begin{theorem}[Strojnowski]
	\label{theorem:Strojnowski}
	\index{Strojonowski's theorem}
	Let $G$ be a group. The following statements are equivalent:
	\begin{enumerate}
		\item $G$ has the double property of unique products. 
		\item Every non-empty finite subset $A\subseteq G$ contains at least one unique product 
			in $AA=\{a_1a_2:a_1,a_2\in A\}$.
		\item $G$ has the unique product property.
	\end{enumerate}
\end{theorem}

\begin{proof}
	It is trivial that $1)\implies2)$.  
	
	Let us prove that 
	$2)\implies3)$. If $G$ does not have the unique product property, 
	there exist finite non-empty subsets 
	$A,B\subseteq G$ such that every element of 
	$AB$ admists at least two representations. Let $C=AB$. Every element $c\in C$ is 
	of the form $c=(a_1b_1)(a_2b_2)$ for some $a_1,a_2\in A$ and $b_1,b_2\in B$. Since 
	$a_2^{-1}b_1^{-1}\in AB$, there exist $a_3\in A\setminus\{a_2\}$ and
	$b_3\in B\setminus\{b_1\}$ such that 
	$a_2^{-1}b_1^{-1}=a_3^{-1}b_3^{-1}$. Thus $b_1a_2=b_3a_3$ and hence 
	\[
	c=(a_1b_1)(a_2b_2)=(a_1b_3)(a_3b_2)
	\]
	has two different representations in $AB$, 
	as $a_2\ne a_3$ and $b_1\ne b_3$.

	We now prove that $3)\implies1)$. Let us assume that $G$ has the unique product property 
	but it is not a group with double unique products. Then there exist 
	finite non-empty subsets 
	$A,B\subseteq G$ with $|A|+|B|>2$ such that 
	in $AB$ there exists a unique element $ab$ with a unique representation in $AB$.
	Let $C=a^{-1}A$ and $D=Bb^{-1}$. Then $1\in C\cap D$ and the identity 
	$1$ admits a unique representation in $CD$ (because $1=cd$ with 
	$c=a^{-1}a_1\ne 1$ and $d=b_1b^{-1}\ne 1$ imply that $ab=a_1b_1$ with $a\ne
	a_1$ and $b\ne b_1)$. Let $E=D^{-1}C$ and $F=DC^{-1}$. Every element of the set $EF$
	can be written as $(d_1^{-1}c_1)(d_2c_2^{-1})$. If either $c_1\ne 1$ or $d_2\ne 1$, 
	then $c_1d_2=c_3d_3$ for some elements $c_3\in C\setminus\{c_1\}$ and  
	$d_3\in D\setminus\{d_2\}$. Thus 
	$(d_1^{-1}c_1)(d_2c_2^{-1})=(d_1^{-1}c_3)(d_3c_2^{-1})$ are two different representations 
    for $(d_1^{-1}c_1)(d_2c_2^{-1})$. If either $c_2\ne 1$
	or $d_1\ne 1$, then $c_2d_1=c_4d_4$ for some $d_4\in D\setminus\{d_1\}$
	and some $c_4\in C\setminus\{c_2\}$. Since 
	$d_1^{-1}c_2^{-1}=d_4^{-1}c_4^{-1}$, it follows that 
	\[
	(d_1^{-1}1)(1c_2^{-1})=(d_4^{-1}1)(1c_4^{-1}).
	\]
	Since $|C|+|D|>2$, either $C$ or 
	$D$ contains $c\ne1$. Thus $(1\cdot 1)(1\cdot 1)=(1\cdot
	c)(1\cdot c^{-1})$. Therefore every element of $EF$ 
	admits at least two representations. 
\end{proof}

% passman lema 1.9 pag 589
\begin{exercise}
	Prove that if a group $G$ satisfies the unique product property, then 
    $K[G]$ contains only trivial units.
\end{exercise}

In general it is extremely hard to check whether a given group
has the unique product property. As a geometrical way to 
attack this problem, Bowditch introduced \emph{diffuse groups}. If
$G$ is a torsion-free group and 
$A\subseteq G$ is a subset, we say that $A$ is antisymmetric 
if $A\cap A^{-1}\subseteq\{1\}$, where $A^{-1}=\{a^{-1}:a\in
A\}$. The set of \textbf{extremal elements} of $A$ is defined as 
$\Delta(A)=\{a\in A:Aa^{-1}\text{ is antisymmetric}\}$. 
Thus 
\[
	a\in A\setminus\Delta(A)
	\Longleftrightarrow
	\text{there existes $g\in G\setminus\{1\}$ such that $ga\in A$ and $g^{-1}a\in A$}.
\]

\begin{definition}
	\index{Group!diffuse}
	A group $G$ is \textbf{diffuse} if for every subset $A\subseteq
	G$ such that $2\leq |A|<\infty$ one has $|\Delta(A)|\geq2$.
\end{definition}

This means that a group $G$ is diffuse if for every finite non-empty subset $A\subseteq G$ there exists 
$a\in A$ such that for all $g\in G\setminus\{1\}$ either $ga\not\in A$ or $g^{-1}a\not\in A$. 

\begin{proposition}
	Left-ordered groups are diffuse.	
\end{proposition}

\begin{proof}
    Let $G$ be a left-ordered group and $A=\{a_1,\dots,a_n\}$ be such that
    \[
    a_1<a_2<\cdots<a_n.
    \]
    We claim that 
	$\{a_1,a_n\}\subseteq\Delta(A)$. If $a_1\in
	A\setminus\Delta(A)$, there exists $g\in G\setminus\{1\}$ such that $ga_1\in A$ and
	$g^{-1}a_1\in A$. Thus  $a_1\leq ga_1$ and $a_1\leq g^{-1}a_1$. It follows that
	$1\leq a^{-1}ga_1$ and $1\leq a_1^{-1}g^{-1}a_1=(a_1^{-1}ga_1)^{-1}$, 
	a contradiction. Similarly, $a_n\in \Delta(A)$.
\end{proof}

There are diffuse groups that are not left-ordered, see
\cite{MR3548136}. 

\begin{proposition}
	\label{pro:difuso=>2up}
    Diffuse groups have double unique products.  
\end{proposition}

\begin{proof}
    Let $G$ be a diffuse group that does not have double unique products. 
    There exist non-empty subsets $A,B\subseteq G$ with $|A|+|B|>2$ such that 
	$C=AB$ admits at most one unique product. Then $|C|\geq2$. Since $G$ is diffse, 
	$|\Delta(C)|\geq2$. If $c\in\Delta(C)$, then $c$ admits a unique 
	expression of the form $c=ab$ with $a\in A$ and $b\in B$ (otherwise, if 
	$c=a_0b_0=a_1b_1$ with $a_0\ne a_1$ and $b_0\ne b_1$). If $g=a_0a_1^{-1}$,
	then $g\ne 1$, 
	\[
	gc=a_0a_1^{-1}a_1b_1=a_0b_1\in C.
	\]
	Moreover, 
	$g^{-1}c=a_1a_0^{-1}a_0b_0=a_1b_0\in C$. Hence $c\not\in\Delta(c)$, a contradiction.
\end{proof}

\begin{problem}
	Find a non-diffuse group with the unique product property.
\end{problem}

%Un grupo $G$ se dice \textbf{débilmente difuso} si para todo subconjunto
%finito $A\subseteq G$ no vacío se tiene $\Delta(A)\ne\emptyset$. La técnica
%usada para demostrar el lema~\ref{lemma:difuso=>2up} puede usarse para
%demostrar que un grupo débilmente difuso posee la propiedad del producto
%único. El teorema~\ref{theorem:Strojnowski} sugiere entonces la siguiente
%pregunta: 
%
%\begin{problem}
%	¿Existe un grupo débilmente difuso que no sea difuso?
%\end{problem}
%
%\section{El grupo de Promislow}
%
%Veremos un ejemplo concreto de un grupo sin torsión que no es ordenable, no es
%difuso y no tiene la propiedad del producto único.
%
%\begin{exercise}
%	\label{exercise:Dinfty}
%	Demuestre que $G=\langle x,y:x^2=y^2=1\rangle$ es isomorfo al grupo diedral infinito.
%\end{exercise}
%
%\begin{definition}
%	Se define el grupo de Promislow como 
%	\[
%		G=\langle x,y:x^{-1}y^2x=y^{-2},\,y^{-1}x^2y=x^{-2}\rangle.
%	\]
%\end{definition}
%
%\begin{proposition}
%	\label{proposition:Promislow}
%	El grupo de Promislow es libre de torsión y no satisface la propiedad del
%	producto único. 
%\end{proposition}
%
%\begin{proof}
%	
%\end{proof}


% \section{18/04/2024}

\subsection{Wielandt's zipper theorem}

\begin{theorem}[Wielandt]
	\index{Wielandt's!zipper theorem}
	\label{thm:zipper}
	Let $G$ be a finite group and $S$ be a subgroup of $G$ subnormal in every 
    proper subgroup of $G$ containing $S$. If $S$ is not subnormal in $G$, 
    then there exists a unique maximal subgroup of $G$ containing $S$. 
\end{theorem}

\begin{proof}
	We proceed by induction on $(G:S)$. If $S$ is not subnormal in $G$, then 
	$S\ne G$ and the case where $(G:S)=1$ holds. 

	Since $S$ is not subnormal in $G$, $N_G(S)\ne G$. Then $S\subseteq
	N_G(S)\subseteq M$ for some maximal subgroup $M$ of $G$. Assume that 
	$S\subseteq K$ for some maximal subgroup $K$ of $G$. We claim that 
	que $K=M$. Since $S\subseteq K\ne G$, $S$ is subnormal in $K$. If $S$ is 
	normal in $K$, then $K\subseteq N_G(S)\subseteq M$. Hence $K=M$ by the maximality of $K$. 
	If $S$ is not normal in $K$, there exist a sequence 
	$S_0,\dots,S_m$ of subgroups of $K$ such that 
	\[
		S=S_0\subseteq S_1\subseteq\cdots\subseteq S_m=K,
	\]
    where $S_i$ is normal in $S_{i+1}$ for all $i$ and 
	$S$ is not normal in $S_2$. Let $x\in S_2$ be such that $xSx^{-1}\ne S$ and 
	$T=\langle S,xSx^{-1}\rangle\subseteq K$. 

	Since $xSx^{-1}\subseteq xS_1x^{-1}=S_1\subseteq N_G(S)$, we obtain that 
	$T\subseteq N_G(S)\subseteq M$. Moreover, $S$ is normal in $T$. Thus $T\ne G$. 

	We claim that $T$ satisfies the theorem's assumptions. If $T$ is subnormal in $G$, then, since 
	$S$ is normal in $T$, $S$ is subnormal in $G$. If $H$ is a proper subgroup of $G$ 
    containing $T$, then, since 
	$S\subseteq H$, $S$ is subnormal in $H$. Moreover, $xSx^{-1}$ is subnormal in $H$. Hence 
	$T$ is subnormal in $H$ by Theorem~\ref{thm:STsubnormal}.

	Since $S\subsetneq T$, $(G:T)<(G:S)$. By the inductive hypothesis, $T$ is contained in a unique maximal subgroup of $G$. Therefore
	$K=M$, since $T\subseteq M$ and 
	$T\subseteq K$.
\end{proof}

Before giving an application, we need a lemma. 

\begin{lemma}
	\label{lem:H=G}
	Let $G$ be a group and $H$ be a subgroup of $G$. If $(xHx^{-1})H=G$ for some 
	$x\in G$, then $H=G$.
\end{lemma}

\begin{proof}
	Write $x=uv$ for some $u\in xHx^{-1}$ and $v\in H$. Since $u\in xHx^{-1}$ and
	$u^{-1}x=v\in H$, we obtain that $H=vHv^{-1}=u^{-1}(xHx^{-1})u=xHx^{-1}$. Thus
	$G=H$. 
\end{proof}

Recall that two subgroups $S$ and $T$ of a group $G$ are said to be
\textbf{permutable} if $ST=TS$. 

\begin{theorem}
	Let $G$ be a finite group and $S$ be a subgroup of $G$ permutable with any of
	its conjugates. Then $S$ is subnormal in $G$. 
\end{theorem}

\begin{proof}
	We proceed by induction on $|G|$. Assume that $S$ is subnormal in 
	every subgroup $H$ such that $S\subseteq H\subsetneq G$.  If $S$ is not subnormal in $G$, 
	then, by Theorem~\ref{thm:zipper}, there exists a unique maximal subgroup $M$ of $G$ 
	such that $S\subseteq M$. Let $x\in G$ and 
	$T=xSx^{-1}$. By Lemma~\ref{lem:H=G}, $ST\ne G$ (because $S\ne G$). Thus 
	$ST$ is contained in some maximal subgroup of $G$. Since 
	$S\subseteq ST$ and $S$ is contained in a unique maximal subgroup of $G$, we conclude that 
	$T\subseteq ST\subseteq M$.  Since $S^G=\langle xSx^{-1}:x\in
	G\rangle\subseteq M\ne G$, the inductive hypothesis implies that $S$ is subnormal in
	$S^G$. Hence $S$ is subnormal in $G$ since $S^G$ is normal in $G$, a contradiction. 
\end{proof}

\subsection{Baer's theorem}

\begin{theorem}[Baer]
	\index{Baer's theorem}
	\label{thm:Baer}
	Let $G$ be a finite group and $H$ be a subgroup of $G$. Then $H\subseteq
	F(G)$ if and only if $\langle H,xHx^{-1}\rangle$ is nilpotent for all 
	$x\in G$.
\end{theorem}

\begin{proof}
	If $H\subseteq F(G)$, then $xHx^{-1}\subseteq F(G)$ for all $x\in G$, since
	$F(G)$ is normal in $G$. Thus $\langle H,xHx^{-1}\rangle$ is nilpotent, as it
	is a subgroup of $F(G)$.

	Conversely, assume that $\langle H,xHx^{-1}\rangle$ is nilpotent for all  $x\in
	G$. Since $H\subseteq \langle H,xHx^{-1}\rangle$, $H$ is nilpotent. By
	Theorem~\ref{thm:F(G)subnormalidad}, it is enough to see that $H$ is subnormal
	in $G$. We proceed by induction on $|G|$. Suppose that $H$ is not subnormal in
	$G$. If $H$ is properly contained in some subgroup $K$, then, since $\langle
	H,kHk^{-1}\rangle$ is nilpotent for all $k\in K$, $H$ is subnormal in $K$ by
	the inductive hypothesis. By Theorem~\ref{thm:zipper}, there exists a unique
	maximal subgroup $M$ of $G$ containing $H$. There are two cases to consider.
    
    Assume first that $G=\langle H,xHx^{-1}\rangle$ for some $x\in G$. Since $G$
    is nilpotent, $H$ subnormal in $G$ by Theorem~\ref{thm:subnormal}, a
    contradiction. 

    Assume now that $\langle H,xHx^{-1}\rangle\ne G$ for all $x\in G$. For each 
	$x\in G$, there exists a maximal subgroup containing $\langle
	H,xHx^{-1}\rangle$. Since $H\subseteq \langle H,xHx^{-1}\rangle$ and $H$
	is contained in a unique maximal subgroup, we conclude that $\langle
	H,xHx^{-1}\rangle\subseteq M$ for all $x\in G$. In particular, the normal closure 
	$H^G$ of $H$ is properly contained in $G$. By the inductive hypothesis, 
	$H$ is subnormal in $H^G$ and $H^G$ is normal in $G$, we conclude that 
	$H$ is subnormal in $G$, a contradiction. 
\end{proof}

\subsection{Zenkov's theorem}

\begin{theorem}[Zenkov]
    \index{Zenkov's!theorem}
    \label{thm:Zenkov}
    Let $G$ be a finite group and $A$ and $B$ be abelian subgroups of $G$. Let
    $M\in\{A\cap gBg^{-1}:g\in G\}$ such that no $A\cap gBg^{-1}$ is properly
    contained in $M$. Then $M\subseteq F(G)$.
\end{theorem}

\begin{proof}
	Without loss of generality, we may assume that $M=A\cap B$. Using induction on 
	$|G|$, we prove that $M\subseteq F(G)$.

	Assume first that $G=\langle A,gBg^{-1}\rangle$ for some $g\in G$. Since $A$
	and $B$ are both abelian, 
 \[
 A\cap gBg^{-1}\subseteq Z(G)
 \]
 and hence 
	\[
		A\cap gBg^{-1}=g^{-1}(A\cap gBg^{-1})g\subseteq A\cap B=M.
	\]
	By the minimality of $M$, 
    \[
    M=A\cap gBg^{-1}\subseteq Z(G)\subseteq F(G)
    \]
	by Corollary~\ref{cor:Z(G)subsetF(G)}.

	Assume now that $G\ne \langle A,gBg^{-1}\rangle$ for all $g\in G$.
	Let $g\in G$, $H=\langle A,gBg^{-1}\rangle\ne G$ and $C=B\cap H$.
	Since $A\subseteq H$, we obtain that 
 	$M=A\cap B=A\cap C$ and 
	$A\cap hCh^{-1}=A\cap hBh^{-1}$
	for all $h\in H$. This implies that no 
	$A\cap hCh^{-1}$ is properly contained in $A\cap C$. 
    By the inductive hypothesis on $H$, 
 	\[
		M=A\cap B=A\cap C\subseteq F(H).
	\]

    We now prove that every Sylow $p$-subgroup $P$ of $M$ is contained in $F(G)$. 
	Since $M$ is generated by its Sylow subgroups, 
    $M\subseteq F(G)$.
	If $P\in\Syl_p(M)$, then $P\subseteq M\subseteq F(H)$. Since $O_p(H)$ is 
    the only Sylow $p$-subgroup of $F(H)$, $P\subseteq O_p(H)$. Since 
	$P\subseteq M\subseteq B$, 
	\[
	gPg^{-1}\subseteq gBg^{-1}\subseteq H
	\]
	for all $g\in G$. Thus $O_p(H)(gPg^{-1})$ is a $p$-subgroup of $H$ containing 
	$\langle P,gPg^{-1}\rangle$. Hence $\langle P,gPg^{-1}\rangle$
	is nilpotent for all $g\in G$, since it is a $p$-group. By Baer's theorem~\ref{thm:Baer}, 
    $P\subseteq F(G)$ for all Sylow $p$-subgroup $P$ of $M$. 
\end{proof}

\begin{corollary}
	\label{cor:Zenkov}
	Let $G$ be a non-trivial finite group and $A$ be an abelian subgroup of $G$ such that 
 	$|A|\geq(G:A)$. Then $A\cap F(G)\ne\{1\}$.
\end{corollary}

\begin{proof}
	Let $g\in G$. We may assume that $G\ne A$. Then $(gAg^{-1})A\ne G$ by Lemma~\ref{lem:H=G}. Since 
	$|gAg^{-1}||A|=|A|^2\geq |A|(G:A)=|G|$, 
	\[
		|G|>|gAg^{-1}A|
		=\frac{|A||gAg^{-1}|}{|A\cap gAg^{-1}|}
		\geq \frac{|G|}{|A\cap gAg^{-1}|}.
	\]
	Hence $A\cap gAg^{-1}\ne 1$ for all $g\in G$. In particular, no
	$A\cap gAg^{-1}$ is properly contained in $A$. By 
	Zenkov's theorem~\ref{thm:Zenkov}, $A\subseteq F(G)$.
\end{proof}

\begin{corollary}
	Let $G=NA$ be a finite group, where $N$ is a normal subgroup of $G$, $A$ is an abelian subgroup of $G$ and 
	$C_A(N)=\{1\}$. If $F(N)=\{1\}$, then $|A|<|N|$. 
\end{corollary}

\begin{proof}
	Since $N$ is normal in $G$, 
	\[
    N\cap F(G)=F(N)=\{1\}
    \]
    by Corollary~\ref{cor:McapF(G)}. Thus $[N,F(G)]=\{1\}$, since 
	both $N$ and $F(G)$ are normal in $G$. Since 
	\[
	|A|\geq |N|\geq \frac{|N|}{|N\cap A|}=(NA:A)=(G:A),
	\]
	$A\cap F(G)\ne\{1\}$ by Corollary~\ref{cor:Zenkov}. If $1\ne a\in
	A\cap F(G)$, then $a\in C_A(N)=1$, a contradiction. 
\end{proof}

\subsection{Brodkey's theorem}

\begin{theorem}[Brodkey]
	\index{Brodkey's!theorem}
	\label{thm:Brodkey}
	Let $G$ be a finite group such that there exists an abelian $P\in\Syl_p(G)$. Then
    there exist $S,T\in\Syl_p(G)$ such that $S\cap T=O_p(G)$.
\end{theorem}

\begin{proof}
	By applying Zenkov's theorem (Theorem~\ref{thm:Zenkov}) with $A=B=P$, 
	\[
    P\cap gPg^{-1}\subseteq F(G)
    \]
    for some $g\in G$. Since $O_p(G)$ is the only Sylow $p$-subgroup of 
	$F(G)$, $P\cap gPg^{-1}\subseteq O_p(G)$.
	Hence $P\cap gPg^{-1}=P_p(G)$, since $O_p(G)$ is contained in every Sylow $p$-subgroup 
	of $G$. 
\end{proof}

\begin{corollary}
	\label{corollary:GP2}
	Let $G$ be a finite group. If there exists an abelian $P\in \Syl_p(G)$, 
	\[
	(G:O_p(G))\leq (G:P)^2. 
	\]
\end{corollary}

\begin{proof}
	By Brodkey's theorem, there exist $S,T\in\Syl_p(G)$
	such that $S\cap T=O_p(G)$. Then 
	\[
		|G|\geq |ST|=\frac{|S||T|}{|S\cap T|}=\frac{|P|^2}{|O_p(G)|},
	\]
	which implies the claim. 
\end{proof}

\begin{corollary}
	Let $G$ be a finite group. If there exists an abelian $P\in\Syl_p(G)$ such that 
	$|P|<\sqrt{|G|}$, then $O_p(G)\ne\{1\}$.
\end{corollary}

\begin{proof}
	Since $(G:P)^2<|G|$, the previous corollary implies that 
	$O_p(G)\ne\{1\}$.
\end{proof}

% \begin{exercise}
% 	\label{xca:G/Z(G)}
% 	Sea $G$ un grupo y sea Sea $K\subseteq Z(G)$. Demuestre que si $G/K$ es
% 	cíclico entonces $G$ es abeliano.
% \end{exercise}

% \begin{sol}{xca:G/Z(G)}
% 	Sean $g,h\in G$ y sea $\pi\colon G\to G/K$ el morfismo canónico. Como $G/K$
% 	es cíclico, existe $x\in G$ tal que $G/K=\langle xK\rangle$. Sean $k,l$ tales que 
% 	$\pi(g)=x^k$, $\pi(h)=x^l$. Entonces existen $z_1,z_2\in K$ tales que 
% 	$g=x^kz_1$, $h=x^lz_2$. Luego $[g,h]=[x^k,x^l]=1$. 
% \end{svgraybox}


\subsection{Lucchini's theorem}

\begin{theorem}[Lucchini]
	\index{Lucchini's!theorem}
	\label{thm:Lucchini}
	Let $G$ be a finite group and $A$ be a proper cyclic subgroup of $G$. If 
	$K=\Core_G(A)$, then $(A:K)<(G:A)$.
\end{theorem}

\begin{proof}
	We proceed by induction on $|G|$. Let $\pi\colon G\to G/K$ be the canonical map. Note that $\Core_{G/K}\pi(A)$ is trivial. 

	Assume first that $K\ne\{1\}$. Since $\pi(A)$ is a proper cyclic subgroup of 
	$G/K$ and $K\subseteq A$, the inductive hypothesis implies that 
	\[
		(A:K)=|\pi(A)|=(\pi(A):\pi(K))<(\pi(G):\pi(A))=\frac{(G:K)}{(A:K)}=(G:A).
	\]

	Assume now that $K=\{1\}$. We want to prove that $|A|<(G:A)$. Suppose that 
	$|A|\geq (G:A)$. Since $A\ne G$, $A\cap F(G)\ne\{1\}$ by Corollary~\ref{cor:Zenkov}. In particular, $F(G)\ne\{1\}$. Let $E$ be a minimal normal subgroup of 
	such that $E\subseteq F(G)$. By Theorem~\ref{thm:Hirsch}, $E\cap Z(F(G))\ne\{1\}$.  Since 
	$E\cap Z(F(G))$ is normal in $G$ and $E$ is minimal, $E\cap Z(F(G))=E$, that is 
	$E\subseteq Z(F(G))$. In particular, $E$ is abelian. By the minimality of 
	$E$, there is a prime number $p$ such that $x^p=1$ for all $x\in E$. 

	\begin{claim}
		$A\cap F(G)$ is a normal subgroup of $EA$.
	\end{claim}

	Since $E$ is normal in $G$, $EA$ is a subgroup of $G$. Since $A\cap
	F(G)\subseteq A$, $A\cap F(G)$ is a subgroup of $EA$.  Since $F(G)$ is 
	normal in $G$, $a(A\cap F(G))a^{-1}=A\cap F(G)$ for all $a\in A$. Moreover, 
    $E\subseteq Z(F(G))$ and $A\cap F(G)\subseteq F(G)$ imply that 
	$x(A\cap F(G))x^{-1}=A\cap F(G)$ for all $x\in E$. 

	\begin{claim}
		$EA\ne G$.
	\end{claim}

	If $G=EA$, then, since $A\cap F(G)$ is a normal subgroup of $G$
	contained in $A$, we conclude that $\{1\}\ne A\cap F(G)\subseteq K=1$, a 
	contradiction. 
%como $F(G)$ es normal en $G$, 
%	\[
%	A\cap F(G)=g(A\cap F(G))g^{-1}=gAg^{-1}\cap F(G)\subseteq gAg^{-1}
%	\]
	%para todo $g\in G$. 
%    Luego $1\ne A\cap F(G)\subseteq K$, una contradicción pues $K=1$.

	\medskip
	Let $p\colon G\to G/E$ the canonical map. By the correspondence theorem,
	there exists a normal subgroup $M$ of $G$ such that $E\subseteq M$ and 
	$p(M)=\Core_{G/E}(p(A))$. Since $EA\ne G$, $p(A)$ is a proper cyclic subgroup 
	of $p(G)$. Since $p(A)\simeq A/A\cap E\simeq EA/E$ and $p(M)\simeq
	M/E$, the inductive hypothesis implies that 
	$(EA:M)<(G:EA)$, as 
	\[
	\frac{|EA/E|}{|M/E|}
	=(p(A):p(M))
	<(p(G):p(A))
	=\frac{|G/E|}{|EA/E|}.
	\]

	\begin{claim}
		$MA=EA$. 
	\end{claim}

	Since $E\subseteq M$, $EA\subseteq MA$. Conversely, if $m\in M$, 
	then, since $p(m)\in\Core_{G/E}(p(A))$, we obtain that 
	$p(m)\in p(A)$. Thus $m\in EA$. 

	\medskip
	Let $B=A\cap M$. Since $(AE:M)<(G:EA)$, 
	\[
	(A:B)=|A/A\cap M|=|AM/M|=(EA:M).
	\]
	By the inductive hypothesis, 
	\begin{equation}
		\label{eq:(M:B)leq|B|}
	\begin{aligned}
		(M:B)&=(M:A\cap M)=(MA:A)\\
		&=(EA:A)
		=\frac{(G:A)}{(G:EA)}
		<\frac{(G:A)}{(AE:M)}
		=\frac{(G:A)}{(A:B)}\leq |B|, 
	\end{aligned}
	\end{equation}
	as $|A|\geq (G:A)$. 

	\begin{claim}
		$M=EB$.
	\end{claim}

	Since $E\cup B\subseteq M$, $EB\subseteq M$. Conversely, if 
	$m\in M$, then $m=ea$ for some $e\in E$ and $a\in A$. Since $e^{-1}m=a\in
	A\cap M=B$ (because $E\subseteq M$), $m\in EB$.

	\begin{claim}
		$M$ is not abelian. 
	\end{claim}

	Suppose that $M$ is abelian. The map $f\colon M\to M$, $m\mapsto
	m^p$, is a group homomorphism such that $E \subseteq\ker f$. Since $M=EB$,
	$f(M)\subseteq f(B)\subseteq B\subseteq A$. Since $M$ is normal in $G$,
	$f(M)$ is normal in $G$. Thus $f(M)=\{1\}$, as $K=\Core_G(A)=\{1\}$ is the largest normal subgroup of $G$ contained in $A$. In particular, since $B$ is normal in $M=EB$, $M/B$ is a $p$-group. Since $B\subseteq M$,  $f(B)=\{1\}$. Moreover, since 
	$B\subseteq A$ is cyclic, $|B|\leq p$. By using~\eqref{eq:(M:B)leq|B|}, 
	$(M:B)<|B|\leq p$. This implies that $M=B\subseteq A$ and $M=E=1$ (because 
	$M$ is normal in $G$ and $\Core_G(A)=K=\{1\}$ is the largest normal subgroup of $G$ containing $A$), a contradiction. 
	
	\begin{claim}
		$Z(M)$  is cyclic. 
	\end{claim}

	Since $M$ is not abelian and $M/E=EB/E\simeq B/E\cap B$ is cyclic,
	$E\not\subseteq Z(M)$, that is $E\cap
	Z(M)\subsetneq E$. Thus  
	\begin{equation}
		\label{equation:EcapZ(M)}
		E\cap Z(M)=\{1\}
	\end{equation}
	by the minimality of $E$. Hence  
	\[
	Z(M)=Z(M)/Z(M)\cap E\simeq p(Z(M))\subseteq p(M)=\Core_{G/E}p(A)\subseteq p(A)
	\]
	and therefore $Z(M)$ is cyclic, since $p(A)$ is cylic. 

	\medskip
	Since $B\subseteq A$ is abelian and $(M:B)<|B|$, $B\cap F(M)\ne1$ by Corollary~\ref{cor:Zenkov}. Then $[E,F(M)]=1$ (because $E\subseteq
	Z(F(G))$ and $F(M)\subseteq F(G)$ by Corollary~\ref{cor:McapF(G)}).
	Hence $B\cap F(M)\subseteq Z(M)$, since $M=BE$, $[B\cap F(M),E]\subseteq
	[F(M),E]=1$ and $[B\cap F(M),B]=1$ as $B$ is abelian. Since 
	$Z(M)$ is cyclic, $B\cap F(M)$ is characteristic in $Z(M)$. Since 
	$Z(M)$ is normal in $G$, $\{1\}\ne B\cap F(M)$ is a normal subgroup of $G$
	contained in $A$, a contradiction. 
\end{proof}

\subsection{Horosevskii's theorem}

To conclude this section, we present a striking application of Lucchini's theorem.

\begin{corollary}[Horosevskii]
	\index{Horosevskii's!theorem}
	Let $G$ be a finite non-trivial group and $\sigma\in\Aut(G)$. Then 
	$|\sigma|<|G|$.
\end{corollary}

\begin{proof}
	Let $A=\langle\sigma\rangle$ act by automorphisms on $G$ and 
	 $\Gamma=G\rtimes A$. The group operation of $\Gamma$ is 
	\[
	(g,\sigma^k)(h,\sigma^l)=(g\sigma^k(h),\sigma^{k+l}).
	\]
	Identity $A$ with $\{1\}\times A$ and $G$ with $G\times\{1\}$. 
	Since $K\cap G\subseteq A\cap G=\{1\}$ and $A\cap C_{\Gamma}(G)=\{1\}$, 
	\[
		K\subseteq A\cap C_{\Gamma}(G)=\{1\}.
	\]
	If $k\in K$ and $g\in G$, then $gkg^{-1}k^{-1}\in G\cap K=\{1\}$ (because
	$K$ and $G$ are both normal in $\Gamma$). By Lucchini's theorem, 
    $(A:K)<(\Gamma:A)$, that is
	\[
		|\sigma|=|A|=(A:K)<(\Gamma:A)=|G|.\qedhere
	\]
\end{proof}

\subsection{*Wielandt's automorphism tower theorem}

We now present without proof a beautiful theorem of Wielandt. 
For $G$ a finite group with trivial center, let 
$A_1=G$ and $A_{k+1}=\Aut(A_k)$ for $k\geq1$. Note that 
identifying $G$ with $\Inn(G)$, one gets a sequence
\begin{equation}
\label{eq:automorphism_groups}
A_1\subseteq A_2\subseteq A_3\subseteq\cdots 
\end{equation}
where $A_i$ is normal in $A_{i+1}$. 

\begin{definition}
    \index{Group!complete}
    A group $G$ is said to be \textbf{complete} if $Z(G)=\{1\}$ and
    $\Aut(G)=\Inn(G)$. 
\end{definition}

\begin{example}
For example, the group $\Sym_3$ is complete:
\begin{lstlisting}
gap> G := SymmetricGroup(3);;
gap> IsTrivial(Center(G));
true
gap> AutomorphismGroup(G)=InnerAutomorphismGroup(G);
true    
\end{lstlisting}
In particular, the sequence \eqref{eq:automorphism_groups}
stabilizes. 
\end{example}

\begin{example}
    Let $G=\Sym_3\times\Sym_3$. Then $|G|=36$ and $Z(G)=\{1\}$. Moreover, 
    $|\Aut(G)|=72$ and $|\Aut(\Aut(G))|=144$. Since the group
    $\Aut(\Aut(G))$ is complete, the sequence \eqref{eq:automorphism_groups} stabilizes. Let us do this 
    with the computer: 
\begin{lstlisting}
gap> G := DirectProduct(SymmetricGroup(3),SymmetricGroup(3));;
gap> A1 := G;;
gap> A2 := AutomorphismGroup(G);;
gap> A3 := AutomorphismGroup(A2);;
gap> Order(A2);
72
gap> Order(A3);
144
gap> IsTrivial(Center(A3));
true
gap> AutomorphismGroup(A3)=InnerAutomorphismGroup(A3);
true    
\end{lstlisting}
\end{example}

Let $G$ be a group and $g\in G$. Let
$\gamma_g\colon G\to G$, $x\mapsto gxg^{-1}$, denote the
conjugation map. Then 
\[
\Inn(G)=\{\gamma_g:g\in G\}
\]
is a normal subgroup of $\Aut(G)$. Moreover, $G/Z(G)\simeq\Inn(G)$. 

% Rotman 
\begin{exercise}
\label{xca:Aut}
    Let $G$ be a non-abelian simple group, $A=\Aut(G)$ and
    $I=\Inn(G)$. Prove the following statements:
    \begin{enumerate}
        \item $C_A(I)=\{\id\}$. 
        \item $f(I)=I$ for all $f\in \Aut(A)$.
        \item Every $f\in\Aut(A)$ is inner. 
        \item $\Aut(G)$ is complete. 
    \end{enumerate}    
\end{exercise}

% \begin{sol}{xca:Aut}
%     \begin{enumerate}
%     \item 
%     \end{enumerate}
% \end{sol}

The following result is known as the \textbf{Wielandt automorphism tower theorem}. 

\begin{theorem}[Wielandt]
    \index{Wielandt's automorphism tower theorem}
    \label{thm:Wielandt:automorphism}
    Let $G$ be a finite group with trivial center. Up to isomorphism, there
    are finitely many groups among the terms of the sequence \eqref{eq:automorphism_groups}.  
\end{theorem}

\begin{proof}
    See \cite[Theorem 9.10]{MR2426855}.
\end{proof}
% \chapter{}

\topic{Radical rings and solutions}

Let $S$ be a non-unitary ring. Consider $S_1=\Z\times S$ with the addition defined component-wise and  multiplication
\[
(k,a)(l,b)=(kl,kb+la+ab)
\]
for all $k,l\in\Z$ and $a,b\in S$. 
Then $S_1$ is a ring and $(1,0)$ is its unit element. 
Furthermore, $\{0\}\times S$ is an ideal of $S_1$. 
Note that $\{0\}\times S\simeq S$ as non-unitary rings. Also 
note that if $(k,x)\in S_1$ is invertible, 
then $k\in\{-1,1\}$. 

\begin{definition}
    A non-unitary ring $S$ is a (Jacobson) \textbf{radical ring} 
    if it is isomorphic to the Jacobson radical of a unitary ring.
\end{definition}

Let $R$ be a ring. The (Jacobson) \textbf{radical} $J(R)$ of $R$ is defined as the intersection
of all maximal left ideals of $R$. One proves that $J(R)$ is an ideal of $R$. Moreover, 
$x\in J(R)$ if and only if $1+rx$ is invertible for all $r\in R$.

\begin{proposition}
\label{pro:radical}
	Let $S$ be a non-unitary ring. The following statements are equivalent.
	\begin{enumerate}
		\item $S$ is a radical ring.
		\item For all $a\in S$ there exists a unique $b\in S$ such that $a+b+ab=a+b+ba=0$.
		\item $S\simeq J(S_1)$. 
	\end{enumerate}
\end{proposition}  
	
\begin{proof}
    Let us first prove that $1)\implies2)$. Let $R$ be a unitary ring such that 
    $S\simeq J(R)$ and let $\psi\colon S\rightarrow R$ be an injective homomorphism 
    of non-unitary rings $\psi(S)=J(R)$. Let $a\in S$. Since  
    $1+\psi(a)\in R$ is invertible, there exists $c\in R$ such that 
    \[
    (1+\psi(a))(1+c)=(1+c)(1+\psi(a))=1.
    \]
    Thus 
    $c=-\psi(a)c-\psi(a)\in J(R)$. 
    Hence there exists $b\in S$ such that $\psi(b)=c$. Therefore 
    \[
    a+b+ab=a+b+ba=0.
    \]
    It is an exercise to prove that $b$ is unique. 
    
    We now prove that $2)\implies 3)$. We first note that if 
    $a\in S$, then there exists $b\in S$ such that $a+b+ab=a+b+ba=0$. 
    Thus every 
    $(1,a)\in S_1$ is invertible, as 
    \[
    (1,a)(1,b)=(1,0)=(1,b)(1,a).
    \]

    We claim that $J(S_1)=\{0\}\times S$. Let us prove that 
    $J(S_1)\supseteq \{0\}\times S$. If $(k,a)\in J(S_1)$, then, in particular, 
    \[
    (1+3k,3a)=(1,0)+(3,0)(k,a)
    \]
    is invertible, which implies that either $1+3k=1$ or $1+3k=-1$. Since
    $k\in\Z$, it follows that $k=0$ and hence $(k,a)=(0,a)\in\{0\}\times S$. 
    To prove that 
    $J(S_1)\supseteq \{0\}\times S$ note that
    if $(0,x)\in\{0\}\times S$, then
    \[
    (1,0)+(k,a)(0,x)=(1,kx+ax)
    \]
    is invertible, as $kx+ax\in S$. 
    
    The implication $3)\implies1)$ is trivial.
\end{proof}

A \textbf{nil ring} is a non-unitary ring $S$ such that every 
element of $S$ is nilpotent. Every nil ring is a radical ring.

\begin{example} 
    $X\C[\![X]\!]$ is a radical ring and it is not a nil ring.
\end{example}

Let $S$ be a ring (unitary or non-unitary, it is not important here). 
Define on $S$ the binary operation 
\[
(a,b)\mapsto a\circ b=a+b+ab
\]
for all $a,b\in S$. Then $(S,\circ)$ is a monoid with neutral element $0$.
Note that $S$ is a radical ring if and only if $(S,\circ)$ is a group. 
If $a\in S$ is invertible in the monoid $(S,\circ)$, we will denote by $a'$ its inverse.

\begin{example}
	For $n>1$ let $A=\left\{\frac{nx}{ny+1}:x,y\in\Z\right\}\subseteq \Q$. 
	Note that $A$ is a (non-unitary) subring of $\Q$. In fact, $A$ is a radical ring. A straightforward computation shows that 
	\[
	\left(\frac{nx}{ny+1}\right)'=\frac{-nx}{n(x+y)+1}.
	\]
\end{example}

We now go back to study solutions to the YBE and discuss the intriguing interplay
between radical rings and involutive solutions. 

\begin{definition}
	\index{Solution!involutive}
	A solution $(X,r)$ is said to be \emph{involutive} if $r^2=\id$. 
\end{definition}

Note that if $(X,r)$ is an  involutive solution, then 
\[
(x,y)=r^2(x,y)=r(\sigma_x(y),\tau_y(x))=(\sigma_{\sigma_x(y)}\tau_y(x),\tau_{\tau_y(x)}\sigma_x(y)).
\]
Hence 
\begin{equation}
	\label{eq:involutive}
	\tau_y(x)=\sigma_{\sigma_x(y)}^{-1}(x),
	\quad
	\sigma_x(y)=\tau_{\tau_y(x)}^{-1}(y)
\end{equation}
for all $x,y\in X$. Thus for involutive solutions
it is enough to know $\{\sigma_x:x\in X\}$, as from this we obtain the
set $\{\tau_x:x\in X\}$.

\begin{example}
	Let $X$ be a non-empty set and $\sigma$ be a bijection on $X$. Then 
	$(X,r)$, where $r(x,y)=(\sigma(y),\sigma^{-1}(x))$, is an involutive solution. 
\end{example}

\index{Jacobson!radical ring}
\index{Radical ring}
We now present a very important family of involutive solutions. 

\begin{theorem}[Rump]
	\label{thm:Rump}
	\index{Rump's theorem}
	Let $R$ be a radical ring. Then $(R,r)$, where 
	\[
	r(x,y)=( -x+x\circ y,(-x+x\circ y)'\circ x\circ y)
	\]
	is an involutive solution.
\end{theorem}

The proposition can demonstrated using Theorem~\ref{thm:LYZ}. We will
prove a more general result later. 

\topic{Skew braces}

By convention, an additive group $A$ will be a (not necessarily abelian) group 
with a binary operation $(a,b)\mapsto a+b$. The 
identity of $A$ will be denoted by $0$ 
and the inverse of an element $a$ will be denoted by $-a$. 

\begin{definition}
	\index{Skew brace}
	\index{Skew brace!multiplicative group}
	\index{Skew brace!additive group}
	A \emph{skew left brace} is a triple $(A,+,\circ)$, where $(A,+)$ and $(A,\circ)$ 
	are (not necessarily abelian) 
	groups and 
	\begin{equation}
	    \label{eq:compatibility}
	    a\circ(b+c)=(a\circ b)-a+(a\circ c)
	\end{equation}
	holds for all $a,b,c\in A$. The groups 
	$(A,+)$ and $(A,\circ)$ are respectively 
	the \emph{additive} and \emph{multiplicative} group
	of the skew left brace $A$.
\end{definition}

We write $a'$ to denote the inverse of $a$ with respect to the circle operation $\circ$. 

Skew right braces are defined similarly, one needs 
to replace~\eqref{eq:compatibility} by 
\[
(a+b)\circ c=a\circ c-c+b\circ c.
\]

\begin{exercise}
Prove that there is a bijective correspondence between 
skew left braces and skew right braces. 
\end{exercise}

A skew brace will always mean a skew left brace. 

\begin{definition}
    Let $\mathcal{X}$ be a family of groups. A skew brace $A$ is said to be
    of $\mathcal{X}$-type if its additive group belongs to $\mathcal{X}$.
\end{definition}

One particularly interesting family of skew braces is the family of \emph{skew braces of abelian type}, 
that is skew braces with abelian additive group. 
In the literature, skew braces of abelian type are called \emph{braces}. 

\begin{example}
	\label{exa:trivial}
	\index{Skew brace!trivial}
	Let $A$ be an additive group. Then $A$ is a skew brace with
	$a\circ b=a+b$ for all $a,b\in A$. 
	A skew brace $(A,+,\circ)$ such that $a\circ b=a+b$ for all $a,b\in A$ is
    said to be \emph{trivial}. 
	Similarly, the
   operation $a\circ b=b+a$ turns $A$ into a skew brace. 
\end{example}

\begin{example}
	\label{exa:times}
	\index{Direct product!of skew braces}
	Let $A$ and $B$ be skew braces. Then $A\times B$ with 
	\[
		(a,b)+(a_1,b_1)=(a+a_1,b+b_1),\quad
		(a,b)\circ (a_1,b_1)=(a\circ a_1,b\circ b_1),
	\]
	is a skew brace. This is the {\em direct product} of the skew braces $A$ and $B$. 
\end{example}

\begin{example}
	\label{exa:sd}
	Let $A$ and $M$ be additive groups and let $\alpha\colon A\to\Aut(M)$ be a
	group homomorphism. Then $M\times A$ with 
	\[
	(x,a)+(y,b)=(x+y,a+b),
	\quad
	(x,a)\circ (y,b)=(x+\alpha_a(y),a+b)
	\]
	is a skew brace. Similarly, $M\times A$ with
	\[
	(x,a)+(y,b)=(x+\alpha_a(y),a+b),\quad
	(x,a)\circ (y,b)=(x+y,b+a)
	\]
	is a skew brace. 
\end{example}

\begin{example}
    \label{exa:WX}
    Let $A$ be an additive group
	and $B$ and $C$ be subgroups of $A$ such that $B\cap C=\{ 0\}$ and $A=B+C$. In this case, one says that $A$ admits an {\em exact factorization} through the subgroups $B$ and $C$.  Thus each $a\in A$ can be written in a unique
	way as $a=b+c$, for some $b\in B$ and $c\in C$.  The map
	\[
		B\times C\to A,\quad
		(b,c)\mapsto b-c,
	\]
	is bijective. Using this map we transport the group structure of the direct
	product $B\times C$ into the set $A$. That is, for $a=b+c\in A$, where $b\in B$ and $c\in C$, and
	$a_1\in A$, let 
	\begin{align*}
		a\circ a_1&=b+a_1+c.
	\end{align*}
	Then $(A,\circ)$ is a group isomorphic to $B\times C$. Moreover, if $x,y\in A$, 
	then 
	\begin{align*}
	a\circ x-a+a\circ y=b+x+c-(b+c)+b+y+c=b+x+y+c=a\circ (x+y),
	\end{align*}
	and therefore $(A,+,\circ)$ is a skew brace. 
\end{example}

% \begin{proof} The map $\eta\colon B\times C\to A$, $\eta(b,c)=bc^{-1}$, is
%   bijective.  Since $\eta$ is bijective and $a\circ
%   a'=\eta(\eta^{-1}(a)\eta^{-1}(a'))$, it follows that $(A,\circ)$ is a group
%   isomorphic to the direct product $B\times C$. To prove that $A$ is a skew
%   brace it remains to show~\eqref{eq:compatibility}. Let $a=bc\in BC$ and
%   $a',a''\in A$. Then \begin{align*} (a\circ a')a^{-1}(a\circ a'')
%     &=(ba'c)a^{-1}(ba''c)\\ &=ba'c(c^{-1}b^{-1})ba''c\\ &=ba'a''c\\ &=a\circ
%     (a'a'').  \end{align*} This completes the proof.  \end{proof}

We now give concrete some examples of the previous construction. 

\begin{example}
  \label{exa:QR}
  Let $n$ be a positive integer. 
  The group $\GL_n(\C)$ admits an
  exact factorization through the subgroups $U(n)$ and $T(n)$, where 
  \[
  U(n)=\{ A\in\GL_n(\C): AA^*=I\}
  \]
  is the unitary group and $T(n)$ is the group of upper triangular matrices
  with positive diagonal entries.  Therefore there exists a skew brace with additive group 
  isomorphic to $\GL_n(\C)$ and multiplicative group isomorphic to $U(n)\times T(n)$.  
\end{example}

The following examples appeared in the theory of 
Hopf--Galois structures.

\begin{example} 
	\label{exa:a5a4c5}
	The alternating simple group $\Alt_5$ admits an exact factorization
  through the subgroups 
  $A=\langle (123),(12)(34)\rangle\cong\Alt_4$ and 
  $B=\langle(12345)\rangle\cong C_5$.  
  There exists a skew brace with additive group isomorphic to $\Alt_5$ and multiplicative
  group isomorphic to $\Alt_4\times C_5$. 
\end{example}

Let us review some basic properties of skew braces. 

\begin{exercise}
\label{xca:0=1}
Let $A$ be a skew brace. Then the following properties hold:
\begin{enumerate}
    \item The neutral element of the additive group of $A$ coincides with 
    the neutral element of the multiplicative group of $A$. It will be denoted
    by $0$. 
    \item $a\circ(-b+c)=a-(a\circ b)+(a\circ c)$, for all $a,b,c\in A$.
    \item $a\circ(b-c)=(a\circ b)-(a\circ c)+a$, for all $a,b,c\in A$.
\end{enumerate}
\end{exercise}

\begin{exercise}
\label{xca:lambda}
    Let $A$ be a skew brace. For each $a\in A$, the map
    \[
        \lambda_a\colon A\to A,\quad
        b\mapsto -a+(a\circ b),
    \]
    is an automorphism of $(A,+)$. Moreover, the map 
    $\lambda\colon (A,\circ)\to\Aut(A,+)$, $a\mapsto\lambda_a$, is a group homomorphism. 
\end{exercise}

\begin{exercise}
\label{xca:mu}
    Let $A$ be a skew brace. For each $a\in A$, the map
    \[
        \mu_a\colon A\to A,\quad
        b\mapsto \lambda_b(a)'\circ b\circ a,
    \]
    is bijective. Moreover, the map 
    $\mu\colon (A,\circ)\to\Sym_A$, $a\mapsto\mu_a$, satisfies $\mu_b\circ\mu_a=\mu_{a\circ b}$, for all $a,b\in A$. 
\end{exercise}

Let $A$ be a skew brace. 
Exercise \ref{xca:lambda} implies that 
\begin{align}
\label{eq:formulas}
&a\circ b = a+\lambda_a(b),
&&a+b=a\circ \lambda^{-1}_a(b),
&&\lambda_a(a')=-a
\end{align}
hold for $a,b\in A$. Moreover, if 
\[
    a*b=\lambda_a(b)-b=-a+a\circ b-b,
\]
then the following identities are easily verified:
\begin{align}
&a*(b+c)=a*b+b+a*c-b,\\
&(a\circ b)*c=(a*(b*c))+b*c+a*c.
\end{align}

 \begin{definition}
 	\index{Homomorphism!of skew braces}
 	A map $f\colon A\to B$ between two skew braces $A$ and $B$ is a {\em homomorphism of skew braces} 
 	if $f(x\circ y)=f(x)\circ f(y)$ and $f(x+y)=f(x)+f(y)$ for all $x,y\in A$.  The \emph{kernel} of $f$ is
     \[
         \ker f=\{a\in A:f(a)=0\}.
     \]
 \end{definition}

A bijective homomorphism of skew braces is an isomorphism. An automorphism of a skew brace $A$ is an isomorphism from the skew brace $A$ to it self. Two skew braces $A$ and $B$ are isomorphic if there exist an isomorphism $f\colon A\rightarrow B$. We write $A\simeq B$ to denote that the skew braces $A$ and $B$ are isomorphic.

\begin{definition}
    \index{Skew brace!two sided}
	A skew brace $A$ is said to be \textbf{two-sided} if 
	\begin{equation}
	\label{eq:right_compatibility}
	(a+b)\circ c=a\circ c-c+b\circ c
	\end{equation}
	holds for all $a,b,c\in A$. 
\end{definition}

If $A$ is a skew two-sided brace, then 
\begin{align}
\label{eq:2sided}
&a\circ(-b)=a-a\circ b+a,
&&(-a)\circ b=b-a\circ b+b    
\end{align}
hold for all $a,b\in A$. The first equality holds for every skew brace and follows 
from the compatibility condition. 
The second equality follows from~\eqref{eq:right_compatibility}. 

\begin{example}
  Any skew brace with abelian multiplicative group is 
  two-sided.
\end{example}

\begin{exercise}
\label{xca:2sided}
	Let $A$ be a skew brace of abelian type such that $\lambda_a(a)=a$ for all $a\in A$.
	Prove that $A$ is two-sided.
\end{exercise}

\index{Jacobson!radical ring}
\index{Radical ring}
Two-sided skew braces of abelian type form an interesting family of non-unitary rings.
Thus skew braces form a far reaching generalization of radical rings. 

\begin{theorem}[Rump]
\label{thm:radical}
\index{Rump's theorem}
    A skew brace of abelian type is two-sided if and only if it is a radical ring. 
\end{theorem}

\begin{proof}
    Assume first that $A$ is a skew two-sided brace of abelian type. Then $(A,+)$ is an abelian group. 
    Let us prove that the operation
    \[
    a*b=-a+a\circ b-b
    \]
    turns $A$ into a radical ring. Left distributivity follows from the compatibility condition:
    \begin{align*}
    a*(b+c)&=-a+a\circ (b+c)-(b+c)
    =-a+a\circ b-a+a\circ c-c-b=a*b+a*c.
    \end{align*}
    Similarly, since $A$ is two-sided, one proves $(a+b)*c=a*c+b*c$. It remains to show that the operation $*$
    is associative. On the one hand, using the first equality of~\eqref{eq:2sided} 
    and the compatibility condition, we write
    \begin{align*}
    a*(b*c)&=a*(-b+b\circ c-c)\\
    &=-a+a\circ(-b+b\circ c-c)-(-b+b\circ c-c)\\
    &=-a+a\circ (-b)-a+a\circ(b\circ c)-a+a\circ (-c)+c-b\circ c+b\\
    &=a\circ (b\circ c)-a\circ b-a\circ c-b\circ c+a+b+c,
    \end{align*}
    since the group $(A,+)$ is abelian. On the other hand, the second equality of~\eqref{eq:2sided} and
    Equality~\eqref{eq:right_compatibility} imply that
    \begin{align*}
    (a*b)*c &= (-a+a\circ b-b)*c=-(-a+a\circ b-b)+(-a+a\circ b-b)\circ c-c\\
    &=b-a\circ b+a+(-a)\circ c-c+(a\circ b)\circ c-c+(-b)\circ c-c\\
    &=(a\circ b)\circ c-a\circ b-a\circ c-b\circ c+a+b+c.
    \end{align*}
    It then follows that the operation $*$ is associative. 
    
    Conversely, if $A$ is a radical ring, say with ring multiplication $(a,b)\mapsto ab$, 
    then $a\circ b=a+ab+b$ turns $A$ into a skew two-sided brace 
    of abelian type. In fact, since $A$ is a radical ring, then 
    $(A,+)$ is an abelian group and $(A,\circ)$ is a group. Moreover, 
    \begin{align*}
        a\circ (b+c)=a+a(b+c)+(b+c)=a+ab+ac+b+c=a\circ b-a+a\circ c.
    \end{align*}
    Similarly ones proves $(a+b)\circ c=a\circ c-c+b\circ c$.
\end{proof}

A natural question arises: Does one need radical rings? Surprisingly, 
radical rings are just the tip of the iceberg. 

\begin{theorem}
\label{thm:YB}
Let $A$ be a skew brace. Then 
$(A,r)$, where 
\[
r\colon A\times A\to A\times A,\quad
r(x,y)=( -x+x\circ y,(-x+x\circ y)'\circ x\circ y),
\]
is a solution to the YBE. 
\end{theorem}

\begin{proof}
    By Theorem~\ref{thm:LYZ}, 
    since $x\circ y=(-x+x\circ y)\circ ((-x+x\circ y)'\circ x\circ y)$ for all $x,y\in A$, 
    we only need to check that 
    $x\rhd y=\lambda_x(y)=-x+x\circ y$ 
    is a left action of $(A,\circ)$ on the set $A$ 
    and that $x\lhd y=\mu_y(x)=(-x+x\circ y)'\circ x\circ y$ 
    is a right action of $(A,\circ)$ on the set $A$. For the left action we use 
    Exercise~\ref{xca:lambda} and for the right action we use Exercise~\ref{xca:mu}.
\end{proof}

\begin{exercise}
Let $A$ be a skew brace. 
Prove that 
\[
\mu_b(a)=\lambda^{-1}_{\lambda_a(b)}(-a\circ b+a+a\circ b).
\]
\end{exercise}

In Theorem~\ref{thm:YB} it is possible to prove that the solution 
is involutive if and only if the additive group of the brace is abelian. We will
prove a generalization of this result. For that purpose, we need a lemma. 

\begin{lemma}
\label{lem:|r|}
Let $A$ be a skew brace and $r$ be its associated solution.  Then
  \begin{align} 
  \nonumber
  r^{2n}(a,b)&=(-n(a\circ b)+a+n(a\circ
    b),\\
    \label{eq:r^2n}
    &\phantom{=(-n(a\circ b)+}(-n(a\circ b)+a+n(a\circ b))'\circ a\circ b),\\
  \nonumber
  r^{2n+1}(a,b)&=(-n(a\circ b)-a+(n+1)(a\circ
    b),\\
    \label{eq:r^2n+1}
    &\phantom{=(-n(a\circ b)+}(-n(a\circ b)-a+(n+1)(a\circ b))'\circ a\circ b),
    \end{align} 
    for all $n\geq0$.  Moreover, the following statements hold:
  \begin{enumerate} 
  \item $r^{2n}=\id$ if and only if $a+nb=nb+a$ for all $a,b\in A$.  
      \item $r^{2n+1}=\id$ if and only if $\lambda_a(b)=n(a\circ
	b)+a-n(a\circ b)$ for all $a,b\in A$.  
	\end{enumerate} 
\end{lemma}

\begin{proof} 
First we shall prove~\eqref{eq:r^2n} and~\eqref{eq:r^2n+1} by induction on $n$. The case $n=0$ is trivial for~\eqref{eq:r^2n}
  and~\eqref{eq:r^2n+1}. Assume that the claim holds for some $n\geq 0$. By applying the map $r$ to Equation~\eqref{eq:r^2n+1} 
  we obtain that 
  \begin{align*} 
  r^{2(n+1)}(a,b) &=r\left( -n(a\circ b)-a+(n+1)(a\circ b),\right.\\
    &\phantom{=(-n(a\circ b)+} \left. (-n(a\circ b)-a+(n+1)(a\circ
    b))'\circ a\circ b\right)\\
    &=\left( -(n+1)(a\circ b)+a+(n+1)(a\circ b),\right.\\
    &\phantom{=(-n(a\circ b)+} \left. (-(n+1)(a\circ b)+a+(n+1)(a\circ
    b))'\circ a\circ b\right).
    \end{align*} 
    By applying $r$ again to this equality, we get 
    \begin{align*} 
  r^{2(n+1)+1}(a,b) &= r\left(-(n+1)(a\circ b)+a+(n+1)(a\circ
    b),\right.\\
    &\phantom{=(-n(a\circ b)+} \left. (-(n+1)(a\circ b)+a+(n+1)(a\circ b))'\circ a\circ b\right)\\
    &=\left( -(n+1)(a\circ b)-a+(n+2)(a\circ b),\right.\\
    &\phantom{=(-n(a\circ b)+} \left. (-(n+1)(a\circ b)-a+(n+2)(a\circ
    b))'\circ a\circ b\right).
    \end{align*} 
   Thus Equations~\eqref{eq:r^2n} and~\eqref{eq:r^2n+1} hold by induction.  The other claims follow easily from
    Equations~\eqref{eq:r^2n} and~\eqref{eq:r^2n+1}.
\end{proof}

%\begin{thm} \label{pro:depth_even} Let $A$ be a skew brace of finite depth
%with more than one element and let $r_A$ be its associated solution. Then the
%order of $r_A$ is an even number.  \end{thm}
%
%\begin{proof} Let $n$ be such that $r^{2n+1}=\id$. By applying
%Lemma~\ref{lem:depth} one obtains that $a^{-1}(a\circ b)^{n+1}=(a\circ b)^na$
%for all $a,b\in A$. In particular, if $b=1$, then $a=1$.  \end{proof}

Recall that the (minimal) \emph{exponent} $\exp(G)$ of a 
finite group $G$ is the least positive integer $n$ such that 
$g^n=1$ for all $g\in G$. 

\begin{theorem} 
\label{thm:|r|} 
  Let $A$ be a finite skew brace with more than one
  element and let $K$ be the additive group of $A$. 
  If $r$ is the solution associated with $A$, 
  then, as a permutation, $r$ has order $2\exp(K/Z(K))$.
\end{theorem}

\begin{proof} 
  Suppose that $r$ has odd order $2n+1$. Since $r^{2n+1}=\id$, 
  Lemma~\ref{lem:|r|} implies that $-a+(n+1)(a\circ b)=n(a\circ b)+a$
  for all $a,b\in A$. In particular, for $b=0$, we get $a=0$, 
  for all $a\in A$, a contradiction. 
  Therefore we may assume that the order of the permutation $r$ is
  $2n$, where 
  \[
  n=\min\{k\in\Z: k>0\text{ and }kb+a=a+kb\;\text{ for all }a,b\in A\}.
  \]
  Now one computes
  \begin{align*} 
  n&=\min\{k\in\Z: k>0\text{ and }kb\in Z(G)\text{ for all }b\in A\}\\ 
  &=\min\{k\in\Z: k>0\text{ and }k(b+Z(G)) = Z(G)\text{ for all }b\in A\} =\exp(G/Z(G)).\qedhere
  \end{align*}
\end{proof}

An inmmediate consequence is the following result.

\begin{corollary}
    Let $A$ be a finite skew brace and $r$ be its associated solution. Then 
    $r$ is involutive if and only if $A$ is of abelian type.
    \end{corollary}

% \begin{proof}
%     It follows immediately from Theorem~\ref{thm:|r|}.
% \end{proof}

%\begin{exa} A skew brace has depth one if and only if its additive group is
%  abelian.  \end{exa}

% \begin{exa} 
%   \label{exa:2p} 
%   Let $p$ be an odd prime number and let $A$ be a non-classical skew brace of
%   size $2p$. Then the additive group of $A$ is isomorphic to the dihedral group
%   $\D_{2p}$ of size $2p$.  Since $Z(\D_{2p})=1$ and the exponent of $\D_{2p}$
%   is $2p$, the order of $r_A$ is $4p$.
% \end{exa}

\topic{Ideals}

\begin{definition}
\index{Subbrace}
Let $A$ be a skew brace. A \emph{subbrace} of $A$ is a non-empty 
subset $B$ of $A$ such that $(B,+)$ is a subgroup of $(A,+)$ and $(B,\circ)$ is a subgroup of $(A,\circ)$. 
\end{definition}

\begin{definition}
    \index{Left!ideal}
    \index{Strong!left ideal}
    Let $A$ be a skew brace. A \emph{left ideal} of $A$ is a subgroup $(I,+)$ of
	$(A,+)$ such that $\lambda_a(I)\subseteq I$ for all $a\in A$, i.e. $\lambda_a(x)\in I$ for all $a\in A$ and $x\in I$. A \emph{strong left ideal} of $A$ 
	is a left ideal $I$ of $A$ such that $(I,+)$ is a normal subgroup of $(A,+)$. 
\end{definition}

\begin{example}
    Let $A$ be a skew brace and $I$ be a characteristic subgroup 
    of the additive group of $A$. Then 
    $I$ is a left ideal of $A$. 
\end{example}

Recall that skew two-sided braces of abelian type 
are equivalent to radical rings. 
One can prove that under this equivalence, 
(left) ideals of the radical ring correspond 
to (left) ideals of the associated brace. 

\begin{proposition}
    A left ideal $I$ of a skew brace $A$ is a subbrace of $A$. 
\end{proposition}

\begin{proof}
    We need to prove that $(I,\circ)$ is a subgroup of $(A,\circ)$. Clearly $I$ is non-empty, 
    as it is an additive subgroup of $A$. If $x,y\in I$, then
    $x\circ y=x+\lambda_x(y)\in I+I\subseteq I$ and $x'=-\lambda_{x'}(x)\in I$. 
\end{proof}

\begin{example}
    Let $A$ be a skew brace. Then 
    \[
    \Fix(A)=\{a\in A:\lambda_x(a)=a\text{ for all $x\in A$}\}
    \]
    is a left ideal of $A$. 
\end{example}

\begin{definition}
    \index{Ideal}
    An \emph{ideal} of a skew brace $A$ is a strong left ideal $I$ of $A$ such that 
	$(I,\circ)$ is a normal subgroup of $(A,\circ)$. 
\end{definition}

In general 
\[
\{\text{subbraces}\}\supsetneq \{\text{left ideals}\}\supsetneq\{\text{strong left ideals}\}\supsetneq\{\text{ideals}\}.
\]
For example, $\Fix(A)$ is not a strong left ideal of $A$.

\begin{example}
    Consider the semidirect product $A=\Z/(3)\rtimes \Z/(2)$ of the
    trivial skew braces $\Z/(3)$ and $\Z/(2)$
    via the non-trivial action of $\Z/(2)$ over $\Z/(3)$.
    Then 
    \[
    \lambda_{(x,y)}(a,b)=-(x,y)+(x,y)\circ(a,b)=-(x,y)+(x+(-1)^ya,y+b)=((-1)^ya,b).
    \]
    Then $\Fix(A)=\{(0,0),(0,1)\}$ is not a 
    normal subgroup of $(A,+)$ and hence $\Fix(A)$ is not a strong left 
    ideal of $A$.
\end{example}

\begin{example}
    \index{Kernel}
	Let $f\colon A\to B$ be a homomorphism of skew braces. Then $\ker f$ 
	is an ideal of $A$.
\end{example}

If $X$ and $Y$ are subsets of a brace $A$, $X*Y$ is defined as the 
subgroup of $(A,+)$ generated by elements of the form $x*y$, $x\in X$ and $y\in Y$, i.e.
\[
X*Y=\langle x*y:x\in X\,,y\in Y\rangle_+.
\]

\begin{proposition}
    \label{pro:A*I}
    Let $A$ be a skew brace. A subgroup $I$ of $(A,+)$ is 
    a left ideal of $A$ if and only if $A*I\subseteq I$.
\end{proposition}

\begin{proof}
    Let $a\in A$ and $x\in I$. If $I$ is a
    left ideal, then $a*x=\lambda_a(x)-x\in I$. Conversely, if $A*I\subseteq
    I$, then $\lambda_a(x)=a*x+x\in I$.
\end{proof}

\begin{proposition}
    \label{pro:I*A}
    Let $A$ be a skew brace. A normal subgroup $I$ of $(A,+)$
    is an ideal of $A$ if and only $\lambda_a(I)\subseteq I$, for all $a\in A$, and
    $I*A\subseteq I$.
\end{proposition}

\begin{proof}
    Let $x\in I$ and $a\in A$.  Assume first that $I$ is invariant under the
    action of $\lambda$ and that $I*A\subseteq I$. Then
    \begin{equation}
    \label{eq:trick:I*A}
        \begin{aligned}
        a\circ x\circ a' &=a+\lambda_a(x\circ a')\\
        &=a+\lambda_a(x+\lambda_x(a'))
        =a+\lambda_a(x)+\lambda_a\lambda_x(a')+a-a\\
        &=a+\lambda_a(x+\lambda_x(a')-a')-a
        =a+\lambda_a(x+x*a')-a\in I,
    \end{aligned}
    \end{equation}
    and hence $I$ is an ideal.

    Conversely, assume that $I$ is an ideal. Then $I*A\subseteq I$ since
    \begin{align*}
        x*a&=-x+x\circ a-a\\
        &=-x+a\circ(a'\circ x\circ a)-a
        =-x+a+\lambda_a(a'\circ x\circ a)-a\in I.\qedhere
    \end{align*}
\end{proof}


Let $I$ and $J$ be ideals
of a skew brace $A$. Then $I\cap J$ is an ideal of $A$.  
The sum $I+J$ of $I$ and $J$ is defined as the
additive subgroup of $A$ generated by all the 
elements of the form
$u+v$, $u\in I$ and $v\in J$. 

\begin{proposition}
Let $A$ be a skew brace and let
$I$ and $J$ be ideals of $A$. Then $I+J$ is an ideal of $A$.
\end{proposition}

\begin{proof}
    Since $I$ and $J$ are normal subgroups of $A$, we have that
    \[ I+J=\{ u+v \mid u\in I,\; v\in J \}.\]
    First note that $I+J$ is a normal subgroup of $(A,+)$ since
    \[
        a+(u+v)-a=(a+u-a)+(a+v-a)\in I+J
    \]
    for all $u\in I$, $v\in J$ and $a\in A$.
    Let $a\in A$, $u\in I$ and $v\in J$. Then $\lambda_a(u+v)=\lambda_a(u)+\lambda_a(v)\in I+J$ and
    hence it follows that $\lambda_a(I+J)\subseteq I+J$. Moreover, by Propositions~\ref{pro:A*I} and~\ref{pro:I*A},
        \[
        (u+v)*a=(u\circ\lambda^{-1}_u(v))*a
        =u*(\lambda^{-1}_u(v)*a)+\lambda^{-1}_u(v)*a+u*a\in I+J.
    \]
    Hence $(I+J)*A\subseteq I+J$. Therefore the result follows by Proposition~\ref{pro:I*A}.
\end{proof}


\begin{definition}
	\index{Socle}
	Let $A$ be a skew brace. The subset 
	$\Soc(A)=\ker\lambda\cap Z(A,+)$
	is the \emph{socle} of $A$.
\end{definition}

We will use the following exercise several times. 

\begin{exercise}
    \label{xca:socle}
    Let $A$ be a skew brace and $a\in\Soc(A)$. Prove that  
    \[
    b+b\circ a=b\circ a+b\quad\text{and}\quad
    \lambda_b(a)=b\circ a\circ b'
    \]
    hold 
    for all $b\in A$.
\end{exercise}

\begin{exercise}
\label{xca:Bachiller1}
    Prove that the socle of a skew brace $A$ is the kernel of the 
    group homomorphism $(A,\circ)\to\Aut(A,+)\times\Sym_A$, $a\mapsto (\lambda_a,\mu_a^{-1})$. 
\end{exercise}

\begin{exercise}
\label{xca:Bachiller2}
    Prove that the socle of a skew brace $A$ is the kernel of the 
    group homomorphism 
    \[
    (A,\circ)\to\Aut(A,+)\times\Aut(A,+),
    \quad
    a\mapsto (\lambda_a,\xi_a),
    \]
    where
    $\xi_a(b)=a+\lambda_a(b)-a$. 
\end{exercise}

\begin{proposition}
	\label{pro:socle}
	Let $A$ be a skew brace. Then $\Soc(A)$ is an ideal of $A$.
\end{proposition}

	
	\begin{proof}
		Clearly $0\in\Soc(A)$, since $\lambda$ is a group homomorphism. Let $a,b\in\Soc(A)$ and $c\in A$. Since 
		$b\circ (-b)=b+(-b)=0$, it follows that 
		$b'=-b\in\Soc(A)$. The calculation 
		\[
		\lambda_{a-b}(c)=\lambda_{a\circ b'}(c)=\lambda_a\lambda^{-1}_b(c)=c,
		\]
 		implies that $a-b\in\ker\lambda$. Since $a-b\in Z(A,+)$, it follows that 
        $(\Soc(A),+)$ is a normal subgroup of $(A,+)$. 
        
        For each $d\in A$, $a+c'\circ d=c'\circ d+a$.  By Exercise~\ref{xca:socle}, we have 
        \begin{align*}
        d+\lambda_c(a) &= d-c+c\circ a
        =c\circ (c'\circ d+a)\\
        &= c\circ (a+c'\circ d)
        = c\circ a-c+d
        = -c+c\circ a+d
        = \lambda_c(a)+d,
        \end{align*}
        that is $\lambda_c(a)$ is central in $(A,+)$. Moreover, again by Exercise~\ref{xca:socle},
        \begin{align*}
            \lambda_c(a)+d &= -c+c\circ a+d 
            = c\circ a-c+d\\
            &= c\circ (a+(c'\circ d)
            = c\circ a\circ c'\circ d=\lambda_c(a)\circ d
        \end{align*}
        and hence 
        \[
        \lambda_{\lambda_c(a)}(d)=-\lambda_c(a)+\lambda_c(a)\circ d=-\lambda_c(a)+\lambda_c(a)+d=d.
        \]
        Therefore $\Soc(A)$ is a strong left ideal of $A$. In fact, $\Soc(A)$ is an ideal of $A$,
        as $c\circ a\circ c'=\lambda_c(a)\in\Soc(A)$.  
	\end{proof}

As a corollary we obtain that the socle of a skew 
brace $A$ is a trivial skew brace of abelian type. 

\begin{proposition}
    \label{pro:soc_kernels}
    Let $A$ be a skew brace. Then $\Soc(A)=\ker\lambda\cap\ker\mu$.
\end{proposition}

\begin{proof}
    Let $a\in\Soc(A)$ and $b\in A$. Then $\lambda_a=\id$ and $a\in Z(A,+)$. By Exercise~\ref{xca:socle}, \[\mu_a(b)=\lambda_b(a)'\circ b\circ a=(b\circ a\circ b')'\circ b\circ a=b.\]  Thus $a\in\ker\lambda\cap\ker\mu$. 
    
    Conversely, let $a\in\ker\lambda\cap\ker\mu$ and $b\in A$. Then $b'=\mu_a(b')=\lambda_{b'}(a)'\circ b'\circ a$, so
    $\lambda_{b'}(a)=b'\circ a\circ b$. Now 
    \[
    b+a=b\circ\lambda^{-1}_b(a)=b\circ\lambda_{b'}(a)=b\circ b'\circ a\circ b=a\circ b=a+\lambda_a(b)=a+b
    \]
    implies that $a\in\Soc(A)$. 
\end{proof}

\begin{definition}
\index{Annihilator}
Let $A$ be a skew brace. The \emph{annihilator} of $A$ is 
defined as the set $\Ann(A)=\Soc(A)\cap Z(A,\circ)$. 
\end{definition}

Note that $\Ann(A)\subseteq\Fix(A)$. 

\begin{proposition}
The annihilator of a skew brace $A$ is an ideal of $A$. 
\end{proposition}

\begin{proof}
    Let $x,y\in\Ann(A)$. Note that $x-y=x\circ y'\in Z(A,\circ)$. Hence $\Ann(A)$ is a subbrace of $A$. Since $\Ann(A)\subseteq Z(A,+)\cap Z(A,\circ)$, 
    we only need to note that $\lambda_a(x)=x\in\Ann(A)$, for all $a\in A$. 
\end{proof}



%Clearly $\Soc(A)=\ker(\lambda)\cap Z(A,+)$. In \cite[Lemma~2.5]{MR3647970} it
%is proved that $\Soc(A)$ is an ideal of $A$.

\index{Quotient brace}
If $A$ is a skew brace and $I$ is an ideal of $A$, then $a+I=a\circ I$ for all $a\in A$. Indeed, 
$a\circ x=a+\lambda_a(x)\in a+I$ and 
$a+x=a\circ\lambda_a^{-1}(x)=a\circ\lambda_{a'}(x)\in a\circ I$ 
for all $a\in A$ and $x\in I$. 
This allows us to prove that there exists a unique skew brace structure over $A/I$ such that
the map 
\[
\pi\colon A\to A/I,
\quad
a\mapsto a+I=a\circ I,
\]
is a homomorphism of skew braces. The skew brace $A/I$ 
is the \textbf{quotient brace} of $A$ modulo $I$. 

\begin{exercise}
\label{xca:iso1}
    Let $f\colon A\to B$ be a homomorphism of skew braces. Prove that $A/\ker f\simeq f(A)$. 
\end{exercise}

\begin{exercise}
\label{xca:iso2}
    Let $A$ be a skew brace and let $B$ be a subbrace of $A$. Prove that if $I$ is an ideal of $A$, 
    then $B\circ I$ is a subbrace of $A$, 
    $B\cap I$ is an ideal of $B$ and $(B\circ I)/I\cong B/(B\cap I)$. 
\end{exercise}

\begin{exercise}
\label{xca:iso3}
Let $A$ be a skew brace and $I$ and $J$ be ideals of $A$. Prove that if $I\subseteq J$, then
$A/J\cong (A/I)/(J/I)$. 
\end{exercise}

\begin{exercise}
\label{xca:correspondence}
Let $A$ be a skew brace and let $I$ be an ideal of $A$. Prove that there is a bijective correspondence between (left) ideals 
of $A$ containing $I$ and (left) ideals of $A/I$. 
\end{exercise}




% \chapter{}

\topic{Braces and 1-cocycles}

Let $K$ and $Q$ be groups.
An {\em extension} of $K$ by $Q$ is a
short exact sequence of group homomorphisms 
\[
\begin{tikzcd}
	1 & K & G & Q & 1
	\arrow[from=1-1, to=1-2]
	\arrow["f", from=1-2, to=1-3]
	\arrow["g", from=1-3, to=1-4]
	\arrow[from=1-4, to=1-5]
\end{tikzcd}
\]
This means that $f$ is injective, $g$ is surjective and $\ker g=\im f$. Note that in this case, $K$ is isomorphic to $f(K)$, which is a normal subgroup of $G$ and $G/f(K)\simeq Q$. We also say that $G$ is an extension of $K$ by $Q$.

\begin{example}
	$C_6$ and $\Sym_3$ are both extensions of $C_3$ by $C_2$.
\end{example}

\begin{example}
	$C_6$ is an extension of $C_2$ by $C_3$.
\end{example}

\begin{example}
    The direct product $K\times Q$ of the groups $K$ and $Q$ 
    is an extension of $K$ by $Q$ and an extension of $Q$ by $K$. 
\end{example}

\begin{example}
Let $G$ be an extension of $K$ by $Q$. If $L$ is a subgroup of $G$ containing $K$, 
then $L$ is an extension
of $K$ by $L/K$.
\end{example}

\index{Lifting} 
Let $E:
\begin{tikzcd}
	1 & K & G & Q & 1
	\arrow[from=1-1, to=1-2]
	\arrow[from=1-2, to=1-3]
	\arrow["p", from=1-3, to=1-4]
	\arrow[from=1-4, to=1-5]
\end{tikzcd}$
be an extension of groups. A {\em lifting} of $E$ is a map $\ell\colon
Q\to G$ such that $p(\ell(x))=x$, for all $x\in Q$. 

\index{Split!extension}
An extension $E$ {\em splits} if there is a lifting of $E$ that it is a group
homomorphism. 

\begin{exercise}
	\label{xca:lifting}
	Let $E:
	\begin{tikzcd}
	1 & K & G & Q & 1
	\arrow[from=1-1, to=1-2]
	\arrow[from=1-2, to=1-3]
	\arrow["p", from=1-3, to=1-4]
	\arrow[from=1-4, to=1-5]
    \end{tikzcd}$
	be an extension. 
	\begin{enumerate}
		\item If $\ell\colon Q\to G$ is a lifting, then $\ell(Q)$
			is a transversal of $\ker p$ in $G$.
		\item Each transversal of $\ker p$ in $G$ induces a lifting $\ell\colon
			Q\to G$.
		\item If $\ell\colon Q\to G$ is a lifting, then 
			$\ell(xy)\ker p=\ell(x)\ell(y)\ker p$.
	\end{enumerate}
\end{exercise}


\index{Derivation}
\index{$1$-cocycle}
Let $Q$ and $K$ be groups. Assume that $Q$ acts by automorphisms on $K$, that is there is a group homomorphism $\alpha\colon Q\rightarrow \Aut(k)$. We write $\alpha_x=\alpha(x)$, for all $x\in Q$.
A map $\varphi\colon Q\to K$ is said to be a {\em $1$-cocycle} (or a derivation) if
\[
		\varphi(xy)=\varphi(x)\alpha_x(\varphi(y)),
\]
for all $x,y\in Q$.  The set of 1-cocycles $Q\to K$ is defined as 
\[
\Der(Q,K)=Z^1(Q,K)=\{\delta\colon Q\to K:\text{$\delta$ is $1$-cocycle}\}.
\]

\begin{example}
	Let $K$ and $Q$ be groups. Let $\alpha\colon Q\rightarrow \Aut(K)$ be a group homomorphism.  
	For each $k\in K$, the map 
	$\delta_k\colon Q\to K$, $x\mapsto \delta_k(x)=k\alpha_x(k)^{-1}$, is a derivation. One typically writes
	$\delta_k(x)=[k,x]$. 
\end{example}

\begin{exercise}
	\label{xca:1cocycle}
	Let $\varphi\colon Q\to K$ be a 1-cocycle. 
	\begin{enumerate}
		\item $\varphi(1)=1$.
		\item $\varphi(y^{-1})=(y^{-1}\cdot\phi(y))^{-1}=y^{-1}\cdot\phi(y)^{-1}$.
		\item The set $\ker\varphi=\{x\in Q:\varphi(x)=1\}$ is a subgroup of $Q$. 
	\end{enumerate}
\end{exercise}


% \begin{svgraybox}
% 	Para $k\in K$ y $x\in Q$ escribimos $\delta_k(x)=[k,x]$. Entonces 
% 	\[
% 	\delta_k(x)(x\delta_k(y)x^{-1})
% 	=kxk^{-1}x^{-1}xkyk^{-1}y^{-1}x^{-1}
% 	=k(xy)k^{-1}(xy)^{-1}
% 	=\delta_k(xy).
% 	\]
% \end{svgraybox}


A subgroup $K$ of a group $G$ admits a {\em complement} $Q$ if $G$ admits an exact factorization 
through $K$ and $Q$, i.e. $G=KQ$ with $K\cap Q=\{1\}$. 
A classical example is the (inner) semidirect product $G=K\rtimes Q$, where $K$ is a normal subgroup of $G$ 
and $Q$ is a subgroup of $G$ such that $K\cap Q=\{1\}$. 

Let $K$ and $Q$ be groups. Let $\alpha\colon Q\rightarrow \Aut(K),\,x\mapsto \alpha_x$ be a group homomorphism. Consider the set $K\times Q$. We define a multiplication on this set by
\[(k_1,q_1)(k_2,q_2)=(k_1\alpha_{q_1}(k_2),q_1q_2),\]
for all $k_1,k_2\in K$ and $q_1,q_2\in Q$. Then $K\times Q$ with this multiplication is a group, called the semidirect product of $K$ by $Q$ via $\alpha$, and denoted by $K\rtimes_{\alpha}Q$. Note that $K\times\{ 1\}$ is a normal subgroup of $K\rtimes_{\alpha}Q$, and $\{1\}\times Q$ is a subgroup of $K\rtimes_{\alpha}Q$. Thus $K\rtimes_{\alpha}Q$ is the inner semidirect product of $K\times\{ 1\}$ by $\{1\}\times Q$. Since $K\simeq K\times\{ 1\}$ and $Q\simeq \{1\}\times Q$, we identify these groups, i. e. $k\equiv (k,1)$ and $q\equiv (1,q)$ for all $k\in K$ and $q\in Q$. Note that
\[ 
qkq^{-1}\equiv (1,q)(k,1)(1,q^{-1})=(\alpha_q(k),q)(1,q^{-1})=(\alpha_q(k),1)=\alpha_q(k)
\]
for all $k\in K$ and $q\in Q$.

\begin{theorem}
	\label{thm:complements}
	Let $Q$ and $K$ be groups and let $\alpha\colon Q\rightarrow \Aut (K)$ be a
	group homomorphism. Then there exists a bijective correspondence
	\[
	\{\text{complements of $K$ in $K\rtimes Q$}\}\leftrightarrow\Der(Q,K).
	\]
	%between
	%the set $\mathcal{C}$ of complements of $K$ in $K\rtimes Q$ and the set
	%$\Der(Q,K)$ of 1-cocycles $Q\to K$.
\end{theorem}

\begin{proof}
    Let $\mathcal{C}$ be the set of complements of $K$ in $K\rtimes Q$. 
	Since $Q$ acts by conjugation on $K$, it follows that $\delta\in\Der(Q,K)$ if and only if 
	$\delta(xy)=\delta(x)x\delta(y)x^{-1}$ for all $x,y\in Q$. In this case, 
	one obtains that 
	$\delta(1)=1$ and $\delta(x^{-1})=x^{-1}\delta(x)^{-1}x$.
	
	Let 
	$C\in\mathcal{C}$. If $x\in Q$, then there exist unique elements  
	$k\in K$ and $c\in C$ such that $x=k^{-1}c$. Hence the  map 
	$\delta_C\colon Q\to K$, $x\mapsto k$, is well-defined and 
	$\delta_C(x)x=c\in C$. 
	
	We claim that $\delta_C\in\Der(Q,K)$. If $x,x_1\in Q$, we write $x=k^{-1}c$
	and $x_1=k_1^{-1}c_1$ for $k,k_1\in K$ and $c,c_1\in C$. Since $K$ is a normal subgroup of 
	the semidirect product $K\rtimes Q$, we can write $xx_1$ as $xx_1=k_2c_2$, where 
	$k_2=k^{-1}(ck_1^{-1}c^{-1})\in K$, $c_2=cc_1\in C$. Thus  
	$\delta_C(xx_1)xx_1=cc_1=\delta_C(x)x\delta_C(x_1)x_1$ 
	implies that $\delta_C(xx_1)=\delta_C(x)x\delta_C(x_1)x^{-1}$. 
	So there is a map $F\colon\mathcal{C}\to\Der(Q,K)$, $F(C)=\delta_C$.

	We now construct a map $G\colon\Der(Q,K)\to\mathcal{C}$. 
	For each 
	$\delta\in\Der(Q,K)$ we find a complement $\Delta$ of $K$ in $K\rtimes Q$. Let 
	$\Delta=\{\delta(x)x:x\in Q\}$. 
	We claim that $\Delta$ is a subgroup of $K\rtimes Q$. Since $\delta(1)=1$,
	$1\in \Delta$. If $x,y\in Q$, then 
	\[
	\delta(x)x\delta(y)y=\delta(x)x\delta(y)x^{-1}xy=\delta(xy)xy\in \Delta.
	\]
	Finally, if $x\in Q$, then 
	\[
	(\delta(x)x)^{-1}=x^{-1}\delta(x)^{-1}xx^{-1}=\delta(x^{-1})x^{-1}\in \Delta.
	\]
	Thus $\Delta$ is a subgroup of $K\rtimes Q$.
	We claim that $\Delta\cap K=\{1\}$. If $x\in Q$ is such that $\delta(x)x\in K$, then 
    since $\delta(x)\in K$, it follows that $x\in K\cap Q=\{1\}$. If $g\in G$, then 
	there are unique $k\in K$ and $x\in Q$ such that $g=kx$. We write 
	$g=k\delta(x)^{-1}\delta(x)x$. Since $k\delta(x)^{-1}\in K$ and $\delta(x)x\in
	\Delta$, we conclude that $G=K\Delta$. Thus there is a well-defined map 
	$G\colon\Der(Q,K)\to\mathcal{C}$, $G(\delta)=\Delta$.

	We claim that $G\circ F=\id_{\mathcal{C}}$. 
	Let $C\in\mathcal{C}$. Then  
	\[
	G(F(C))=G(\delta_C)=\{\delta_C(x)x:x\in
	Q\}=C,
	\]
	by construction. (We know that $\delta_C(x)x\in C$. Conversely, if $c\in
	C$, we write $c=kx$ for unique elements $k\in K$ and $x\in Q$. Thus $x=k^{-1}c$
	and hence $c=\delta_c(x)x$.)

	Finally, we prove that $F\circ G=\id_{\Der(Q,K)}$. Let $\delta\in\Der(Q,K)$.
    Then	
    \[
	F(G(\delta))=F(\Delta)=\delta_{\Delta}.
	\]
	Finally, we need to show that $\delta_\Delta=\delta$.  Let $x\in Q$. There exists 
	$\delta(y)y\in\Delta$ for some $y\in Q$ such that $x=k^{-1}\delta(y)y$.
	Thus $\delta_{\Delta}(x)x=\delta(y)y$ and hence $x=y$ and $\delta_{\Delta}(x)=\delta(y)$ by
	the uniqueness. Therefore, $\delta_{\Delta}=\delta$, and the result follows. 
\end{proof}

\index{Derivation!inner}
\index{$1$-coboundary} 
Let $K$ and $Q$ be groups and let $\alpha\colon Q\rightarrow \Aut(K)$ be a group homomorphism.
A derivation $\delta\in\Der(Q,K)$ is said to be {\em inner} if there exists $k\in K$ 
such that $\delta(x)=[k,x]$ for all $x\in Q$. The set of 
{\em inner derivations} will be denoted by 
\[
		\Inn(Q,K)=B^1(Q,K)=\{\delta\in\Der(Q,K):\text{$\delta$ is inner}\}.
\]
An inner derivation is also called a {\em $1$-coboundary}.

\begin{theorem}[Sysak]
	\index{Sysak's theorem}
	\label{theorem:Sysak}
	Let $K$ and $Q$ be groups and let $\alpha\colon Q\rightarrow \Aut(K)$ be a group homomorphism. Let
	$\delta\in\Der(Q,K)$.
	\begin{enumerate}
		\item $\Delta=\{\delta(x)x:x\in Q\}$ is a complement of $K$ in $K\rtimes Q$.
		\item $\delta\in\Inn(Q,K)$ if and only if $\Delta=kQ k^{-1}$ for some $k\in K$.
		\item $\ker\delta=Q\cap\Delta$.
		\item $\delta$ is surjective if and only if $K\rtimes Q=\Delta Q$.
	\end{enumerate}
\end{theorem}

\begin{proof}
	In the proof of Theorem~\ref{thm:complements} we 
	found that $\Delta$ is a complement of $K$ in $K\rtimes Q$. 

	Let us prove the second statement. If $\delta$ is inner, then there exists 
    $k\in K$ such that $\delta(x)=[k,x]=kxk^{-1}x^{-1}$ for all $x\in
	Q$. Since $\delta(x)x=kxk^{-1}$ for all $x\in Q$,  $\Delta=kQk^{-1}$.
	Conversely, if there exists $k\in K$ such that $\Delta=kQk^{-1}$, for each 
	$x\in Q$ there exists $y\in Q$ such that $\delta(x)x=kyk^{-1}$. Since
	$[k,y]=kyk^{-1}y^{-1}\in K$, $\delta(x)\in K$ and $\delta(x)x=[k,y]y\in KQ$,
	we conclude that  $x=y$ and hence $\delta(x)=[k,x]$. 

	Let us prove the third statement. If $x\in Q$ is such that $\delta(x)x=y\in
	Q$, then \[
	\delta(x)=yx^{-1}\in K\cap Q=\{1\}.
	\]
	Conversely, if $x\in Q$
	is such that $\delta(x)=1$, then $x=\delta(x)x\in Q\cap\Delta$. 

	Finally we prove the fourth statement. If $\delta$ is surjective, then for each 
	$k\in K$ there exits $y\in Q$ such that $\delta(y)=k$. Thus $K\rtimes Q\subseteq
	\Delta Q$, as 
	\[
	kx=\delta(y)x=(\delta(y)y)y^{-1}x\in \Delta Q.
	\]
	Since $\Delta$ and $Q$ are subgroups of $K\rtimes Q$, we have that 
	$\Delta Q\subseteq K\rtimes Q$, and therefore $\Delta Q=K\rtimes Q$.
	Conversely, if $k\in K$ and $x\in Q$ there exist  
	$y,z\in Q$ such that $kx=\delta(y)yz$. Then it follows that 
	$k=\delta(y)$. 
\end{proof}

\begin{exercise}
	\label{xca:ker1cocycle}
	Let $\delta\in\Der(Q,K)$. 
	\begin{enumerate}
	\item Prove that $\delta$ is injective if and only if 
	$\ker\delta=\{1\}$.
	\item Prove that if $\delta$ is bijective, then  
	$K$ admits a complement 
	$\Delta$ in $K\rtimes Q$ such that $K\rtimes Q=K\rtimes\Delta=\Delta Q$ and 
	$Q\cap\Delta=\{1\}$.
	\end{enumerate}
\end{exercise}


A group $G$ admits a {\em triple factorization} if there are subgroups 
$A$, $B$ and $M$ such that $G=MA=MB=AB$ and $A\cap M=B\cap M=\{1\}$.
The following result is an immediate consequence of Sysak's theorem.

\begin{corollary}
	If the group $Q$ acts by automorphisms on $K$ and $\delta\in\Der(Q,K)$ is
	surjective, then $G=K\rtimes Q$ admits a triple factorization. 
\end{corollary}

% \begin{proof}
% 	Sean $x,y\in Q$ tales que $\delta(x)=\delta(y)$. Como $\delta(x^{-1}y)=1$
% 	pues 
% 	\[
% 	\delta(x^{-1}y)=\delta(x^{-1})(x^{-1}\delta(y)x)=\delta(x^{-1})x^{-1}\delta(x)x=\delta(x^{-1}x)=\delta(1)=1
% 	\]
% 	y $\delta$ es inyectiva, $x^{-1}y=1$. La afirmación recíproca es trivial.
% \end{proof}

% \begin{corollary}
% 	Si $\delta\in\Der(Q,K)$ es biyectivo entonces $K$ admite un complemento
% 	$\Delta$ en $K\rtimes Q$ tal que $K\rtimes Q=K\rtimes\Delta=\Delta Q$ y
% 	$Q\cap\Delta=1$.
% \end{corollary}

% \begin{proof}
% 	Vimos en el teorema de Sysak que $\delta$ es sobreyectiva si y
% 	sólo si $K\rtimes Q=\Delta Q$ y que $\ker\delta=Q\cap\Delta$.
% \end{proof}

%\section{Aplicación: subespacios invariantes}
%
%Sea $A$ un grupo que actúa por automorfismos en un grupo $G$. Definimos
%\[
%C_G(A)=\{g\in G:g\cdot a=a\text{ para todo $a\in A$}\}.
%\]
%
%Como aplicación de la teoría de Schur--Zassenhaus vamos a demostrar los
%teoremas de Sylow para subespacios $A$-invariantes.
%Necesitamos el siguiente lema:
%
%\begin{lemma}
%	\label{lemma:Glauberman}
%	Sean $A$ y $G$ grupos finitos de órdenes coprimos. Supongamos que $A$ actúa
%	por automorfismos en $G$ y que $A$ o $G$ es resoluble. Supongamos que $A$
%	actúa en un conjunto $X$ y que $G$ actúa transitivamente en $X$ de forma tal que
%	\begin{equation}
%		\label{equation:Glauberman:compatibilidad}
%		a\cdot (g\cdot x)=(aga^{-1})\cdot (a\cdot x)
%	\end{equation}
%	para todo $a\in A$, $g\in G$, $x\in X$. Valen las siguientes afirmaciones:
%	\begin{enumerate}
%		\item Existe un $x\in X$ invariante por la acción de $A$.
%		\item Si $x,y\in X$ son invariantes por la acción de $A$ entonces
%			existe $c\in C_G(A)$ tal que $c\cdot x=y$.
%	\end{enumerate}
%\end{lemma}
%
%\begin{proof}
%	Sea $\Gamma=G\rtimes A$ el producto semidirecto. Todo $\gamma$ se escribe
%	en forma única como $\gamma=ga$ con $g\in G$, $a\in A$. Veamos que $\Gamma$
%	actúa en $X$ por
%	\[
%		\gamma\cdot x=(ga)\cdot x=g\cdot (a\cdot x).
%	\]
%	Es fácil ver que es una acción pues la igualdad
%	\[
%	(ga)\cdot ((hb)\cdot x)=((ga)(hb))\cdot x=(gaha^{-1})\cdot ((ab)\cdot x)
%	\]
%	es consecuencia de la relación de
%	compatibilidad~\eqref{equation:Glauberman:compatibilidad}.\framebox{completar}
%
%\end{proof}
%\begin{theorem}
%	\label{theorem:Sylow_Ainv}
%\end{theorem}
%
%}

Let $A$ be an additive group (note that we do not assume that an additive group is abelian) 
and $G$ be a group and let 
$G\times A\to A$, $(g,a)\mapsto g\cdot a$,
be a left action of $G$ on $A$ by automorphisms. This means that the action of $G$ on $A$ satisfies 
$g\cdot (a+b)=g\cdot a+g\cdot b$ for all $g\in G$ and $a,b\in A$.
A \emph{bijective
$1$-cocyle} is a bijective map $\pi\colon G\to A$ such that 
\begin{equation}
    \label{eq:1cocycle}
    \pi(gh)=\pi(g)+g\cdot \pi(h)
\end{equation}
for all $g,h\in G$. To simplify the notation we 
just say that the pair $(G,\pi)$ is a bijective 1-cocycle with values
on $A$. 

\begin{theorem}
	\label{thm:1cocycle}
	Let $A$ be an additive group. There exists a bijective
	correspondence
	\[
		\{\text{bijective 1-cocycles with values on $A$}\}
	\leftrightarrow
	\{\text{braces with additive group $A$}\}
	\]
%    Over any additive group $A$ the following data are equivalent:
%    \begin{enumerate}
%        \item A group $G$ and a bijective
%            1-cocycle $\pi\colon G\to A$. 
%        \item A brace structure over $A$. 
%    \end{enumerate}

    \begin{proof}
        Consider on $A$ a second group structure given by 
        \[
		a\circ b=\pi(\pi^{-1}(a)\pi^{-1}(b))=a+\pi^{-1}(a)\cdot b
		\]
		for all
        $a,b\in A$.  Since $G$ acts on $A$ by
        automorphisms, 
        \begin{align*}
            a\circ (b+c)&=\pi(\pi^{-1}(a)\pi^{-1}(b+c))=a+\pi^{-1}(a)\cdot (b+c)\\
            &=a+ \pi^{-1}(a)\cdot b+\pi^{-1}(a)\cdot c
            =a\circ b-a+a\circ c
        \end{align*}
        holds for all $a,b,c\in A$.
        
        Conversely, assume that the additive group $A$ has a brace structure. Let $G$ be the multiplicative group of $A$
        and $\pi=\id$. By
        Exercise~\ref{xca:lambda}, $a\mapsto\lambda_a$ is a group homomorphism from $G$ to $\Aut(A,+)$ and 
        hence $G$ acts on $A$ by automorphisms. Then~\eqref{eq:1cocycle} holds
        and therefore $\pi\colon G\to A$ is a bijective 1-cocycle. 
    \end{proof}
\end{theorem}

The construction of the previous theorem is functorial.

\begin{exercise}
\label{xca:1cocycle}
Let $\pi\colon G\to A$ and $\eta\colon H\to B$ be bijective 1-cocycles.  A
\emph{homorphism} between these bijective 1-cocycles is a pair $(f,g)$ of group
homomorphisms  $f\colon G\to H$, $g\colon A\to B$ such that
\begin{align*}
&\eta f=g\pi,\\
&g(h\cdot a)=f(h)\cdot g(a),&&a\in A,\;h\in G.
\end{align*}
Bijective 1-cocycles and homomorphisms form a category. 
For a given additive group $A$ 
the full subcategory of the category of bijective 1-cocycles with objects
$\pi\colon G\to A$ is equivalent to the full subcategory of the category of
braces with additive group $A$. 
\end{exercise}




\begin{example}
	\label{exa:d8q8}
	Let 
	\[
	D_4=\langle r,s:r^4=s^2=1,srs=r^{-1}\rangle
	\]
	be the dihedral group of eight elements and let
	\[
	Q_8=\{1,-1,i,-i,j,-j,k,-k\}
	\]
	be the quaternion group of eight elements.  Let
	$\pi:Q_8\to D_4$ be given by 
	\begin{align*}
		1\mapsto 1 &, & -1\mapsto r^2 &,  & -k\mapsto r^3s &,&  k\mapsto rs &,\\
		i\mapsto s &, & -i\mapsto r^2s &, &  j\mapsto r^3 &, & -j\mapsto r &.
	\end{align*}
	Since $\pi$ is bijective, 
	a straightforward calculation shows that $D_4$ with 
	\[
	  x+y=xy,\quad 
	  x\circ y=\pi(\pi^{-1}(x)\pi^{-1}(y))
	\]
	is a two-sided brace with additive group isomorphic to $D_4$ and multiplicative group
	isomorphic to $Q_8$. 
\end{example}






% \section{16/05/2024}
% \section{24/05/2024}

\subsection{*The Deaconescu--Walls theorem}

Let $A$ be a group acting on automorphisms on a finite group $G$. Then 
$C_{G}(A)=\{g\in G:a\cdot g=g\,\,\forall a\in A\}$ acts by left multiplication 
on the set of 
$A$-orbits by 
\[
  c(A\cdot g)
  =\{c(a\cdot g):a\in A\}
  =\{(a\cdot c)(a\cdot g):a\in A\}
  =\{a\cdot (cg):a\in A\}
  =A\cdot (cg)
\]
for all $g\in G$ and $c\in C_G(A)$.

%The following theorem first appeared in~\cite{MR2164558}. 
%The proof presented here goes back to Isaacs, see~\cite{MR2922681}. 

\begin{theorem}[Deaconescu--Walls]
	\index{Deaconescu--Walls theorem}
	\label{thm:DeaconescuWalls}
	Let $A$ be a group acting by automorphisms on a finite group $G$. Let
	$C=C_{G}(A)$ and $N=C\cap [A,G]$,
	where $[A,G]$ is the subgroup of $G$ generated by $[a,g]=(a\cdot g)g^{-1}$,
	$a\in A$, $g\in G$.  Then $(C:N)$ divides the number of $A$-orbits of 
	$G$. 
\end{theorem}

\begin{proof}
  The group $C$ acts by left multiplication on the set $\Omega$ of 
  $A$-orbits of $G$. Let $X=A\cdot g\in\Omega$ and $C_X$ be the stabilizer of 
  $C$ in $X$. If $c\in C_X$, then $cX=X$. In particular, if $c\in C_X$, then 
  $cg=a\cdot g$ for some $a\in A$, that is $c=(a\cdot
  g)g^{-1}=[a,g]\in [A,G]$. Thus $C_X\subseteq N$.

  To show that $(C:N)$ divides $|\Omega|$, it is enough to show that 
  $(C:N)$ divides the size of each $C$-orbit. If $X\in\Omega$, then $C\cdot
  X$ has size 
  \[
	(C:C_X)=(C:N)(N:C_X).
  \]
  Hence $(C:N)$ divides the size of the orbit $C\cdot X$.
\end{proof}

\begin{corollary}
	\label{cor:Z(G)subset[G,G]}
  Let $G$ be a non-trivial finite group with $k(G)$ conjugacy classes. 
  If the order of $Z(G)$ is coprime with $k(G)$, then  
  $Z(G)\subseteq[G,G]$.
\end{corollary}

\begin{proof}
	The group $A=G$ acts on $G$ by conjugation. Since $C_G(A)=Z(G)$ and 
	$[A,G]=[G,G]$, Theorem~\ref{thm:DeaconescuWalls} implies that the index 
	$(Z(G):Z(G)\cap [G,G])$ divides $k(G)$. Since $k(G)$ and $|Z(G)|$ are coprime, we conclude that $Z(G)=Z(G)\cap [G,G]\subseteq [G,G]$. 
\end{proof}

\begin{definition}
	\index{Central automorphism}
 	Let $G$ be a group and $f\in\Aut(G)$. We say that $f$ is \textbf{central} if 
	$f(x)x^{-1}\in Z(G)$ for all $x\in G$.
\end{definition}

An automorphism $f$ of a group $G$ 
is central if and only if $f\in C_{\Aut(G)}(\Inn(G))$.

\begin{corollary}
	Let $G$ be a finite group with $k(G)$ conjugacy classes and $c(G)$
	central automorphisms. If $\gcd(|G|,k(G)c(G))=1$, then 
	$[G,G]=Z(G)$.
\end{corollary}

\begin{proof}
	By Corollary~\ref{cor:Z(G)subset[G,G]}, $Z(G)\subseteq [G,G]$. Conversely, let 
	$A=C_{\Aut(G)}(\Inn(G))$. Since $|G|$ and $k(G)c(G)$ are coprime 
	and $(C_G(A):C_G(A)\cap [A,G])$ divides $c(G)$ by 
	Theorem~\ref{thm:DeaconescuWalls}, we obtain that $C_G(A)=C_G(A)\cap [A,G]$. 
	Since 
	\[
		a\cdot [x,y]=[(a\cdot x)x^{-1}x,(a\cdot y)y^{-1}y]=[x,y]
	\]
	for all $a\in A$ and $x,y\in G$, 
    $[G,G]\subseteq C_G(A)$. Moreover, 
    $[A,G]\subseteq Z(G)$. Thus 
	\[
	[G,G]\subseteq C_G(A)=C_G(A)\cap [A,G]\subseteq [A,G]\subseteq Z(G).\qedhere 
	\]
\end{proof}

\begin{exercise}
    Let $p$ be a prime number and $G$ be a group with $p$ conjugacy classes. 
    Prove that either $Z(G)\subseteq[G,G]$ or $|G|=p$. 
\end{exercise}

% \begin{proof}
%   Hacemos actuar a $G$ en $G$ por conjugación.  Como cada elemento de $C=Z(G)$
%   es una clase de conjugación, $|C|\leq p$. Si $|C|=p$ entonces $G=C=Z(G)$
%   tiene orden $p$. Si no, $|C|$ es coprimo con $p$ y luego $C\subseteq
%   N=[G,G]$.
% \end{proof}


\subsection{*The Chermak--Delgado subgroup}

\begin{definition}
\index{Chermak--Delgado!measure}
Let $G$ be a finite group and $H$ a subgroup of $G$. 
The \textbf{Chermak--Delgado measure} of $H$ 
is the number 
$m_G(H)=|H||C_G(H)|$.
\end{definition}

\begin{example}
If $G$ is abelian and $H$ is a subgroup of $G$, then 
$m_G(H)=|H||G|$.
\end{example}

\begin{example}
Let $G=\Sym_3$. The subgroups of $G$ are 
	\[
		H_0=1,\quad
		H_1=\langle (23)\rangle,\quad
		H_2=\langle (12)\rangle,\quad
		H_3=\langle (13)\rangle,\quad
		H_4=\langle (123)\rangle,\quad
		H_5=\Sym_3.
	\]
	A direct calculation shows that 
	\[
		m_G(H_j)=\begin{cases}
			6 & \text{if $j\in\{0,5\}$},\\
			4 & \text{if $j\in\{1,2,3\}$},\\
			9 & \text{if $j=4$}.
		\end{cases}
	\]
\end{example}

\begin{lemma}
\label{lem:CD1}
Let $G$ be a finite group and $H$ be a subgroup of $G$. Then 
\[
m_G(H)\leq m_G(C_G(H)).
\]
If the equality holds, then $H=C_G(C_G(H))$.
\end{lemma}

\begin{proof}
Let $C=C_G(H)$. 
Since $H\subseteq C_G(C)$, 
\[
m_G(C)=|C||C_G(C)|\geq |C||H|=m_G(H). 
\]
If $m_G(H)=m_G(C_G(H))$, then $|H|=|C_G(C_G(H))|$ and 
hence $H=C_G(C_G(H))$, as $H\subseteq C_G(C_G(H))$. 
\end{proof}

\begin{lemma}
	Let $G$ be a finite group and 
	$H$ and $,K$ be subgroups of $G$. Let $D=H\cap K$ and
    $J=\langle H,K\rangle$. Then 
	\[
		m_G(H)m_G(K)\leq m_G(D)m_G(J).
	\]
	If the equality holds, then $J=HK$ and $C_G(D)=C_G(H)C_G(K)$.
	\label{lem:CD2}
\end{lemma}

\begin{proof}
	Let $C_H=C_G(H)$, $C_K=C_G(K)$, $C_D=C_G(D)$, and $C_J=C_G(J)$. Then
	$C_J=C_H\cap C_K$ and $C_H\cup C_K\subseteq C_D$. Since 
	\[
		|J|\geq |HK|=\frac{|H||K|}{|D|},
		\quad
		|C_D|\geq |C_HC_K|=\frac{|C_H||C_K|}{|C_J|},
	\]
	we obtain that 
	\[
		m_G(D)
		=|D||C_D|\geq \frac{|H||K|}{|J|}\frac{|C_H||C_K|}{|C_J|}
		=\frac{m_G(H)m_G(K)}{m_G(J)}.
	\]
	The second claim is clear. 
\end{proof}

\begin{definition}
\index{Lattice of subgroups}
Let $G$ be a finite group and $\mathcal{L}$ be a collection of subgroups of $G$. We say that $\mathcal{L}$ is a \textbf{lattice} if for every $H,K\in\mathcal{L}$ one has that
$H\cap K\in\mathcal{L}$ and $\langle H,K\rangle\in\mathcal{L}$. 
\end{definition}

Since $G$ is finite, it makes sense to consider the set $\mathcal{L}(G)$ of 
subgroups of $G$ $G$ where the Chermak--Delgado gets its largest value,
say $M_G$. 

\begin{exercise}
	\label{xca:M_S}
	Let $G$ be a finite group and $H$ be a subgroup of $G$. Prove that 
	$M_H\leq M_G$.
\end{exercise}

% \begin{svgraybox}
% 	Sabemos que existe algún subgrupo $K$ de $H$ tal que $M_H=m_H(K)$. Como
% 	$C_H(K)\subseteq C_G(K)$, 
% 	\[
% 		M_H=m_H(K)=|H||C_H(K)|\leq |H||C_G(K)|\leq m_G(H)\leq M_G.
% 	\]
% \end{svgraybox}

\begin{example}
	\label{exa:D8_CD}
    Let $G=\D_8=\langle r,s:r^4=s^2=1,srs=r^{-1}\rangle$ be the dihedral group
    of eight elements. In the subgroups 
    \[
		G,
		\quad
		Z(G)=\{1,r^2\},\quad
		A=\{1,r,r^2,r^3\},\quad
		B=\{1, s,sr^2,r^2\},\quad
		C=\{1,sr,sr^3,r^2\},
	\]
	the Chermak--Delgado measure is $16$ and this is the largest possible value. Thus and $M_G=16$ and $\mathcal{L}(G)=\{G,Z(G),A,B,C\}$. 
	\begin{lstlisting}
gap> ChermakDelgado := function(group, subgroup)
> return Size(subgroup)\
> *Size(Centralizer(group, subgroup));
> end;
function( group, subgroup ) ... end
gap> gr := DihedralGroup(IsPermGroup, 8);;
gap> r := gr.1;;
gap> s := gr.2;;
gap> ChermakDelgado(gr, Subgroup(gr, [r]));
16
gap> ChermakDelgado(gr, Subgroup(gr, [s*r,s*r^3]));
16
gap> ChermakDelgado(gr, Subgroup(gr, [s,s*r^2]));
16
gap> ChermakDelgado(gr, Subgroup(gr, [r^2]));
16
gap> List(AllSubgroups(gr), x->ChermakDelgado(gr, x));
[ 8, 16, 8, 8, 8, 8, 16, 16, 16, 16 ]
	\end{lstlisting}
\end{example}

\begin{theorem}
	Let $G$ be a finite group. The following statements hold: 
	\begin{enumerate}
		\item $\mathcal{L}(G)$ is a lattice. 
		\item If $H,K\in\mathcal{L}(G)$, then $\langle H,K\rangle=HK$.
		\item If $H\in\mathcal{L}(G)$, then $C_G(H)\in\mathcal{L}(G)$ and $C_G(C_G(H))=H$.
	\end{enumerate}
	\label{thm:lattice}
\end{theorem}

\begin{proof}
	If $H,K\in\mathcal{L}(G)$, then $m_G(H)=m_G(K)=M_G$. Let $D=H\cap K$ and $J=\langle
	H,K\rangle$. By Lemma~\ref{lem:CD2}, 
	\[
		M_G^2=m_G(H)m_G(K)\leq m_G(D)m_G(J).
	\]
	Since $m_G(D)\leq M_G$ and $m_G(J)\leq M_G$ (because $M_G$ is the largest possible value), we conclude that $m_G(D)=m_G(J)=M_G$. Hence $\mathcal{L}(G)$ is a lattice. 

	Since $m_G(H)m_G(K)=m_G(D)m_G(J)=M_G^2$, Lemma~\ref{lem:CD2} implies that 
	$J=HK$. 

	By Lemma~\ref{lem:CD1}, 
	\[
	M_G=m_G(H)\leq m_G(C_G(H)).
	\]
	Since $M_G$ is the largest possible value, $m_G(C_G(H))=M_G$. Thus 
    $C_G(H)\in\mathcal{L}(G)$.  By Lemma~\ref{lem:CD1}, $C_G(C_G(H))=H$.
\end{proof}

\index{Chermak--Delgado!subgroup}
Theorem~\ref{thm:lattice} implies the existence 
of the \textbf{Chermak--Delgado subgroup}.

\begin{corollary}
	\label{cor:ChermakDelgado}
	Let $G$ be a finite group. There exists a unique subgroup $M$ minimal 
    suc that $m_G(M)$ is the largest possible value among all the subgroups 
    of $G$. Moreover, $M$ is characteristic, abelian and $Z(G)\subseteq M$. 
\end{corollary}

% f(C_G(H))=C_G(f(H))$ para todo $H$ y todo $f\in\Aut(G)$.

\begin{proof}
	By Theorem~\ref{thm:lattice}, $\mathcal{L}(G)$ is a lattice. Let 
	\[
		M=\bigcap_{H\in\mathcal{L}(G)}H\in\mathcal{L}(G).
	\]
	By Theorem~\ref{thm:lattice},  
	\[
    C_G(M)\in\mathcal{L}(G)
    \text{ and }M=C_G(C_G(M))\supseteq Z(G).
    \]Since $C_G(M)\in\mathcal{L}(G)$, $M\subseteq C_G(M)$. Hence $M$ is abelian. Moreover, $M$ is characteristic in $G$ because $f(M)\in\mathcal{L}(G)$
	for all $f\in\Aut(G)$.
\end{proof}

\begin{example}
	Let $G=\D_8$ be the dihedral group of eight elements. The Chermak--Delgado subgroup of $G$ is $Z(G)\simeq C_2$. See Example~\ref{exa:D8_CD}.
\end{example}

\begin{theorem}[Chermak--Delgado]
	\index{Chermak--Delgado!theorem}
 	\label{thm:ChermakDelgado}
	Let $G$ be a finite group. Then $G$ has an abelian characteristic subgroup $M$ such that $(G:M)\leq (G:A)^2$ for every abelian subgroup 
	$A$ of $G$. 
\end{theorem}

\begin{proof}
	Let $M$ be the Chermak--Delgado subgroup of Corollary~\ref{cor:ChermakDelgado} 
    and $A$ be an abelian subgroup of 
	$G$. Then $A\subseteq C_G(A)$. Hence 
	\[
		m_G(M)\geq m_G(A)=|A||C_G(A)|\geq|A|^2
	\]
	and 
	\[
	(G:A)^2
	=\frac{|G|^2}{|A|^2}\geq\frac{|G|^2}{m_G(M)}
	=\frac{|G|}{|M|}\frac{|G|}{|C_G(M)|}
	=\frac{|G|}{|M|}
	=(G:M).\qedhere 
	\]
\end{proof}

\begin{corollary}
	Let $G$ be a non-abelian finite group and $H$ be a subgroup of $G$ such that 
	\[
	|H||C_G(H)|>|G|.
	\]
	Then $G$ is not simple. 
\end{corollary}

\begin{proof}
	Let $M$ be the Chermak--Delgado subgroup of $G$. 
	Then 
	\begin{equation}
		\label{equation:mG}
	m_G(M)\geq m_G(H)>|G|.
	\end{equation}
	This implies that $M\ne\{1\}$, since $m_G(M)=m_G(1)=|G|$. If $G$ is simple, then $G=M$ is abelian. 
\end{proof}

\begin{corollary}
	Let $G$ be a non-abelian finite group and $P\in\Syl_p(G)$. If $P$ is abelian and $|P|^2>|G|$, then $G$ is not simple. 
\end{corollary}

\begin{proof}
	Let $M$ be the Chermak--Delgado subgroup of $G$. Since $P$ is abelian, 
    \[
    (G:M)\leq (G:P)^2<|G|
    \]
    by Theorem~\ref{thm:ChermakDelgado}. Hence 
    $M\ne\{1\}$. If $G$ is simple, then $G=M$ is abelian. 
\end{proof}

We now discuss an application of the Wielandt zipper theorem 
to the Chermak--Delgado lattice. 

\begin{lemma}
	\label{lem:L(G)L(S)}
	Let $G$ be a finite group, $H\in\mathcal{L}(G)$ and  $S$ be a subgroup of $G$ such that 
	$HC_G(H)\subseteq S$. Then $H\in\mathcal{L}(S)$.
\end{lemma}

\begin{proof}
	Since $C_G(H)\subseteq S$, $C_G(H)=C_S(H)$. By Exercise~\ref{xca:M_S},
    $M_S\leq M_G$. Thus $M_G=M_S$, since 
	\[
		M_G=m_G(H)=|H||C_G(H)|=|H||C_S(H)|=m_S(H)\leq M_S.\qedhere 
	\]
\end{proof}

\begin{theorem}
	\label{thm:L(G)subnormal}
	Let $G$ be a finite group. Every $H\in\mathcal{L}(G)$ is subnormal in $G$.
\end{theorem}

\begin{proof}
	We proceed by induction on $|G|$. If $|G|=1$, the result is trivial. So assume the group $G$ is non-trivial. Let $H\in\mathcal{L}(G)$ and $K=HC_G(H)$. Since $H$ is normal in $K$, by the inductive hypotehsis, 
    it is enough to show that 
	$K$ is subnormal in $G$. If $K=G$, the claim holds. So assume that 
	$K\ne G$. 

	Assume that $K$ is not subnormal in $G$. By the inductive hypothesis and 
    Wielandt's zipper theorem (Theorem~\ref{thm:zipper}), there exists a unique 
    maximal subgroup $M$ containing $K$. By Theorem~\ref{thm:lattice},
	$C_G(H)\in\mathcal{L}(G)$ and $K=HC_G(H)\in\mathcal{L}(G)$. By Lemma~\ref{lem:L(G)L(S)},
	$H\in\mathcal{L}(M)$. Hence $K\in\mathcal{L}(M)$. By the inductive hypothesis, $K$ is subnormal in $M$. We claim that $M$ is normal in $G$. Let $x\in G$. Since 
	$m_G(xKx^{-1})=m_G(K)$, the subgroup $xKx^{-1}\in\mathcal{L}(G)$. Hence 
	$K(xKx^{-1})\in\mathcal{L}(G)$. 
	
	If $K(xKx^{-1})=G$, then, since there exist $k_1,k_2\in K$ such that 
	$k_1(xk_2x^{-1})=x^{-1}$, we obtain that $x\in K$, since $x^{-1}=k_2k_1\in K$. This implies that $xKx^{-1}\subseteq K$. Therefore $K=G$, a contradiction.

	Since $K(xKx^{-1})\ne G$, there exists a maximal subgroup $N$ such that 
	$K(xKx^{-1})\subseteq N$. Since $K\subseteq N$, $N=M$ because $M$ is the unique
	maximal subgroup containing $K$. Since $xKx^{-1}\subseteq M$, $K\subseteq
	x^{-1}Mx$. Hence $x^{-1}Mx=M$, because $x^{-1}Mx$ is a maximal subgroup containing $K$ and $M$ is the only maximal subgroup containing $K$. 
\end{proof}

\begin{corollary}
	Let$G$ be a non-abelian finite Then $\mathcal{L}(G)=\{1,G\}$. 
\end{corollary}

\begin{proof}
	Let $K\in\mathcal{L}(G)$. Then $K$ is subnormal in $G$ by Theorem~\ref{thm:L(G)subnormal}. Hence $K\in\{1,G\}$. Now the claim follows from $m_G(1)=m_G(G)$. 
\end{proof}

\begin{exercise}
	Let $n\geq5$. Prove that $\mathcal{L}(\Sym_n)=\{1,\Sym_n\}$. 
\end{exercise}

% \begin{proof}
% 	Let $G=\Sym_n$ y sea $K\in\mathcal{L}(G)$. Por el
% 	teorema~\ref{thm:L(G)subnormal}, $K$ es subnormal en $G$. Si $K\ne G$
% 	entonces se tiene una sucesión estrictamente creciente de subgrupos 
% 	\[
% 	K=K_1\triangleleft
% 	K_2\triangleleft\cdots\triangleleft K_{n-1}\triangleleft K_n=G.
% 	\]
% 	Como $K_{n-1}$ es normal en $G$, $K_{n-1}\in\{1,\Alt_n\}$ y luego $K=1$. 
% 	El corolario queda demostrado al observar que $m_G(1)=m_G(G)$. 
% \end{proof}



\subsection{Miller's double cosets theorem}

\index{Double coset}
Let $G$ be a group and $H$ and $K$ be subgroups of $G$. 
The group $L=H\times K$ acts on $G$ by
\[
(h,k)\cdot g=hgk^{-1},\quad h\in H,k\in K,g\in G.
\]
The orbits of this action are the set of the form 
\[
HgK=\{hgk:h\in H,\,k\in K\}.
\]
A set of the form $HgK$ for some $g\in G$ is called a \textbf{double coset} modulo $(H,K)$ 
with representative $g$. In particular, 
any two double cosets are either disjoint or equal, and $G$ decomposes
as a disjoint union 
\[
G=\bigcup_{i\in I}Hg_iK,
\]
for some set $I$. Let 
\[
L_g=\{(h,k)\in H\times K:hgk^{-1}=g\}=\{(h,g^{-1}hg)\in H\times K\}.
\]
Then
$|L_g|=|H\cap gKg^{-1}|$, 
because there is a bijection $L_g\to H\cap gKg^{-1}$.  
By the fundamental counting principle, 
\[
|HgK|=(L:L_g)=\frac{|H\times K|}{|H\cap gKg^{-1}|}=\frac{|H||K|}{|H\cap gKg^{-1}|}.
\]

We need a lemma. 

\begin{lemma}
\label{lem:Miller}
    Let $G$ be a finite group, $x\in G$, and $H$ and $K$ be subgroups of $G$. Then
    \[
    \#\{zK:zK\subseteq HxK\}=(H:xKx^{-1}\cap H).
    \]
\end{lemma}

\begin{proof}
    Let $L=xKx^{-1}\cap H$ and 
    \[
    \varphi\colon H/L\to\{yK:yK\subseteq H\times K\},\quad 
    hL\mapsto hxK.
    \]

    The map $\varphi$ is well-defined. If $hL=h_1L$, then $h^{-1}h_1\in L$. Thus 
    $h^{-1}h_1=xkx^{-1}$ for some $k\in K$. This means that
    \[
    (h_1x)^{-1}(hx)=x^{-1}h_1^{-1}hx=k\in K,
    \]
    that is $\varphi(hL)=hxK=h_1xK=\varphi(h_1L)$. 

    The map $\varphi$ is surjective: If $zK$ is such that $zK\subseteq HxK$, then 
    $z=hxk$ for some $k\in K$. In particular, 
    $zK=hxK$. Now $\varphi(hL)=hxK=zK$.

    The map $\varphi$ is injective: If $hxK=h_1xK$, then 
    $x^{-1}h_1^{-1}hx\in K$. Moreover, 
    $h_1^{-1}h\in xKx^{-1}\cap H=L$. Thus $h_1L=hL$. 
\end{proof}

\begin{exercise}
\label{xca:Miller}
    Let $G$ be a finite group, $H$ and $K$ be subgroups of $G$, and $x\in G$. Prove 
    that 
    \[
    \#\{Hy:Hy\subseteq HxK\}=(K:xHx^{-1}\cap K).
    \]
\end{exercise}

\begin{theorem}[Miller]
\index{Miller' theorem}
    Let $G$ be a finite group and $H$ and $K$ be subgroups of $G$ 
    of the same index. Then there exists a common complete set
    of representatives for the right cosets of $H$ in $G$ and the 
    left cosets of $K$ in $G$. 
\end{theorem}

\begin{proof}
    Let $Hy$ be a right coset and $zK$ be a left coset. Note that 
    $Hy$ and $zK$ have a common representative
    if and only if $Hy\cap zK\ne\emptyset$, as 
    \[
    Hy=Hx\text{ and }zK=xK
    \Longleftrightarrow 
    xy^{-1}\in H\text{ and }z^{-1}x\in K
    \Longleftrightarrow x\in Hy\cap zK.
    \]

    The group $G$ decomposes as a  
    disjoint union of finitely many double cosets. Each doble coset
    $HxK$ is a disjoint union of finitely many right cosets of $H$ 
    and a disjoint union of finitely many left cosets of $K$. Thus 
    \[
    HxK=\bigcup_{i=1}^kHy_i=\bigcup_{j=1}^lz_jK, 
    \]
    where the unions are disjoint. 
    Since $|H|=|K|$, by applying cardinality, it follows that $k=l$. To prove the theorem
    it is enough to show that each $Hy_i$ intersects every $z_jK$. 
    
    Note that for each $i\in\{1,\dots,k\}$ there exists $j\in\{1,\dots,k\}$ such that
    $Hy_i\cap z_jK\ne\emptyset$. 
    Without loss of generality, we may assume (reordering if needed) that 
    $Hy_1\cap z_jK\ne\emptyset$ for all $j\in\{1,\dots,m\}$, where $1\leq m\leq k$. Then
    \[
    Hy_1\subseteq\bigcup_{j=1}^mz_jK. 
    \]
    Then
    \[
    Hy_1K\subseteq\bigcup_{j=1}^mz_jK\subseteq \bigcup_{j=1}^kz_jK=HxK.
    \]
    Since $Hy_1K$ and $HxK$ are double cosets with non-empty intersection, 
    they are equal. Thus 
    \[
    |HxK|=|Hy_1K|\leq \sum_{j=1}^m|z_jK|=m|K|.
    \]
    By Lemma~\ref{lem:Miller}, 
    \[
    k=\#\{z_jK:z_jK\subseteq HxK\}=(H:xKx^{-1}\cap H). 
    \]
    Therefore
    \[
    k|K|=\frac{|H||K|}{|H\cap xKx^{-1}|}=|HxK|\leq m|K|
    \]
    and hence $k=m$. 
\end{proof}

\begin{exercise}[Hall]
\label{xca:Hall:cosets}
    Let $G$ be a finite group and $H$ be a subgroup of $G$ with $(G:H)=n$. 
    Then there exists $x_1,\dots,x_n\in G$ such that 
    $\{Hx_1,Hx_2,\dots,Hx_n\}=\{x_1H,x_2H,\dots,x_nH\}$. 
\end{exercise}



\backmatter

%\addcontentsline{lec}{chapter}{Some hints}
%\include{exercises}
%\include{hints}

%\addcontentsline{lec}{chapter}{Some solutions}
%\chapter*{Some solutions}

\pagestyle{plain}
\fancyhf{}
\fancyhead[LE,RO]{Rings and modules}
\fancyhead[RE,LO]{Some solutions}
\fancyfoot[CE,CO]{\leftmark}
\fancyfoot[LE,RO]{\thepage}

\addcontentsline{toc}{chapter}{Some solutions}

\begin{sol}{xca:A1Bm}
    Let $A=\{a\}$ and $B=\{b_1,\dots,b_m\}$. 
    If $AB$ admits a non-unique product, say $x=ab_i=ab_j$ with $i\ne j$, 
    then $b_i=b_j$, a contradiction.
\end{sol}

\begin{sol}{xca:gABh}
    If $AB$ admits a non-unique product $x=ab=a_1b_1$, then so does $(gA)(Bh)$, as
    $(ga)(bh)=(ga_1)(b_1h)$ is a non-unique product of $(gA)(Bh)$. 
    The converse is trivial. 
\end{sol}

\begin{sol}{xca:A2Bm}
    Assume that $AB$ contains no unique products. 
    By Exercise \ref{xca:gABh} we may assume that $A=\{1,a\}$ and $B=\{1,b_2,\dots,b_m\}$. 
    We claim that $a^k\in B$ for all $k\geq0$. We proceed by induction on $k$. The case
    $k=0$ is easy, as $a^0=1\in B$. Now if $a^k\in B$, then $a^{k+1}=aa^{k}\in AB$. Since $G$ has no torsion and 
    $AB$ contains no unique products, 
    $a^{k+1}=b_j$ for some $j$. It follows that $\{a^k:k\geq0\}\subseteq B$, a contradiction. 
\end{sol}

\begin{sol}{xca:0=1}
The first claim follows from the compatibility condition~\eqref{eq:compatibility} with
$c=1$.  To prove the second claim let $d=b+c$.
Then~\eqref{eq:compatibility} becomes 
\[
	a\circ d =a\circ b-a+a\circ (-b+d)
\]
and the claim follows. The third claim is
proved similarly.
\end{sol}

\begin{sol}{xca:lambda}
The inverse of $\lambda_a$ is given by $\lambda^{-1}_a\colon A\to A$, $b\mapsto a'\circ (a+b)$. To prove
that $\lambda_a\in\Aut(A,+)$ we note that
\[
\lambda_a(b+c)=-a+a\circ(b+c)=-a+a\circ b-a+a\circ c=\lambda_a(b)+\lambda_a(c).
\]
Note that $\lambda_a(b)=-a+a\circ b=a\circ (a'+b)$, for all $a,b\in A$. Hence 
\begin{align*}
\lambda_a(\lambda_b(c))&=a\circ (a'+b\circ (b'+c))=-a+a\circ b\circ (b'+c)\\
&=-a+a-a\circ b+a\circ b\circ c=-a\circ b+a\circ b\circ c=\lambda_{a\circ b}(c).\qedhere    
\end{align*}
\end{sol}

\begin{sol}{xca:mu}
    Note that
    \[\mu_a(b)=\lambda_b(a)'\circ b\circ a=(b\circ (b'+a))'\circ b\circ a=(b'+a)'\circ a,\]
    for all $a,b\in A$. Hence $\mu_a$ is bijective and
    \[\mu_a^{-1}(b)=((b\circ a')'-a)'=(a\circ b'-a)'=(b'+a')'\circ a',\]
    for all $a,b\in A$. Now we have
    \begin{align*}
        \mu_b(\mu_a(c))&=\mu_b((c'+a)'\circ a)=(a'\circ (c'+a)+b)'\circ b\\
        &=(a'\circ c'-a'+b)'\circ b=(a'\circ (c'+a\circ b))'\circ b\\
        &=(c'+a\circ b)'\circ a\circ b=\mu_{a\circ b}(c),
    \end{align*}
    for all $a,b,c\in A$. Therefore the result follows.
\end{sol}

\begin{sol}{xca:socle}
    Let $b\in A$ and $a\in\Soc(A)$. Since   
    \[
    b'\circ (b\circ a+b)=a-b'
    \text{ and }
    b'\circ (b+b\circ a)=-b'+a,
    \]
    the first claim follows since
    $a\in Z(A,+)$.
    Now we prove the second claim:
    \[
    b\circ a\circ b'=b\circ (a\circ b')=b\circ (a+b')=b\circ a-b=-b+b\circ
    a=\lambda_b(a).\qedhere
    \]
\end{sol}

\begin{sol}{xca:bijective}
	We first prove that restriction restriction $\pi_1|_G$ of $\pi_1$ onto $G$ is injective. Let $(a,f)\in G$ and $(b,g)\in G$
	be such that 
	$\pi_1(a,f)=\pi_1(b,g)$. Then $a=b$. Since $G$ is a
	subgroup, 
	\[
		(-f^{-1}(a),f^{-1})=(a,f)^{-1}\in G,
	    \quad
		(-g^{-1}(a),g^{-1})=(a,g)^{-1}\in G,
	\]
	and hence $f=g$ since
	\[
	(-f^{-1}(a),f^{-1})\cdot a=0=(-g^{-1}(a),g^{-1})\cdot a 
	\]
	and $G$ is a regular subgroup.
	Now we prove that $\pi_1|_G$ is surjective. Let $a\in A$. 
	Since $G$ is regular, there exists $(x,f)\in G$ such that $x+f(a)=(x,f)\cdot a=0$, so $(-f(a),f)\in G$ for some $f\in\Aut(A)$. 
	Then $(a,f^{-1})=(-f(a),f)^{-1}\in G$ and $\pi_2|_G(a,f^{-1})=a$. 
\end{sol}




%\addcontentsline{lec}{chapter}{References}
\bibliographystyle{abbrv}
\bibliography{refs}


\printindex     
%\phantom{Trick}
%\addcontentsline{lec}{chapter}{\indexname}

\end{document}





