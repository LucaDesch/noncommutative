\chapter{}


\topic{Connel's theorem}

When $K[G]$ is prime? Connel's theorem gives a full answer to this natural
question in the case where $K$ is of characteristic zero. 

%\begin{lemma}
%	\label{lemma:Dfg}
%	Sea $H$ un subgrupo finitamente generado de $\Delta(G)$.
%	\begin{enumerate}
%		\item $(G:C_G(H))$ es finito.
%		\item $(H:Z(H))$ es finito.
%		\item $[H,H]$ es finito.
%		\item Si $H_0$ es el conjunto de elementos de torsión de $H$, $H_0$ es
%			un subgrupo normal finito de $H$ y $H/H_0$ es finitamente generado,
%			abeliano y libre de torsión.
%	\end{enumerate}
%\end{lemma}
%
%\begin{proof}
%	Veamos la primera afirmación: Si $H=\langle
%	h_1,\dots,h_n\rangle\subseteq\Delta(G)$, entonces $(G:C_G(h_i))$ es finito
%	para todo $i\in\{1,\dots,n\}$. Como $C_G(H)=\cap_{i=1}^nC_G(h_i)$, se
%	concluye que $(G:C_G(H))$ es finito.
%
%	Para demostrar la segunda afirmación basta observar que $Z(H)=H\cap C_G(H)$
%	y luego $(H:Z(H))\leq(G:C_G(H)<\infty$. % necesito dos lemas
%
%	La tercera afirmación es consecuencia de la segunda gracias a un teorema de
%	Schur.
%
%	Por último, demostremos la cuarta afirmación.  El grupo $H/[H,H]$ es
%	abeliano y finitamente generado y luego, sus elementos de torsión forman un
%	grupo finito. Como $[H,H]$ es finito, $[H,H]$ es un subgrupo normal de
%	$H_0$. Vamos a demostrar que la torsión de $H/[H,H]$ es igual a
%	$H_0/[H,H]$. La inclusión $\supseteq$ es trivial. Veamos entonces que vale
%	$\subseteq$: so $(x[H,H])^k=1$, entonces $x^k\in[H,H]$. Luego $(x^k)^m=1$ y
%	luego $x\in H_0$. Tenemos entonces que 
%	\[
%		H/[H,H]\simeq\Z^r\times\operatorname{tor}(H/[H,H])\simeq\Z^r\times H_0/[H,H]
%	\]
%	y luego $H/H_0$ es finitamente generado, abeliano y libre de torsión.
%
%\end{proof}
%
%\begin{lemma}
%	\label{lemma:K[abelian]}
%	Si $G$ un grupo abeliano finitamente generado y sin torsión, entonces
%	$K[G]$ es un dominio. 
%\end{lemma}
%
%\begin{proof}
%	Por el teorema
%	de estructura de grupos abelianos finitamente generados podemos escribir
%	$G=\langle x_1\rangle\times\cdots\langle x_n\rangle$, donde
%	$\langle x_j\rangle\simeq\Z$ para todo $j\in\{1,\dots,n\}$. Todo elemento
%	de $G$ se escribe unívocamente como $x_1^{m_1}\cdots x_n^{m_n}$ y
%	luego la función 
%	\[
%		\iota\colon K[X_1,\dots,X_n]\to K[G],\quad
%		X_j\mapsto x_j,
%	\]
%	es un
%	morfismo de anillos inyectivo. Si $\alpha\in K[G]$, entonces existe
%	$m\in\N$ suficientemente grande tal que $\iota((X_1\cdots X_n)^m)\alpha\in
%	\iota(K[X_1,\dots,X_n])\simeq K[X_1,\dots,X_n]$. Luego $K[G]\subseteq
%	K(X_1,\dots,X_n)$ y $K[G]$ es un dominio.
%\end{proof}

%\begin{lemma}
%	Si $G$ es un grupo, entonces
%	$\Delta(G)/\Delta^+(G)$ es abeliano y libre de torsión.
%%	Valen las siguientes afirmaciones:
%%	\begin{enumerate}
%%		%\item $\Delta^+(G)$ está generado por los subgrupos normales finitos de $G$.
%%		\item 
%%		\item Si $\Delta^+(G)=1$, entonces $K[\Delta(G)]$ es un dominio.
%%	\end{enumerate}
%\end{lemma}
%
%\begin{proof}
%%	Demostremos la primera afirmación. 
%	Sean $y_1,\dots,y_n\in\Delta(G)$ y sea $L=\langle y_1,\dots,y_n\rangle$.
%	Como $[L,L]$ es finito por el lema~\ref{lemma:Dfg}, $[L,L]\subseteq\Delta^+(G)$. Luego
%	$\Delta(G)/\Delta^+(G)$ es abeliano y libre de torsión.
%%
%%	Para demostrar la segunda afirmación basta observar que si $\Delta^+(G)=1$
%%	entonces, por el primer ítem, $\Delta(G)$ es abeliano, finitamente generado
%%	y libre de torsión. Luego $K[\Delta(G)]$ es un dominio por el
%%	lema~\ref{lemma:K[abelian]}. 
%\end{proof}

If $S$ is a finite subset of a group $G$, then we define 
$\widehat{S}=\sum_{x\in S}x$. 

\begin{lemma}
	\label{lemma:sumN}
	Let $N$ be a finite normal subgroup of $G$. Then $\widehat{N}=\sum_{x\in N}x$ is central
	in $K[G]$ and $\widehat{N}(\widehat{N}-|N|1)=0$.
\end{lemma}

\begin{proof}
	Assume that $N=\{n_1,\dots,n_k\}$. Let 
	$g\in G$. Since $N\to N$, $n\mapsto gng^{-1}$, is bijective, 
	\[
		g\widehat{N}g^{-1}=g(n_1+\cdots+n_k)g^{-1}=gn_1g^{-1}+\cdots+gn_kg^{-1}=\widehat{N}.
	\]
	Since $nN=N$ if $n\in N$, it follows that $n\widehat{N}=\widehat{N}$. Thus 
	$\widehat{N}\widehat{N}=\sum_{j=1}^k n_j\widehat{N}=|N|\widehat{N}$.
\end{proof}

If $G$ is a group, let 
\begin{align*}
	&\Delta^+(G)=\{x\in \Delta(G):\text{$x$ has finite order}\}.
\end{align*}
An application of Dietzmann's theorem:

\begin{proposition}
	\label{lem:DcharG}
	If $G$ is a group, then $\Delta^+(G)$ is a characteristic subgroup of $G$.
\end{proposition}

\begin{proof}
	Clearly, $1\in\Delta^+(G)$. 
	Let $x,y\in\Delta^+(G)$ and $H$ be the subgroup of $G$ generated by the set 
	$C$ formed by all finite conjugates of $x$ and $y$. If $|x|=n$ and 
	$|y|=m$, then $c^{nm}=1$ for all $c\in C$. 
	Since $C$ is finite and closed under conjugation, Dietzmann's theorem 
	implies that $H$ is finite and hence 
	$H\subseteq\Delta^+(G)$. In particular, $xy^{-1}\in\Delta^+(G)$. It is now clear
	that $\Delta^+(G)$ is a characteristic subgroup, as for 
	every $f\in\Aut(G)$ and $x\in\Delta^+(G)$ it follows that $f(x)\in\Delta^+(G)$. 
\end{proof}

To prove Connel's theorem we need a lemma. 

\begin{lemma}
	\label{lem:Connel}
	Let $G$ be a group and  $x\in\Delta^+(G)$. There exists a finite normal subgroup
	$H$ of $G$ such that $x\in H$.
\end{lemma}

\begin{proof}
	Let $H$ be the subgroup generated by the conjugates of $x$. Since $x$ has finitely many conjugates, 
	$H$ is finitely generated. Moreover, $H$ is normal in $G$ and it is generated by torsion elements. 
	All these generators of $H$ have the same order, say $n$. By Dietzmann's theorem, 
	$H$ is finite. 
\end{proof}

\index{Ring!prime}
Recall that a ring $R$ is said to be \textbf{prime} 
if for $x,y\in R$ such that $xRy=\{0\}$ it follows that $x=0$ or $y=0$. Prime rings
are non-commutative analogs of domains. 

\begin{theorem}[Connell]
	\label{thm:Connel}
	\index{Connel's theorem}
	Let $K$ be a field of characteristic zero. Let 
	$G$ be a group. The following statements are equivalent: 
	\begin{enumerate}
		\item $K[G]$ is prime.
		\item $Z(K[G])$ is prime.
		\item $G$ does not contain non-trivial finite normal subgroups. 
		\item $\Delta^+(G)=\{1\}$.
	\end{enumerate}
\end{theorem}

\begin{proof}
	We first prove that $1)\implies2)$. Since $Z(K[G])$ is commutative, we need to prove that 
	there are no non-trivial zero divisors. Let $\alpha,\beta\in Z(K[G])$ be such that 
	$\alpha\beta=0$. Let $A=\alpha K[G]$ and $B=\beta K[G]$. Since both $\alpha$ and 
	$\beta$ are central, both $A$ and $B$ are ideals of $K[G]$. Since $AB=\{0\}$,
	it follows that either $A=\{0\}$ or $B=\{0\}$, as $K[G]$ is prime by assumption. 
	Thus either $\alpha=0$ or 
	$\beta=0$.

	We now prove that $2)\implies3)$. Let $N$ be a normal finite subgroup of $G$. 
	By Lemma~\ref{lemma:sumN}, $\widehat{N}=\sum_{x\in N}x$ is central in 
	$K[G]$ and $\widehat{N}(\widehat{N}-|N|1)=0$. Since $\widehat{N}\ne 0$ (recall that 
	$K$ has characteristic zero) and $Z(K[G])$ is a domain,
	$\widehat{N}=|N|1$, that is $N=\{1\}$.

	Let us prove that $3)\implies4)$. Let $x\in\Delta^+(G)$. By Lemma~\ref{lem:Connel},
	there exists a finite normal subgroup $H$ of $G$ that contains $x$. By assumption, $H$ is trivial. 
	Hence $x=1$.

	Finally, let us prove that $4)\implies1)$. Let $A$ and $B$ be ideals of 
	$K[G]$ such that $AB=\{0\}$. Assume that $B\ne\{ 0\}$ and let $\beta\in
	B\setminus\{0\}$.  If $\alpha\in A$, then, since 
	\[
	\alpha K[G]\beta\subseteq \alpha B\subseteq AB=\{0\},
	\]
	Passman's lemma ~\ref{lem:Passman}
	implies that $\pi_{\Delta(G)}(\alpha)\pi_{\Delta(G)}(\beta)=\{0\}$.
	By assumption, $\Delta^+(G)$ is trivial. Thus $\Delta(G)$ is torsion-free 
	and hence $\Delta(G)$ is abelian by Proposition~\ref{pro:FCabeliano}. It follows that
	$K[\Delta(G)]$ has no zero divisors and therefore $\alpha=0$. 
\end{proof}

\index{Hopkins--Levitzky's theorem}
We now need to recall Hopkins--Levitzky's theorem. The theorem
states that unitary left artinian rings are left noetherian.

\begin{theorem}[Connel]
\index{Connel's theorem}
	Let $K$ be a field of characteristic zero. If $G$ is a group, then 
	$K[G]$ is left artinian if and only if $G$ is finite.
\end{theorem}

\begin{proof}
	If $G$ is finite, $K[G]$ is left artinian, as it is a finite-dimensional algebra.
	
	Let us assume that $K[G]$ is left artinian. 
	If $K[G]$ is prime, Wedderburn's theorem implies that 
	$K[G]$ is simple and hence 
	$G$ is trivial (otherwise, $K[G]$ is not simple as the augmentation ideal 
	is a non-zero ideal of $K[G]$).

	Since $K[G]$ is left artinian, it is left noetherian by Hopkins--Levitzky's theorem. Thus $K[G]$
	admits a composition series. We proceed by induction on the length of this composition series of 
	$K[G]$. If the length is one, $\{0\}$ is the only ideal of $K[G]$ and hence the result follows as 
	$K[G]$ is prime. If we assume the result holds for length $n$ and $K[G]$ is not prime, then, 
	Connel's theorem implies that $G$ contains a finite non-trivial normal subgroup $H$. The canonical map
	$K[G]\to K[G/H]$ implies that $K[G/H]$ is left artinian and has length 
	$<n$. By using the inductive hypothesis, $G/H$
	is a finite group. Since $H$ is also finite, it follows that $G$ is finite. 
\end{proof}


\topic{The Yang--Baxter equation}

We now briefly discuss set-theoretic solutions to the Yang--Baxter equation. 

\begin{definition}
\index{Solution}
\index{Solution!finite}
A \emph{set-theoretic solution} to the Yang--Baxter equation (YBE) is a pair $(X,r)$, 
where $X$ is a non-empty set and $r\colon X\times X\to X\times X$ is a bijective map that satisfies 
\[
(r\times\id)(\id\times r)(r\times\id)=
(\id\times r)(r\times\id)(\id\times r),
\]
where, if $r(x,y)=(\sigma_x(y),\tau_y(x))$, then 
\begin{align*}
& r\times\id\colon X\times X\times X\to X\times X\times X, &&(r\times\id)(x,y,z)=(\sigma_x(y),\tau_y(x),z),\\
& \id\times r\colon X\times X\times X\to X\times X\times X, &&(\id\times r)(x,y,z)=(x,\sigma_y(z),\tau_z(y)).
\end{align*}
The solution $(X,r)$ is said to be \emph{finite} if $X$ is a finite set. 
\end{definition}

\begin{figure}
\centering
\begin{tikzpicture}
\pic[
  braid/.cd,
  number of strands=3,
  height=.5cm,
  width=.5cm,
  ultra thick,
  gap=0.1,
  name prefix=braid,
] {braid={a_{1}a_2a_1}};
\end{tikzpicture}
\hspace{1cm}
\begin{tikzpicture}
\pic[
  braid/.cd,
  number of strands=3,
  height=.5cm,
  width=.5cm,
  ultra thick,
  gap=0.1,
  name prefix=braid,
] {braid={a_{2}a_1a_2}};
\end{tikzpicture}
\caption{The Yang--Baxter equation.}
\label{fig:braid}
\end{figure}


\begin{example}
Let $X$ be a non-empty set. Then $(X,\id)$ is a set-theoretic 
solution to the YBE. 	
\end{example}

\begin{example}
\index{Solution!trivial}
Let $X$ be a non-empty set. Then $(X,r)$, where $r(x,y)=(y,x)$, is a  set-theoretic solution to the YBE. This solution 
is known as the \emph{trivial solution} over the set $X$. 
\end{example}

By convention, we write
\[
r(x,y)=(\sigma_x(y),\tau_y(x)).
\]

\begin{lemma}
    \label{lem:YB}
    Let $X$ be a non-empty set and $r\colon X\times X\to X\times X$ be a bijective map.
    Then $(X,r)$ is a set-theoretic solution to the YBE if and only if 
    \begin{align*}
        &\sigma_x\sigma_y = \sigma_{\sigma_x(y)}\sigma_{\tau_y(x)},&
        &\sigma_{\tau_{\sigma_y(z)}(x)}\tau_z(y)=\tau_{\sigma_{\tau_y(x)}(z)}\sigma_x(y),&
        &\tau_z\tau_y=\tau_{\tau_z(y)}\tau_{\sigma_y(z)}
    \end{align*}
    for all $x,y,z\in X$. 
\end{lemma}

\begin{proof}
    We write $r_1=r\times\id$ and $r_2=\id\times r$. We first compute
    \begin{align*}
        r_1r_2r_1(x,y,z)&=r_1r_2(\sigma_x(y),\tau_y(x),z)
        =r_1(\sigma_x(y),\sigma_{\tau_y(x)}(z),\tau_z\tau_y(x))\\
        &=\left(\sigma_{\sigma_x(y)}\sigma_{\tau_y(x)}(z),\tau_{\sigma_{\tau_y(x)}(z)}\sigma_x(y),\tau_z\tau_y(x)\right).
    \end{align*}
    Then we compute
    \begin{align*}
        r_2r_1r_2(x,y,z)&=r_2r_1(x,\sigma_y(z),\tau_z(y))
        =r_2(\sigma_x\sigma_y(z),\tau_{\sigma_y(z)}(x),\tau_z(y))\\
        &=\left(\sigma_x\sigma_y(z),\sigma_{\tau_{\sigma_y(z)}(x)}\tau_z(y),\tau_{\tau_z(y)}\tau_{\sigma_y(z)}(x)\right)
    \end{align*}
    and the claim follows.    
\end{proof}

If $(X,r)$ is a set-theoretic solution, by definition the map $r\colon X\times X\to X\times X$ is 
invertible. By convention, we write 
 \[
 r^{-1}(x,y)=(\widehat{\sigma}_x(y),\widehat{\tau}_y(x)).
 \]
 Note that this implies that
 \[
 x=\widehat{\sigma}_{\sigma_x(y)}\tau_y(x),\quad
 y=\widehat{\tau}_{\tau_y(x)}\sigma_x(y).
 \]
 It is easy to check that $(X,r^{-1})$ is a set-theoretic solution to the YBE. Thus Lemma~\ref{lem:YB} implies that 
 the following formulas hold:
 \[
 \widehat{\tau}_y\widehat{\tau_x}=\widehat{\tau}_{\tau_y(x)}\widehat{\tau}_{\sigma_x(y)},
 \quad
 \widehat{\sigma}_x\widehat{\sigma_y}=\widehat{\sigma}_{\sigma_x(y)}\widehat{\sigma}_{\tau_y(x)}.
 \]
% Since $r(\tau^{-1}_y(x),y)=(\sigma_{\tau^{-1}_y(x)}(y),x)$, 
% it follows that 
% \[
% \widehat{\tau}_x\sigma_{\tau^{-1}_y(x)}(y)=y.
% \]
% for all $x,y\in X$. Moreover, 
% \[
% x\triangleright y=\tau_x\sigma_{\tau^{-1}_y(x)}(y)=\tau_x\widehat{\tau}^{-1}_x(y)
% \]
% for all $x,y\in X$. 

\begin{example}
Let $X=\{1,2,3,4\}$ and $r(x,y)=(\sigma_x(y),\tau_y(x))$, where
\begin{align*}
&\sigma_1=(132),&&
\sigma_2=(124),&&
\sigma_3=(143),&&
\sigma_4=(234),\\
&\tau_1=(12)(34),&&
\tau_2=(12)(34),&&
\tau_3=(12)(34),&&
\tau_4=(12)(34).
\end{align*}
Then $r$ is invertible with $r^{-1}(x,y)=(\widehat{\sigma}_x(y),\widehat{\tau}_y(x))$ given by
\begin{align*}
&\widehat{\sigma}_1=(12)(34), &&
\widehat{\sigma}_2=(12)(34), &&
\widehat{\sigma}_3=(12)(34), &&
\widehat{\sigma}_4=(12)(34),\\
&\widehat{\tau}_1=(142),&&
\widehat{\tau}_2=(123),&&
\widehat{\tau}_3=(243),&&
\widehat{\tau}_4=(134).
\end{align*}
\end{example}

\begin{definition}
A \emph{homomorphism} between the set-theoretic solutions $(X,r)$ and
$(Y,s)$ is a map $f\colon X\to Y$ such that the diagram 
\[\begin{tikzcd}
	{X\times X} & {X\times X} \\
	{Y\times Y} & {Y\times Y}
	\arrow["r", from=1-1, to=1-2]
	\arrow["{f\times f}"', from=1-1, to=2-1]
	\arrow["{f\times f}", from=1-2, to=2-2]
	\arrow["s"', from=2-1, to=2-2]
\end{tikzcd}
\]
is commutative, that is $s (f\times f)=(f\times f) r$. An \emph{isomorphism} of solutions is a bijective
homomorphism of solutions.
\end{definition}

Since we are interested in studying the combinatorics behind set-theoretic solutions to the YBE,
it makes sense to study the following family of solutions. 

\begin{definition}
\index{Solution!non-degenerate}
We say that a set-theoretic solution $(X,r)$ to the YBE 
is \emph{non-degenerate} if the maps $\sigma_x$ and $\tau_x$ are 
permutations of $X$. 
\end{definition}

By convention, a \emph{solution} we will mean a non-degenerate {\bf set-theoretic} solution to the YBE.

\begin{lemma}
\label{lem:LYZ}
Let $(X,r)$ be a solution. 
\begin{enumerate}
    \item Given $x,u\in X$, there exist unique $y,v\in X$ such that $r(x,y)=(u,v)$. 
    \item Given $y,v\in X$, there exist unique $x,u\in X$ such that $r(x,y)=(u,v)$. 
\end{enumerate}
\end{lemma}

\begin{proof}
    For the first claim take $y=\sigma_x^{-1}(u)$ and $v=\tau_y(x)$. 
    For the second, $x=\tau_y^{-1}(v)$ and $u=\sigma_x(y)$. 
\end{proof}

The bijectivity of $r$ means that any row determines the whole square. Lemma~\ref{lem:LYZ}
means that any column also determines the whole square, see Figure~\ref{fig:square}.

\begin{figure}
\centering
\begin{tikzpicture}
\pic[
  braid/.cd,
  number of strands=2,
  ultra thick,
  gap=0.1,
  name prefix=braid,
] {braid={a_{1}^{-1}}};
\node[] at (-.25,-.12) {$x$};
\node[] at (1.25,-.12) {$y$};
\node[] at (-.25,-1.4) {$u$};
\node[] at (1.25,-1.4) {$v$};
\node[] at (-.25,-.75) {$r$};
\end{tikzpicture}
\begin{tikzpicture}
\pic[
  braid/.cd,
  number of strands=2,
  ultra thick,
  gap=0.1,
  name prefix=braid,
] {braid={a_{1}}};
\node[] at (-.25,-.12) {$x$};
\node[] at (1.25,-.12) {$y$};
\node[] at (-.25,-1.4) {$u$};
\node[] at (1.25,-1.4) {$v$};
\node[] at (-.25,-.75) {$r^{-1}$};
\end{tikzpicture}
\caption{Any row or column determines the whole square.}
\label{fig:square}
\end{figure}

\begin{example}
If the map $(x,y)\mapsto(\sigma_x(y),\tau_y(x))$ satisfies the Yang--Baxter equation, then 
so does $(x,y)\mapsto (\tau_x(y),\sigma_y(x))$. 
\end{example}

\begin{example}
\label{exa:Lyubashenko}
Let $X$ be a non-empty set and $\sigma$ and $\tau$ be 
bijections on $X$ such that $\sigma\circ\tau=\tau\circ\sigma$. Then 
$(X,r)$, where $r(x,y)=(\sigma(y),\tau(x))$, is a non-degenerate solution. 
This is known as the \emph{permutation solution} associated
with permutations $\sigma$ and $\tau$. 
%The solution $(X,r)$ is involutive 
%if and only if $\tau^{-1}=\sigma$. 
\end{example}
%
%\begin{example}
%\label{exa:Wada}
%Let $G$ be a group. Then $(G,r)$, where $r(x,y)=(xy^{-1}x^{-1},xy^2)$, is a solution. 
%\end{example}

\begin{example}
\label{exa:Venkov}
Let $G$ be a group. Then $(G,r)$, where $r(x,y)=(xyx^{-1},x)$, is a solution. 
\end{example}

\begin{example}
Let $n\geq2$ and $X=\Z/(n)$ be the ring of integers modulo $n$. Prove that
the map $r(x,y)=(2x-y,x)$ satisfies the the set-theoretic YBE.  
\end{example}

\begin{theorem}[Lu--Yan--Zhu]
\label{thm:LYZ}
Let $G$ be a group, $\xi\colon G\times G\to G$, $\xi(x,y)=x\rhd y$,
be a left action of the group $G$ on itself as a set and 
$\eta\colon G\times G\to G$, $\eta(x,y)=x\lhd y$, 
be a right action of the group $G$ on itself as a set. If the compatibility condition
\[
uv=(u\rhd v)(u\lhd v)
\]
holds for all $u,v\in G$, then the pair $(G,r)$, where 
\[
r\colon G\times G\to G\times G,\quad
r(u,v)=(u\rhd v,u\lhd v)
\]
is a solution. Moreover, 
if $r(x,y)=(u,v)$, then 
\[
r(x^{-1},y^{-1})=(u^{-1},v^{-1}),
\quad
r(x^{-1},u)=(y,v^{-1}),
\quad
r(v,y^{-1})=(u^{-1},x).
\]
\end{theorem}

\begin{proof}
We write $r_1=r\times\id$ and $r_2=\id\times r$. Let
\[
r_1r_2r_1(u,v,w)=(u_1,v_1,w_1),\quad
r_2r_1r_2(u,v,w)=(u_2,v_2,w_2).
\]
The compatibility condition implies that $u_1v_1w_1=u_2v_2w_2$. 
So we need to prove that $u_1=u_2$ and $w_1=w_2$. We note that
\begin{align*}
&u_1=(u\rhd v)\rhd ( (u\lhd v)\rhd w),
&&w_1=(u\lhd v)\lhd w,\\
&u_2=u\rhd (v\rhd w),
&&w_2=(u\lhd (v\rhd w))\lhd (v\lhd w).
\end{align*}
Using the compatibility condition and the fact that $\xi$ is a left action, 
\begin{align*}
    &u_1=((u\rhd v)(u\lhd v))\rhd w=(uv)\rhd w=u\rhd (v\rhd w)=u_2.
\end{align*}
Similarly, since $\eta$ is a right action, 
\[
w_2=u\lhd ((v\rhd w)(v\lhd w))=u\lhd (vw)=(u\lhd v)\lhd w=w_1.
\]

To prove that $r$ is invertible we proceed as follows. 
Write $r(u,v)=(x,y)$, thus $u\rhd v=x$, $u\lhd v=y$ and $uv=xy$. Since 
\begin{align*}
& (y\rhd v^{-1})u=(y\rhd v^{-1})(y\lhd v^{-1})=yv^{-1}=x^{-1}u,
\end{align*}
it follows that $y\rhd v^{-1}=x^{-1}$, i.e. $v^{-1}=y^{-1}\rhd x^{-1}$. Similarly, 
\[
v(u^{-1}\lhd x)=(u^{-1}\rhd x)(u^{-1}\lhd x)=u^{-1}x=vy^{-1}
\]
implies that $u^{-1}=y^{-1}\lhd x^{-1}$. Clearly 
$r^{-1}=\zeta (i\times i) r (i\times i) \zeta$,
is the inverse of $r$, where $\zeta(x,y)=(y,x)$ and $i(x)=x^{-1}$. 
\end{proof}

\begin{proposition}
Under the assumptions of Theorem~\ref{thm:LYZ}, 
if $r(x,y)=(u,v)$, then 
\[
r(v^{-1},u^{-1})=(y^{-1},x^{-1}),
\quad
r(x^{-1},u)=(y,v^{-1}),
\quad
r(v,y^{-1})=(u^{-1},x).
\]
\end{proposition}

\begin{proof}
In the proof of Theorem~\ref{thm:LYZ} we found that 
the inverse of the map $r$ is given by $r^{-1}=\zeta (i\times i) r (i\times i) \zeta$,
where $\zeta(x,y)=(y,x)$ and $i(x)=x^{-1}$. Hence 
\[
r^{-1}(y^{-1},x^{-1})=\zeta (i\times i) r (i\times i) \zeta(y^{-1},x^{-1})=\zeta (i\times i) r (x,y)=(v^{-1},u^{-1}).
\]
It follows that $r(v^{-1},u^{-1})=(y^{-1},x^{-1})$.  
To prove the equality $r(x^{-1},u)=(y,v^{-1})$ we proceed as follows. Since $r(x,y)=(u,v)$, it 
follows that $x\triangleright y=u$. Then $x^{-1}\triangleright u=y$ and
hence $r(x^{-1},u)=(y,z)$ for some $z\in G$. 
Since $xy=uv$ and $x^{-1}u=yz$, it immediately follows that $yt=yv^{-1}$. Then 
$z=v^{-1}$. Similarly one proves $r(v,y^{-1})=(u^{-1},x)$.
\end{proof}
%
%\begin{proposition}
%Under the assumptions of Theorem~\ref{thm:LYZ}, 
%if $r(x,y)=(u,v)$, then 
%\[
%r(x^{-1},y^{-1})=(u^{-1},v^{-1}),
%\quad
%r(x^{-1},u)=(y,v^{-1}),
%\quad
%r(v,y^{-1})=(u^{-1},x).
%\]
%\end{proposition}
%
%\begin{proof}
%In the proof of Theorem~\ref{thm:LYZ} we found that 
%the inverse of the map $r$ is given by $r^{-1}=\zeta (i\times i) r (i\times i) \zeta$,
%where $\zeta(x,y)=(y,x)$ and $i(x)=x^{-1}$. It follows that $r(x^{-1},y^{-1})=(u^{-1},v^{-1})$.  
%To prove the equality $r(x^{-1},u)=(y,v^{-1})$ we proceed as follows. Since $r(x,y)=(u,v)$, it 
%follows that $x\triangleright y=u$. Then $x^{-1}\triangleright u=y$ and
%hence $r(x^{-1},u)=(y,z)$ for some $z\in G$. 
%Since $xy=uv$ and $x^{-1}u=yz$, it follows that $yt=yv^{-1}$. Then 
%$z=v^{-1}$. Similarly one proves $r(v,y^{-1})=(u^{-1},x).$
%\end{proof}



