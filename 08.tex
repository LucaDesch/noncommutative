\section{}

\subsection{Wielandt's zipper theorem}

\begin{theorem}[Wielandt]
	\index{Wielandt's zipper theorem}
	\label{thm:zipper}
	Let $G$ be a finite group and $S$ be a subgroup of $G$ subnormal in every 
    proper subgroup of $G$ containing $S$. If $S$ is not subnormal in $G$, 
    then there exists a unique maximal subgroup of $G$ containing $S$. 
\end{theorem}

\begin{proof}
	We proceed by induction on $(G:S)$. If $S$ is not subnormal in $G$, then 
	$S\ne G$ and the case where $(G:S)=1$ holds. 

	Since $S$ is not subnormal in $G$, $N_G(S)\ne G$. Then $S\subseteq
	N_G(S)\subseteq M$ for some maximal subgroup $M$ of $G$. Assume that 
	$S\subseteq K$ for some maximal subgroup $K$ of $G$. We claim that 
	que $K=M$. Since $S\subseteq K\ne G$, $S$ is subnormal in $K$. If $S$ is 
	normal in $K$, then $K\subseteq N_G(S)\subseteq M$. Hence $K=M$ by the maximality of $K$. 
	If $S$ is not normal in $K$, there exist a sequence 
	$S_0,\dots,S_m$ of subgroups of $K$ such that 
	\[
		S=S_0\subseteq S_1\subseteq\cdots\subseteq S_m=K,
	\]
    where $S_i$ is normal in $S_{i+1}$ for all $i$ and 
	$S$ is not normal in $S_2$. Let $x\in S_2$ be such that $xSx^{-1}\ne S$ and 
	$T=\langle S,xSx^{-1}\rangle\subseteq K$. 

	Since $xSx^{-1}\subseteq xS_1x^{-1}=S_1\subseteq N_G(S)$, we obtain that 
	$T\subseteq N_G(S)\subseteq M$. Moreover, $S$ is normal in $T$. Thus $T\ne G$. 

	We claim that $T$ satisfies the theorem's assumptions. If $T$ is subnormal in $G$, then, since 
	$S$ is normal in $T$, $S$ is subnormal in $G$. If $H$ is a proper subgroup of $G$ 
    containing $T$, then, since 
	$S\subseteq H$, $S$ is subnormal in $H$. Moreover, $xSx^{-1}$ is subnormal in $H$. Hence 
	$T$ is subnormal in $H$ by Theorem~\ref{thm:STsubnormal}.

	Since $S\subsetneq T$, $(G:T)<(G:S)$. By the inductive hypothesis, $T$ is contained in a unique maximal subgroup of $G$. Therefore
	$K=M$, since $T\subseteq M$ and 
	$T\subseteq K$.
\end{proof}

Before giving an application, we need a lemma. 

\begin{lemma}
	\label{lem:H=G}
	Let $G$ be a group and $H$ be a subgroup of $G$. If $(xHx^{-1})H=G$ for some 
	$x\in G$, then $H=G$.
\end{lemma}

\begin{proof}
	Write $x=uv$ for some $u\in xHx^{-1}$ and $v\in H$. Since $u\in xHx^{-1}$ and
	$u^{-1}x=v\in H$, we obtain that $H=vHv^{-1}=u^{-1}(xHx^{-1})u=xHx^{-1}$. Thus
	$G=H$. 
\end{proof}

Recall that two subgroups $S$ and $T$ of a group $G$ are said to be
\textbf{permutable} if $ST=TS$. 

\begin{theorem}
	Let $G$ be a finite group and $S$ be a subgroup of $G$ permutable with any of
	its conjugates. Then $S$ is subnormal in $G$. 
\end{theorem}

\begin{proof}
	We proceed by induction on $|G|$. Assume that $S$ is subnormal in 
	every subgroup $H$ such that $S\subseteq H\subsetneq G$.  If $S$ is not subnormal in $G$, 
	then, by Theorem~\ref{thm:zipper}, there exists a unique maximal subgroup $M$ of $G$ 
	such that $S\subseteq M$. Let $x\in G$ and 
	$T=xSx^{-1}$. By Lemma~\ref{lem:H=G}, $ST\ne G$ (because $S\ne G$). Thus 
	$ST$ is contained in some maximal subgroup of $G$. Since 
	$S\subseteq ST$ and $S$ is contained in a unique maximal subgroup of $G$, we conclude that 
	$T\subseteq ST\subseteq M$.  Since $S^G=\langle xSx^{-1}:x\in
	G\rangle\subseteq M\ne G$, the inductive hypothesis implies that $S$ is subnormal in
	$S^G$. Hence $S$ is subnormal in $G$ since $S^G$ is normal in $G$, a contradiction. 
\end{proof}

\subsection{Baer's theorem}

\begin{theorem}[Baer]
	\index{Baer's theorem}
	\label{thm:Baer}
	Let $G$ be a finite group and $H$ be a subgroup of $G$. Then $H\subseteq
	F(G)$ if and only if $\langle H,xHx^{-1}\rangle$ is nilpotent for all 
	$x\in G$.
\end{theorem}

\begin{proof}
	If $H\subseteq F(G)$, then $xHx^{-1}\subseteq F(G)$ for all $x\in G$, since
	$F(G)$ is normal in $G$. Thus $\langle H,xHx^{-1}\rangle$ is nilpotent, as it
	is a subgroup of $F(G)$.

	Conversely, assume that $\langle H,xHx^{-1}\rangle$ is nilpotent for all  $x\in
	G$. Since $H\subseteq \langle H,xHx^{-1}\rangle$, $H$ is nilpotent. By
	Theorem~\ref{thm:F(G)subnormalidad}, it is enough to see that $H$ is subnormal
	in $G$. We proceed by induction on $|G|$. Suppose that $H$ is not subnormal in
	$G$. If $H$ is properly contained in some subgroup $K$, then, since $\langle
	H,kHk^{-1}\rangle$ is nilpotent for all $k\in K$, $H$ is subnormal in $K$ by
	the inductive hypothesis. By Theorem~\ref{thm:zipper}, there exists a unique
	maximal subgroup $M$ of $G$ containing $H$. There are two cases to consider.
    
    Assume first that $G=\langle H,xHx^{-1}\rangle$ for some $x\in G$. Since $G$
    is nilpotent, $H$ subnormal in $G$ by Theorem~\ref{thm:subnormal}, a
    contradiction. 

    Assume now that $\langle H,xHx^{-1}\rangle\ne G$ for all $x\in G$. For each 
	$x\in G$, there exists a maximal subgroup containing $\langle
	H,xHx^{-1}\rangle$. Since $H\subseteq \langle H,xHx^{-1}\rangle$ and $H$
	is contained in a unique maximal subgroup, we conclude that $\langle
	H,xHx^{-1}\rangle\subseteq M$ for all $x\in G$. In particular, the normal closure 
	$H^G$ of $H$ is properly contained in $G$. By the inductive hypothesis, 
	$H$ is subnormal in $H^G$ and $H^G$ is normal in $G$, we conclude that 
	$H$ is subnormal in $G$, a contradiction. 
\end{proof}

\subsection{Zenkov's theorem}

\begin{theorem}[Zenkov]
    \index{Zenkov's!theorem}
    \label{thm:Zenkov}
    Let $G$ be a finite group and $A$ and $B$ be abelian subgroups of $G$. Let
    $M\in\{A\cap gBg^{-1}:g\in G\}$ such that no $A\cap gBg^{-1}$ is properly
    contained in $M$. Then $M\subseteq F(G)$.
\end{theorem}

\begin{proof}
	Without loss of generality, we may assume that $M=A\cap B$. Using induction on 
	$|G|$, we prove that $M\subseteq F(G)$.

	Assume first that $G=\langle A,gBg^{-1}\rangle$ for some $g\in G$. Since $A$
	and $B$ are both abelian, 
 \[
 A\cap gBg^{-1}\subseteq Z(G)
 \]
 and hence 
	\[
		A\cap gBg^{-1}=g^{-1}(A\cap gBg^{-1})g\subseteq A\cap B=M.
	\]
	By the minimality of $M$, 
    \[
    M=A\cap gBg^{-1}\subseteq Z(G)\subseteq F(G)
    \]
	by Corollary~\ref{cor:Z(G)subsetF(G)}.

	Assume now that $G\ne \langle A,gBg^{-1}\rangle$ for all $g\in G$.
	Let $g\in G$, $H=\langle A,gBg^{-1}\rangle\ne G$ and $C=B\cap H$.
	Since $A\subseteq H$, we obtain that 
 	$M=A\cap B=A\cap C$ and 
	$A\cap hCh^{-1}=A\cap hBh^{-1}$
	for all $h\in H$. This implies that no 
	$A\cap hCh^{-1}$ is properly contained in $A\cap C$. 
    By the inductive hypothesis on $H$, 
 	\[
		M=A\cap B=A\cap C\subseteq F(H).
	\]

    We now prove that every Sylow $p$-subgroup $P$ of $M$ is contained in $F(G)$. 
	Since $M$ is generated by its Sylow subgroups, 
    $M\subseteq F(G)$.
	If $P\in\Syl_p(M)$, then $P\subseteq M\subseteq F(H)$. Since $O_p(H)$ is 
    the only Sylow $p$-subgroup of $F(H)$, $P\subseteq O_p(H)$. Since 
	$P\subseteq M\subseteq B$, 
	\[
	gPg^{-1}\subseteq gBg^{-1}\subseteq H
	\]
	for all $g\in G$. Thus $O_p(H)(gPg^{-1})$ is a $p$-subgroup of $H$ containing 
	$\langle P,gPg^{-1}\rangle$. Hence $\langle P,gPg^{-1}\rangle$
	is nilpotent for all $g\in G$, since it is a $p$-group. By Baer's theorem~\ref{thm:Baer}, 
    $P\subseteq F(G)$ for all Sylow $p$-subgroup $P$ of $M$. 
\end{proof}

\begin{corollary}
	\label{cor:Zenkov}
	Let $G$ be a non-trivial finite group and $A$ be an abelian subgroup of $G$ such that 
 	$|A|\geq(G:A)$. Then $A\cap F(G)\ne\{1\}$.
\end{corollary}

\begin{proof}
	Let $g\in G$. We may assume that $G\ne A$. Then $(gAg^{-1})A\ne G$ by Lemma~\ref{lem:H=G}. Since 
	$|gAg^{-1}||A|=|A|^2\geq |A|(G:A)=|G|$, 
	\[
		|G|>|gAg^{-1}A|
		=\frac{|A||gAg^{-1}|}{|A\cap gAg^{-1}|}
		\geq \frac{|G|}{|A\cap gAg^{-1}|}.
	\]
	Hence $A\cap gAg^{-1}\ne 1$ for all $g\in G$. In particular, no
	$A\cap gAg^{-1}$ is properly contained in $A$. By 
	Zenkov's theorem~\ref{thm:Zenkov}, $A\subseteq F(G)$.
\end{proof}

\begin{corollary}
	Let $G=NA$ be a finite group, where $N$ is a normal subgroup of $G$, $A$ is an abelian subgroup of $G$ and 
	$C_A(N)=\{1\}$. If $F(N)=\{1\}$, then $|A|<|N|$. 
\end{corollary}

\begin{proof}
	Since $N$ is normal in $G$, 
	\[
    N\cap F(G)=F(N)=\{1\}
    \]
    by Corollary~\ref{cor:McapF(G)}. Thus $[N,F(G)]=\{1\}$, since 
	both $N$ and $F(G)$ are normal in $G$. Since 
	\[
	|A|\geq |N|\geq \frac{|N|}{|N\cap A|}=(NA:A)=(G:A),
	\]
	$A\cap F(G)\ne\{1\}$ by Corollary~\ref{cor:Zenkov}. If $1\ne a\in
	A\cap F(G)$, then $a\in C_A(N)=1$, a contradiction. 
\end{proof}

\subsection{Brodkey's theorem}

\begin{theorem}[Brodkey]
	\index{Teorema!de Brodkey}
	\index{Brodkey!teorema de}
	\label{theorem:Brodkey}
	Sea $G$ un grupo finito tal que existe $P\in\Syl_p(G)$ abeliano. Entonces
	existen $S,T\in\Syl_p(G)$ tales que $S\cap T=O_p(G)$.
\end{theorem}

\begin{proof}
	Al aplicar el teorema de Zenkov~\ref{theorem:Zenkov} con $A=B=P$ se tiene
	que $P\cap gPg^{-1}\subseteq F(G)$ para algún $g\in G$. Como $O_p(G)$ es el
	único $p$-subgrupo de Sylow de $F(G)$, $P\cap gPg^{-1}\subseteq O_p(G)$.
	Luego $P\cap gPg^{-1}=P_p(G)$ pues $O_p(G)$ está contenido en todo
	$p$-subgrupo de Sylow de $G$.
\end{proof}

\begin{corollary}
	\label{corollary:GP2}
	Sea $G$ un grupo finito. Si existe $P\in \Syl_p(G)$ abeliano,
	\[
	(G:O_p(G))\leq (G:P)^2. 
	\]
\end{corollary}

\begin{proof}
	Por el teorema de Brodkey~\ref{theorem:Brodkey}, existen $S,T\in\Syl_p(G)$
	tales que $S\cap T=O_p(G)$. Entonces
	\[
		|G|\geq |ST|=\frac{|S||T|}{|S\cap T|}=\frac{|P|^2}{|O_p(G)|},
	\]
	que implica el corolario.
\end{proof}

\begin{corollary}
	Sea $G$ un grupo finito. Si existe un subgrupo $P\in\Syl_p(G)$ abeliano tal que
	$|P|<\sqrt{|G|}$ entonces $O_p(G)\ne1$.
\end{corollary}

\begin{proof}
	Como $(G:P)^2<|G|$, el corolario~\ref{corollary:GP2} implica que
	$O_p(G)\ne1$.
\end{proof}

\begin{exercise}
	\label{exercise:G/Z(G)}
	Sea $G$ un grupo y sea Sea $K\subseteq Z(G)$. Demuestre que si $G/K$ es
	cíclico entonces $G$ es abeliano.
\end{exercise}

\begin{svgraybox}
	Sean $g,h\in G$ y sea $\pi\colon G\to G/K$ el morfismo canónico. Como $G/K$
	es cíclico, existe $x\in G$ tal que $G/K=\langle xK\rangle$. Sean $k,l$ tales que 
	$\pi(g)=x^k$, $\pi(h)=x^l$. Entonces existen $z_1,z_2\in K$ tales que 
	$g=x^kz_1$, $h=x^lz_2$. Luego $[g,h]=[x^k,x^l]=1$. 
\end{svgraybox}



\subsection{Lucchini's theorem}

\begin{theorem}[Lucchini]
	\index{Lucchini!teorema de}
	\index{Teorema de!Lucchini}
	\label{theorem:Lucchini}
	Sea $G$ un grupo finito y sea $A$ un subgrupo cíclico propio. Si
	$K=\Core_G(A)$ entonces $(A:K)<(G:A)$.
\end{theorem}

\begin{proof}
	Procederemos por inducción en $|G|$. Sea $\pi\colon G\to G/K$ el morfismo
	canónico. Observemos que $\Core_{G/K}\pi(A)$ es trivial. 

	Supongamos primero que $K\ne 1$. Como $\pi(A)$ es un subgrupo cíclico propio de
	$G/K$ y $K\subseteq A$, la hipótesis inductiva implica que 
	\[
		(A:K)=|\pi(A)|=(\pi(A):\pi(K))<(\pi(G):\pi(A))=\frac{(G:K)}{(A:K)}=(G:A).
	\]

	Supongamos ahora que $K=1$. Queremos demostrar que $|A|<(G:A)$. Supongamos
	entonces que $|A|\geq (G:A)$. Como $A\ne G$, $A\cap F(G)\ne1$ por el
	corolario~\ref{corollary:Zenkov}. En particular, $F(G)\ne 1$. Sea $E$ un
	subgrupo minimal-normal de $G$ tal que $E\subseteq F(G)$. Por el
	teorema~\ref{theorem:Z(nilpotent)}, $E\cap Z(F(G))\ne 1$.  Luego, como
	$E\cap Z(F(G))$ es normal en $G$ y $E$ es minimal, $E\cap Z(F(G))=E$, es
	decir $E\subseteq Z(F(G))$. En particular, $E$ es abeliano y luego, por la
	minimalidad de $E$, existe un primo $p$ tal que $x^p=1$ para todo $x\in E$. 

	\begin{claim}
		$A\cap F(G)$ es un subgrupo normal de $EA$.
	\end{claim}

	Como $E$ es normal en $G$, $EA$ es un subgrupo de $G$. Como $A\cap
	F(G)\subseteq A$, $A\cap F(G)$ es un subgrupo de $EA$.  Como $F(G)$ es
	normal en $G$, $a(A\cap F(G))a^{-1}=A\cap F(G)$ para todo $a\in A$. Por
	otro lado $E\subseteq Z(F(G))$ y $A\cap F(G)\subseteq F(G)$ implican que
	$x(A\cap F(G))x^{-1}=A\cap F(G)$ para todo $x\in E$. 

	\begin{claim}
		$EA\ne G$.
	\end{claim}

	Si $G=EA$ entonces, como $A\cap F(G)$ es un subgrupo normal de $G$
	contenido en $A$, se concluye que $1\ne A\cap F(G)\subseteq K=1$, una
	contradicción.
%como $F(G)$ es normal en $G$, 
%	\[
%	A\cap F(G)=g(A\cap F(G))g^{-1}=gAg^{-1}\cap F(G)\subseteq gAg^{-1}
%	\]
	para todo $g\in G$. Luego $1\ne A\cap F(G)\subseteq K$, una contradicción pues $K=1$.

	\medskip
	Sea $p\colon G\to G/E$ el morfismo canónico. Por la correspondencia, existe
	un subgrupo normal $M$ de $G$ con $E\subseteq M$ tal que
	$p(M)=\Core_{G/E}(p(A))$. Como $EA\ne G$, $p(A)$ es un subgrupo cíclico
	propio de $p(G)$. Como $p(A)\simeq A/A\cap E\simeq EA/E$ y $p(M)\simeq
	M/E$, la hipótesis inductiva implica que 
	$(EA:M)<(G:EA)$ pues 
	\[
	\frac{|EA/E|}{|M/E|}
	=(p(A):p(M))
	<(p(G):p(A))
	=\frac{|G/E|}{|EA/E|}.
	\]

	\begin{claim}
		$MA=EA$. 
	\end{claim}

	Como $E\subseteq M$ entonces $EA\subseteq MA$. Recíprocamente, si $m\in M$
	entonces, como $p(m)\in\Core_{G/E}(p(A))$, en particular 
	$p(m)\in p(A)$. Luego $m\in EA$. 

	\medskip
	Sea $B=A\cap M$. Al usar que $(AE:M)<(G:EA)$, 
	que
	\[
	(A:B)=|A/A\cap M|=|AM/M|=(EA:M)
	\]
	y la  hipótesis inductiva
	obtenemos
	\begin{equation}
		\label{eq:(M:B)leq|B|}
	\begin{aligned}
		(M:B)&=(M:A\cap M)=(MA:A)\\
		&=(EA:A)
		=\frac{(G:A)}{(G:EA)}
		<\frac{(G:A)}{(AE:M)}
		=\frac{(G:A)}{(A:B)}\leq |B|
	\end{aligned}
	\end{equation}
	pues $|A|\geq (G:A)$. 

	\begin{claim}
		$M=EB$.
	\end{claim}

	Como $E\cup B\subseteq M$ entonces $EB\subseteq M$. Recíprocamente, si
	$m\in M$ entonces $m=ea$ para algún $e\in E$, $a\in A$. Como $e^{-1}m=a\in
	A\cap M=B$ pues $E\subseteq M$ entonces $m\in EB$.

	\begin{claim}
		$M$ es no abeliano.
	\end{claim}

	Supongamos que $M$ es abeliano. La función $f\colon M\to M$, $m\mapsto
	m^p$, es un morfismo de grupos tal que $E \subseteq\ker f$. Como $M=EB$,
	$f(M)\subseteq f(B)\subseteq B\subseteq A$. Como $M$ es normal en $G$,
	$f(M)$ es normal en $G$. Luego $f(M)=1$ pues $K=\Core_G(A)=1$ es el
	mayor subgrupo normal de $G$ contenido en $A$; en particular, como $B$ es
	normal en $M=EB$, $M/B$ es un $p$-grupo. Como $B\subseteq M$,  $f(B)=1$;
	además como $B\subseteq A$ es cíclico, $|B|\leq p$. Luego, por la
	fórmula~\eqref{eq:(M:B)leq|B|}, 
	$(M:B)<|B|\leq p$. Esto implica que $M=B\subseteq A$ y que $M=E=1$ (porque
	$M$ es normal en $G$ y $\Core_G(A)=K=1$ es el mayor subgrupo normal de $G$
	que contiene a $A$), una contradicción.

	\begin{claim}
		$Z(M)$ es cíclico.	
	\end{claim}

	Como $M$ es no abeliano y $M/E=EB/E\simeq B/E\cap B$ es ciclico,
	$E\not\subseteq Z(M)$ (ejercicio~\ref{exercise:G/Z(G)}), es decir $E\cap
	Z(M)\subsetneq E$.  Luego 
	\begin{equation}
		\label{equation:EcapZ(M)}
		E\cap Z(M)=1
	\end{equation}
	por la minimalidad de $E$. Entonces
	\[
	Z(M)=Z(M)/Z(M)\cap E\simeq p(Z(M))\subseteq p(M)=\Core_{G/E}p(A)\subseteq p(A)
	\]
	y luego $Z(M)$ es cíclico pues $p(A)$ es cíclico.

	\medskip
	Como $B\subseteq A$ es abeliano y $(M:B)<|B|$, $B\cap F(M)\ne1$ por el
	corolario~\ref{corollary:Zenkov}. Entonces $[E,F(M)]=1$ (pues $E\subseteq
	Z(F(G))$ y $F(M)\subseteq F(G)$ por el corolario~\ref{corollary:McapF(G)}).
	Luego $B\cap F(M)\subseteq Z(M)$ pues $M=BE$, $[B\cap F(M),E]\subseteq
	[F(M),E]=1$ y también $[B\cap F(M),B]=1$ porque $B$ es abeliano. Como
	$Z(M)$ es cíclico, $B\cap F(M)$ es característico en $Z(M)$. Luego, como
	$Z(M)$ es normal en $G$, $1\ne B\cap F(M)$ es un subgrupo normal de $G$
	contenido en $A$, una contradicción.
\end{proof}

Para terminar la sección veamos una aplicación del teorema de Lucchini.

\begin{corollary}[Horosevskii]
	\index{Horosevskii!teorema de}
	\index{Teorema de!Horosevskii}
	Sea $G\ne1$ un grupo finito y sea $\sigma\in\Aut(G)$. Entonces
	$|\sigma|<|G|$.
\end{corollary}

\begin{proof}
	Sea $A=\langle\sigma\rangle$. Como $A$ actúa por automorfismos en $G$,
	podemos considerar el grupo $\Gamma=G\rtimes A$ con la operación 
	\[
	(g,\sigma^k)(h,\sigma^l)=(g\sigma^k(h),\sigma^{k+l}).
	\]
	Identificamos $A$ con $1\times A$ y $G$ con $G\times 1$. 
	Como $K\cap G\subseteq A\cap G=1$ y $A\cap C_{\Gamma}(G)=1$, 
	\[
		K\subseteq A\cap C_{\Gamma}(G)=1
	\]
	pues si $k\in K$ y $g\in G$ entonces $gkg^{-1}k^{-1}\in G\cap K=1$ (porque
	$K$ y $G$ son normales en $\Gamma$).  Por el teorema de
	Lucchini~\ref{theorem:Lucchini}, $(A:K)<(\Gamma:A)$, es decir
	\[
		|\sigma|=|A|=(A:K)<(\Gamma:A)=|G|.
	\]
\end{proof}