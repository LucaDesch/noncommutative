\section{15/02/2024}

\subsection{Solvable groups}

\index{Subgroup!characteristic}
A subgroup $H$ of $G$ is said to be \textbf{characteristic} if 
$f(H)\subseteq H$ for all $f\in\Aut(G)$. 
The center and the commutator subgroup are
characteristic subgroups. 
Every characteristic subgroup is normal, 
as the maps $x\mapsto gxg^{-1}$ are automorphisms. 

\begin{exercise}
    Prove that if $H$ is characteristic in $K$ and $K$ is normal in $G$, then
    $H$ is normal in $G$. 
\end{exercise}

\index{Derived series}
For a group $G$, let 
$G^{(0)}=G$ and 
$G^{(i+1)}=[G^{(i)},G^{(i)}]$ for $i\geq0$.
The \textbf{derived series} of $G$ is the sequence
\[
G=G^{(0)}\supseteq G^{(1)}\supseteq G^{(2)}\supseteq\cdots
\]
Each $G^{(i)}$ is a characteristic subgroup of $G$. We say that 
$G$ is \textbf{solvable} if $G^{(n)}=\{1\}$ for some $n$.  

\begin{example}
	Abelian groups are solvable. 
\end{example}

\begin{example}
	The group $\SL_2(3)$ is solvable, as the derived series is 
	\[
	\SL_2(3)\supseteq Q_8\supseteq C_4\supseteq \{1\}.
	\]
\end{example}

\begin{example}
	Non-abelian simple groups cannot be solvable. 
\end{example}

\begin{exercise}
	\label{xca:solvable}
	Let $G$ be a group. Prove the following statements:
	\begin{enumerate}
		\item A subgroup $H$ of $G$ is solvable, when $G$ is solvable.
		\item Let $K$ be a normal subgroup of $G$. 
		    Then $G$ is solvable if and only if $K$ and $G/K$ are solvable.
	\end{enumerate}
\end{exercise}

\begin{example}
    For $n\geq5$ the group $\Alt_n$ is simple and non-abelian. Hence it 
    is not solvable. It follows that 
    $\Sym_n$ is not solvable for $n\geq5$. 
\end{example}

\begin{exercise}
\label{xca:Robinson:5.4.1}
    Let $p$, $q$ and $r$ be prime numbers. Prove that groups
    of order $p^\alpha q$, $p^2q^2$ and $pqr$ are solvable. 
\end{exercise}

\begin{exercise}
\label{xca:less60}
    Prove that groups of order $<60$ are solvable. 
\end{exercise}

\begin{exercise}
\label{xca:pgroups_solvable}
Let $p$ be a prime number. Prove that finite $p$-groups are solvable.
\end{exercise}

\begin{theorem}[Burnside]
	\index{Burnside's theorem}
        \label{thm:Burnside}
	Let $p$ be a prime number. If $G$ is a finite group that has 
        a conjugacy class of $G$ with $p^k>1$ elements, then $G$ 
	is not simple.
\end{theorem}

The easiest way to prove Theorem \ref{thm:Burnside}
is using character theory. 

\begin{theorem}[Burnside]
  \index{Burnside's theorem}
  Let $p$ and $q$ be prime numbers. If $G$ has order $p^aq^b$, then $G$ is solvable.
\end{theorem}

\begin{proof}
	If $G$ is abelian, then it is solvable.
	Suppose now $G$ is non-abelian.
	Let us assume that the theorem is not true. Let $G$ be a group
	of minimal order $p^aq^b$
	that is not solvable. Since $|G|$ is minimal, $G$ is a non-abelian simple group.
	By the previous theorem, 
	$G$ has no conjugacy classes of size $p^k$ nor 
	conjugacy classes of size $q^l$ with $k,l\geq1$. The size
	of every conjugacy class of $G$ is one or divisible by $pq$. 
	Note that, since $G$ is a non-abelian simple group,
	the center of $G$ is trivial.
	Thus there is only one conjugacy class of size one.
	By the class
	equation,
	\[
		|G|=1+\sum_{C:|C|>1}|C|\equiv 1 \bmod pq,
	\]
	where the sum is taken over all conjugacy classes 
	with more than one element, a contradiction.
\end{proof}

A recent generalization of Burnside's theorem
is based on \emph{word maps}. A word map
of a group $G$ is a map 
\[
G^k\to G,\quad 
(x_1,\dots,x_k)\mapsto w(x_1,\dots,x_k)
\]
for some word $w(x_1,\dots,x_k)$ of the free group $F_k$ of rank $k$. 
Some word maps are surjective in certain families of groups. For example, 
Ore's conjecture is precisely the surjectivity of the word map
$(x,y)\mapsto [x,y]=xyx^{-1}y^{-1}$ in every finite non-abelian simple 
group. 

\begin{theorem}[Guralnick--Liebeck--O'Brien--Shalev--Tiep]
    Let $a,b\geq0$, $p$ and $q$ be prime numbers and $N=p^aq^b$. The map 
    $(x,y)\mapsto x^Ny^N$ is surjective in every non-abelian 
    finite simple group. 
\end{theorem}

The proof appears in~\cite{MR3827208}. 

The theorem implies Burnside's theorem. Let $G$ be a group of order
$N=p^aq^b$. Assume that $G$ is not solvable. 
Fix a composition series of $G$. There is a non-abelian factor $S$ 
of order that divides $N$. Since 
$S$ is simple non-abelian and $s^N=1$, it follows that the word map
$(x,y)\mapsto x^Ny^N$ has trivial image in $S$, a contradiction 
to the theorem. 

\begin{theorem}[Feit--Thompson]
    \index{Feit--Thompson theorem}
    Groups of odd order are solvable. 
\end{theorem}

The proof of Feit--Thompson theorem is extremely hard. 
It occupies a full volume of the 
\emph{Pacific Journal of Mathematics}~\cite{MR166261}. 
A formal verification of the proof 
(based on the computer software Coq) 
was announced in~\cite{MR3111271}.  This motivates a natural problem: To formally verify 
the classification of finite simple groups.  
Will mathematics move away from depending on just humans to verify proofs? Formal verification with computer-proof assistants 
could become the new standard for rigor in mathematics. 

The proof of the Feit-Thompson theorem is notably hard, spanning an entire volume of the \emph{Pacific Journal of Mathematics}\cite{MR166261}. The theorem was recently verified using Coq, a computer-proof assistant; see \cite{MR3111271}. 
This prompts a natural question: Can we extend the practice of formal verification to the classification of finite simple groups? And this prompts another question: Will mathematics move away from depending on just humans to verify proofs? Formal verification with computer-proof assistants 
could become the new standard for rigor in mathematics. 

Back in the day, it was believed that if a certain divisibility 
conjecture is true, 
the proof of Feit--Thompson theorem 
could be simplified. 

\begin{conjecture}[Feit--Thompson]
\index{Feit--Thompson conjecture}
    There are no prime numbers $p$ and $q$ such that
    $\frac {p^{q}-1}{p-1}$ divides $\frac{q^{p} - 1}{q - 1}$. 
\end{conjecture}

The conjecture remains open. However, now we know that 
proving the conjecture will not simplify further
the proof of Feit--Thompson theorem. 

In 2012, Le proved that the conjecture is true for $q=3$, see 
\cite{MR2900154}. 

In~\cite{MR297686} 
Stephens proved that a certain stronger version of the conjecture 
does not hold, as the integers 
$\frac {p^{q}-1}{p-1}$ and $\frac{q^{p} - 1}{q - 1}$ 
could have common factors. In fact, if $p=17$ and $q=3313$, 
then 
\[
\gcd\left(\frac {p^{q}-1}{p-1},\frac{q^{p} - 1}{q - 1}\right)=112643.
\]
Nowadays we can check this easily on almost every desktop computer:
\begin{lstlisting}
gap> Gcd((17^3313-1)/16,(3313^17-1)/3312);
112643
\end{lstlisting}
No other counterexamples have been found of Stephen’s 
stronger version of the conjecture.

\begin{definition}
\index{Subgroup!elementary abelian}
Let $p$ be a prime number. A $p$-group $P$ is said to be 
\textbf{elementary abelian} if $x^p=1$ for all $x\in P$.
\end{definition}

\begin{definition}
\index{Subgroup!minimal normal}
A subgroup $M$ of $G$ is said to be \textbf{minimal normal} if $M\ne\{1\}$,
$M$ is normal in $G$ and the only normal 
subgroup of $G$ properly contained in $M$ is $\{1\}$. 
\end{definition}

\begin{example}
    If a normal subgroup $M$ is minimal (with respect to the inclusion), 
    then it is minimal and normal. However, 
    the converse statement does not hold. For example, the subgroup
    of $\Alt_4$ generated by $(12)(34)$, $(13)(24)$ and $(14)(23)$ is minimal normal in $\Alt_4$ 
    but it is not minimal. 
\end{example}

\begin{exercise}
Prove that every finite group contains a minimal normal subgroup. 
\end{exercise}

\begin{example}
    Let $G=\D_{6}=\langle r,s:r^6=s^2=1,\,srs=r^{-1}\rangle$ the dihedral group of twelve elements. The subgroups
    $S=\langle r^2\rangle$ 
    and $T=\langle r^3\rangle$ are (the only) minimal normal in $G$.
\end{example}

\begin{example}
    Let $G=\SL_2(3)$. The only minimal normal subgroup of $G$ is its center
    $Z(\SL_2(3))\simeq C_2$.
\end{example}


The following lemma will be very useful later. 

\begin{lemma}
\label{lemma:minimal_normal}
Let $M$ be a minimal normal subgroup of $G$. If $M$ is solvable and finite, 
then $M$ is an elementary abelian $p$-group for some prime number $p$. 
\end{lemma}

\begin{proof}
Since $M$ is solvable, $[M,M]\subsetneq M$. Moreover, $[M,M]$ is normal in $G$, as
$[M,M]$ is characterisitic in $M$ and $M$ is normal in $G$. Since $M$ is minimal normal, 
$[M,M]=\{1\}$. Hence $M$ is abelian. 
	
Since $M$ is finite, there is a prime number $p$ such that $\{1\}\ne P=\{x\in
M:x^p=1\}\subseteq M$.  Since $P$ is characteristic in $M$, $P$ is normal in 
$G$. By minimality, $P=M$.
\end{proof}

\begin{theorem}
	Let $G$ be a finite non-trivial solvable group. Then 
		every maximal subgroup of $G$ has index $p^\alpha$ for some prime number $p$. 
\end{theorem}

\begin{proof}
	We proceed by induction on $|G|$.
	If $|G|$ is a prime power, the claim is clear. Assume that $|G|\geq6$ and let $M$ be a maximal subgroup of $G$. 
        Let $N$ be a minimal normal subgroup of $G$ and $\pi\colon G\to G/N$ the canonical map. 
	If $N=G$, then $N=G$ is a $p$-group and we are done. Assume then that 
	$N\ne G$. Since $M\subseteq NM\subseteq G$,
	either $M=NM$ or $NM=G$ (by the maximality of $M$).  If 
	$M=NM\supseteq N$, then $\pi(M)$ is a maximal subgroup of $\pi(G)=G/N$. Hence 
	\[
	(G:M)=(\pi(G):\pi(M))
	\]
	is a prime power by the inductive hypothesis. If 
	$NM=G$, then 
	\[
	(G:M)=\frac{|G|}{|M|}=\frac{|NM|}{|N|}=\frac{|N|}{|N\cap M|}
	\]
	is a prime power, because $N$ is a $p$-group by the previous lemma. 
\end{proof}

\begin{exercise}
    Let $G$ be a finite non-trivial solvable group. Prove that 
    there exists a prime number $p$ such that $G$ 
    contains a minimal normal $p$-subgroup. 
\end{exercise}

\begin{example}
	Let $G=\Sym_4$. The $2$-subgroup 
	\[
	K=\{\id,(12)(34),(13)(24),(14)(23)\}\simeq
	C_2\times C_2
	\]
	is minimal normal. Note that $G$ does not have minimal normal $3$-subgroups. 
\end{example}

\begin{theorem}
Let $G$ be a finite non-trivial group. Then $G$ is solvable if and only if 
every non-trivial quotient of $G$ contains an abelian non-trivial normal subgroup. 
\end{theorem}

\begin{proof}
Every quotient of $G$ is solvable and therefore contains an abelian 
minimal normal subgroup. To prove the converse we proceed by induction on $|G|$. Let 
$N$ be a normal abelian subgroup of $G$. If $N=G$, then $G$ is solvable (because it is abelian). If 
$N\ne G$, then $|G/N|<|G|$. Since every quotient of $G/N$ is a quotient of $G$, the group 
$G/N$ satisfies the assumptions of the theorem. Hence $G/N$ is solvable by the inductive hypothesis. 
Now $N$ and $G/N$ are solvable, so is $G$. 
\end{proof}

% Since aplicación del teorema~\ref{theorem:SchurZassenhaus} de
% Schur--Zassenhaus, in el capítulo~\ref{extensiones},
% teorema~\ref{theorem:solvable_maximal}, demostraremos que si $G$ is un
% grupo finite solvable no trivial and $p$ is un primo que divide al orden de
% $G$, existe un subgrupo maximal de índice una potencia de $p$. 
% Otra aplicación del teorema de Schur--Zassenhaus: la teoría de Hall, 
% una generalización de
% la teoría de Sylow para grupos solvables. 

\begin{exercise}
	\label{exercise:solvable}
	Let $G$ be a group. Prove that $G$ is solvable if and only if there is a sequence
	\[
		\{1\}=N_0\subseteq N_1\subseteq\cdots\subseteq N_k=G
	\]
        of normal subgroups such that every quotient 
	$N_i/N_{i-1}$ is abelian.
\end{exercise}

% \begin{svgraybox}
% 	Si existe una sucesión de subgrupos normales $1=N_0\subseteq
% 	N_1\subseteq\cdots\subseteq N_k=G$ tales que cada cociente $N_i/N_{i-1}$ es
% 	abeliano, entonces $[N_i,N_i]\subseteq N_{i-1}$ para cada $i$. Demostremos
% 	por inducción que $G^{(m)}\subseteq N_{n-m}$ para todo $m\leq n$. El caso
% 	$m=0$ is trivial pues $G^{(0)}=G\subseteq N_{n}=G$; si suponemos que el
% 	resultado vale para $m$ entonces, por hipótesis inductiva, 
% 	\[
% 	G^{(m+1)}=[G^{(m)},G^{(m)}]\subseteq [N_{n-m},N_{n-m}]\subseteq N_{n-m-1}.
% 	\]
% 	Luego $G^{(n)}\subseteq N_{0}=1$ and $G$ is solvable.

% 	Supongamos ahora que $G$ is solvable. Entonces existe $n\in\N$ tal que
% 	$G^{(n)}=1$. Since cada $G^{(i)}$ is característico in $G$, in particular
% 	cada $G^{(i)}$ is normal in $G$. Además 
% 	\[
% 		1=G^{(n)}\subseteq G^{(n-1)}\subseteq\cdots\subseteq G^{(0)}=G.
% 	\]
% \end{svgraybox}

\subsection{Hall's theorem}

We start with an extremely simple and useful tool. 

\begin{lemma}[Frattini's argument]
    \label{lem:Frattini_argument}
    \index{Frattini's argument}
    Let $G$ be a finite group and $K$ be a normal subgroup of $G$. If 
    $P\in\Syl_p(K)$ for some prime number $p$, then $G=KN_G(P)$.
\end{lemma}

\begin{proof}
	Let $g\in G$. Since $gPg^{-1}\subseteq gKg^{-1}=K$, because $K$ is normal in $G$, 
	and $gPg^{-1}\in\Syl_p(K)$, there exists $k\in K$ such that 
	$kPk^{-1}=gPg^{-1}$. Hence $k^{-1}g\in N_G(P)$, as
	$P=(k^{-1}g)P(k^{-1}g)^{-1}$. Therefore $g=k(k^{-1}g)\in
	KN_G(P)$.
\end{proof}

\begin{theorem}[Hall]
	\label{theorem:Hall}
        \index{Hall's theorem}
	Let $G$ be a finite group such that every maximal subgroup of $G$ 
        has a prime or a prime-square index. Then $G$ is solvable. 
\end{theorem}

\begin{proof}
We proceed by induction on $|G|$. Let $p$ be the largest prime divisor of $|G|$. Let 
$S\in\Syl_p(G)$ and $N=N_G(S)$. 

If $N=G$, then $S$ is normal in $G$. Since every 
maximal subgroup of $G/S$ has prime or a prime-square index, 
$G/S$ is solvable by the inductive hypothesis. Since $S$ is a $p$-group, it is solvable. 
Therefore $G$ is solvable. 

Assume now that $N\ne G$. Let $H$ be a maximal subgroup of $G$ containing $N$. Then 
\[
N=N_H(S)=N_G(S),
\]
the number of Sylow $p$-subgroups of $G$ is $(G:N)\equiv1\bmod p$  
and the number of Sylow $p$-subgroups of $H$ is $(G:H)\equiv1\bmod p$ 
by the third Sylow's theorem. By assumption, there exists a prime number $q$ 
such that 
$(G:H)\in\{q,q^2\}$. Since $q$ divides $|G|$, it follows that $p<q$. If $(G:H)=q$, then 
$p$ divides $q-1$ and therefore $p\leq q-1<p$, a contradiction. Thus $q^2=(G:H)\equiv 1\bmod p$.
From this it follows that $q\equiv -1\bmod p$ and hence $q=2$ and 
$p=q+1=3$. 

Therefore $G$ has order $2^\alpha 3^ \beta$. 
If we apply Burnside's theorem, we are done. Instead, we will finish the proof with an elementary argument. Let $K$ be a minimal normal subgroup of $G$.
By Frattini's argument (Lemma~\ref{lem:Frattini_argument}),  $G=KN=KH$. Since $H$ is maximal, 
\[
(K:K\cap H)(G:H)=4, 
\]
as $(G:H)=|G|/|H|=|KH|/|H|=|K|/|K\cap H|$. 
Since $(K:K\cap H)=4$, letting $K$ act on $K/K\cap H$ by left multiplication, there exists 
a non-trivial group homomorphism $\rho\colon K\to\Sym_4$. Since $[K,K]$ is characteristic in $K$ and $K$ is normal in $G$, $[K,K]\subseteq K$ is normal in $G$. Since $K$ is minimal normal in $G$, 
there are two possible cases: either $[K,K]=\{1\}$ or 
$[K,K]=K$. If $[K,K]=K$, 
since $\Sym_4$ is solvable, $\rho(K)$ is solvable. Then
\[
\rho(K)=\rho([K,K])=[\rho(K),\rho(K)],
\]
a contradiction. Therefore $[K,K]=\{1\}$ and $K$ is solvable (as it is abelian).
\end{proof}

% \begin{proof}
% 	We proceed by induction on $|G|$. Let $N$ be a minimal normal subgroup of 
% 	$G$ and $p$ be the largest prime divisor of $|N|$. Let $P\in\Syl_p(N)$ and 
% 	$L=N_G(P)$. If $L=G$, then $P$ is normal in $G$. Since $P$ and 
% 	$G/P$ are solvable by the inductive hypothesis, $G$ is solvable. Assume now that 
% 	$L\ne G$. Let $M$ be a maximal subgroup of $G$ containing $L$. 
% 	    By Frattini's argument (Lemma~\ref{lem:Frattini_argument}),  
% 	$G=NL=NM$. Since $M$ is maximal, 
% 	there exists a prime number $q$ such that  
% 	\[
% 	(N:N\cap M)=(G:M)\in\{q,q^2\}, 
% 	\]
% 	as $(G:M)=|G|/|M|=|NM|/|M|=|N|/|N\cap M|$.  
% 	Thus $q$ divides $|N|$ and hence $q\leq p$. In particular, $q\not\equiv
% 	1\bmod p$ (otherwise, if $q\equiv1\bmod p$, 
%         then $p\mid q-1$ and thus $p\leq q-1<q\leq p$, a contradiction). 
% 	If $g\in G$, then  
% 	\[
% 	gPg^{-1}\subseteq gNg^{-1}=N
% 	\]
% 	and hence $gPg^{-1}\in\Syl_p(N)$. Let $G$ act by conjugation on 
% 	$\Syl_p(N)$. The number of $p$-subgroups of $N$ is then 
% 	\[
% 		(G:N_G(P))=(G:L)\equiv 1\bmod p.	
% 	\]
% 	Since $L\subseteq M$, $L=N_M(P)$. 
	
% 	%Supongamos que $|P|=p^{\alpha}$.  
% 	Since $P\subseteq M$, $P$ acts 
% 	on the set $X=\{mPm^{-1}:m\in M\}$. We claim that $\{P\}$ is the only one-element orbit. 
%         If $\{P_1\}$ is an orbit, then $P_1=mPm^{-1}$ for some $m\in M$ and 
%  	$P=m^{-1}P_1m=P_1$. 
% % 	es un subgrupo normal de $\langle P,P_1\rangle$ y
% % 	luego $P_1P$ is un subgrupo de $M$ de orden $p^{\beta}$ con
% % 	$\beta\leq\alpha$. Since $P\subseteq P_1P$, se concluye que $P_1P=P$ and luego
% % 	$P=P_1$. 

%     Decompose $X$ as a disjoint union of orbits: 
% 	\[
% 	X=\{P\}\cup O(P_1)\cup\cdots\cup O(P_k),
% 	\]
% 	where $\{P\}$ is the only one-element orbit and 
%         each $O(P_j)$ has size divisible by $p$. 
% 	Then 
% 	\[
% 		(M:N_M(P))=(M:L)=|X|\equiv1\bmod p.
% 	\]
% 	From $(G:L)=(G:M)(M:L)$ we conclude that 
%         $(G:M)\equiv1\bmod p$.
% 	Therefore 
%  \[
%  (G:M)=q^2\equiv 1\bmod p.
%  \]
%  This implies that $q\equiv -1\bmod
% 	p$. Hence $q=2$ and $p=q+1=3$.

% 	Since $(N:N\cap M)=4$, letting $N$ act on $N/N\cap M$ by left multiplication, there exists 
%         a non-trivial group homomorphism $\rho\colon
% 	N\to\Sym_4$. Since $[N,N]$ is characteristic in $N$ and $N$ is normal in $G$,
% 	there are two cases: either $[N,N]=\{1\}$ or 
% 	$[N,N]=N$. If $[N,N]=N$, then
% 	\[
% 	\rho(N)=\rho([N,N])=[\rho(N),\rho(N)].
% 	\]
% 	Since $\Sym_4$ is solvable, $\rho(N)$ is solvable. Hence $\rho(N)=\{1\}$, a contradiction. 
% 	Therefore $[N,N]=\{1\}$. Since $N$ is solvable (it is abelian) 
% 	and $G/N$ is solvable by the inductive hypothesis, $G$ is solvable.
% %	El grupo $N$ tiene orden $2^a3^b$, and entonces $N$ is solvable por el
% %	teorema de Burnside.  Since $N$ is solvable and $G/N$ is solvable por
% %	hipótesis inductiva, $G$ is solvable por el
% %	teorema~\ref{theorem:solvable}. 
% \end{proof}

