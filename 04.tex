\chapter{
}
\section{Bi-ordered groups}

Based on Example~\ref{example:k[Z]} we will study 
some properties of groups. 

Recall that a \textbf{total order} is a partial order in which any two elements are comparable. This
means that a total order is a binary relation $\leq$ on some set $X$ 
such that for all $x,y,z\in X$ one has 
\begin{enumerate}
    \item $x\leq x$.
    \item $x\leq y$ and $y\leq z$ imply $x\leq z$.
    \item $x\leq y$ and $y\leq x$ imply $x=y$.
    \item $x\leq y$ or $y\leq x$. 
\end{enumerate}
A set equipped with a total order is a \textbf{totally ordered set}. 

\begin{definition}
	\index{Group!bi-ordered}
	A group $G$ is \textbf{bi-ordered} if there exists a total order 
	$<$ in $G$
	such that $x<y$ implies that $xz<yz$ and $zx<zy$ for all $x,y,z\in G$.
\end{definition}

\begin{example}
	The group $\R_{>0}$ of positive real numbers is bi-ordered. 
\end{example}

\begin{exercise}
	Let $G$ be a bi-ordered group and $x,x_1,y,y_1\in G$. Prove that
	$x<y$ and $x_1<y_1$ imply $xx_1<yy_1$.
\end{exercise}

\begin{exercise}
	Let $G$ be a bi-ordered group and $g,h\in G$. Prove that $g^n=h^n$
	for some $n>0$ implies $g=h$.
\end{exercise}

\begin{definition}
	Let $G$ be a bi-ordered group. The \textbf{positive cone} of $G$  
	is the set $P(G)=\{x\in
	G:1<x\}$.
\end{definition}

Let us state some properties of positive cones. 

\begin{proposition}
	\label{pro:biordenableP1}
	Let $G$ be a bi-ordered group and let $P$ be its positive cone. 
	\begin{enumerate}
		\item $P$ is closed under multiplication, i.e. $PP\subseteq P$. 
		\item $G=P\cup P^{-1}\cup \{1\}$ (disjoint union).
		\item $xPx^{-1}=P$ for all $x\in G$.
	\end{enumerate}
\end{proposition}

\begin{proof}
	If $x,y\in P$ and $z\in G$, then, since $1<x$ and $1<y$, it follows that 
	$1<xy$.  Thus $1=z1z^{-1}<zxz^{-1}$. It remains to prove the second claim.  
	If $g\in G$, then $g=1$ or $g>1$ or $g<1$. Note that $g<1$ if and only if 
	$1<g^{-1}$, so the claim follows. 
\end{proof}

The previous proposition admits a converse statement. 

\begin{proposition}
	\label{pro:biordenableP2}
	Let $G$ be a group and $P$ be a subset of $G$ such that 
	$P$ is closed under multiplication, $G=P\cup P^{-1}\cup \{1\}$ (disjoint union) and
	$xPx^{-1}=P$ for all $x\in G$. Let $x<y$ whenever  
	$yx^{-1}\in P$. Then $G$ is bi-ordered with positive
	cone is $P$.
\end{proposition}

\begin{proof}
	Let $x,y\in G$. Since $yx^{-1}\in G$ and $G=P\cup
	P^{-1}\cup\{1\}$ (disjoint union), 
	either $yx^{-1}\in P$ or $xy^{-1}=(yx^{-1})^{-1}\in
	P$ or $yx^{-1}=1$. Thus either $x<y$ or $y<x$ or $x=y$. If $x<y$ and $z\in
	G$, then $zx<zy$, as $(zy)(zx)^{-1}=z(yx^{-1})z^{-1}\in P$ and  
	$zPz^{-1}=P$. Moreover, $xz<yz$ since $(yz)(xz)^{-1}=yx^{-1}\in P$. To prove
	that $P$ is the positive cone of $G$ note that 
	$x1^{-1}=x\in P$ if and only if $1<x$. 
\end{proof}

An important property:

\begin{proposition}
	\label{pro:BOsintorsion}
	Bi-ordered groups are torsion-free.
\end{proposition}

\begin{proof}
	Let $G$ be a bi-ordered group and $g\in G\setminus\{1\}$. 
	If $g>1$, then
	$1<g<g^2<\cdots$. If $g<1$, then $1>g>g^2>\cdots$. Hence $g^n\ne 1$ 
	for all $n\ne 0$. 
\end{proof}

The converse of the previous proposition does not hold. 

\begin{exercise}
Let $G=\langle x,y:yxy^{-1}=x^{-1}\rangle$. 
\begin{enumerate}
    \item Prove that $x$ and $y$ are torsion-free. 
    \item Prove that $G$ is torsion-free. 
    \item Prove that $G\simeq \langle a,b:a^2=b^2\rangle$.
\end{enumerate}
\end{exercise}

% To prove that $x$ is torsion-free, use the group homomorphism $G\to\Z$
% given by $x\mapsto 1$, $y\mapsto 0$. To prove that $y$ is torsion-free
% use the group homomorphism $x\mapsto\begin{pmatrix}-1&0\\0&1\end{pmatrix}$,
% $y\mapsto\begin{pmatrix}1&1\\0&1\end{pmatrix}$. 

% 	We first show that $G\simeq\langle a,b:a^2=b^2\rangle$. For that purpose, 
% 	it is enough to check that 
% 	the map $G\to \langle a,b:a^2=b^2\rangle$, $x\mapsto a$, $y\mapsto ab^{-1}$, 
% 	is a well-defined group homomorphism with inverse
% 	$\langle a,b:a^2=b^2\rangle\to G$, $a\mapsto x$, $b\mapsto y^{-1}x$. 

\begin{example}
	The torsion-free group $G=\langle x,y:yxy^{-1}=x^{-1}\rangle$ is not bi-ordered. 
	If not, let $P$ 
	be the positive cone. If $x\in P$, 
	then $yxy^{-1}=x^{-1}\in P$, a contradiction. Hence $x^{-1}\in P$
	and $x=y^{-1}x^{-1}y\in P$, a contradiction.
\end{example}

\begin{theorem}
	\label{thm:BO}
	Let $G$ be a bi-ordered group. Then $K[G]$ is a domain such that
	only has trivial units. Moreover, if $G$ is non-trivial, 
	then $J(K[G])=\{0\}$. 
\end{theorem}

\begin{proof}
	Let $\alpha,\beta\in K[G]$ be such that  
	\begin{align*}
		\alpha&=\sum_{i=1}^m a_ig_i, && g_1<g_2<\cdots<g_m,&& a_i\ne 0 && \forall i\in\{1,\dots,m\},\\
		\beta&=\sum_{j=1}^n b_jh_j, && h_1<h_2<\cdots<h_n, && b_j\ne 0 && \forall j\in\{1,\dots,n\}.
	\end{align*}
	Then 
	\[
		g_1h_1\leq g_ih_j\leq g_mh_n
	\]
	for all $i,j$. Moreover, $g_1h_1=g_ih_j$ if and only if $i=j=1$. The
	coefficient of $g_1h_1$ in $\alpha\beta$ is $a_1b_1\ne 0$. In particular, 
	$\alpha\beta\ne0$. If $\alpha\beta=\beta\alpha=1$, then the coefficient of
	$g_mh_n$ in $\alpha\beta$ is $a_mb_n$. Hence $m=n=1$ and therefore 
	$\alpha=a_1g_1$ and $\beta=b_1h_1$ with $a_1b_1=b_1a_1=1$ in $K$ and $g_1h_1=1$
	in $G$.
\end{proof}

\begin{theorem}[Levi]
	\label{thm:Levi}
	\index{Levi's theorem}
	Let $A$ be an abelian group. Then $A$ is bi-ordered if and only
	if $A$ is torsion-free.
\end{theorem}

\begin{proof}
	If $A$ is bi-ordered, then $A$ is torsion-free. Let us prove the non-trivial implication, 
	so assume that
	$A$ is torsion-free abelian. Let $\mathcal{S}$ be the class 
	of subsets $P$ of $A$ such that $0\in P$, are closed under 
	the addition of
	$A$ and satisfy the following property: if $x\in P$ and $-x\in P$,
	then $x=0$.
	Clearly, $\mathcal{S}\ne\emptyset$, as 
	$\{0\}\in\mathcal{S}$.  The inclusion turns $\mathcal{S}$ into a partially ordered set  
	and $\bigcup_{i\in I}P_i$ is an upper bound for the chain 
	$\{P_i:i\in I\}$. By Zorn's lemma, 
	$\mathcal{S}$ admits a maximal element $P\in\mathcal{S}$.

	\begin{claim}
		If $x\in A$ is such that $kx\in P$ for some $k>0$, then  $x\in P$.		
	\end{claim}

	Let $Q=\{x\in A:kx\in P\text{ for some 
	$k>0$}\}$. We will show that $Q\in\mathcal{S}$.  Clearly, $0\in Q$. Moreover, $Q$
	is closed under addition, as $k_1x_1\in P$ and $k_2x_2\in P$ imply 
	$k_1k_2(x_1+x_2)\in P$. Let $x\in A$ be such that $x\in Q$ and $-x\in Q$. Thus 
	$kx\in P$ and $l(-x)\in P$ for some $l>0$. Since $klx\in P$ and 
	$kl(-x)\in P$, it follows that $klx=0$, a contradiction since $A$ is torsion-free. 
	Hence $x\in Q\subseteq P$. 

	\begin{claim}
		If $x\in A$ is such that $x\not\in P$, then $-x\in P$. 	
	\end{claim}

	Assume that $-x\not\in P$ and let $P_1=\{y+nx:y\in P,\,n\geq0\}$. We will
	show that  $P_1\in\mathcal{S}$.  Clearly, $0\in P_1$ and $P_1$ is closed under
	addition. If $P_1\not\in S$, there exists 
	\[
		0\ne y_1+n_1x=-(y_2+n_2x),
	\]
	where $y_1,y_2\in P$ and $n_1,n_2\geq0$. Thus $y_1+y_2=-(n_1+n_2)x$. If 
	$n_1=n_2=0$, then $y_1=-y_2\in P$ and $y_1=y_2=0$, so it follows that
	$y_1+n_1x=0$, a contradiction. If $n_1+n_2>0$, then, since 
	\[
		(n_1+n_2)(-x)=y_1+y_2\in P,
	\]
	it follows from the first claim that $-x\in P$, a contradiction. 
	Let us show that $P_1\in\mathcal{S}$. 
	Since $P\subseteq P_1$, the maximality of $P$ implies that 
	$x\in P=P_1$.

	\medskip
	By Proposition~\ref{pro:biordenableP2}, 
	$P^*=P\setminus\{0\}$ is the positive cone of a bi-order in $A$. 
	In fact, $P^*$ is closed under addition, as $x,y\in
	P^*$ implies that $x+y\in P$ and $x+y=0$ implies $x=y=0$, as $x=-y\in P$. Moreover,
	$G=P^*\cup -P^*\cup\{0\}$ (disjoint union), as 
	the second claim states that $x\not\in P^*$ implies 
	$-x\in P$. 
\end{proof}

\begin{corollary}
	Let $A$ be a non-trivial torsion-free abelian group. Then $K[A]$ 
	is a domain that only admits trivial units and $J(K[A])=\{0\}$. 
\end{corollary}

\begin{proof}
	Apply Levi's theorem and Theorem~\ref{thm:BO}.
\end{proof}

Some exercises.

\begin{exercise}
    Let $N$ be a central subgroup of $G$. If $N$ and $G/N$ are bi-ordered, 
    then $G$ is bi-ordered. Prove with an example that $N$ needs to be central, normal 
    is not enough. 
\end{exercise}

\begin{exercise}
    Let $G$ be a group that admits 
    a sequence 
    \[
    \{1\}=G_0\subseteq G_1\subseteq\cdots\subseteq G_n=G
    \]
    such that
    each $G_k$ is normal in $G_{k+1}$ and each quotient $G_{k+1}/G_k$ is 
    torsion-free abelian. Prove that $G$ is bi-ordered.  
\end{exercise}

\begin{exercise}
    Prove that torsion-free nilpotent groups are bi-ordered. 
\end{exercise}


\topic{Left-ordered groups}

\begin{definition}
	\index{Grupo!ordenable a derecha}
	Un grupo $G$ se dice \textbf{ordenable a derecha} si existe un orden total
	$<$ en $G$ tal que si $x<y$ entonces $xz<yz$ para todo $x,y,z\in G$.
\end{definition}

%\begin{corollary}
%	Sea $G$ un grupo abeliano. 
%	\begin{enumerate}
%		\item Si $K$ es de característica cero, entonces $J(K[G])=0$.
%		\item Si $K$ es de característica $p$, entonces $J(K[G])=0$ si y sólo si $G$ no tiene elementos de orden $p$.
%	\end{enumerate}
%\end{corollary}
%
%\begin{proof}
%	
%\end{proof}

Si $G$ es un grupo ordenable a derecha, se define el cono positivo de $G$ como
el subconjunto $P(G)=\{x\in G:1<x\}$. 

\begin{exercise}
	\index{Cono positivo!de un grupo ordenable a derecha}
	Sea $G$ un grupo ordenable a derecha con cono positivo $P$. Demuestre las
	siguientes afirmaciones:
	\begin{enumerate}
		\item $P$ es cerrado por multiplicación.
		\item $G=P\cup P^{-1}\cup \{1\}$ (unión disjunta).
	\end{enumerate}
\end{exercise}

\begin{exercise}
	Sea $G$ un grupo y sea $P$ un subconjunto cerrado por multiplicación y tal
	que $G=P\cup P^{-1}\cup \{1\}$ (unión disjunta). Demuestre que si se define $x<y$ si y
	sólo si $yx^{-1}\in P$, entonces $G$ es ordenable a derecha con cono
	positivo $P$.
\end{exercise}

\begin{lemma}
	Sea $G$ un grupo y sea $N$ un subgrupo normal de $G$.  Si $N$ y $G/N$ son
	ordenables a derecha, entonces $G$ también lo es. 
%		\item Si $N$ y $G/N$ son biordenables y $N$ es central, entonces $G$ es
%	biordenable.
%	\end{enumerate}
\end{lemma}
%
\begin{proof}
	Como $N$ y $G/N$ son ordenables a derecha, existen los conos positivos
	$P(N)$ y $P(G/N)$. Sea $\pi\colon G\to G/N$ el morfismo canónico y sea
	\[
		P(G)=\{x\in G:\pi(x)\in P(G/N)\text{ o bien }x\in N\}.
	\]	
	Dejamos como ejercicio demostrar que $P(G)$ es cerrado por la
	multiplicación y que $G=P(G)\cup P(G)^{-1}\cup \{1\}$ (unión disjunta).
	Luego $G$ es ordenable a derecha. 
\end{proof}
%
%\begin{theorem}
%	\label{theorem:}
%	Si $G$ tiene una serie finita subnormal $1=G_0\triangleleft
%	G_1\triangleleft\cdots\triangleleft G_n=G$ y cada cociente $G_{i+1}/G_i$ es
%	abeliano libre de torsión, entonces $G$ es ordenable a derecha. Si además
%	$G$ es libre de torsión y nilpotente, entonces $G$ es biordenable.
%\end{theorem}

Para dar un criterio de ordenabilidad necesitamos un lema:

\begin{lemma}
	\label{lemma:fg}
	Sea $G$ un grupo finitamente generado y sea $H$ un subgrupo de índice
	finito. Entonces $H$ es finitamente generado.
\end{lemma}

\begin{proof}
	Supongamos que $G$ está generado por $\{g_1,\dots,g_m\}$ y supongamos que
	para cada $i$ existe $k$ tal que $g_i^{-1}=g_k$. Sea $t_1,\dots,t_n$ un
	conjunto de representantes de $G/H$. Para $i\in\{1,\dots,n\}$,
	$j\in\{1,\dots,m\}$, escribimos
	\[
		t_ig_j=h(i,j)t_{k(i,j)}.
	\]
	Vamos a demostrar que $H$ está generado por los $h(i,j)$. Sea $x\in H$.
	Escribamos 
	\begin{align*}
	x &=g_{i_1}\cdots g_{i_s}\\
	&= (t_1g_{i_1})g_{i_2}\cdots g_{i_s}\\
	&= h(1,i_1)t_{k_1}g_{i_2}\cdots g_{i_s}\\
	&= h(1,i_1)h(k_1,i_2)t_{k_2}g_{i_3}\cdots g_{i_s}\\
	&= h(1,i_1)h(k_1,i_2)\cdots h(k_{s-1},g_{i_s})t_{k_s},
	\end{align*}
	donde $k_1,\dots,k_{s-1}\in\{1,\dots,n\}$. Como $t_{k_s}\in H$,
	$t_{k_s}=t_1\in H$ y luego $x\in H$.
\end{proof}

El siguiente teorema nos da un criterio de ordenabilidad a derecha:

\begin{theorem}
	Sea $G$ un grupo libre de torsión y finitamente generado. Si $A$ es un
	subgrupo normal abeliano tal que $G/A$ es finito y cíclico, entonces $G$ es
	ordenable a derecha.
\end{theorem}

\begin{proof}
	Primero observemos que si $A$ es trivial, entonces $G$ también es trivial.
Supongamos entonces que $A\ne 1$.  Como $A$ tiene índice finito, es finitamente
generado. Procederemos por inducción en la cantidad de generadores de $A$. Como
$G/A$ es cíclico, existe $x\in G$ tal que $G=\langle A,x\rangle$. Luego
$[x,A]=\langle [x,a]:a\in A\rangle$ es un subgrupo normal de $G$ tal que
$A/C_A(x)\simeq [x,A]$ (pues $a\mapsto [x,a]$ es un morfismo de grupos $A\to A$
con imagen $[x,A]$ y núcleo $C_A(x)$). Si $\pi\colon G\to G/[x,A]$, entonces
$G/[x,A]=\langle \pi(A),\pi(x)\rangle$ y luego $G/[x,A]$ es abeliano pues
$[\pi(x),\pi(A)]=\pi[x,A]=1$. Además $G/[x,A]$ es finitamente generado pues $G$
es finitamente generado. Como $(G:A)=n$ y $G$ no tiene torsión, $1\ne x^n\in
A$.  Luego $x^n\in C_A(x)$ y entonces $1\leq \rank C_A(x)<\rank A$ (si $\rank
C_A(x)=\rank A$, entonces $[x,A]$ sería un subgrupo de torsión de $A$, una
contradicción pues $x\not\in A$).  Luego 
\[
\rank[x,A]=\rank (A/C_A(x))\leq\rank A-1
\]
y entonces $\rank (A/[x,A])\geq 1$. Demostramos así que $A/[x,A]$ es infinito y
luego $G/[x,A]$ es también infinito. 

Como $G/[x,A]$ es un grupo abeliano finitamente generado e infinito, existe un
un subgrupo normal $H$ de $G$ tal que $[x,A]\subseteq H$ y $G/H\simeq\Z$. El
subgrupo $B=A\cap H$ es abeliano, normal en $H$ y cumple que $H/B$ es cíclico
(pues puede identificarse con un subgrupo de $G/A$). Como $\rank B<\rank A$, la
hipótesis inductiva implica que $H$ es ordenable a derecha y luego $G$ también
es ordenable a derecha.
\end{proof}

\begin{exercise}[Malcev--Neumann]
	\index{Teorema!de Malcev--Neumann}
	Sea $G$ un grupo ordenable a derecha. Demuestre que $K[G]$ no tiene divisores de
	cero ni unidades no triviales.	
\end{exercise}

%\begin{proof}
%	Recordemos que si $\alpha=\sum_{i=1}^na_ig_i\in K[G]$ y
%	$\beta=\sum_{j=1}^mb_jh_j\in K[G]$ entonces
%	\begin{equation}
%		\label{eq:producto}
%		\alpha\beta=\sum_{i=1}^n\sum_{j=1}^ma_ib_j(g_ih_j).
%	\end{equation}
%	Sin pérdida de generalidad podemos suponer que los $a_i$ y los $b_j$ son no
%	nulos y que además $g_1<g_2<\cdots<g_n$. Sean $i,j$ tales que 
%	\[
%		g_ih_j=\min\{g_ih_j:1\leq i\leq n,1\leq j\leq m\}.
%	\]
%	Afirmamos que entonces $i=1$ (pues si $i>1$ entonces tendríamos que
%	$g_1h_j<g_ih_j$, una contradicción). Además, como $g_1h_j\ne g_1h_k$ si
%	$k\ne j$, existe un único elemento minimal en el miembro derecho de la
%	ecuación~\eqref{eq:producto}. El mismo argumento muestra que existe un
%	único elemento maximal en la ecuación~\eqref{eq:producto}. Luego
%	$\alpha\beta\ne 0$ (pues $a_1b_j\ne 0$) y entonces $K[G]$ no tiene
%	divisores de cero. Si además $n>1$ o $m>1$ entonces en la
%	expresión~\eqref{eq:producto} hay al menos dos términos que no se cancelan
%	y luego $\alpha\beta\ne1$. Luego las unidades de $K[G]$ son triviales.
%\end{proof}

En 1973 Formanek demostró que la conjetura de los divisores de cero es
verdadera para grupos super resolubles sin torsión. En 1976 Brown e independientemente
Farkas y Snider demostraron que la conjetura es verdadera para grupos policíclicos-por-finitos sin torsión.

