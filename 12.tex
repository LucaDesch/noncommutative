\section{}

\subsection{Miller's double cosets theorem}

\index{Double coset}
Let $G$ be a group and $H$ and $K$ be subgroups of $G$. 
The group $L=H\times K$ acts on $G$ by
\[
(h,k)\cdot g=hgk^{-1},\quad h\in H,k\in K,g\in G.
\]
The orbits of this action are the set of the form 
\[
HgK=\{hgk:h\in H,\,k\in K\}.
\]
A set of the form $HgK$ for some $g\in G$ is called a \textbf{double coset} modulo $(H,K)$ 
with representative $g$. In particular, 
any two double cosets are either disjoint or equal, and $G$ decomposes
as a disjoint union 
\[
G=\bigcup_{i\in I}Hg_iK,
\]
for some set $I$. Let 
\[
L_g=\{(h,k)\in H\times K:hgk^{-1}=g\}=\{(h,g^{-1}hg)\in H\times K\}.
\]
Then
$|L_g|=|H\cap gKg^{-1}|$, 
because there is a bijection $L_g\to H\cap gKg^{-1}$.  
By the fundamental counting principle, 
\[
|HgK|=(L:L_g)=\frac{|H\times K|}{|H\cap gKg^{-1}|}=\frac{|H||K|}{|H\cap gKg^{-1}|}.
\]

We need a lemma. 

\begin{lemma}
\label{lem:Miller}
    Let $G$ be a finite group, $x\in G$, and $H$ and $K$ be subgroups of $G$. Then
    \[
    \#\{zK:zK\subseteq HxK\}=(H:xKx^{-1}\cap H).
    \]
\end{lemma}

\begin{proof}
    Let $L=xKx^{-1}\cap H$ and 
    \[
    \varphi\colon H/L\to\{yK:yK\subseteq H\times K\},\quad 
    hL\mapsto hxK.
    \]

    The map $\varphi$ is well-defined. If $hL=h_1L$, then $h^{-1}h_1\in L$. Thus 
    $h^{-1}h_1=xkx^{-1}$ for some $k\in K$. This means that
    \[
    (h_1x)^{-1}(hx)=x^{-1}h_1^{-1}hx=k\in K,
    \]
    that is $\varphi(hL)=hxK=h_1xK=\varphi(h_1L)$. 

    The map $\varphi$ is surjective: If $zK$ is such that $zK\subseteq HxK$, then 
    $z=hxk$ for some $k\in K$. In particular, 
    $zK=hxK$. Now $\varphi(hL)=hxK=zK$.

    The map $\varphi$ is injective: If $hxK=h_1xK$, then 
    $x^{-1}h_1^{-1}hx\in K$. Moreover, 
    $h_1^{-1}h\in xKx^{-1}\cap H=L$. Thus $h_1L=hL$. 
\end{proof}

\begin{exercise}
\label{xca:Miller}
    Let $G$ be a finite group, $H$ and $K$ be subgroups of $G$, and $x\in G$. Prove 
    that 
    \[
    \#\{Hy:Hy\subseteq HxK\}=(K:xHx^{-1}\cap K).
    \]
\end{exercise}

\begin{theorem}[Miller]
\index{Miller' theorem}
    Let $G$ be a finite group and $H$ and $K$ be subgroups of $G$ 
    of the same index. Then there exists a common complete set
    of representatives for the right cosets of $H$ in $G$ and the 
    left cosets of $K$ in $G$. 
\end{theorem}

\begin{proof}
    Let $Hy$ be a right coset and $zK$ be a left coset. Note that 
    $Hy$ and $zK$ have a common representative
    if and only if $Hy\cap zK\ne\emptyset$, as 
    \[
    Hy=Hx\text{ and }zK=xK
    \Longleftrightarrow 
    xy^{-1}\in H\text{ and }z^{-1}x\in K
    \Longleftrightarrow x\in Hy\cap zK.
    \]

    The group $G$ decomposes as a  
    disjoint union of finitely many double cosets. Each doble coset
    $HxK$ is a disjoint union of finitely many right cosets of $H$ 
    and a disjoint union of finitely many left cosets of $K$. Thus 
    \[
    HxK=\bigcup_{i=1}^kHy_i=\bigcup_{j=1}^lz_jK, 
    \]
    where the unions are disjoint. 
    Since $|H|=|K|$, by applying cardinality, it follows that $k=l$. To prove the theorem
    it is enough to show that each $Hy_i$ intersects every $z_jK$. 
    
    Note that for each $i\in\{1,\dots,k\}$ there exists $j\in\{1,\dots,k\}$ such that
    $Hy_i\cap z_jK\ne\emptyset$. 
    Without loss of generality, we may assume (reordering if needed) that 
    $Hy_1\cap z_jK\ne\emptyset$ for all $j\in\{1,\dots,m\}$, where $1\leq m\leq k$. Then
    \[
    Hy_1\subseteq\bigcup_{j=1}^mz_jK. 
    \]
    Then
    \[
    Hy_1K\subseteq\bigcup_{j=1}^mz_jK\subseteq \bigcup_{j=1}^kz_jK=HxK.
    \]
    Since $Hy_1K$ and $HxK$ are double cosets with non-empty intersection, 
    they are equal. Thus 
    \[
    |HxK|=|Hy_1K|\leq \sum_{j=1}^m|z_jK|=m|K|.
    \]
    By Lemma~\ref{lem:Miller}, 
    \[
    k=\#\{z_jK:z_jK\subseteq HxK\}=(H:xKx^{-1}\cap H). 
    \]
    Therefore
    \[
    k|K|=\frac{|H||K|}{|H\cap xKx^{-1}|}=|HxK|\leq m|K|
    \]
    and hence $k=m$. 
\end{proof}

\begin{exercise}[Hall]
\label{xca:Hall:cosets}
    Let $G$ be a finite group and $H$ be a subgroup of $G$ with $(G:H)=n$. 
    Then there exists $x_1,\dots,x_n\in G$ such that 
    $\{Hx_1,Hx_2,\dots,Hx_n\}=\{x_1H,x_2H,\dots,Hx_n\}$. 
\end{exercise}

