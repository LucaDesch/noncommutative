\chapter{}

\topic{Radical rings and solutions}

Let $S$ be a non-unitary ring. Consider $S_1=\Z\times S$ with the addition defined component-wise and  multiplication
\[
(k,a)(l,b)=(kl,kb+la+ab)
\]
for all $k,l\in\Z$ and $a,b\in S$. 
Then $S_1$ is a ring and $(1,0)$ is its unit element. 
Furthermore, $\{0\}\times S$ is an ideal of $S_1$. 
Note that $\{0\}\times S\simeq S$ as non-unitary rings. Also 
note that if $(k,x)\in S_1$ is invertible, 
then $k\in\{-1,1\}$. 

\begin{definition}
    A non-unitary ring $S$ is a (Jacobson) \textbf{radical ring} 
    if it is isomorphic to the Jacobson radical of a unitary ring.
\end{definition}

Let $R$ be a ring. The (Jacobson) \textbf{radical} $J(R)$ of $R$ is defined as the intersection
of all maximal left ideals of $R$. One proves that $J(R)$ is an ideal of $R$. Moreover, 
$x\in J(R)$ if and only if $1+rx$ is invertible for all $r\in R$.

\begin{proposition}
\label{pro:radical}
	Let $S$ be a non-unitary ring. The following statements are equivalent.
	\begin{enumerate}
		\item $S$ is a radical ring.
		\item For all $a\in S$ there exists a unique $b\in S$ such that $a+b+ab=a+b+ba=0$.
		\item $S\simeq J(S_1)$. 
	\end{enumerate}
\end{proposition}  
	
\begin{proof}
    Let us first prove that $1)\implies2)$. Let $R$ be a unitary ring such that 
    $S\simeq J(R)$ and let $\psi\colon S\rightarrow R$ be an injective homomorphism 
    of non-unitary rings $\psi(S)=J(R)$. Let $a\in S$. Since  
    $1+\psi(a)\in R$ is invertible, there exists $c\in R$ such that 
    \[
    (1+\psi(a))(1+c)=(1+c)(1+\psi(a))=1.
    \]
    Thus 
    $c=-\psi(a)c-\psi(a)\in J(R)$. 
    Hence there exists $b\in S$ such that $\psi(b)=c$. Therefore 
    \[
    a+b+ab=a+b+ba=0.
    \]
    It is an exercise to prove that $b$ is unique. 
    
    We now prove that $2)\implies 3)$. We first note that if 
    $a\in S$, then there exists $b\in S$ such that $a+b+ab=a+b+ba=0$. 
    Thus every 
    $(1,a)\in S_1$ is invertible, as 
    \[
    (1,a)(1,b)=(1,0)=(1,b)(1,a).
    \]

    We claim that $J(S_1)=\{0\}\times S$. Let us prove that 
    $J(S_1)\supseteq \{0\}\times S$. If $(k,a)\in J(S_1)$, then, in particular, 
    \[
    (1+3k,3a)=(1,0)+(3,0)(k,a)
    \]
    is invertible, which implies that either $1+3k=1$ or $1+3k=-1$. Since
    $k\in\Z$, it follows that $k=0$ and hence $(k,a)=(0,a)\in\{0\}\times S$. 
    To prove that 
    $J(S_1)\supseteq \{0\}\times S$ note that
    if $(0,x)\in\{0\}\times S$, then
    \[
    (1,0)+(k,a)(0,x)=(1,kx+ax)
    \]
    is invertible, as $kx+ax\in S$. 
    
    The implication $3)\implies1)$ is trivial.
\end{proof}

A \textbf{nil ring} is a non-unitary ring $S$ such that every 
element of $S$ is nilpotent. Every nil ring is a radical ring.

\begin{example} 
    $X\C[\![X]\!]$ is a radical ring and it is not a nil ring.
\end{example}

Let $S$ be a ring (unitary or non-unitary, it is not important here). 
Define on $S$ the binary operation 
\[
(a,b)\mapsto a\circ b=a+b+ab
\]
for all $a,b\in S$. Then $(S,\circ)$ is a monoid with neutral element $0$.
Note that $S$ is a radical ring if and only if $(S,\circ)$ is a group. 
If $a\in S$ is invertible in the monoid $(S,\circ)$, we will denote by $a'$ its inverse.

\begin{example}
	For $n>1$ let $A=\left\{\frac{nx}{ny+1}:x,y\in\Z\right\}\subseteq \Q$. 
	Note that $A$ is a (non-unitary) subring of $\Q$. In fact, $A$ is a radical ring. A straightforward computation shows that 
	\[
	\left(\frac{nx}{ny+1}\right)'=\frac{-nx}{n(x+y)+1}.
	\]
\end{example}

We now go back to study solutions to the YBE and discuss the intriguing interplay
between radical rings and involutive solutions. 

\begin{definition}
	\index{Solution!involutive}
	A solution $(X,r)$ is said to be \emph{involutive} if $r^2=\id$. 
\end{definition}

Note that if $(X,r)$ is an  involutive solution, then 
\[
(x,y)=r^2(x,y)=r(\sigma_x(y),\tau_y(x))=(\sigma_{\sigma_x(y)}\tau_y(x),\tau_{\tau_y(x)}\sigma_x(y)).
\]
Hence 
\begin{equation}
	\label{eq:involutive}
	\tau_y(x)=\sigma_{\sigma_x(y)}^{-1}(x),
	\quad
	\sigma_x(y)=\tau_{\tau_y(x)}^{-1}(y)
\end{equation}
for all $x,y\in X$. Thus for involutive solutions
it is enough to know $\{\sigma_x:x\in X\}$, as from this we obtain the
set $\{\tau_x:x\in X\}$.

\begin{example}
	Let $X$ be a non-empty set and $\sigma$ be a bijection on $X$. Then 
	$(X,r)$, where $r(x,y)=(\sigma(y),\sigma^{-1}(x))$, is an involutive solution. 
\end{example}

\index{Jacobson!radical ring}
\index{Radical ring}
We now present a very important family of involutive solutions. 

\begin{theorem}[Rump]
	\label{thm:Rump}
	\index{Rump's theorem}
	Let $R$ be a radical ring. Then $(R,r)$, where 
	\[
	r(x,y)=( -x+x\circ y,(-x+x\circ y)'\circ x\circ y)
	\]
	is an involutive solution.
\end{theorem}

The proposition can demonstrated using Theorem~\ref{thm:LYZ}. We will
prove a more general result later. 

\topic{Skew braces}

By convention, an additive group $A$ will be a (not necessarily abelian) group 
with a binary operation $(a,b)\mapsto a+b$. The 
identity of $A$ will be denoted by $0$ 
and the inverse of an element $a$ will be denoted by $-a$. 

\begin{definition}
	\index{Skew brace}
	\index{Skew brace!multiplicative group}
	\index{Skew brace!additive group}
	A \emph{skew left brace} is a triple $(A,+,\circ)$, where $(A,+)$ and $(A,\circ)$ 
	are (not necessarily abelian) 
	groups and 
	\begin{equation}
	    \label{eq:compatibility}
	    a\circ(b+c)=(a\circ b)-a+(a\circ c)
	\end{equation}
	holds for all $a,b,c\in A$. The groups 
	$(A,+)$ and $(A,\circ)$ are respectively 
	the \emph{additive} and \emph{multiplicative} group
	of the skew left brace $A$.
\end{definition}

We write $a'$ to denote the inverse of $a$ with respect to the circle operation $\circ$. 

Skew right braces are defined similarly, one needs 
to replace~\eqref{eq:compatibility} by 
\[
(a+b)\circ c=a\circ c-c+b\circ c.
\]

\begin{exercise}
Prove that there is a bijective correspondence between 
skew left braces and skew right braces. 
\end{exercise}

A skew brace will always mean a skew left brace. 

\begin{definition}
    Let $\mathcal{X}$ be a family of groups. A skew brace $A$ is said to be
    of $\mathcal{X}$-type if its additive group belongs to $\mathcal{X}$.
\end{definition}

One particularly interesting family of skew braces is the family of \emph{skew braces of abelian type}, 
that is skew braces with abelian additive group. 
In the literature, skew braces of abelian type are called \emph{braces}. 

\begin{example}
	\label{exa:trivial}
	\index{Skew brace!trivial}
	Let $A$ be an additive group. Then $A$ is a skew brace with
	$a\circ b=a+b$ for all $a,b\in A$. 
	A skew brace $(A,+,\circ)$ such that $a\circ b=a+b$ for all $a,b\in A$ is
    said to be \emph{trivial}. 
	Similarly, the
   operation $a\circ b=b+a$ turns $A$ into a skew brace. 
\end{example}

\begin{example}
	\label{exa:times}
	\index{Direct product!of skew braces}
	Let $A$ and $B$ be skew braces. Then $A\times B$ with 
	\[
		(a,b)+(a_1,b_1)=(a+a_1,b+b_1),\quad
		(a,b)\circ (a_1,b_1)=(a\circ a_1,b\circ b_1),
	\]
	is a skew brace. This is the {\em direct product} of the skew braces $A$ and $B$. 
\end{example}

\begin{example}
	\label{exa:sd}
	Let $A$ and $M$ be additive groups and let $\alpha\colon A\to\Aut(M)$ be a
	group homomorphism. Then $M\times A$ with 
	\[
	(x,a)+(y,b)=(x+y,a+b),
	\quad
	(x,a)\circ (y,b)=(x+\alpha_a(y),a+b)
	\]
	is a skew brace. Similarly, $M\times A$ with
	\[
	(x,a)+(y,b)=(x+\alpha_a(y),a+b),\quad
	(x,a)\circ (y,b)=(x+y,b+a)
	\]
	is a skew brace. 
\end{example}

\begin{example}
    \label{exa:WX}
    Let $A$ be an additive group
	and $B$ and $C$ be subgroups of $A$ such that $B\cap C=\{ 0\}$ and $A=B+C$. In this case, one says that $A$ admits an {\em exact factorization} through the subgroups $B$ and $C$.  Thus each $a\in A$ can be written in a unique
	way as $a=b+c$, for some $b\in B$ and $c\in C$.  The map
	\[
		B\times C\to A,\quad
		(b,c)\mapsto b-c,
	\]
	is bijective. Using this map we transport the group structure of the direct
	product $B\times C$ into the set $A$. That is, for $a=b+c\in A$, where $b\in B$ and $c\in C$, and
	$a_1\in A$, let 
	\begin{align*}
		a\circ a_1&=b+a_1+c.
	\end{align*}
	Then $(A,\circ)$ is a group isomorphic to $B\times C$. Moreover, if $x,y\in A$, 
	then 
	\begin{align*}
	a\circ x-a+a\circ y=b+x+c-(b+c)+b+y+c=b+x+y+c=a\circ (x+y),
	\end{align*}
	and therefore $(A,+,\circ)$ is a skew brace. 
\end{example}

% \begin{proof} The map $\eta\colon B\times C\to A$, $\eta(b,c)=bc^{-1}$, is
%   bijective.  Since $\eta$ is bijective and $a\circ
%   a'=\eta(\eta^{-1}(a)\eta^{-1}(a'))$, it follows that $(A,\circ)$ is a group
%   isomorphic to the direct product $B\times C$. To prove that $A$ is a skew
%   brace it remains to show~\eqref{eq:compatibility}. Let $a=bc\in BC$ and
%   $a',a''\in A$. Then \begin{align*} (a\circ a')a^{-1}(a\circ a'')
%     &=(ba'c)a^{-1}(ba''c)\\ &=ba'c(c^{-1}b^{-1})ba''c\\ &=ba'a''c\\ &=a\circ
%     (a'a'').  \end{align*} This completes the proof.  \end{proof}

We now give concrete some examples of the previous construction. 

\begin{example}
  \label{exa:QR}
  Let $n$ be a positive integer. 
  The group $\GL_n(\C)$ admits an
  exact factorization through the subgroups $U(n)$ and $T(n)$, where 
  \[
  U(n)=\{ A\in\GL_n(\C): AA^*=I\}
  \]
  is the unitary group and $T(n)$ is the group of upper triangular matrices
  with positive diagonal entries.  Therefore there exists a skew brace with additive group 
  isomorphic to $\GL_n(\C)$ and multiplicative group isomorphic to $U(n)\times T(n)$.  
\end{example}

The following examples appeared in the theory of 
Hopf--Galois structures.

\begin{example} 
	\label{exa:a5a4c5}
	The alternating simple group $\Alt_5$ admits an exact factorization
  through the subgroups 
  $A=\langle (123),(12)(34)\rangle\cong\Alt_4$ and 
  $B=\langle(12345)\rangle\cong C_5$.  
  There exists a skew brace with additive group isomorphic to $\Alt_5$ and multiplicative
  group isomorphic to $\Alt_4\times C_5$. 
\end{example}

Let us review some basic properties of skew braces. 

\begin{exercise}
\label{xca:0=1}
Let $A$ be a skew brace. Then the following properties hold:
\begin{enumerate}
    \item The neutral element of the additive group of $A$ coincides with 
    the neutral element of the multiplicative group of $A$. It will be denoted
    by $0$. 
    \item $a\circ(-b+c)=a-(a\circ b)+(a\circ c)$, for all $a,b,c\in A$.
    \item $a\circ(b-c)=(a\circ b)-(a\circ c)+a$, for all $a,b,c\in A$.
\end{enumerate}
\end{exercise}

\begin{exercise}
\label{xca:lambda}
    Let $A$ be a skew brace. For each $a\in A$, the map
    \[
        \lambda_a\colon A\to A,\quad
        b\mapsto -a+(a\circ b),
    \]
    is an automorphism of $(A,+)$. Moreover, the map 
    $\lambda\colon (A,\circ)\to\Aut(A,+)$, $a\mapsto\lambda_a$, is a group homomorphism. 
\end{exercise}

\begin{exercise}
\label{xca:mu}
    Let $A$ be a skew brace. For each $a\in A$, the map
    \[
        \mu_a\colon A\to A,\quad
        b\mapsto \lambda_b(a)'\circ b\circ a,
    \]
    is bijective. Moreover, the map 
    $\mu\colon (A,\circ)\to\Sym_A$, $a\mapsto\mu_a$, satisfies $\mu_b\circ\mu_a=\mu_{a\circ b}$, for all $a,b\in A$. 
\end{exercise}

Let $A$ be a skew brace. 
Exercise \ref{xca:lambda} implies that 
\begin{align}
\label{eq:formulas}
&a\circ b = a+\lambda_a(b),
&&a+b=a\circ \lambda^{-1}_a(b),
&&\lambda_a(a')=-a
\end{align}
hold for $a,b\in A$. Moreover, if 
\[
    a*b=\lambda_a(b)-b=-a+a\circ b-b,
\]
then the following identities are easily verified:
\begin{align}
&a*(b+c)=a*b+b+a*c-b,\\
&(a\circ b)*c=(a*(b*c))+b*c+a*c.
\end{align}

 \begin{definition}
 	\index{Homomorphism!of skew braces}
 	A map $f\colon A\to B$ between two skew braces $A$ and $B$ is a {\em homomorphism of skew braces} 
 	if $f(x\circ y)=f(x)\circ f(y)$ and $f(x+y)=f(x)+f(y)$ for all $x,y\in A$.  The \emph{kernel} of $f$ is
     \[
         \ker f=\{a\in A:f(a)=0\}.
     \]
 \end{definition}

A bijective homomorphism of skew braces is an isomorphism. An automorphism of a skew brace $A$ is an isomorphism from the skew brace $A$ to it self. Two skew braces $A$ and $B$ are isomorphic if there exist an isomorphism $f\colon A\rightarrow B$. We write $A\simeq B$ to denote that the skew braces $A$ and $B$ are isomorphic.

\begin{definition}
    \index{Skew brace!two sided}
	A skew brace $A$ is said to be \textbf{two-sided} if 
	\begin{equation}
	\label{eq:right_compatibility}
	(a+b)\circ c=a\circ c-c+b\circ c
	\end{equation}
	holds for all $a,b,c\in A$. 
\end{definition}

If $A$ is a skew two-sided brace, then 
\begin{align}
\label{eq:2sided}
&a\circ(-b)=a-a\circ b+a,
&&(-a)\circ b=b-a\circ b+b    
\end{align}
hold for all $a,b\in A$. The first equality holds for every skew brace and follows 
from the compatibility condition. 
The second equality follows from~\eqref{eq:right_compatibility}. 

\begin{example}
  Any skew brace with abelian multiplicative group is 
  two-sided.
\end{example}

\begin{exercise}
\label{xca:2sided}
	Let $A$ be a skew brace of abelian type such that $\lambda_a(a)=a$ for all $a\in A$.
	Prove that $A$ is two-sided.
\end{exercise}

\index{Jacobson!radical ring}
\index{Radical ring}
Two-sided skew braces of abelian type form an interesting family of non-unitary rings.
Thus skew braces form a far reaching generalization of radical rings. 

\begin{theorem}[Rump]
\label{thm:radical}
\index{Rump's theorem}
    A skew brace of abelian type is two-sided if and only if it is a radical ring. 
\end{theorem}

\begin{proof}
    Assume first that $A$ is a skew two-sided brace of abelian type. Then $(A,+)$ is an abelian group. 
    Let us prove that the operation
    \[
    a*b=-a+a\circ b-b
    \]
    turns $A$ into a radical ring. Left distributivity follows from the compatibility condition:
    \begin{align*}
    a*(b+c)&=-a+a\circ (b+c)-(b+c)
    =-a+a\circ b-a+a\circ c-c-b=a*b+a*c.
    \end{align*}
    Similarly, since $A$ is two-sided, one proves $(a+b)*c=a*c+b*c$. It remains to show that the operation $*$
    is associative. On the one hand, using the first equality of~\eqref{eq:2sided} 
    and the compatibility condition, we write
    \begin{align*}
    a*(b*c)&=a*(-b+b\circ c-c)\\
    &=-a+a\circ(-b+b\circ c-c)-(-b+b\circ c-c)\\
    &=-a+a\circ (-b)-a+a\circ(b\circ c)-a+a\circ (-c)+c-b\circ c+b\\
    &=a\circ (b\circ c)-a\circ b-a\circ c-b\circ c+a+b+c,
    \end{align*}
    since the group $(A,+)$ is abelian. On the other hand, the second equality of~\eqref{eq:2sided} and
    Equality~\eqref{eq:right_compatibility} imply that
    \begin{align*}
    (a*b)*c &= (-a+a\circ b-b)*c=-(-a+a\circ b-b)+(-a+a\circ b-b)\circ c-c\\
    &=b-a\circ b+a+(-a)\circ c-c+(a\circ b)\circ c-c+(-b)\circ c-c\\
    &=(a\circ b)\circ c-a\circ b-a\circ c-b\circ c+a+b+c.
    \end{align*}
    It then follows that the operation $*$ is associative. 
    
    Conversely, if $A$ is a radical ring, say with ring multiplication $(a,b)\mapsto ab$, 
    then $a\circ b=a+ab+b$ turns $A$ into a skew two-sided brace 
    of abelian type. In fact, since $A$ is a radical ring, then 
    $(A,+)$ is an abelian group and $(A,\circ)$ is a group. Moreover, 
    \begin{align*}
        a\circ (b+c)=a+a(b+c)+(b+c)=a+ab+ac+b+c=a\circ b-a+a\circ c.
    \end{align*}
    Similarly ones proves $(a+b)\circ c=a\circ c-c+b\circ c$.
\end{proof}

A natural question arises: Does one need radical rings? Surprisingly, 
radical rings are just the tip of the iceberg. 

\begin{theorem}
\label{thm:YB}
Let $A$ be a skew brace. Then 
$(A,r)$, where 
\[
r\colon A\times A\to A\times A,\quad
r(x,y)=( -x+x\circ y,(-x+x\circ y)'\circ x\circ y),
\]
is a solution to the YBE. 
\end{theorem}

\begin{proof}
    By Theorem~\ref{thm:LYZ}, 
    since $x\circ y=(-x+x\circ y)\circ ((-x+x\circ y)'\circ x\circ y)$ for all $x,y\in A$, 
    we only need to check that 
    $x\rhd y=\lambda_x(y)=-x+x\circ y$ 
    is a left action of $(A,\circ)$ on the set $A$ 
    and that $x\lhd y=\mu_y(x)=(-x+x\circ y)'\circ x\circ y$ 
    is a right action of $(A,\circ)$ on the set $A$. For the left action we use 
    Exercise~\ref{xca:lambda} and for the right action we use Exercise~\ref{xca:mu}.
\end{proof}

\begin{exercise}
Let $A$ be a skew brace. 
Prove that 
\[
\mu_b(a)=\lambda^{-1}_{\lambda_a(b)}(-a\circ b+a+a\circ b).
\]
\end{exercise}

In Theorem~\ref{thm:YB} it is possible to prove that the solution 
is involutive if and only if the additive group of the brace is abelian. We will
prove a generalization of this result. For that purpose, we need a lemma. 

\begin{lemma}
\label{lem:|r|}
Let $A$ be a skew brace and $r$ be its associated solution.  Then
  \begin{align} 
  \nonumber
  r^{2n}(a,b)&=(-n(a\circ b)+a+n(a\circ
    b),\\
    \label{eq:r^2n}
    &\phantom{=(-n(a\circ b)+}(-n(a\circ b)+a+n(a\circ b))'\circ a\circ b),\\
  \nonumber
  r^{2n+1}(a,b)&=(-n(a\circ b)-a+(n+1)(a\circ
    b),\\
    \label{eq:r^2n+1}
    &\phantom{=(-n(a\circ b)+}(-n(a\circ b)-a+(n+1)(a\circ b))'\circ a\circ b),
    \end{align} 
    for all $n\geq0$.  Moreover, the following statements hold:
  \begin{enumerate} 
  \item $r^{2n}=\id$ if and only if $a+nb=nb+a$ for all $a,b\in A$.  
      \item $r^{2n+1}=\id$ if and only if $\lambda_a(b)=n(a\circ
	b)+a-n(a\circ b)$ for all $a,b\in A$.  
	\end{enumerate} 
\end{lemma}

\begin{proof} 
First we shall prove~\eqref{eq:r^2n} and~\eqref{eq:r^2n+1} by induction on $n$. The case $n=0$ is trivial for~\eqref{eq:r^2n}
  and~\eqref{eq:r^2n+1}. Assume that the claim holds for some $n\geq 0$. By applying the map $r$ to Equation~\eqref{eq:r^2n+1} 
  we obtain that 
  \begin{align*} 
  r^{2(n+1)}(a,b) &=r\left( -n(a\circ b)-a+(n+1)(a\circ b),\right.\\
    &\phantom{=(-n(a\circ b)+} \left. (-n(a\circ b)-a+(n+1)(a\circ
    b))'\circ a\circ b\right)\\
    &=\left( -(n+1)(a\circ b)+a+(n+1)(a\circ b),\right.\\
    &\phantom{=(-n(a\circ b)+} \left. (-(n+1)(a\circ b)+a+(n+1)(a\circ
    b))'\circ a\circ b\right).
    \end{align*} 
    By applying $r$ again to this equality, we get 
    \begin{align*} 
  r^{2(n+1)+1}(a,b) &= r\left(-(n+1)(a\circ b)+a+(n+1)(a\circ
    b),\right.\\
    &\phantom{=(-n(a\circ b)+} \left. (-(n+1)(a\circ b)+a+(n+1)(a\circ b))'\circ a\circ b\right)\\
    &=\left( -(n+1)(a\circ b)-a+(n+2)(a\circ b),\right.\\
    &\phantom{=(-n(a\circ b)+} \left. (-(n+1)(a\circ b)-a+(n+2)(a\circ
    b))'\circ a\circ b\right).
    \end{align*} 
   Thus Equations~\eqref{eq:r^2n} and~\eqref{eq:r^2n+1} hold by induction.  The other claims follow easily from
    Equations~\eqref{eq:r^2n} and~\eqref{eq:r^2n+1}.
\end{proof}

%\begin{thm} \label{pro:depth_even} Let $A$ be a skew brace of finite depth
%with more than one element and let $r_A$ be its associated solution. Then the
%order of $r_A$ is an even number.  \end{thm}
%
%\begin{proof} Let $n$ be such that $r^{2n+1}=\id$. By applying
%Lemma~\ref{lem:depth} one obtains that $a^{-1}(a\circ b)^{n+1}=(a\circ b)^na$
%for all $a,b\in A$. In particular, if $b=1$, then $a=1$.  \end{proof}

Recall that the (minimal) \emph{exponent} $\exp(G)$ of a 
finite group $G$ is the least positive integer $n$ such that 
$g^n=1$ for all $g\in G$. 

\begin{theorem} 
\label{thm:|r|} 
  Let $A$ be a finite skew brace with more than one
  element and let $K$ be the additive group of $A$. 
  If $r$ is the solution associated with $A$, 
  then, as a permutation, $r$ has order $2\exp(K/Z(K))$.
\end{theorem}

\begin{proof} 
  Suppose that $r$ has odd order $2n+1$. Since $r^{2n+1}=\id$, 
  Lemma~\ref{lem:|r|} implies that $-a+(n+1)(a\circ b)=n(a\circ b)+a$
  for all $a,b\in A$. In particular, for $b=0$, we get $a=0$, 
  for all $a\in A$, a contradiction. 
  Therefore we may assume that the order of the permutation $r$ is
  $2n$, where 
  \[
  n=\min\{k\in\Z: k>0\text{ and }kb+a=a+kb\;\text{ for all }a,b\in A\}.
  \]
  Now one computes
  \begin{align*} 
  n&=\min\{k\in\Z: k>0\text{ and }kb\in Z(G)\text{ for all }b\in A\}\\ 
  &=\min\{k\in\Z: k>0\text{ and }k(b+Z(G)) = Z(G)\text{ for all }b\in A\} =\exp(G/Z(G)).\qedhere
  \end{align*}
\end{proof}

An inmmediate consequence is the following result.

\begin{corollary}
    Let $A$ be a finite skew brace and $r$ be its associated solution. Then 
    $r$ is involutive if and only if $A$ is of abelian type.
    \end{corollary}

% \begin{proof}
%     It follows immediately from Theorem~\ref{thm:|r|}.
% \end{proof}

%\begin{exa} A skew brace has depth one if and only if its additive group is
%  abelian.  \end{exa}

% \begin{exa} 
%   \label{exa:2p} 
%   Let $p$ be an odd prime number and let $A$ be a non-classical skew brace of
%   size $2p$. Then the additive group of $A$ is isomorphic to the dihedral group
%   $\D_{2p}$ of size $2p$.  Since $Z(\D_{2p})=1$ and the exponent of $\D_{2p}$
%   is $2p$, the order of $r_A$ is $4p$.
% \end{exa}

\topic{Ideals}

\begin{definition}
\index{Subbrace}
Let $A$ be a skew brace. A \emph{subbrace} of $A$ is a non-empty 
subset $B$ of $A$ such that $(B,+)$ is a subgroup of $(A,+)$ and $(B,\circ)$ is a subgroup of $(A,\circ)$. 
\end{definition}

\begin{definition}
    \index{Left!ideal}
    \index{Strong!left ideal}
    Let $A$ be a skew brace. A \emph{left ideal} of $A$ is a subgroup $(I,+)$ of
	$(A,+)$ such that $\lambda_a(I)\subseteq I$ for all $a\in A$, i.e. $\lambda_a(x)\in I$ for all $a\in A$ and $x\in I$. A \emph{strong left ideal} of $A$ 
	is a left ideal $I$ of $A$ such that $(I,+)$ is a normal subgroup of $(A,+)$. 
\end{definition}

\begin{example}
    Let $A$ be a skew brace and $I$ be a characteristic subgroup 
    of the additive group of $A$. Then 
    $I$ is a left ideal of $A$. 
\end{example}

Recall that skew two-sided braces of abelian type 
are equivalent to radical rings. 
One can prove that under this equivalence, 
(left) ideals of the radical ring correspond 
to (left) ideals of the associated brace. 

\begin{proposition}
    A left ideal $I$ of a skew brace $A$ is a subbrace of $A$. 
\end{proposition}

\begin{proof}
    We need to prove that $(I,\circ)$ is a subgroup of $(A,\circ)$. Clearly $I$ is non-empty, 
    as it is an additive subgroup of $A$. If $x,y\in I$, then
    $x\circ y=x+\lambda_x(y)\in I+I\subseteq I$ and $x'=-\lambda_{x'}(x)\in I$. 
\end{proof}

\begin{example}
    Let $A$ be a skew brace. Then 
    \[
    \Fix(A)=\{a\in A:\lambda_x(a)=a\text{ for all $x\in A$}\}
    \]
    is a left ideal of $A$. 
\end{example}

\begin{definition}
    \index{Ideal}
    An \emph{ideal} of a skew brace $A$ is a strong left ideal $I$ of $A$ such that 
	$(I,\circ)$ is a normal subgroup of $(A,\circ)$. 
\end{definition}

In general 
\[
\{\text{subbraces}\}\supsetneq \{\text{left ideals}\}\supsetneq\{\text{strong left ideals}\}\supsetneq\{\text{ideals}\}.
\]
For example, $\Fix(A)$ is not a strong left ideal of $A$.

\begin{example}
    Consider the semidirect product $A=\Z/(3)\rtimes \Z/(2)$ of the
    trivial skew braces $\Z/(3)$ and $\Z/(2)$
    via the non-trivial action of $\Z/(2)$ over $\Z/(3)$.
    Then 
    \[
    \lambda_{(x,y)}(a,b)=-(x,y)+(x,y)\circ(a,b)=-(x,y)+(x+(-1)^ya,y+b)=((-1)^ya,b).
    \]
    Then $\Fix(A)=\{(0,0),(0,1)\}$ is not a 
    normal subgroup of $(A,+)$ and hence $\Fix(A)$ is not a strong left 
    ideal of $A$.
\end{example}

\begin{example}
    \index{Kernel}
	Let $f\colon A\to B$ be a homomorphism of skew braces. Then $\ker f$ 
	is an ideal of $A$.
\end{example}

If $X$ and $Y$ are subsets of a brace $A$, $X*Y$ is defined as the 
subgroup of $(A,+)$ generated by elements of the form $x*y$, $x\in X$ and $y\in Y$, i.e.
\[
X*Y=\langle x*y:x\in X\,,y\in Y\rangle_+.
\]

\begin{proposition}
    \label{pro:A*I}
    Let $A$ be a skew brace. A subgroup $I$ of $(A,+)$ is 
    a left ideal of $A$ if and only if $A*I\subseteq I$.
\end{proposition}

\begin{proof}
    Let $a\in A$ and $x\in I$. If $I$ is a
    left ideal, then $a*x=\lambda_a(x)-x\in I$. Conversely, if $A*I\subseteq
    I$, then $\lambda_a(x)=a*x+x\in I$.
\end{proof}

\begin{proposition}
    \label{pro:I*A}
    Let $A$ be a skew brace. A normal subgroup $I$ of $(A,+)$
    is an ideal of $A$ if and only $\lambda_a(I)\subseteq I$, for all $a\in A$, and
    $I*A\subseteq I$.
\end{proposition}

\begin{proof}
    Let $x\in I$ and $a\in A$.  Assume first that $I$ is invariant under the
    action of $\lambda$ and that $I*A\subseteq I$. Then
    \begin{equation}
    \label{eq:trick:I*A}
        \begin{aligned}
        a\circ x\circ a' &=a+\lambda_a(x\circ a')\\
        &=a+\lambda_a(x+\lambda_x(a'))
        =a+\lambda_a(x)+\lambda_a\lambda_x(a')+a-a\\
        &=a+\lambda_a(x+\lambda_x(a')-a')-a
        =a+\lambda_a(x+x*a')-a\in I,
    \end{aligned}
    \end{equation}
    and hence $I$ is an ideal.

    Conversely, assume that $I$ is an ideal. Then $I*A\subseteq I$ since
    \begin{align*}
        x*a&=-x+x\circ a-a\\
        &=-x+a\circ(a'\circ x\circ a)-a
        =-x+a+\lambda_a(a'\circ x\circ a)-a\in I.\qedhere
    \end{align*}
\end{proof}


Let $I$ and $J$ be ideals
of a skew brace $A$. Then $I\cap J$ is an ideal of $A$.  
The sum $I+J$ of $I$ and $J$ is defined as the
additive subgroup of $A$ generated by all the 
elements of the form
$u+v$, $u\in I$ and $v\in J$. 

\begin{proposition}
Let $A$ be a skew brace and let
$I$ and $J$ be ideals of $A$. Then $I+J$ is an ideal of $A$.
\end{proposition}

\begin{proof}
    Since $I$ and $J$ are normal subgroups of $A$, we have that
    \[ I+J=\{ u+v \mid u\in I,\; v\in J \}.\]
    First note that $I+J$ is a normal subgroup of $(A,+)$ since
    \[
        a+(u+v)-a=(a+u-a)+(a+v-a)\in I+J
    \]
    for all $u\in I$, $v\in J$ and $a\in A$.
    Let $a\in A$, $u\in I$ and $v\in J$. Then $\lambda_a(u+v)=\lambda_a(u)+\lambda_a(v)\in I+J$ and
    hence it follows that $\lambda_a(I+J)\subseteq I+J$. Moreover, by Propositions~\ref{pro:A*I} and~\ref{pro:I*A},
        \[
        (u+v)*a=(u\circ\lambda^{-1}_u(v))*a
        =u*(\lambda^{-1}_u(v)*a)+\lambda^{-1}_u(v)*a+u*a\in I+J.
    \]
    Hence $(I+J)*A\subseteq I+J$. Therefore the result follows by Proposition~\ref{pro:I*A}.
\end{proof}


\begin{definition}
	\index{Socle}
	Let $A$ be a skew brace. The subset 
	$\Soc(A)=\ker\lambda\cap Z(A,+)$
	is the \emph{socle} of $A$.
\end{definition}

We will use the following exercise several times. 

\begin{exercise}
    \label{xca:socle}
    Let $A$ be a skew brace and $a\in\Soc(A)$. Prove that  
    \[
    b+b\circ a=b\circ a+b\quad\text{and}\quad
    \lambda_b(a)=b\circ a\circ b'
    \]
    hold 
    for all $b\in A$.
\end{exercise}

\begin{exercise}
\label{xca:Bachiller1}
    Prove that the socle of a skew brace $A$ is the kernel of the 
    group homomorphism $(A,\circ)\to\Aut(A,+)\times\Sym_A$, $a\mapsto (\lambda_a,\mu_a^{-1})$. 
\end{exercise}

\begin{exercise}
\label{xca:Bachiller2}
    Prove that the socle of a skew brace $A$ is the kernel of the 
    group homomorphism 
    \[
    (A,\circ)\to\Aut(A,+)\times\Aut(A,+),
    \quad
    a\mapsto (\lambda_a,\xi_a),
    \]
    where
    $\xi_a(b)=a+\lambda_a(b)-a$. 
\end{exercise}

\begin{proposition}
	\label{pro:socle}
	Let $A$ be a skew brace. Then $\Soc(A)$ is an ideal of $A$.
\end{proposition}

	
	\begin{proof}
		Clearly $0\in\Soc(A)$, since $\lambda$ is a group homomorphism. Let $a,b\in\Soc(A)$ and $c\in A$. Since 
		$b\circ (-b)=b+(-b)=0$, it follows that 
		$b'=-b\in\Soc(A)$. The calculation 
		\[
		\lambda_{a-b}(c)=\lambda_{a\circ b'}(c)=\lambda_a\lambda^{-1}_b(c)=c,
		\]
 		implies that $a-b\in\ker\lambda$. Since $a-b\in Z(A,+)$, it follows that 
        $(\Soc(A),+)$ is a normal subgroup of $(A,+)$. 
        
        For each $d\in A$, $a+c'\circ d=c'\circ d+a$.  By Exercise~\ref{xca:socle}, we have 
        \begin{align*}
        d+\lambda_c(a) &= d-c+c\circ a
        =c\circ (c'\circ d+a)\\
        &= c\circ (a+c'\circ d)
        = c\circ a-c+d
        = -c+c\circ a+d
        = \lambda_c(a)+d,
        \end{align*}
        that is $\lambda_c(a)$ is central in $(A,+)$. Moreover, again by Exercise~\ref{xca:socle},
        \begin{align*}
            \lambda_c(a)+d &= -c+c\circ a+d 
            = c\circ a-c+d\\
            &= c\circ (a+(c'\circ d)
            = c\circ a\circ c'\circ d=\lambda_c(a)\circ d
        \end{align*}
        and hence 
        \[
        \lambda_{\lambda_c(a)}(d)=-\lambda_c(a)+\lambda_c(a)\circ d=-\lambda_c(a)+\lambda_c(a)+d=d.
        \]
        Therefore $\Soc(A)$ is a strong left ideal of $A$. In fact, $\Soc(A)$ is an ideal of $A$,
        as $c\circ a\circ c'=\lambda_c(a)\in\Soc(A)$.  
	\end{proof}

As a corollary we obtain that the socle of a skew 
brace $A$ is a trivial skew brace of abelian type. 

\begin{proposition}
    \label{pro:soc_kernels}
    Let $A$ be a skew brace. Then $\Soc(A)=\ker\lambda\cap\ker\mu$.
\end{proposition}

\begin{proof}
    Let $a\in\Soc(A)$ and $b\in A$. Then $\lambda_a=\id$ and $a\in Z(A,+)$. By Exercise~\ref{xca:socle}, \[\mu_a(b)=\lambda_b(a)'\circ b\circ a=(b\circ a\circ b')'\circ b\circ a=b.\]  Thus $a\in\ker\lambda\cap\ker\mu$. 
    
    Conversely, let $a\in\ker\lambda\cap\ker\mu$ and $b\in A$. Then $b'=\mu_a(b')=\lambda_{b'}(a)'\circ b'\circ a$, so
    $\lambda_{b'}(a)=b'\circ a\circ b$. Now 
    \[
    b+a=b\circ\lambda^{-1}_b(a)=b\circ\lambda_{b'}(a)=b\circ b'\circ a\circ b=a\circ b=a+\lambda_a(b)=a+b
    \]
    implies that $a\in\Soc(A)$. 
\end{proof}

\begin{definition}
\index{Annihilator}
Let $A$ be a skew brace. The \emph{annihilator} of $A$ is 
defined as the set $\Ann(A)=\Soc(A)\cap Z(A,\circ)$. 
\end{definition}

Note that $\Ann(A)\subseteq\Fix(A)$. 

\begin{proposition}
The annihilator of a skew brace $A$ is an ideal of $A$. 
\end{proposition}

\begin{proof}
    Let $x,y\in\Ann(A)$. Note that $x-y=x\circ y'\in Z(A,\circ)$. Hence $\Ann(A)$ is a subbrace of $A$. Since $\Ann(A)\subseteq Z(A,+)\cap Z(A,\circ)$, 
    we only need to note that $\lambda_a(x)=x\in\Ann(A)$, for all $a\in A$. 
\end{proof}



%Clearly $\Soc(A)=\ker(\lambda)\cap Z(A,+)$. In \cite[Lemma~2.5]{MR3647970} it
%is proved that $\Soc(A)$ is an ideal of $A$.

\index{Quotient brace}
If $A$ is a skew brace and $I$ is an ideal of $A$, then $a+I=a\circ I$ for all $a\in A$. Indeed, 
$a\circ x=a+\lambda_a(x)\in a+I$ and 
$a+x=a\circ\lambda_a^{-1}(x)=a\circ\lambda_{a'}(x)\in a\circ I$ 
for all $a\in A$ and $x\in I$. 
This allows us to prove that there exists a unique skew brace structure over $A/I$ such that
the map 
\[
\pi\colon A\to A/I,
\quad
a\mapsto a+I=a\circ I,
\]
is a homomorphism of skew braces. The skew brace $A/I$ 
is the \textbf{quotient brace} of $A$ modulo $I$. 

\begin{exercise}
\label{xca:iso1}
    Let $f\colon A\to B$ be a homomorphism of skew braces. Prove that $A/\ker f\simeq f(A)$. 
\end{exercise}

\begin{exercise}
\label{xca:iso2}
    Let $A$ be a skew brace and let $B$ be a subbrace of $A$. Prove that if $I$ is an ideal of $A$, 
    then $B\circ I$ is a subbrace of $A$, 
    $B\cap I$ is an ideal of $B$ and $(B\circ I)/I\cong B/(B\cap I)$. 
\end{exercise}

\begin{exercise}
\label{xca:iso3}
Let $A$ be a skew brace and $I$ and $J$ be ideals of $A$. Prove that if $I\subseteq J$, then
$A/J\cong (A/I)/(J/I)$. 
\end{exercise}

\begin{exercise}
\label{xca:correspondence}
Let $A$ be a skew brace and let $I$ be an ideal of $A$. Prove that there is a bijective correspondence between (left) ideals 
of $A$ containing $I$ and (left) ideals of $A/I$. 
\end{exercise}



