\chapter{}

\topic{Locally indicable groups}

\begin{definition}
\index{Group!indicable}
    A group $G$ is \textbf{indicable} if
    there exists a non-trivial group homomorphism $G\to\Z$.
\end{definition}

We know that braid groups are indicable.
The free group $F_n$ in $n$ letters is indicable. 

\begin{definition}
\index{Group!locally indicable}
    A group $G$ is \textbf{locally indicable} if every 
    non-trivial finitely generated subgroup is indicable.
\end{definition}

\index{Burns--Hale's theorem}
Burns--Hale's theorem (see \cite[Theorem 1.50]{MR3560661}) states that a group $G$ is left-ordered if and only if
for every non-trivial finitely generated subgroup $H$ of $G$ there exists 
a left-ordered group $L$ and a non-trivial group homomorphism $H\to L$. As a consequence, 
locally indicable groups are left-ordered. 

\begin{example}
    Since subgroups of free groups are free, 
    it follows that $F_n$ is locally indicable. 
\end{example}

\begin{proposition}
\label{pro:LI_exact}
    Let
    \[\begin{tikzcd}
	1 & K & G & Q & 0
	\arrow[from=1-1, to=1-2]
	\arrow["\alpha", from=1-2, to=1-3]
	\arrow["\beta", from=1-3, to=1-4]
	\arrow[from=1-4, to=1-5]
\end{tikzcd}
\]
    be an exact sequence of groups. 
    If $K$ and $Q$ are
    locally indicable, then $G$ is locally indicable.
\end{proposition}

\begin{proof}
    Let $g_1,\dots,g_n\in G$ and $L=\langle g_1,\dots,g_n\rangle$. 
    Assume first that $\beta(L)\ne\{1\}$. Since $Q$ is locally indicable, 
    there exists a non-trivial group homomorphism $\beta(L)\to\Z$. Then the 
    composition $L\to\beta(Q)\to\Z$ is then a non-trivial group homomorphism. Assume now
    that $\beta(L)=\{1\}$. Then there exist $k_1,\dots,k_n\in K$ 
    such that $\alpha(k_i)=g_i$ for all $i\in\{1,\dots,n\}$. Note that
    $\alpha\colon \langle k_1,\dots,k_n\rangle\to L$ is a group isomorphism. Since
    $K$ is locally indicable, there exits a non-trivial group 
    homomorphism $\langle k_1,\dots,k_n\rangle\to\Z$. 
    Thus the composition $L\to\langle k_1,\dots,k_n\rangle\to\Z$ is a non-trivial
    group homomorphism 
    and hence $G$ is locally indicable. 
\end{proof}

As a consequence of the previous proposition, 
if $G$ and $H$ are locally indicable groups and 
$\sigma\colon G\to\Aut(H)$ is a group homomorphism, then 
$G\rtimes_\sigma H$ is locally indicable. In particular, the 
direct product of locally indicable groups is locally indicable.

\begin{example}
    The group $G=\langle x,y:x^{-1}yx=y^{-1}\rangle$ is
    locally indicable. We know that $G$ is torsion-free. Let 
    $K=\langle y\rangle\simeq\Z$. Then $G/K\simeq\Z$ and 
    then, since there is an exact sequence
    $1\to\Z\to G\to\Z\to1$ 
    it follows from Proposition \ref{pro:LI_exact} 
    that $G$ is locally indicable.
\end{example}

% $(x,y)\mapsto(x+1,y)$, $(x,y)\mapsto(-x,y+1)$
% A=2312, B=200-2
% Let $K=\langle y\rangle$. Then $G/K\simeq\Z$ and 
% the second result follows from the previous proposition. 

\begin{exercise}
\label{xca:B3_LI}
    Prove that $\B_3$ is locally indicable. 
\end{exercise}

The previous exercise uses the fact that $[\B_3,\B_3]$ is isomorphic to the free group in two letters, see
Exercise \ref{xca:derivedB3}.
An alternative solution to the previous fact goes as follows: $\B_3$ is the fundamental group
of the trefoil knot and fundamental groups of knots are locally indicable. 

\begin{exercise}
    Prove that $\B_4$ is locally indicable.
\end{exercise}

The previous exercise might be harder than Exercise \ref{xca:B3_LI}. One possible solution
is based on using the Reidemeister--Schreier method to prove that 
$[\B_4,\B_4]$ is a certain semidirect product 
between free groups in two generators. Another solution: Let 
$f\colon\B_4\to\B_3$ be the group homomorphism given by $f(\sigma_1)=f(\sigma_3)=\sigma_1$ 
and $f(\sigma_2)=\sigma_2$. Then $\ker f=\langle \sigma_1\sigma_3^{-1},\sigma_2\sigma_1\sigma_3^{-1}\sigma_2^{-1}\rangle$ 
is isomorphic to the free group in two letters. Now use the exact sequence
$1\to \ker f\to\B_4\to\B_3\to1$. 

\begin{exercise}
\label{xca:relations}
    Let $n\geq5$. Consider the elements of $\B_n$ given by 
    \begin{align*}
        &\beta_1=\sigma_1^{-1}\sigma_2,
        &&\beta_2=\sigma_2\sigma_1^{-1}, 
        &&\beta_3=\sigma_1\sigma_2\sigma_1^{-2},
        &&\beta_4=\sigma_3\sigma_1^{-1}, 
        &&\beta_5=\sigma_4\sigma_1^{-1}.
    \end{align*}
    Prove the following relations:
    \begin{enumerate}
        \item $\beta_1\beta_5=\beta_5\beta_2$.
        \item $\beta_2\beta_5=\beta_5\beta_3$.
        \item $\beta_1\beta_3=\beta_2$.
        \item $\beta_1\beta_4\beta_3=\beta_4\beta_2\beta_4$.
        \item $\beta_4\beta_5\beta_4=\beta_5\beta_4\beta_5$.
    \end{enumerate}
\end{exercise}

\begin{exercise}
    Let $n\geq 5$. 
    Prove that $\B_n$ is not locally indicable.
\end{exercise}

For the previous exercise one needs to show that
every group homomorphism $f\colon \langle\beta_1,\dots,\beta_5\rangle\to\Z$ is trivial. Hint: consider
the abelianization of $\langle\beta_1,\dots,\beta_5\rangle$. 

\topic{Unique product groups}

Let $G$ be a group and $A,B\subseteq G$ be non-empty subsets. 
An element $g\in G$ is a \textbf{unique product} in $AB$ if $g=ab=a_1b_1$ for some
$a,a_1\in
A$ and $b,b_1\in B$ implies that $a=a_1$ and $b=b_1$.

\begin{definition}
	\index{Group!unique product}
	A group $G$ has the \textbf{unique product property} if 
	for every finite non-empty subsets $A,B\subseteq G$ there exists at least one
	unique product in $AB$.
\end{definition}

\begin{proposition}
    Left-ordered groups have the unique product property.
\end{proposition}

\begin{proof}
    Let $G$ be a left-ordered group. 
	Let $A$ be a non-empty finite subset of $G$ and $B=\{b_1,\dots,b_n\}\subseteq G$. 
	Assume that $b_1<b_2<\cdots<b_n$. Let $c\in A$ be such that $cb_1$ is the 
	minimum of $Ab_1=\{ab_1:a\in A\}$. We claim that $cb_1$ admits a unique
	representation of the form $\alpha\beta$ with $\alpha\in A$ and 
	$\beta\in B$. If $cb_1=ab$, then, since $ab=cb_1\leq ab_1$, it follows that 
	$b\leq b_1$. Hence $b=b_1$ and $a=c$. 
\end{proof}

\begin{exercise}
	Prove that groups with the unique product property are
	torsion-free.
\end{exercise}

The converse does not hold. 
Promislow's group is a celebrated counterexample.

\begin{theorem}[Promislow]
\index{Promislow's theorem}
    The group $G=\langle a,b:a^{-1}b^2a=b^{-2},b^{-1}a^2b=a^{-2}\rangle$
    does not have the unique product property.
\end{theorem}

\begin{proof}
    Let 
    \begin{multline}
    \label{eq:Promislow}
    S=\{ a^2b,
    b^2a,
    aba^{-1},
    (b^2a)^{-1},
    (ab)^{-2},
    b,
    (ab)^2a,
    (ab)^2,
    (aba)^{-1},\\
    bab,
    b^{-1},
    a,
    aba,
    a^{-1}
    \}.
    \end{multline}
    We use \textsf{GAP} and the representation $G\to\GL(4,\Q)$ given by 
    \[
a\mapsto\begin{pmatrix}
1 & 0 & 0 & 1/2\\
0 & -1 & 0 & 1/2\\
0 & 0 & -1 & 0\\
0 & 0 & 0 & 1
\end{pmatrix},
\quad
b\mapsto\begin{pmatrix}
-1 & 0 & 0 & 0\\
0 & 1 & 0 & 1/2\\
0 & 0 & -1 & 1/2\\
0 & 0 & 0 & 1
\end{pmatrix}
\]
    to check that 
    $G$ does not have
    unique product property, as each 
    \[
    s\in S^2=\{s_1s_2:s_1,s_2\in S\}
    \]
    admits at least two different decompositions of the 
    form $s=xy=uv$ for $x,y,u,v\in S$. 
    We first create the matrix representations of $a$ and $b$.
\begin{lstlisting}
gap> a := [[1,0,0,1/2],[0,-1,0,1/2],[0,0,-1,0],[0,0,0,1]];;
gap> b := [[-1,0,0,0],[0,1,0,1/2],[0,0,-1,1/2],[0,0,0,1]];;
\end{lstlisting}
    Now we create
    a function that produces the set $S$.
\begin{lstlisting}
gap> Promislow := function(x, y)
> return Set([
> x^2*y,
> y^2*x,
> x*y*Inverse(x),
> (y^2*x)^(-1),
> (x*y)^(-2),
> y,
> (x*y)^2*x,
> (x*y)^2,
> (x*y*x)^(-1),
> y*x*y,
> y^(-1),
> x,
> x*y*x,
> x^(-1)
]);
end;;
\end{lstlisting}
So the set $S$ of \eqref{eq:Promislow} 
will be \lstinline{Promislow(a,b)}. We now
create a function that checks whether
every element of a Promislow subset 
admits more than one representation.
\begin{lstlisting}
gap> is_UPP := function(S)
> local l,x,y;
> l := [];
> for x in S do
> for y in S do
> Add(l,x*y);
> od;
> od;
> if ForAll(Collected(l), x->x[2] <> 1) then
> return false;
> else
> return fail;
> fi;
> end;;
\end{lstlisting}
Finally, we check whether every element of 
$S$ admits more than one representation.
\begin{lstlisting}
gap> S := Promislow(a,b);;
gap> is_UPP(S);
false
\end{lstlisting}
This completes the proof. 
\end{proof}

There are other examples. 

% Let $G=\langle x,y:x^{-1}y^2xy^2=x^{-2}yx^{-2}y^3=1\rangle$. 
% We first construct the group and a certain normal
% subgroup $N$ of index eight. 

% \begin{lstlisting}
% gap> f := FreeGroup(2);;
% gap> x := f.1;;
% gap> y := f.2;;
% gap> rels := [Inverse(x)*y^2*x*y^2, Inverse(x^2)*y*Inverse(x^2)*y^3];;
% gap> G := f/rels;;
% gap> x := G.1;;
% gap> y := G.2;;
% gap> N := Subgroup(G, [y*x*Inverse(x*y), y^2, x^4]);
% gap> IsNormal(N,G);
% true
% gap> StructureDescription(G/N);
% "C4 x C2"
% \end{lstlisting}
% The subgroup $N$ has a nice presentation. It can be presented
% as the group 
% \[
% N=\langle x_1,x_2,x_3:[x_1,x_2]=[x_1,x_3]=1,\,x_3x_2=x_2x_3x_1^8\rangle.
% \]
% \begin{lstlisting}
% gap> g := IsomorphismFpGroup(N);
% gap> RelatorsOfFpGroup(Image(g));
% [ F3*F1*F3^-1*F1^-1, F2*F1*F2^-1*F1^-1, F1*F3^-1*F1^5*F2*F1^2*F3*F2^-1 ]
% \end{lstlisting}
% From these relations
% one proves by induction that 
% \begin{align}
%     x_3^bx_2^a=x_2^ax_3^bx_1^{8ab}
% \end{align}
% for all $a,b\in\Z$. 
% % First induction in $n_1$, then induction in $n_2$. 
% It follows that every element of $N$ 
% is of the form $x_1^{n_1}x_2^{n_2}x_3^{n_3}$ 
% for $n_1,n_2,n_3\in\Z$. Moreover, 
% \[
% (x_1^{n_1}x_2^{n_2}x_3^{n_3})^k=x_1^{kn_1+(k-1)8n_2n_3}x_2^{kn_2}x_3^{kn_3}
% \]
% for all $k\in\Z$. Note that $x_1^8$ is a commutator. Moreover, 
% $N/[N,N]\simeq\Z\times\Z\times\Z/8$. This implies that
% $N$ is torsion-free. Let us prove that $G$ is torsion-free. 
% Let $\pi\colon G\to G/N$ be the canonical map. Let 
% $g\in G$ be a torsion element, in particular $g\not\in N$
% and hence $\pi(g)\ne 1$. So $\pi(g)$ has order two 
% or four. Without loss of generality we may assume that
% $\pi(g)$ has order two.  
% Then $\pi(g^2)=\pi(g)^2=y^2=1$ and hence 
% $g^2\in N$. Since $N$ is torsion free, it follows that $g=1$. 

\begin{definition}
\index{Group!double unique product}
	A group $G$ has the \textbf{double property of unique products} 
	if for every finite non-empty subsets $A,B\subseteq G$ such that 
	$|A|+|B|>2$ there are at least two unique products in $AB$.
\end{definition}

\begin{theorem}[Strojnowski]
	\label{theorem:Strojnowski}
	\index{Strojonowski's theorem}
	Let $G$ be a group. The following statements are equivalent:
	\begin{enumerate}
		\item $G$ has the double property of unique products. 
		\item Every non-empty finite subset $A\subseteq G$ contains at least one unique product 
			in $AA=\{a_1a_2:a_1,a_2\in A\}$.
		\item $G$ has the unique product property.
	\end{enumerate}
\end{theorem}

\begin{proof}
	It is trivial that $1)\implies2)$.  Let us prove that 
	$2)\implies3)$. If $G$ does not have the unique product property, 
	there exist finite non-empty subsets 
	$A,B\subseteq G$ such that every element of 
	$AB$ admists at least two representations. Let $C=AB$. Every element $c\in C$ is 
	of the form $c=(a_1b_1)(a_2b_2)$ for some $a_1,a_2\in A$ and $b_1,b_2\in B$. Since 
	$a_2^{-1}b_1^{-1}\in AB$, there exist $a_3\in A\setminus\{a_2\}$ and
	$b_3\in B\setminus\{b_1\}$ such that 
	$a_2^{-1}b_1^{-1}=a_3^{-1}b_3^{-1}$. Thus $b_1a_2=b_3a_3$ and hence 
	\[
	c=(a_1b_1)(a_2b_2)=(a_1b_3)(a_3b_2)
	\]
	has two different representations in $AB$, 
	as $a_2\ne a_3$ and $b_1\ne b_3$.

	We now prove that $3)\implies1)$. Let us assume that $G$ has the unique product property 
	but it is not a group with double unique products. Then there exist 
	finite non-empty subsets 
	$A,B\subseteq G$ with $|A|+|B|>2$ such that 
	in $AB$ there exists a unique element $ab$ with a unique representation in $AB$.
	Let $C=a^{-1}A$ and $D=Bb^{-1}$. Then $1\in C\cap D$ and the identity 
	$1$ admits a unique representation in $CD$ (because $1=cd$ with 
	$c=a^{-1}a_1\ne 1$ and $d=b_1b^{-1}\ne 1$ imply that $ab=a_1b_1$ with $a\ne
	a_1$ and $b\ne b_1)$. Let $E=D^{-1}C$ and $F=DC^{-1}$. Every element of the set $EF$
	can be written as $(d_1^{-1}c_1)(d_2c_2^{-1})$. If either $c_1\ne 1$ or $d_2\ne 1$, 
	then $c_1d_2=c_3d_3$ for some elements $c_3\in C\setminus\{c_1\}$ and  
	$d_3\in D\setminus\{d_2\}$. Thus 
	$(d_1^{-1}c_1)(d_2c_2^{-1})=(d_1^{-1}c_3)(d_3c_2^{-1})$ are two different representations 
    for $(d_1^{-1}c_1)(d_2c_2^{-1})$. If either $c_2\ne 1$
	or $d_1\ne 1$, then $c_2d_1=c_4d_4$ for some $d_4\in D\setminus\{d_1\}$
	and some $c_4\in C\setminus\{c_2\}$. Since 
	$d_1^{-1}c_2^{-1}=d_4^{-1}c_4^{-1}$, it follows that 
	\[
	(d_1^{-1}1)(1c_2^{-1})=(d_4^{-1}1)(1c_4^{-1}).
	\]
	Since $|C|+|D|>2$, either $C$ or 
	$D$ contains $c\ne1$. Thus $(1\cdot 1)(1\cdot 1)=(1\cdot
	c)(1\cdot c^{-1})$. Therefore every element of $EF$ 
	admits at least two representations. 
\end{proof}

% passman lema 1.9 pag 589
\begin{exercise}
	Prove that if a group $G$ satisfies the unique product property, then 
    $K[G]$ contains only trivial units.
\end{exercise}

In general it is extremely hard to check whether a given group
has the unique product property. As a geometrical way to 
attack this problem, Bowditch introduced \emph{diffuse groups}. If
$G$ is a torsion-free group and 
$A\subseteq G$ is a subset, we say that $A$ is antisymmetric 
if $A\cap A^{-1}\subseteq\{1\}$, where $A^{-1}=\{a^{-1}:a\in
A\}$. The set of \textbf{extremal elements} of $A$ is defined as 
$\Delta(A)=\{a\in A:Aa^{-1}\text{ is antisymmetric}\}$. 
Thus 
\[
	a\in A\setminus\Delta(A)
	\Longleftrightarrow
	\text{there existes $g\in G\setminus\{1\}$ such that $ga\in A$ and $g^{-1}a\in A$}.
\]

\begin{definition}
	\index{Group!diffuse}
	A group $G$ is \textbf{diffuse} if for every subset $A\subseteq
	G$ such that $2\leq |A|<\infty$ one has $|\Delta(A)|\geq2$.
\end{definition}

\begin{proposition}
	Left-ordered groups are diffuse.	
\end{proposition}

\begin{proof}
    Let $G$ be a left-ordered group and $A=\{a_1,\dots,a_n\}$ be such that
    \[
    a_1<a_2<\cdots<a_n.
    \]
    We claim that 
	$\{a_1,a_n\}\subseteq\Delta(A)$. If $a_1\in
	A\setminus\Delta(A)$, there exists $g\in G\setminus\{1\}$ such that $ga_1\in A$ and
	$g^{-1}a_1\in A$. Thus  $a_1\leq ga_1$ and $a_1\leq g^{-1}a_1$. It follows that
	$1\leq a^{-1}ga_1$ and $1\leq a_1^{-1}g^{-1}a_1=(a_1^{-1}ga_1)^{-1}$, 
	a contradiction. Similarly, $a_n\in \Delta(A)$.
\end{proof}

\begin{proposition}
	\label{pro:difuso=>2up}
    Diffuse groups have double unique products.  
\end{proposition}

\begin{proof}
    Let $G$ be a diffuse group that does not have double unique products. 
    There exist non-empty subsets $A,B\subseteq G$ with $|A|+|B|>2$ such that 
	$C=AB$ admits at most one unique product. Then $|C|\geq2$. Since $G$ is diffse, 
	$|\Delta(C)|\geq2$. If $c\in\Delta(C)$, then $c$ admits a unique 
	expression of the form $c=ab$ with $a\in A$ and $b\in B$ (otherwise, if 
	$c=a_0b_0=a_1b_1$ with $a_0\ne a_1$ and $b_0\ne b_1$). If $g=a_0a_1^{-1}$,
	then $g\ne 1$, 
	\[
	gc=a_0a_1^{-1}a_1b_1=a_0b_1\in C.
	\]
	Moreover, 
	$g^{-1}c=a_1a_0^{-1}a_0b_0=a_1b_0\in C$. Hence $c\not\in\Delta(c)$, a contradiction.
\end{proof}

\begin{problem}
	Find a non-diffuse group with the unique product property.
\end{problem}

%Un grupo $G$ se dice \textbf{débilmente difuso} si para todo subconjunto
%finito $A\subseteq G$ no vacío se tiene $\Delta(A)\ne\emptyset$. La técnica
%usada para demostrar el lema~\ref{lemma:difuso=>2up} puede usarse para
%demostrar que un grupo débilmente difuso posee la propiedad del producto
%único. El teorema~\ref{theorem:Strojnowski} sugiere entonces la siguiente
%pregunta: 
%
%\begin{problem}
%	¿Existe un grupo débilmente difuso que no sea difuso?
%\end{problem}
%
%\section{El grupo de Promislow}
%
%Veremos un ejemplo concreto de un grupo sin torsión que no es ordenable, no es
%difuso y no tiene la propiedad del producto único.
%
%\begin{exercise}
%	\label{exercise:Dinfty}
%	Demuestre que $G=\langle x,y:x^2=y^2=1\rangle$ es isomorfo al grupo diedral infinito.
%\end{exercise}
%
%\begin{definition}
%	Se define el grupo de Promislow como 
%	\[
%		G=\langle x,y:x^{-1}y^2x=y^{-2},\,y^{-1}x^2y=x^{-2}\rangle.
%	\]
%\end{definition}
%
%\begin{proposition}
%	\label{proposition:Promislow}
%	El grupo de Promislow es libre de torsión y no satisface la propiedad del
%	producto único. 
%\end{proposition}
%
%\begin{proof}
%	
%\end{proof}

