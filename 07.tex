\section{}

\subsection{Subnormality}

\begin{definition}
	\index{Subgroup!subnormal}
	Let $G$ be a group. A subgroup $H$ of $G$ is said to be \text{subnormal} in $G$ if there is a sequence 
    of subgroups 
	\[
		H=H_0\subseteq H_1\subseteq\cdots\subseteq H_k=G		
	\]
	with $H_i$ normal in $H_{i+1}$ for all $i\in\{0,\dots,k-1\}$. 
\end{definition}

\begin{example}
	Let $G=\Sym_4$. Then $K=\{\id,(12)(34),(13)(24),(14)(23)\}$ is normal in $G$. 
	The subgroup $L=\{\id,(12)(34)\}$ is subnormal in $G$ (and not normal). 
	es subnormal. 
\end{example}

\begin{exercise}
\label{xca:correspondence_subnormality}
    Prove that the correspondence theorem preserves subnormality. 
\end{exercise}

\begin{theorem}
	\label{thm:subnormal}
	Let $G$ be a finite group. Then $G$ is nilpotent if and only if every subgroup of $G$ is subnormal in $G$. 
\end{theorem}

\begin{proof}
	Assume first that every subgroup of $G$ is subnormal in $G$. Let $H$ be a subnormal subgroup of $G$, where 
	\[
		H=H_0\subseteq H_1\subseteq\cdots\subseteq H_k=G
	\]
	with $H_i$ normal in $H_{i+1}$. Without loss of generality, we may assume that 
	$H\subsetneq H_1$. Since $H\subsetneq H_1\subseteq N_G(H)$, 
	$G$ is nilpotent by Exercise~\ref{xca:normalizadora}.

	Assume now that $G$ is nilpotent. Let $H$ be a subgroup of $G$.
	We proceed by induction on $(G:H)$. If $(G:H)=1$, then $H=G$ and the theorem holds. If 
	$H\ne G$, since $H\subsetneq N_G(H)$ by Lemma~\ref{lem:normalizadora}, 
	\[
		(G:N_G(H))<(G:H).
	\]
	By the inductive hypothesis, $N_G(H)$ is subnormal in $G$. Since $H$ is 
	normal in $N_G(H)$, we conclude that $H$ is subnormal in $G$.
\end{proof}

\begin{corollary}
	Sea $G$ un grupo y sea $K$ un subgrupo de $G$ tal que $K\subseteq Z(G)$.
	Entonces $G$ es nilpotente si y sólo si $G/K$ es nilpotente. 
\end{corollary}

\begin{proof}
	Si $G$ es nilpotente, entonces $G/K$ es nilpotente por el
	teorema~\ref{theorem:nilpotente}. Demostremos la afirmación recíproca. Sea
	$\pi\colon G\to G/K$ el morfismo canónico. Sea $U$ un subgrupo de $G$. Como
	$G/K$ es nilpotente, el teorema~\ref{theorem:subnormal} implica que
	$\pi(U)$ es un subgrupo subnormal de $G/K$. La correspondencia implica que
	$UK$ es un subgrupo subnormal de $G$, y luego, como $K$ es central, $U$ es
	normal en $UZ$. Luego $U$ es subnormal en $G$ y entonces $G$ es nilpotente
	por el teorema~\ref{theorem:subnormal}.
\end{proof}

\begin{theorem}
	\label{theorem:F(G)subnormalidad}
	Sea $G$ un grupo finito y sea $H$ un subgrupo de $G$. Entonces $H$ es
	nilpotente y subnormal en $G$ si y sólo si $H\subseteq F(G)$.
\end{theorem}

\begin{proof}
	Supongamos que $H\subseteq F(G)$. Como $F(G)$ es nilpotente por el
	teorema~\ref{theorem:Fitting}, $H$ es nilpotente por el
	teorema~\ref{theorem:nilpotente}. Además, como $H$ es subnormal en $F(G)$ por la
	teorema~\ref{theorem:subnormal} y $F(G)$ es normal en $G$, $H$ es
	subnormal en $G$.

	Supongamos ahora que $H$ es nilpotente y subnormal en $G$.  Procederemos
	por inducción en $|G|$. Si $H=G$ el resultado es trivialmente cierto.
	Supongamos entonces que $H\ne G$. Como $H$ es subnormal en $G$, 
	\[
		H=H_0\triangleleft H_1\triangleleft\cdots\triangleleft H_k=G.
	\]
	Sea $M=H_{k-1}$. Como $M\ne G$ y $M$ es normal en $G$, 
	$H\subseteq F(M)$ por hipótesis	inductiva. Luego $H\subseteq F(M)=M\cap
	F(G)\subseteq F(G)$ por el corolario~\ref{corollary:McapF(G)}.
\end{proof}

\begin{lemma}
	\label{lemma:McapN=1}
	Sean $M$ y $N$ subgrupos normales de un grupo $G$ tales que $M\cap N=1$.
	Entonces $M\subseteq C_G(N)$.
\end{lemma}

\begin{proof}
	Sean $m\in M$ y $n\in N$. Entonces $[n,m]=(nmn^{-1})m\in M$ pues $M$ es
	normal en $G$ y también $[n,m]=n(mn^{-1}m^{-1})\in N$ pues $N$ es normal en
	$G$. Luego $[n,m]\in M\cap N=1$.
\end{proof}

\begin{theorem}[Wielandt]
	\label{theorem:MsubsetNG(S)}
	Sea $G$ un grupo finito. Si $S$ es un subgrupo subnormal de $G$ 
	y $M$ es un subgrupo minimal-normal de $G$ entonces $M\subseteq N_G(S)$.
\end{theorem}

\begin{proof}
	Procederemos por inducción en $|G|$. Si $S=G$ no hay nada para demostrar.
	Supongamos que $S\ne G$.  Como $S$ es subnormal en $G$, existe una sucesión 
	\[
		S=S_0\triangleleft S_1\triangleleft\cdots\triangleleft S_{k-1}\triangleleft S_k=G.
	\]
	Sea $N=S_{k-1}$. 

	Si $M\cap N\ne 1$ entonces $M\subseteq N$ (pues como $M$ y $N$ son normales
	en $G$, $M\cap N=M$ por la minimalidad de $M$). Vamos a demostrar que
	$M\subseteq\Soc(N)$.  Como $M\ne 1$ y $M$ es normal en $N$,
	$M\cap\Soc(N)\ne 1$. Además $\Soc(N)$ es característico en $N$ y $N$ es
	normal en $G$, entonces $\Soc(N)$ es normal en $G$. Luego $M\cap\Soc(N)$ es
	un subgrupo normal de $G$. Como además $1\ne M\cap\Soc(N)\subseteq M$, se
	concluye que $M\cap\Soc(N)=M$ por la minimalidad de $M$. Por hipótesis inductiva, 
	todo subgrupo minimal-normal de $N$ normaliza a $S$; entonces 
	$\Soc(N)\subseteq N_N(S)\subseteq N_G(S)$ y luego 
	\[
	M\subseteq\Soc(N)\subseteq N_G(S).
	\]
	Si $M\cap N=1$, el lema~\ref{lemma:McapN=1} implica que
	\[
	M\subseteq C_G(N)\subseteq C_G(S)\subseteq N_G(S). 
	\]
\end{proof}

\begin{corollary}
	Sea $G$ finito y sea $S$ un subgrupo subgrupo subnormal de $G$. Entonces
	$\Soc(G)\subseteq N_G(S)$.
\end{corollary}

\begin{proof}
	Como todo subgrupo minimal-normal de $G$ está contenido en $N_G(S)$ por 
	el teorema~\ref{theorem:MsubsetNG(S)}, 
	$\Soc(G)=\langle M:M\text{ subgrupo 
	minimal-normal de $G$}\rangle\subseteq N_G(S)$.
\end{proof}

En el siguiente teorema demostraremos que los subgrupos normales forman un
reticulado.

\begin{theorem}[Wielandt]
	\label{theorem:STsubnormal}
	Sea $G$ un grupo finito y sean $S,T$ subgrupos subnormales. Entonces $S\cap T$ y 
	$\langle S,T\rangle$ son subnormales en $G$.
\end{theorem}

\begin{proof}
	Demostremos primero que $S\cap T$ es subnormal en $G$. Como la
	subnormalidad es transivitiva, basta ver que $S\cap T$ es subnormal en $T$.
	Como $S$ es subnormal en $G$, existe una sucesión 
	\[
		S=S_0\triangleleft S_1\triangleleft \cdots\triangleleft S_k=G.
	\]
	Cada $S_{j-1}\cap T$ es normal en $S_j\cap T$ y luego $S\cap T$ es
	subnormal en $T$.
	
	
	Para demostrar que $\langle S,T\rangle$ es subnormal en $G$ 
	procederemos por inducción en $|G|$. Supongamos que $G\ne1$ y sea $M$ un
	subgrupo minimal-normal de $G$. Sea $\pi\colon G\to G/M$ el morfismo
	canónico. Como $\pi(S)$ y $\pi(T)$ son subnormales en $G/M$ y $|G/M|<|G|$,
	la hipótesis inductiva implica que 
	\[
	\pi(\langle S,T\rangle M)=\pi(\langle S,T\rangle)=\langle \pi(S),\pi(T)\rangle
	\]
	es subnormal en $G/M$. Por la correspondencia, $\langle S,T\rangle M$ es
	subnormal en $G$. Por otro lado, el teorema~\ref{theorem:MsubsetNG(S)}
	implica que $M\subseteq N_G(S)$ y $M\subseteq N_G(T)$. Luego $M\subseteq
	N_G(\langle S,T\rangle)$. Como entonces $\langle S,T\rangle$ es normal en
	$\langle S,T\rangle M$ y $\langle S,T\rangle M$ es subnormal en $G$, se
	concluye que $\langle S,T\rangle$ es subnormal en $G$.
\end{proof}


%\section{Torres de automorfismos}
%
%\begin{exercise}
%	\label{exercise:Inn(G)}
%	Sea $G$ un grupo. Demuestre las siguientes afirmaciones:
%	\begin{enumerate}
%		\item La conjugación $\gamma\colon G\to\Inn(G)$, $g\mapsto \gamma_g$,
%			es morfismo con núcleo $Z(G)$.
%		\item $\Inn(G)$ es normal en $\Aut(G)$.
%		\item Si $Z(G)=1$ entonces $C_{\Aut(G)}(\Inn(G))=1$.
%		\item Si $Z(G)=1$ entonces $Z(\Aut(G))=1$.
%	\end{enumerate}
%\end{exercise}
%
%\begin{svgraybox}
%	\begin{enumerate}
%		\item Es morfismo pues
%			\[
%			\gamma_{gh}(x)=(gh)x(gh)^{-1}=\gamma_g\gamma_h(x)
%			\]
%			y el núcleo es
%			\[
%			\ker\gamma=\{g\in G:\gamma_g=\id\}=\{g\in G:gxg^{-1}=x\text{ para todo $x\in G$}\}=Z(G).
%			\]
%		\item Es trivial pues $f\gamma_gf^{-1}=\gamma_{f(g)}$.
%		\item Sea $f\in C_{\Aut(G)}(\Inn(G))$. Como
%			$f\gamma_gf^{-1}=\gamma_{f(g)}$, 
%	Sea $\sigma\in\Aut(\Aut(G))$. Sabemos por el
%	ejercicio~\ref{exercise:Inn(G)} que $\Inn(G)$ es normal en $\Aut(G)$, y
%	entonces $\sigma(\Inn(G))$ es normal en $\Aut(G)$. El grupo
%	\[
%	\Inn(G)\simeq G/Z(G)\simeq G
%	\]
%	es simple y $\sigma(\Inn(G))\cap \Inn(G)$ es normal en $\Inn(G)$; entonces
%	hay dos posibilidades: $\Inn(G)\cap\sigma(\Inn(G))=1$ o bien
%	$\Inn(G)\cap\sigma(\Inn(G))=\sigma(\Inn(G))$. Basta demostrar que
%	$\Inn(G)\cap\sigma(\Inn(G))\ne1$. 
%
%	Si $\Inn(G)\cap\sigma(\Inn(G))=1$ entonces, al usar la normalidad de
%	$\sigma(\Inn(G)$ y de $\Inn(G)$ en $\Aut(G)$, tendríamos
%	$[\Inn(G),\sigma(\Inn(G))]\in \Inn(G)\cap \sigma(\Inn(G))=1$.  Luego, por
%	el el ejercicio ~\ref{exercise:Inn(G)}, 
%	\[
%	G\simeq \Inn(G)\simeq\sigma(\Inn(G))\subseteq
%	C_{\Aut(G)}(\Inn(G))=1,
%	\]
%	una contradicción.
%\end{proof}
%
%\begin{theorem}
%	\label{theorem:simple=>completo}
%	Sea $G$ un grupo simple no abeliano. Entonces $\Aut(G)$ es completo.
%\end{theorem}
%
%\begin{proof}
%	Como $Z(G)=1$, el ejercicio~\ref{exercise:Inn(G)} implica que
%	$Z(\Aut(G))=1$. Por el lema~\ref{lemma:Inn(G)char}, $\Inn(G)$ es
%	característico en $\Aut(G)$. Queremos demostrar que 
%	$\Aut(G)\subseteq \Inn(G)$.
%	
%	Sean $\sigma\in\Aut(\Aut(G))$, $g\in G$ y $\gamma_g\in\Inn(G)$. Como
%	$\sigma(\Inn(G))\subseteq\Inn(G)$, existe $\alpha(g)\in G$ tal que
%	$\sigma(\gamma_g)=\gamma_{\alpha(g)}$. Queda definida entonces una función
%	$\alpha\colon G\to G$ tal que $\sigma(\gamma_g)=\gamma_{\alpha(g)}$ para
%	todo $g\in G$.
%
%	\begin{claim}
%		$\alpha\in\Aut(G)$. 
%	\end{claim}
%
%	Veamos que $\alpha$ es inyectiva: si $\alpha(g)=\alpha(h)$ entonces
%	\[
%	\sigma(\gamma_g)=\gamma_{\alpha(g)}=\gamma_{\alpha(h)}=\sigma(\gamma_h)
%	\]
%	y luego $\gamma_g=\gamma_h$; esto implica que $h=g$ pues
%	$\ker(\gamma)=Z(G)=1$.  Luego $\alpha$ es biyectiva. Además $\alpha$ es
%	morfismo pues 
%	\[
%	\gamma_{\alpha(gh)}=\sigma(\gamma_{gh})=\sigma(\gamma_h\gamma_h)=\sigma(\gamma_h)\sigma(\gamma_h)=\gamma_{\alpha(g)}\gamma_{\alpha(h)}.
%	\]
%
%	\begin{claim}
%		$\sigma=\gamma_\alpha$.
%	\end{claim}
%
%	Sea $\tau=\sigma\gamma_{\alpha}^{-1}$ y sea $h\in G$. Entonces
%	\[
%		\tau(\gamma_h)=\sigma\gamma_{\alpha}^{-1}\gamma_h
%		=\sigma(\alpha^{-1}\gamma_h\alpha)
%		=\sigma\gamma_{\alpha^{-1}(h)}
%		=\gamma_{\alpha\alpha^{-1}(h)}=\gamma_h.
%	\]
%	Si $\beta\in\Aut(G)$ y $g\in G$ entonces 
%	\[
%		\beta\gamma_g\beta^{-1}
%		=\gamma_{\beta(g)}
%		=\tau(\gamma_{\beta(g)})
%		=\tau(\beta)\tau(\gamma_g)\tau(\beta)^{-1}
%		=\tau(\beta)\gamma_g\tau(\beta)^{-1}.
%	\]
%	Como entonces $\tau(\beta)\beta^{-1}\in C_{\Aut(G)}(\Inn(G))=1$ para todo
%	$\beta$, $\tau=\id$ y luego $\sigma=\gamma_{\alpha}$.
%\end{proof}
%
%Si $G$ es un grupo con centro trivial entonces $G\hookrightarrow\Aut(G)$ pues 
%\[
%G\simeq
%G/Z(G)\simeq\Inn(G)\subseteq\Aut(G).
%\]
%Como también $\Aut(G)$ tiene centro trivial, al iterar este procedimiento
%obtenemos una sucesión
%\begin{equation}
%	\label{equation:Aut(G)}
%G\hookrightarrow\Aut(G)\hookrightarrow\Aut(\Aut(G))\hookrightarrow\cdots
%\end{equation}
%Como aplicación del concepto de subnormalidad veremos un teorema de Wielandt
%que afirma que la sucesión~\eqref{equation:Aut(G)} se estabiliza. 
%
%\begin{lemma}
%	\label{lemma:CG(S)=1}
%	Sea $G$ un grupo y sea $S=S_1\triangleleft
%	S_2\triangleleft\cdots\triangleleft S_r=G$.  Si $C_{S_{i+1}}(S_i)=1$ para
%	todo $i\in\{1,\dots,r-1\}$ entonces $C_G(S)=1$. 
%\end{lemma}
%
%\begin{proof}
%	Procederemos por inducción en $r$. El caso $r=2$ es trivial pues
%	$C_{G}(S)=C_{S_2}(S_1)=1$. Supongamos entonces que $r>2$. Al usar la
%	hipótesis inductiva al grupo $S_{r-1}$ obtenemos 
%	$C_G(S)\cap S_{r-1}=C_{S_{r-1}}(S)=1$.
%	Como $S_1$ es normal en $S_2$, $C_{G}(S)$ también es normal en $S_2$ pues
%	si $x\in C_G(S)$, $s_1\in S_1$, $s_2\in S_2$ entonces $s_2^{-1}s_1s_2\in
%	S_1$ y luego 
%	\[
%		[s_2xs_2^{-1},s_1]=s_2x(s_2^{-1}s_1s_2)x^{-1}s_2^{-1}s_1^{-1}=1.
%	\]
%	La normalidad de $S_{r-1}$ en $G$ implica que 
%	$[C_G(S),S_2]\subseteq C_G(S)\cap S_{r-1}=1$.
%	Luego $C_G(S)\subseteq C_G(S_2)$
%	Al usar la hipótesis inductiva en la sucesión 
%	$S_2\triangleleft\cdots\triangleleft S_r=G$, se concluye que $C_G(S)=1$.
%\end{proof}
%
%\begin{lemma}
%	\label{lemma:CG(N)=Z(N)}
%	Sea $N$ un subgrupo normal de un grupo finito $G$ tal que $C_G(N)\subseteq
%	N$. Entonces $|G|$ divide a $|Z(N)||\Aut(N)|$. En particular, $|G|$ divide
%	al factorial de $|N|$.
%\end{lemma}
%
%\begin{proof}
%	Al hacer actuar a $G$ en $N$ por conjugación obtenemos un morfismo
%	$\rho\colon G\to\Aut(N)$ con núcleo
%	\[
%	\ker\rho=\{g\in G:gng^{-1}=n\text{ para todo $n\in N$}\}=C_G(N).
%	\]
%	Como $C_G(N)\subseteq N$, $\ker\rho=C_G(N)=Z(N)$ y luego $G/Z(N)$ es
%	isomorfo a un subgrupo de $\Aut(N)$. 
%
%	Por el teorema de Lagrange, $|Z(N)|$ divide a $|N|$. Como 
%	$\Aut(N)$ actúa fielmente en el conjunto $N\setminus\{1\}$, se tiene 
%	un morfismo inyectivo $\Aut(N)\to\Sym_{|N|-1}$. Luego $|G|$ divide a $|N|!=|N|(|N|-1)!$.
%	% recordemos que actuar fielmente quiere decir que $f\cdot n=n$ para todo $n$ implica que $f=\id$. 
%\end{proof}
%
%Recordemos que si $G$ es un grupo, existe un único subgrupo normal minimal
%$G^{\infty}$ con la siguiente propiedad: el cociente $G/G^{\infty}$ es
%nilpotente.
%
%\begin{lemma}
%	\label{lemma:Ginf=Sinf}
%	Sea $G$ un grupo finito tal que $G=SF$ para algún subgrupo $S$ subnormal en
%	$G$ y algún subgrupo nilpotente $F$ normal en $G$. Entonces
%	$G^{\infty}=S^{\infty}$.
%\end{lemma}
%
%\begin{proof}
%	Sin perder generalidad podemos suponer que $S\ne G$.
%\end{proof}